\documentclass[a4paper, oneside, dvipsnames, table]{article}
%openany
\usepackage{../templatelatex/Stiletemplate}
\usepackage{hyperref}
\usepackage{fancyhdr}
\usepackage[italian]{babel}
\newcommand{\Data}{2019-12-13}

\newcommand{\Titolo}{Verbale Riunione \Data}

\newcommand{\Gruppo}{qbteam}

\newcommand{\Redattori}{Mattei Christian}

\newcommand{\Verificatori}{Drago Francesco}

\newcommand{\Approvatore}{Azzalin Tommaso \newline Salmaso Enrico}

\newcommand{\Distribuzione}{Vardanega Tullio \newline Cardin Riccardo \newline Gruppo qbteam}

\newcommand{\Uso}{Interno}

\newcommand{\NomeProgetto}{Stalker}

\newcommand{\Mail}{qbteamswe@gmail.com}

\newcommand{\DescrizioneDoc}{Questo documento si occupa di riportare quanto discusso nella riunione del \Data}

\newcommand{\pathimg}{../../../Template/Immagini/qbteam.png}

\newcommand{\Versionedoc}{1.0.0}
\renewcommand\thesection{}


\begin{document}

\copertina{}
\newpage

\section*{Registro delle modifiche}
{
\rowcolors{2}{grigetto}{white}
\renewcommand{\arraystretch}{1.5}
\centering
\begin{longtable}{C{2cm} C{2cm}  C{3cm}  C{3cm} C{4.5cm}}
\rowcolor{rossoep}
\textcolor{white}{\textbf{Versione}} & \textcolor{white}{\textbf{Data}} & \textcolor{white}{\textbf{Nominativo}} & \textcolor{white}{\textbf{Ruolo}} & \textcolor{white}{\textbf{Descrizione}}\\	
\endhead
1.2.0 & 2020-03-07 & \CE{} & Verificatore & Verifica del documento. \\

1.1.4 & 2020-02-16 & \SE{} & Amministratore & Aggiornati 4.3.2 e 4.3.3 VERIFICATO DA \LD{}\\

1.1.3 & 2020-02-15 & \SE{} & Amministratore & Aggiornato 4.2.2 VERIFICATO DA \BR{}\\

1.1.5 & 2020-02-16 & \SE{} & Amministratore & Aggiunto 2.2.5.5 VERIFICATO DA \BR{}. \\

1.1.4 & 2020-02-15 & \SE{} & Amministratore & Aggiornato 4.2.2 VERIFICATO DA \BR{}. \\

1.1.3 & 2020-02-14 & \SE{} & Amministratore & Aggiunta 2.2.6 VERIFICATO DA \LD{}. \\

1.1.2 & 2020-02-12 & \SE{} & Amministratore & Aggiornamento 4.2.4 VERIFICATO DA \LD{}. \\ 

1.1.1 & 2020-02-12 & \BR{} & Amministratore & Aggiornamento 3.1 VERIFICATO DA \LD{}. \\ 

1.1.0 & 2020-02-12 & \LD{} & Verificatore & Verifica VERIFICATO DA \LD{}.  \\ 

1.0.3 & 2020-02-12 & \BR{} & Amministratore & Aggiunto e verificato 3.4.2 VERIFICATO DA \LD{}. \\ 

1.0.2 & 2020-02-12 & \SE{} & Amministratore & Aggiunte e verificate metriche SFIN e SFOUT VERIFICATO DA \LD{}. \\ 

1.0.1 & 2020-02-12 & \SE{} & Amministratore & Modificati e verificati i paragrafi 2.2.4.1 e 2.2.4.2 VERIFICATO DA \LD{}. \\ 

1.0.0 & 2020-01-13 & \AT{} & Amministratore & Approvazione per il rilascio.  \\

0.2.0 & 2020-01-13 & \PF{}, \CE{} & Verificatori & Verifica documento.  \\ 

0.1.9 & 2020-01-13 & \CE{} & Amministratore & Aggiunta del template dei digrammi UML dei casi d'uso. \\

0.1.8 & 2020-01-13 & \BR{} & Amministratore & Modifica dei casi d'uso d'errore, test di sistema. \\

0.1.7 & 2020-01-13 & \AT{} & Amministratore & Revisione Introduzione, Processo di fornitura, sviluppo, attività di codifica e di progettazione, processi organizzativi, documentazione. \\

0.1.6 & 2020-01-12 & \MC{} & Amministratore & Revisione e modifica strutturale dei capitoli del documento. \\

0.1.5 & 2020-01-12 & \AT{} & Amministratore & Modifica Processi Primari. \\

0.1.4 & 2020-01-11 & \MC{} & Amministratore & Stesura capitolo gestione della qualità. \\

0.1.3 & 2020-01-10 & \MC{} & Amministratore & Revisione documentazione nei Processi di supporto. \\

0.1.2 & 2020-01-06 & \AT{} & Amministratore & Modifica del processo di verifica, validazione, piano di qualifica. \\

0.1.1 & 2020-01-06 & \AT{} & Amministratore & Modifica del processo di verifica. \\

0.1.0 & 2019-12-23 & \PF{}, \CE{} & Verificatori & Verifica del documento. \\

0.0.11 & 2019-12-22 & \PF{} & Amministratore & Stesura delle sottosezioni introduzione e scopo della sezione processi di sviluppo. \\

0.0.10 & 2019-12-22 & \PF{}  & Amministratore & Stesura Sviluppo dei processi primari. \\

0.0.9 & 2019-12-21 & \PF{} & Amministratore & Stesura delle sottosezioni Gestione della qualità, Verifica e Validazione della sezione processi di supporto. \\

0.0.8 & 2019-12-20 & \MC{} & Amministratore & Modifica descrizione repository, Studio di fattibilità. \\

0.0.7 & 2019-12-19 & \SE{} & Amministratore & Revisione del documento fino ad ora redatto. \\

0.0.6 & 2019-12-19 & \CE{} & Amministratore & Modificata la sezione dei casi d’uso con le decisioni prese per la loro nomenclatura il 2019-12-18. \\

0.0.5 & 2019-12-15 & \SE{} & Amministratore & Aggiunta parte di progettazione. \\

0.0.4 & 2019-12-15 & \BR{}, \PF{}  & Amministratori & Aggiunta parte processi organizzativi, gestione delle risorse prodotte. \\

0.0.3 & 2019-12-15 & \MC{} & Amministratore & Stesura studio di fattibilità. \\

0.0.2 & 2019-12-14 & \CE{} & Amministratore & Aggiunta parte relativa all’Analisi dei requisiti. \\

0.0.1 & 2019-12-14 & \CE{} & Amministratore & Creato il documento. \\
		
\end{longtable}
}

\fancyglossario{}

\clearpage
\tableofcontents
\clearpage

\section{A}
\textbf{Autenticazione:}\\
È l’azione che conferma la verità di un attributo di un singolo dato o di un’informazione sostenuta vera da un’entità. In informatica è un processo nella quale un computer, sistema informatico o un utente verifica la corretta identità di un altro software, computer o utente che vuole comunicare attraverso una connessione, autorizzandolo  ad utilizzare eventuali servizi associati.


\section{L}
\textbf{Lightweight Directory Access Protocol (LDAP:}\\
È un protocollo standard per l'interrogazione e la modifica dei servizi di directory. Le informazioni vengono raggruppate e possono essere espresse come record di dati ed organizzate in maniera gerarchica.

\section{M}
\textbf{Modalità anonima:}\\
È una specifica progettuale che da la possibilità all’utente che usufruisce l’applicazione di non essere tracciato precisamente dove si trova, ma segnala la sua eventuale presenza in una determinata area. I dati identificativi del soggetto non saranno visibili nel server se si parla di “utente evento pubblico”; al contrario se è un “utente aziendale”, perché è necessario sapere la sua identità per contare le ore lavorative.


\end{document}