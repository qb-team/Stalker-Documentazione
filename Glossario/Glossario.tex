\documentclass[a4paper, oneside, dvipsnames, table]{article}
%openany
\usepackage{../templatelatex/Stiletemplate}
\usepackage{hyperref}
\usepackage{fancyhdr}
\usepackage[italian]{babel}
\newcommand{\Data}{2019-12-13}

\newcommand{\Titolo}{Verbale Riunione \Data}

\newcommand{\Redattori}{\MC{}}

\newcommand{\Verificatori}{\DF{}}

\newcommand{\Approvatore}{\AT{} \newline \SE{}}

\newcommand{\Distribuzione}{\VT{} \newline \CR{} \newline Gruppo \Gruppo{}}

\newcommand{\Uso}{Interno}

\newcommand{\DescrizioneDoc}{Questo documento si occupa di riportare quanto discusso nella riunione del \Data}

\newcommand{\pathimg}{../../../Utilita/Immagini/qbteam.png}

\newcommand{\Versionedoc}{1.0.0}
\renewcommand\thesection{}


\begin{document}

\copertina{}
\newpage

\section*{Registro delle modifiche}
{
\rowcolors{2}{grigetto}{white}
\renewcommand{\arraystretch}{1.5}
\centering
\begin{longtable}{ c c  C{2.3cm} c C{3cm} C{3.2cm}}
\rowcolor{rossoep}
\textcolor{white}{\textbf{Versione}} & \textcolor{white}{\textbf{Data}} & \textcolor{white}{\textbf{Nominativo}} & \textcolor{white}{\textbf{Ruolo}} & 
\textcolor{white}{\textbf{Verificatore}}& \textcolor{white}{\textbf{Descrizione}}\\	


1.0.0 & \Data & \CE{} & Responsabili & \CE{} & Approvazione per il rilascio.  \\
		
0.0.1 & \Data & \DF{} & Analista & \AT{} & Stesura e verifica del documento.  \\
		
		
\end{longtable}
}

\fancyglossario{}

\clearpage
\tableofcontents
\clearpage

\section{A}
\textbf{Autenticazione:}\\
È l’azione che conferma la verità di un attributo di un singolo dato o di un’informazione sostenuta vera da un’entità. In informatica è un processo nella quale un computer, sistema informatico o un utente verifica la corretta identità di un altro software, computer o utente che vuole comunicare attraverso una connessione, autorizzandolo  ad utilizzare eventuali servizi associati.


\section{L}
\textbf{Lightweight Directory Access Protocol (LDAP:}\\
È un protocollo standard per l'interrogazione e la modifica dei servizi di directory. Le informazioni vengono raggruppate e possono essere espresse come record di dati ed organizzate in maniera gerarchica.

\section{M}
\textbf{Modalità anonima:}\\
È una specifica progettuale che da la possibilità all’utente che usufruisce l’applicazione di non essere tracciato precisamente dove si trova, ma segnala la sua eventuale presenza in una determinata area. I dati identificativi del soggetto non saranno visibili nel server se si parla di “utente evento pubblico”; al contrario se è un “utente aziendale”, perché è necessario sapere la sua identità per contare le ore lavorative.


\end{document}