\section{M}
\textbf{Machine Learning}\\
Apprendimento automatico di differenti meccanismi che permettono a una macchina intelligente di migliorare le proprie capacità e prestazioni nel tempo. La macchina, quindi, sarà in grado di imparare a svolgere determinati compiti migliorando, tramite l’esperienza, le proprie capacità, le proprie risposte e funzioni. \\ \\
\textbf{MainNet}\\
Il network originale e principale per le transazioni dei Bitcoin, in cui la criptovaluta ha un reale valore economico. \\ \\
\textbf{Maturità}\\
Capacità di un prodotto software di evitare che si verifichino errori errori o siano prodotti risultati non corretti in fase di esecuzione. \\ \\
\textbf{Milestone}\\
Momento nel ciclo di vita del software in cui è fissato il raggiungimento di un obiettivo specifico, a cui corrisponde una o più baseline.\\ \\
\textbf{Modalità di tracciamento anonimo DA MODIFICARE}\\
È una specifica progettuale che da la possibilità all’utente che usufruisce l’applicazione di non essere tracciato precisamente dove si trova, ma segnala la sua eventuale presenza in una determinata area. I dati identificativi del soggetto non saranno visibili nel server se si parla di “utente evento pubblico”; al contrario se è un “utente aziendale”, perché è necessario sapere la sua identità per contare le ore lavorative. \\ \\

\textbf{Modalità di tracciamento autenticato}\\
\textbf{Modificabilità}\\
Capacità di un prodotto software di consentire lo sviluppo di modifiche al software originale. L'implementazione include modifiche al codice, alla progettazione e alla documentazione.\\ \\
\textbf{Movimento}\\
Per movimento si intende una azione fisica di ingresso o di uscita nei luoghi dell'organizzazione che viene effettuata dall'utente. \\ \\
\clearpage