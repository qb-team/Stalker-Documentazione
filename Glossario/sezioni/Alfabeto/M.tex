\section{M}
\textbf{Machine Learning}\\
Apprendimento automatico di differenti meccanismi che permettono a una macchina intelligente di migliorare le proprie capacità e prestazioni nel tempo. La macchina, quindi, sarà in grado di imparare a svolgere determinati compiti migliorando, tramite l’esperienza, le proprie capacità, le proprie risposte e funzioni. \\ \\
\textbf{MainNet}\\
Il network originale e principale per le transazioni dei Bitcoin, in cui la criptovaluta ha un reale valore economico. \\ \\
\textbf{Maturità}\\
Capacità di un prodotto software di evitare che si verifichino errori errori o siano prodotti risultati non corretti in fase di esecuzione. \\ \\
\textbf{Merge}\\
In italiano "fusione", è un comando di Git che permette di unire due rami (branch), includendo le modifiche eseguite a carico di un ramo in un altro. \\ \\
\textbf{Milestone}\\
Momento nel ciclo di vita del software in cui è fissato il raggiungimento di un obiettivo specifico, a cui corrisponde una o più baseline.\\ \\
\textbf{Modalità di tracciamento anonimo}\\
È una specifica progettuale che da la possibilità all’utente che usufruisce dell’applicazione di essere tracciato all'interno dei luoghi dell'organizzazione ma nascondendo la sua identità fisica (nome, cognome e altri dati personali reali) al servizio. Per poter riconoscere la presenza dell'utente all'interno del luogo, sarà associato un codice univoco che non identifica in alcun modo la sua identità fisica ma solo la sua presenza.\\ \\
\textbf{Modalità di tracciamento autenticato}\\
È una specifica progettuale che da la possibilità all’utente che usufruisce dell’applicazione di essere tracciato all'interno dei luoghi dell'organizzazione. Questa modalità riconosce l'identità fisica dell'utente tracciato (nome, cognome e altri dati personali reali). \\ \\
\textbf{Modificabilità}\\
Capacità di un prodotto software di consentire lo sviluppo di modifiche al software originale. L'implementazione include modifiche al codice, alla progettazione e alla documentazione.\\ \\
\textbf{Movimento}\\
Per movimento si intende un'attività di spostamento fisico di ingresso o di uscita in un luogo di un'organizzazione, effettuata dall'utente anonimo o riconosciuto. \\ \\
\clearpage
