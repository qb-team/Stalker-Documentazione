\section{M}
\TermineGlossario{Machine Learning}
\DefinizioneGlossario{Apprendimento automatico di differenti meccanismi che permettono a una macchina intelligente di migliorare le proprie capacità e prestazioni nel tempo. La macchina, quindi, sarà in grado di imparare a svolgere determinati compiti migliorando, tramite l’esperienza, le proprie capacità, le proprie risposte e funzioni.}

\TermineGlossario{MainNet}
\DefinizioneGlossario{Il network originale e principale per le transazioni dei Bitcoin, in cui la criptovaluta ha un reale valore economico.}

\TermineGlossario{Markdown}
\DefinizioneGlossario{Linguaggio di markup molto semplice, per realizzare con semplicità contenuti testuali. Viene utilizzato dal gruppo per realizzare i manuali utente e manutentore, ma viene utilizzato implicitamente anche su GitHub quando si crea un issue, in quanto linguaggio di default.}

\TermineGlossario{Maturità}
\DefinizioneGlossario{Capacità di un prodotto software di evitare che si verifichino errori errori o siano prodotti risultati non corretti in fase di esecuzione.}

\TermineGlossario{Maven}
\DefinizioneGlossario{È uno strumento di build automation utilizzato prevalentemente nella gestione di progetti Java.}

\TermineGlossario{Merge}
\DefinizioneGlossario{In italiano "fusione", è un comando di Git che permette di unire due rami (branch), includendo le modifiche eseguite a carico di un ramo in un altro.}

\TermineGlossario{Milestone}
\DefinizioneGlossario{Momento nel ciclo di vita del software in cui è fissato il raggiungimento di un obiettivo specifico, a cui corrisponde una o più baseline.}

\TermineGlossario{Mkdocs}
\DefinizioneGlossario{Strumento di generazione di documentazione che permette al gruppo di realizzare i due manuali, utente e manutentore, sotto forma di sito web utilizzando come linguaggio prevalente Markdown. Oltre a Markdown è possibile utilizzare HTML.}

\TermineGlossario{Modalità di tracciamento anonimo}
\DefinizioneGlossario{È una specifica progettuale che da la possibilità all’utente che usufruisce dell’applicazione di essere tracciato all'interno dei luoghi dell'organizzazione ma nascondendo la sua identità fisica (nome, cognome e altri dati personali reali) al servizio. Per poter riconoscere la presenza dell'utente all'interno del luogo, sarà associato un codice univoco che non identifica in alcun modo la sua identità fisica ma solo la sua presenza.}

\TermineGlossario{Modalità di tracciamento autenticato}
\DefinizioneGlossario{È una specifica progettuale che da la possibilità all’utente che usufruisce dell’applicazione di essere tracciato all'interno dei luoghi dell'organizzazione. Questa modalità riconosce l'identità fisica dell'utente tracciato (nome, cognome e altri dati personali reali).}

\TermineGlossario{Modificabilità}
\DefinizioneGlossario{Capacità di un prodotto software di consentire lo sviluppo di modifiche al software originale. L'implementazione include modifiche al codice, alla progettazione e alla documentazione.}

\TermineGlossario{Movimento}
\DefinizioneGlossario{Per movimento si intende una azione fisica di ingresso o di uscita nei luoghi dell'organizzazione che viene effettuata dall'utente.}

\TermineGlossario{MySQL}
\DefinizioneGlossario{È un database relazionale open source composto da un client a riga di comando e un server. Supporta linguaggi come Java, PHP, Python, ecc.}

\TermineGlossario{MyStalkersList}
\DefinizioneGlossario{Nome della sezione dell'applicazione utenti contenente la lista delle organizzazioni abilitate al tracciamento dell'utente all'interno delle loro strutture, altresì detta "Lista dei preferiti".}
\clearpage
