\section{D}
\textbf{Dapps (decentralized application)}\\
È un sistema decentralizzato dove ogni nodo ha la stessa importanza e non esiste un'entità centrale dominante con poteri di decisione. Le applicazioni decentralizzate sono applicazioni eseguite su una rete peer to peer di computer e funzionano su Internet senza essere controllate da una singola entità. I Dapps sono stati resi popolari grazie alle tecnologie di contabilità distribuita come Ethereum blockchain, dove i Dapps sono spesso definiti “smart contract”. \\ \\
\textbf{Dashboard}\\
È un tipo di interfaccia utente grafica che mostra informazioni generiche ed è caratterizzata dalla facilità di lettura ed immediatezza, consentendo al management di agire tempestivamente prendendo decisioni corrette. \\ \\
\textbf{Dati dell'organizzazione\ap{G}}\\
È un sottoinsieme dei parametri dell'organizzazione\ap{G}, nel quale vengono esclusi il perimetro di tracciamento\ap{G}dell'organizzazione\ap{G} e l'insieme di luoghi di tracciamento. \\ \\
\textbf{Disciplinato}\\
Comportamento conforme a regole e strategie ben definite.
\textbf{Dispatcher}\\
È un modulo del sistema operativo che ha la responsabilità di: estrazione dei messaggi in arrivo dai canali sottostanti, della loro conversione in chiamate al metodo nel codice dell'applicazione e della restituzione dei risultati al chiamante. \\ \\
\textbf{Docker}\\
È un progetto open source che automatizza il deployment di applicazioni all'interno di contenitori software, fornendo un'astrazione aggiuntiva grazie alla virtualizzazione a livello di sistema operativo di Linux. \\ \\
\clearpage