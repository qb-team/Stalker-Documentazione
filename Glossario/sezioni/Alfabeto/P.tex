\section{P}
\textbf{Parametri (dell'organizzazione)}\\
Con parametri dell'organizzazione si intende l'insieme di:
\begin{itemize}
    \item Nome dell'organizzazione;
    \item Immagine dell'organizzazione;
    \item Descrizione dell'organizzazione;
    \item Indirizzo dell'organizzazione;
    \item Perimetro di tracciamento dell'organizzazione;
    \item L'insieme di luoghi di tracciamento.
\end{itemize} 
~\\
\textbf{Perimetro di tracciamento dell'organizzazione}\\
È la superficie geografica dove gli utenti dell'applicazione verranno tracciati per l'organizzazione in questione.\\ \\
\textbf{Platform as a Service (PAAS)}\\
È un un servizio cloud tramite il quale un provider mette a disposizione un ambiente di sviluppo e degli appositi strumenti per ideare nuove applicazioni. \\ \\
\textbf{Plugin}\\
È un programma non autonomo che interagisce con un altro programma per ampliarne o estenderne le funzionalità originarie. \\ \\
\textbf{PostgreSQL}\\
È un sistema di gestione di database open source ad oggetti. \\ \\
\textbf{Prestazionale}\\
In relazione a un requisito esprime dei vincoli di economicità che il prodotto deve rispettare.\\ \\
\textbf{Privilegi}\\
I privilegi sono fondamentalmente i vari tipi di amministratore. Il privilegio più basso è visualizzatore, poi gestore e infine proprietario. \\ \\
\textbf{Processo}\\ 
È l'insieme delle attività correlate e coese che trasformano i bisogni in prodotti (il risultato di un processo si chiama prodotto). Opera secondo regole consumando risorse. \\ \\
\textbf{Product Baseline}\\ 
Documentazione che descrive tutte le caratteristiche fisiche e funzionali implementate all'interno del prodotto, essa deve includere i diagrammi di classe, di sequenza e la contestualizzazione dei design pattern adottati. \\ \\
\textbf{Progettista}\\ 
Ha l'incarico di definire l'architettura alla base del sistema del prodotto software. Segue lo sviluppo e non la manutenzione del prodotto. \\ \\
\textbf{Programmatore}\\ 
Partecipa sia alla realizzazione che alla manutenzione del prodotto. È competente nella codifica e nella realizzazione di componenti necessarie all'esecuzione delle prove di verifica e validazione. Il codice prodotto dal programma deve essere mantenibile nel tempo.\\ \\
\textbf{Programmazione concorrente e distribuita}\\
La concorrenza è una caratteristica dei sistemi di elaborazione nei quali può verificarsi che un insieme di processi o sotto-processi (thread) computazionali sia in esecuzione nello stesso istante. La distribuzione indica genericamente una tipologia di sistema informatico costituito da un insieme di processi interconnessi tra loro in cui le comunicazioni avvengono solo esclusivamente tramite lo scambio di opportuni messaggi. \\ \\
\textbf{Proof of Concept (PoC)}\\
È un dimostratore eseguibile con lo scopo di rappresentare la Baseline per lo sviluppo del progetto. Il suo codice può essere usa-e-getta. \\ \\  %adattamento pagina pdf
\textbf{Python}\\
È un linguaggio di programmazione ad alto livello, orientato agli oggetti, adatto, tra gli altri usi, a sviluppare applicazioni distribuite, scripting, computazione numerica e system testing. \\ \\
\clearpage