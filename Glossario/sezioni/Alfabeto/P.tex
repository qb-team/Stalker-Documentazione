\section{P}
\textbf{PAAS (Platform as a Service)}\\
È un un servizio cloud tramite il quale un provider mette a disposizione un ambiente di sviluppo e degli appositi strumenti per ideare nuove applicazioni. \\ \\
\textbf{Parametri (dell'organizzazione\ap{G})}\\
Con parametri dell'organizzazione\ap{G} si intende l'insieme di:
\begin{itemize}
    \item Nome dell'organizzazione\ap{G};
    \item Immagine dell'organizzazione\ap{G};
    \item Descrizione dell'organizzazione\ap{G};
    \item Indirizzo dell'organizzazione\ap{G};
    \item Perimetro di tracciamento\ap{G}dell'organizzazione\ap{G};
    \item L'insieme di luoghi di tracciamento.
\end{itemize} \mbox{} \\ \\

\textbf{Perimetro di tracciamento\ap{G}dell'organizzazione\ap{G}}\\
È la superficie geografica dove gli utenti dell'applicazione verranno tracciati per l'organizzazione\ap{G} in questione.\\ \\
\textbf{Plugin}\\
È un programma non autonomo che interagisce con un altro programma per ampliarne o estenderne le funzionalità originarie. \\ \\
\textbf{PostgreSQL}\\
È un sistema di gestione di database open source ad oggetti. \\ \\
\textbf{Privilegi}\\
I privilegi sono fondamentalmente i vari tipi di amministratore. Il privilegio più basso è visualizzatore, poi gestore e infine proprietario. \\ \\
\textbf{Processo}\\ 
È l'insieme delle attività correlate e coese che trasformano i bisogni in prodotti (il risultato di un processo si chiama prodotto). Opera secondo regole consumando risorse. \\ \\
\textbf{Programmazione concorrente e distribuita}\\
La concorrenza è una caratteristica dei sistemi di elaborazione nei quali può verificarsi che un insieme di processi o sotto-processi (thread) computazionali sia in esecuzione nello stesso istante. La distribuzione indica genericamente una tipologia di sistema informatico costituito da un insieme di processi interconnessi tra loro in cui le comunicazioni avvengono solo esclusivamente tramite lo scambio di opportuni messaggi. \\ \\
\textbf{Python}\\
È un linguaggio di programmazione ad alto livello, orientato agli oggetti, adatto, tra gli altri usi, a sviluppare applicazioni distribuite, scripting, computazione numerica e system testing. \\ \\
\clearpage