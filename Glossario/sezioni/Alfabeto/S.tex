\section{S}
\textbf{Scalabilità orizzontale}\\
È l'abilità di aumentare la capacità collegando più entità hardware o software in modo che funzionino come una singola unità logica. Quando i server sono raggruppati , il server originale viene ridimensionato orizzontalmente. Se un cluster (insieme di computer) richiede più risorse per migliorare le prestazioni e fornire alta disponibilità, un amministratore può scalare aggiungendo più server al cluster. \\ \\
\textbf{Sea-level}\\
Categoria di operazioni con uno scopo unico unico e il fine di soddisfare una o più postcondizioni di UC X. \\ \\
\textbf{Serverless}\\
È un network la cui gestione non viene incentrata su dei server, ma viene dislocata fra i vari utenti che utilizzano il network stesso, quindi il lavoro di gestione del network viene eseguito dagli stessi utilizzatori. \\ \\
\textbf{Smart Contract Ethereum}\\
È il codice che viene eseguito sul Ethereum Virtual Machine (EVM). Esso può accettare ed archiviare Ethereum, dati o la combinazione di entrambi distribuendoli ad altri account o persino ad altri smart contract. \\ \\
\textbf{Snake Case}\\
Formato di scrittura di identificatori (tipicamente di nomi di file o di variabili) in cui le parole che li compongono, in minuscolo, sono separati da trattini bassi.
\textbf{Solidity}\\
È un linguaggio di programmazione orientato agli oggetti per la scrittura di smart contract. Viene utilizzato per implementare smart contract su varie piattaforme di blockchain, in particolare Ethereum. \\ \\
\textbf{Stakeholder}\\
Sono le persone influenti per il prodotto: dicono se una certa opportunità è buona. Possono essere chi usa il prodotto, chi compra il prodotto, chi sostiene i costi di realizzazione, chi verifica le esecuzioni dei processi. \\ \\
\textbf{Stream}\\
È un "canale" tra la sorgente e la destinazione attraverso il quale fluiscono i dati. \\ \\
\textbf{Studio di fattibilità}\\
Documento redatto per analizzare i capitolati d'appalto presentati dai vari proponenti.
In esso vengono descritte le ragioni che portano alla scelta o al rifiuto di intraprendere la fornitura del prodotto richiesto da un capitolato.
\textbf{Support Vector Machine (SVM)}\\
Sono modelli di apprendimento (machine learning) supervisionato con algoritmi di apprendimento associati che analizzano i dati utilizzati per l'analisi di classificazione e regressione. \\ \\
\clearpage