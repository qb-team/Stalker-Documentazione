\section{G}
\textbf{Geolocalizzazione}\\
Processo che permette l'individuazione geografica del luogo in cui si trova un determinato oggetto attraverso l'uso di apparecchiature in grado di trasmettere segnali a un satellite. \\ \\
\textbf{Gherkin}\\
È il linguaggio che Cucumber usa per definire i casi di test. È progettato per essere non tecnico e leggibile dall'uomo e descrive collettivamente i casi d'uso relativi a un sistema software. \\ \\
\textbf{Git}\\
È un software di controllo versione distribuito utilizzabile da interfaccia a riga di comando. \\ \\
\textbf{Git Flow}\\
È un’estensione di git che si basa sul modello di branching di Vincent Driessen. Git Flow descrive un modello di diramazione ben preciso costruito che rispecchia le fasi di rilascio del progetto.\\ \\
\textbf{Github}\\
È un servizio di hosting per progetti software. Il nome deriva dal fatto che è una implementazione dello strumento di controllo versione distribuito Git. \\ \\
\textbf{GitLab}\\
È una piattaforma web open source che permette la gestione di repository Git e di funzioni trouble ticket. \\ \\
\textbf{Global Positioning System (GPS)}\\
È un sistema di posizionamento e navigazione satellitare. Attraverso una rete dedicata di satelliti artificiali in orbita, fornisce a un terminale mobile o ricevitore GPS informazioni sulle sue coordinate geografiche e sul suo orario in ogni posto sulla Terra dove vi sia un contatto privo di ostacoli con almeno quattro satelliti del sistema. \\ \\
\textbf{Grafana}\\
È un sistema di analisi e monitoraggio open source per basi di dati e funziona come un'applicazione web. \\ \\
\textbf{Grafana}\\ 
È un sistema open source per l'automazione dello sviluppo che introduce un domain-specific language (DSL) basato su Groovy. Esso è  stato progettato per sviluppi multi-progetto 
che possono crescere fino a divenire abbastanza grandi e supporta sviluppi incrementali determinando in modo intelligente. \\ \\

\clearpage