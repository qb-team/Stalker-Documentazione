\section{G}
\textbf{Geo-localizzazione}\\
Processo che permette l'individuazione geografica del luogo in cui si trova un determinato oggetto attraverso l'uso di apparecchiature in grado di trasmettere segnali a un satellite. \\ \\
\textbf{Gherkin}\\
È il linguaggio che Cucumber usa per definire i casi di test. È progettato per essere non tecnico e leggibile dall'uomo e descrive collettivamente i casi d'uso relativi a un sistema software. \\ \\
\textbf{Git}\\
È un software di controllo versione distribuito utilizzabile da interfaccia a riga di comando. \\ \\
\textbf{Github}\\
È un servizio di hosting per progetti software. Il nome deriva dal fatto che è una implementazione dello strumento di controllo versione distribuito Git. \\ \\
\textbf{GitLab}\\
È una piattaforma web open source che permette la gestione di repository Git e di funzioni trouble ticket. \\ \\
\textbf{GPS (Global Positioning System)}\\
È un sistema di posizionamento e navigazione satellitare. Attraverso una rete dedicata di satelliti artificiali in orbita, fornisce a un terminale mobile o ricevitore GPS informazioni sulle sue coordinate geografiche e sul suo orario in ogni posto sulla Terra dove vi sia un contatto privo di ostacoli con almeno quattro satelliti del sistema. \\ \\
\textbf{Grafana}\\
È sistema di analisi e monitoraggio open source per basi di dati e funziona come un'applicazione web. \\ \\
\clearpage