\section{I}
\TermineGlossario{Infrastructure as a service (IaaS)} 
\DefinizioneGlossario{È un'infrastruttura informatica istantanea, fornita e gestita su Internet. Essa è un servizio cloud.}

\TermineGlossario{Indirizzo IP}
\DefinizioneGlossario{È un'etichetta numerica che identifica univocamente una macchina collegata ad una rete informatica che utilizza l'Internet Protocol come protocollo di rete per l'instradamento.}

\TermineGlossario{Integrated Development Environment (IDE)}
\DefinizioneGlossario{È un ambiente di sviluppo integrato, cioè è un software che, in fase di programmazione, supporta i programmatori nello sviluppo del codice sorgente di un programma. In fase di scrittura del codice segnala eventuali errori commessi dallo sviluppatore.}

\TermineGlossario{Incontro formale}
\DefinizioneGlossario{Riunione fra i membri del gruppo in cui si discutono temi riguardanti il progetto e si prendono delle decisioni.
In questi incontri, normalmente fisici, viene redatto un verbale con un segretario che si occupa della sua stesura e successiva pubblicazione nel repository della documentazione.}

\TermineGlossario{Ingresso}
\DefinizioneGlossario{Per ingresso presso un luogo di un'organizzazione si intende l'attività di spostamento fisico in cui un'utente passa da una posizione geografica non soggetta a tracciamento ad una interna ad un perimetro che delimita un luogo soggetto a tracciamento.}

\TermineGlossario{Interfaccia grafica (GUI, Graphical User Interface)}
\DefinizioneGlossario{È un tipo di interfaccia utente che consente l'interazione uomo-macchina in modo visuale utilizzando rappresentazioni grafiche piuttosto che riga di comando.}

\TermineGlossario{Interfaccia web}
\DefinizioneGlossario{È un'interfaccia che permette la visualizzazione di siti internet adeguando il computer alle necessità di interazione dell'utente, codificando e decodificando il linguaggio di programmazione e di demarcazione utilizzato nel web.}

\TermineGlossario{Internet of Things (IoT, internet delle cose)}
\DefinizioneGlossario{È un neologismo riferito all'estensione di Internet al mondo degli oggetti e dei luoghi concreti. Il concetto rappresenta una possibile evoluzione dell'uso della rete internet: gli oggetti si rendono riconoscibili e acquisiscono intelligenza grazie al fatto di poter comunicare dati su se stessi e accedere ad informazioni aggregate da parte di altri.}
\clearpage
