\section{I}
\textbf{IDE (Integrated Development Environment)}\\
È un ambiente di sviluppo integrato, cioè è un software che, in fase di programmazione, supporta i programmatori nello sviluppo del codice sorgente di un programma. In fase di scrittura del codice segnala eventuali errori commessi dallo sviluppatore. \\ \\
\textbf{Interfaccia grafica (GUI, Graphical User Interface)}\\
È un tipo di interfaccia utente che consente l'interazione uomo-macchina in modo visuale utilizzando rappresentazioni grafiche piuttosto che riga di comando. \\ \\
\textbf{Interfaccia web}\\
È un'interfaccia che permette la visualizzazione di siti internet adeguando il computer alle necessità di interazione dell'utente, codificando e decodificando il linguaggio di programmazione e di demarcazione utilizzato nel web. \\ \\
\textbf{Internet of things (IoT, internet delle cose)}\\
È un neologismo riferito all'estensione di Internet al mondo degli oggetti e dei luoghi concreti. Il concetto rappresenta una possibile evoluzione dell'uso della rete internet: gli oggetti si rendono riconoscibili e acquisiscono intelligenza grazie al fatto di poter comunicare dati su se stessi e accedere ad informazioni aggregate da parte di altri. \\ \\
\clearpage