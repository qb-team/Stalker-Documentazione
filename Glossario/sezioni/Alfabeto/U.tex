\section{U}
\textbf{UFT-8}\\
È una codifica che assegna a ogni carattere Unicode esistente una specifica sequenza di bit, che può essere letta anche come numero binario. Questo significa che UTF-8 assegna un numero binario fisso ad ogni lettera, numero e simbolo di un numero crescente di lingue. \\ \\
\textbf{UI (User Interface)}\\
È un'interfaccia uomo-macchina, ovvero ciò che si frappone tra una macchina e un utente, consentendone l'interazione reciproca. \\ \\
\textbf{UML (Unified Modeling Language)}\\
In ingegneria del software è un linguaggio di modellazione e di specifica basato sul paradigma orientato agli oggetti. Viene usato per descrivere soluzioni analitiche e progettuali in modo sintetico e comprensibile ad un vasto pubblico (standard industriale unificato). \\ \\
\textbf{Upselling}\\
È una tecnica di vendita in cui un venditore induce il cliente ad acquistare più articoli costosi, aggiornamenti o altri componenti aggiuntivi nel tentativo di effettuare una vendita più redditizia. \\ \\
\clearpage