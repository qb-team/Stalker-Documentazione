\section{A}
\TermineGlossario{Accesso} 
\DefinizioneGlossario{Per accesso si intende l'azione fisica di ingresso-uscita nei luoghi dell'organizzazione che viene effettuata dall'utente.}

\TermineGlossario{Accuratezza}
\DefinizioneGlossario{Capacità di un prodotto software di fornire i risultati o gli effetti attesi con il livello di precisione richiesta
dall'utente.}

\TermineGlossario{Adeguatezza}
\DefinizioneGlossario{Capacità di un prodotto software di fornire un appropriato insieme di funzioni che permettano all'utente di svolgere determinati task e di raggiungere gli obiettivi prefissati.}

\TermineGlossario{Amazon Web Services (AWS)}
\DefinizioneGlossario{Azienda statunitense di proprietà Amazon, che fornisce servizi di cloud computing su un'omonima piattaforma on demand.}

\TermineGlossario{Amministratore}
\DefinizioneGlossario{Ha l'incarico di controllare l'efficienza dell'ambiente di lavoro e di gestire tutti i documenti relativi al progetto. Si occupa, inoltre, della configurazione e versionamento del prodotto.}

\TermineGlossario{Analista}
\DefinizioneGlossario{Ha l'incarico di seguire il progetto dall'inizio fino alla fine e redige i documenti \SdF{} e \AdR{}. Il suo lavoro si basa nel conoscere a fondo il problema e definire i requisiti espliciti ed impliciti.}

\TermineGlossario{Analizzabilità}
\DefinizioneGlossario{Capacità di un prodotto software di facilitare la diagnosi sul software, individuando la causa di errori o malfunzionamenti qualora presenti.}

\TermineGlossario{Angular}
\DefinizioneGlossario{È un framework open source per lo sviluppo di applicazioni web scritto in TypeScript, HTML e CSS.}

\TermineGlossario{Apache Kafka}
\DefinizioneGlossario{È una piattaforma open source di stream processing scritta in Java e Scala, dove è possibile gestire feed di dati in tempo reale con bassa latenza ed alta velocità. Essa è usata principalmente per tutte le applicazioni di elaborazioni di stream.}

\TermineGlossario{API (Application programming interface)}
\DefinizioneGlossario{L'interfaccia di programmazione delle applicazioni sono set di definizioni e protocolli con i quali vengono realizzati ed integrati software applicativi. Consentono ai  prodotti o servizi di comunicare con altri prodotti o servizi senza sapere come vengono implementati, semplificando così lo sviluppo delle applicazioni.}

\TermineGlossario{Applicazione web}
\DefinizioneGlossario{Una applicazione distribuita web-based cioè fruibile via web per mezzo di un network, come Internet, che offre determinati servizi all'utente che la utilizza. Le applicazioni web a differenza delle altre applicazioni non necessitano di venire installate.}

\TermineGlossario{Asincrona}
\DefinizioneGlossario{Sono generiche attività che proseguono indisturbate ed il chiamante non deve attendere la fine dell'attività del codice richiamato.}

\TermineGlossario{Attrattività}
\DefinizioneGlossario{Capacità del prodotto software di risultare piacevole per l'utente. La qualità è relativa alla progettazione dell'aspetto grafico delle interfacce.}

\TermineGlossario{Autenticazione}
\DefinizioneGlossario{È l’azione che conferma la verità di un attributo di un singolo dato o di un’informazione sostenuta vera da un’entità. In informatica è un processo nella quale un computer, sistema informatico o un utente verifica la corretta identità di un altro software, computer o utente che vuole comunicare attraverso una connessione, autorizzandolo  ad utilizzare eventuali servizi associati.}
\clearpage