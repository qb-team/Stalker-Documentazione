\section{L}
\TermineGlossario{Librerie}
\DefinizioneGlossario{E' una raccolta di componenti che offrono servizi ad un livello di astrazione piuttosto basso, ovvero assemblare componenti semplici e predefiniti per ottenere strutture complesse specializzate.}

\TermineGlossario{Lightweight Directory Access Protocol (LDAP)}
\DefinizioneGlossario{È un protocollo standard per l'interrogazione e la modifica dei servizi di directory. Le informazioni vengono raggruppate e possono essere espresse come record di dati ed organizzate in maniera gerarchica.}

\TermineGlossario{Lista degli accessi}
\DefinizioneGlossario{Detta anche "storico degli accessi", contiene una lista degli ingressi o uscite che sono stati fatti dall'utente tracciato nei luoghi dell'organizzazione o più in generale nell'organizzazione. Essa contiene informazioni relative al luogo e all'organizzazione dove e avvenuto l'accesso, la data e l'ora è infine il tempo trascorso all'interno.}

\TermineGlossario{Lista delle organizzazioni}
\DefinizioneGlossario{Si intende un insieme di tutte le organizzazioni che utilizzano il servizio Stalker per tracciare le presenze delle persone all’interno dei propri luoghi, in tale lista ci saranno salvate le informazioni generali di ogni organizzazione.}

\TermineGlossario{Lista delle organizzazioni preferite}
\DefinizioneGlossario{Contiene una lista delle organizzazioni che l'utente ha deciso di inserire tra le sue preferite.}

\TermineGlossario{Logic Layer}
\DefinizioneGlossario{Il “business logic layer” è il punto in cui si affrontano i problemi che il programma è stato creato per risolvere. Nel logic layer, le classi decidono quali informazioni sono necessarie per risolvere i problemi assegnati, richiedono tali informazioni dal livello di accesso, manipolandole come richiesto e restituiscono i risultati finali al livello di presentazione per la formattazione.}

\TermineGlossario{Login}
\DefinizioneGlossario{Procedura di accesso effettuata dall'utente o dall'amministratore per accedere alla applicazione nel caso dell'utente, al Server nel caso dell'amministratore.}

\TermineGlossario{Logout}
\DefinizioneGlossario{Procedura di uscita effettuata dal utente o dall'amministratore per uscire dell'applicazione nel caso dell'utente o dal Server nel caso dell'amministratore.}

\TermineGlossario{Luogo di tracciamento}
\DefinizioneGlossario{Identifica una superficie di estensione contenuta locata geograficamente all'interno del perimetro di tracciamento dell'organizzazione. Un luogo non può fuoriuscire dal perimetro di tracciamento dell'organizzazione. Ciascun luogo è riconducibile ad una organizzazione.}

\clearpage