\section{L}
\textbf{Lightweight Directory Access Protocol (LDAP)}\\
È un protocollo standard per l'interrogazione e la modifica dei servizi di directory. Le informazioni vengono raggruppate e possono essere espresse come record di dati ed organizzate in maniera gerarchica. \\ \\
\textbf{Lista delle organizzazioni\ap{G}}\\
Si intende un insieme di tutte le organizzazioni\ap{G} che utilizzano il servizo Stalker per tracciare le presenze delle persone all’interno dei propri luoghi, in tale lista ci saranno salvate le informazioni generali di ogni organizzazione\ap{G}.\\ \\
\textbf{Lista delle organizzazioni\ap{G} preferite}\\
Contiene una lista delle organizzazioni\ap{G} che l'utente ha deciso di inserire tra le sue preferite.
\textbf{Logic Layer}\\
Il “business logic layer” è il punto in cui si affrontano i problemi che il programma è stato creato per risolvere. Nel logic layer, le classi decidono quali informazioni sono necessarie per risolvere i problemi assegnati, richiedono tali informazioni dal livello di accesso, manipolandole come richiesto e restituiscono i risultati finali al livello di presentazione per la formattazione. \\ \\
\textbf{Login\ap{G}} \\
Procedura di accesso effettuata dall'utente o dall'amministratore per accedere alla applicazione nel caso dell'utente, al Server nel caso dell'amministratore.\\ \\
\textbf{Logout\ap{G}} \\
Procedera di uscita effettuata dal utente o dall'amministartore per uscire dell'applicazione nel caso dell'utente o dal Server nel caso dell'amministratore. \\ \\
\textbf{Luogo di tracciamento}\\
Identifica una superficie di estensione contenuta locata geograficamente all'interno del perimetro di tracciamento\ap{G}dell'organizzazione\ap{G}. Un luogo non può fuoriuscire dal perimetro di tracciamento\ap{G}dell'organizzazione\ap{G}. Ciascun luogo è riconducibile ad una organizzazione\ap{G}.\\ \\
\textbf{Lista degli accessi\ap{G}}\\
Detta anche "storico degli accessi\ap{G}", contiene una lista degli ingressi o uscite che sono stati fatti dall'utente tracciato nei luoghi dell'organizzazione\ap{G} o più in generale nell'organizzazione\ap{G}. Essa contiene informazioni relative al luogo e all'organizzazione\ap{G} dove e avvenuto l'accesso, la data e l'ora è infine il tempo trascorso all'interno.\\ \\
\clearpage