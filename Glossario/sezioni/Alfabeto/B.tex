\section{B}
\textbf{Backend as a service (BAAS)}\\
È un modello per fornire agli sviluppatori di applicazioni web o mobile un modo per collegare le loro applicazioni a un backend cloud storage e API, esposte da applicazioni backend, fornendo allo stesso tempo funzioni quali la gestione degli utenti, le notifiche push, e l'integrazione con servizi di rete sociale. \\ \\
\textbf{Back end (server)}\\
Sono delle interfacce che hanno come destinatario un programma. Una applicazione back end è un programma con il quale l'utente interagisce indirettamente. In una struttura client/server il back end è il server. \\ \\
\textbf{Beacon (Bluetooth)}\\
Sono dei trasmettitori hardware (classe di dispositivi Bluetooth a bassa energia LE) che trasmettono il loro identificatore a dispositivi elettronici portatili vicini. \\ \\
\textbf{Behavior-Driven Development (BDD)}\\
È un processo di sviluppo software agile che incoraggia la collaborazione tra sviluppatori, QA (Quality Assurance) e partecipanti non tecnici o aziendali a un progetto software. Lo sviluppo basato sul comportamento combina le tecniche generali e i principi del TDD (Test-driven development) con idee di progettazione guidata dal dominio e analisi e progettazione orientate agli oggetti per fornire ai team di sviluppo e gestione del software strumenti condivisi e un processo condiviso per collaborare allo sviluppo del software. \\ \\
\textbf{Big Data}\\
Indica in maniera generica un'estesa raccolta di dati da richiedere tecnologie e metodi analitici specifici per l'estrazione di valore o conoscenza. Il termine è utilizzato in riferimento alla capacità di analizzare  e mettere in relazione un'enorme mole di dati eterogenei, strutturati e non, allo scopo di scoprire i legami tra fenomeni diversi e prevedere quelli futuri. \\ \\
\textbf{Bitbucket}\\
È un servizio di hosting web-based per progetti che usano i sistemi di controllo versione Mercurial o Git (è simile a GitHub). \\ \\
\textbf{Bootstrap}\\
È una raccolta di strumenti liberi per la creazione di siti e applicazioni web. Essa contiene modelli di progettazione basati su HTML e CSS, sia per la tipografia, ma anche per le varie componenti dell'interfaccia. \\ \\
\clearpage