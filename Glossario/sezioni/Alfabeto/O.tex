\section{O}
\textbf{OpenShift}\\
È una piattaforma container per le imprese basata su Kubernetes, che offre operazioni automatizzate in tutto lo stack per gestire deployment di cloud ibridi e multi-cloud. \\ \\
\textbf{Operabilità}\\
Capacità di un prodotto software di offrire delle funzioni coerenti con le aspettative dell’utente \\ \\
\textbf{Orange (Canvas)}\\
È un toolkit di visualizzazione dei dati open source, machine learning e data mining. È dotato di un front-end di programmazione visiva per l'analisi dei dati esplorativi e la visualizzazione interattiva dei dati. Orange è costituito da un'interfaccia canvas su cui l'utente posiziona i widget e crea un flusso di lavoro di analisi dei dati. \\ \\
\textbf{Ordinamento per data (decrescente)}\\
Considerando una data maggiore di un'altra più recente si intende che, dato un insieme di date (possibilmente con elementi ripetuti), una data più recente precede una meno recente. \\ \\
\textbf{Ordinamento per data (crescente)}\\
Considerando una data maggiore di un'altra più recente, si intende che, dato un insieme di date (possibilmente con elementi ripetuti), una data meno recente precede una più recente. \\ \\
\textbf{Organizzazione}\\
Soggetto che ha interesse a tracciare le presenze delle persone all’interno dei propri luoghi, in maniera anonima o autenticata. Ogni organizzazione ha associato un proprio nome e il nome e l'indirizzo dei vari luoghi appartenenti a essa \\ \\
\textbf{Organizzazione preferita}\\
Identifica un'organizzazione che è stata inserita all'interno della lista dei preferiti dall'utente. \\ \\
\clearpage