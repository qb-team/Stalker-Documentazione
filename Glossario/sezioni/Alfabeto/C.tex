\section{C}
\textbf{Clean architecture}\\
È una filosofia di progettazione software che separa gli elementi di un progetto in livelli ad anello. \\ \\
\textbf{CLI}\\
Acronimo di Command Line Interface. Si riferisce a un'interfaccia utente che permette di interagire con il sistema operativo e con gli altri programmi mediante la digitazione di comandi testuali. Alcuni esempi di CLI sono cmd o Command Prompt su SO Windows, bash su SO basati su Linux.
\textbf{ClickHouse}\\
È un sistema di gestione di database orientato alle colonne open source in grado di generare in tempo reale report di dati analitici utilizzando query SQL. \\ \\
\textbf{Cloud}\\
È uno spazio di archiviazione dove può essere accessibile in qualsiasi momento ed in ogni luogo utilizzando semplicemente una qualunque connessione ad Internet. \\ \\
\textbf{Cluster}\\
È un insieme di computer connessi tra loro tramite una rete telematica con lo scopo di distribuire un'elaborazione molto complessa tra i vari computer, aumentando la potenza di calcolo del sistema e/o garantendo una maggiore disponibilità di servizio. \\ \\
\textbf{Code generation}\\
Meccanismo dove il compilatore prende in input il codice sorgente e lo converte in codice macchina. \\ \\
\textbf{Commit}\\
Una serie di modifiche che sono stati esplicitamente convalidate. \\ \\
\textbf{Computation as a service (CaaS)}\\
Le risorse di elaborazione vengono fornite su richiesta tramite risorse virtuali o fisiche come servizio. \\ \\
\textbf{Configuration Item}\\
Parte che compone il prodotto software. Ha una propria identità, che è univoca. Le informazioni disponibili per il configuration item sono un identificativo univoco, un nome descrittivo, la data di creazione, l'autore, il registro delle modifiche e il suo stato corrente.\\ \\
\textbf{Criptovaluta}\\
È una rappresentazione digitale di valore basata sulla crittografia. \\ \\
\textbf{CSS3 (Cascading Style Sheets)}\\
È un linguaggio usato per definire la formattazione di documenti HTML, XHTML e XML ad esempio i siti web e relative pagine web; permette una programmazione più chiara e facile da utilizzare, sia per gli autori delle pagine stesse sia per gli utenti, garantendo anche il riutilizzo di codice e facilita la manutenzione. Le specifiche CSS3 sono costituite da sezioni separate dette "moduli" e hanno differenti stati di avanzamento e stabilità. \\ \\
\textbf{Cucumber}\\
È uno strumento software utilizzato dai programmatori di computer che supporta lo sviluppo basato sul comportamento (BDD). Esso consente l'esecuzione della documentazione di funzionalità scritta in testo rivolto alle imprese. Esegue test di accettazione automatizzati scritti in uno stile di sviluppo basato sul comportamento (Behavior-Driven Development, BDD). \\ \\
\clearpage