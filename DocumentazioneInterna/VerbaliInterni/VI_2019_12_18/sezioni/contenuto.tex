\section{Informazioni Generali}
\begin{itemize}
\item \textbf{Luogo:} aula 1BC45, presso Torre Archimede.
\item \textbf{Data:} \Data.
\item \textbf{Ora:} 13:00 - 15:30.
\item \textbf{Partecipanti del gruppo:}
	\begin{itemize}
		\item \AT; 
		\item \BR;
		\item \CE; 
		\item \DF;
		\item \LD;
		\item \MC;
		\item \PF;
		\item \SE.
	\end{itemize} 
\item \textbf{Segretario:} \BR.
\end{itemize}

\section{Ordine del Giorno}
\begin{itemize}
	\item Standardizzazione dei documenti alle norme di progetto e all'analisi dei requisiti;
	\item revisione degli attori dei casi d'uso.
\end{itemize}


\section{Resoconto}
\subsection{Standard della documentazione}
Sono stati decisi gli standard da usare per quanto riguarda le varie documentazioni in modo che i membri del gruppo possano lavorare in modo consistente ed uniforme.\\

\subsection{Attori casi d'uso}
Sono stati rivisti tutti gli attori relativi ai casi d'uso e la rispettiva nomenclatura.\\

\clearpage