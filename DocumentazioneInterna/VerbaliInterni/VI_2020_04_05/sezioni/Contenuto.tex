\section{Informazioni Generali}
\begin{itemize}
\item \textbf{Luogo:} Discord.
\item \textbf{Data:} \Data.
\item \textbf{Ora:} 15:20 - 16:00.
\item \textbf{Partecipanti del gruppo:}
	\begin{itemize}
		\item \AT{}; 
		\item \BR{};
		\item \CE{}; 
		\item \DF{};
		\item \LD{};
		\item \MC{};
		\item \PF{};
		\item \SE{}.
	\end{itemize} 
\item \textbf{Segretario:} \AT{}.
\end{itemize}

\section{Ordine del Giorno}
\begin{itemize}
	\item Modifica alla pianificazione;
	\item Nuovi ruoli di progetto;
	\item Connessione ai server del proponente più discussione sull'affittarne uno personale;
	\item Situazione sviluppo app, web-app, server;
	\item Requisiti soddisfatti sul totale.
\end{itemize}

\section{Resoconto}

\subsection{Modifica alla pianificazione}
Dopo la decisione avanzata al committente in data 2020-03-24 si indica di seguito la nuova pianificazione a cui segue la modifica del \PdP{}.
\begin{itemize}
	\item Nuova data di RQ: 18 maggio;
	\item Nuova data di consegna per RQ: 11 maggio;
	\item Nuova data entro cui terminare i requisiti obbligatori del prodotto: 3 maggio (in teoria dovrebbe coincidere con presentazione PB);
	\item Nuova data di RA: 18 giugno;
	\item Nuova data di consegna per RA: 11 giugno.
\end{itemize}

\subsection{Nuovi ruoli di progetto}
Vengono assegnati i nuovi ruoli di progetto in vigore dal 2020-04-21 fino al 2020-05-18.\newline
I nuovi ruoli principali (fino al 2020-05-03) sono:
\begin{itemize}
	\item \DF{}: \Amministratore{};
	\item \BR{}: \Responsabile{}.
\end{itemize}
Dal 2020-05-04 fino al 2020-05-18 i ruoli si invertiranno:
\begin{itemize}
	\item \BR{}: \Amministratore{};
	\item \DF{}: \Responsabile{};
\end{itemize}

\subsection{Connessione ai server del proponente più discussione sull'affittarne uno personale}
Il proponente, attraverso Emanuele, ha comunicato come potersi collegare al laboratorio aziendale per provare il sistema.
Viene visionato il tutorial fornitoci ma si discute la possibilità di acquistare per il tempo necessario un server esterno.

\subsection{Situazione sviluppo app, web-app, server}
Viene discusso fra tutti i membri del gruppo la situazione dello sviluppo.
Il gruppo dell'app afferma di aver portato avanti i requisiti degli incrementi 1 e 2 e hanno iniziato a vedere come svolgere i requisiti dell'incremento 3.
Si è deciso di adottare il pattern architetturale Model View Presenter dopo aver seguito le lezioni del corso di Ingegneria del Software. Attendono di potersi
collegare a un'istanza del backend funzionante per poter verificare a pieno quanto fatto.\\
Il gruppo della web-app afferma invece che ha, grazie a SonarQube, risolto alcuni errori e/o bug presenti nel codice del PoC che ha conseguentemente utilizzato 
come baseline per lo sviluppo dell'applicazione web. Hanno proseguito abilitando la Continuous Integration con la piattaforma Travis CI e il calcolo della code coverage
con Coveralls, verificandone il funzionamento implementando alcuni test di unità.
Il gruppo del server ha invece ultimato le API seguendo la specifica OpenAPI, ha generato il codice sorgente tramite lo strumento OpenAPI Generator e lo ha utilizzato per l'implementazione del server.
Ha deciso di studiare la funzionalità multi-module di Maven per separare i vari moduli del codice sorgente e ha studiato il funzionamento del pattern Publisher/Subscriber implementato da Redis.
Sta ultimando la progettazione dell'architettura del backend, di cui i diagrammi, e sta provando il funzionamento degli unit test.

\subsection{Requisiti soddisfatti sul totale}
Vengono soddisfatti, oltre a quelli già segnati in precedenza, i seguenti requisiti:
\begin{itemize}
	\item R1FA1.6;
	\item R1FA1.7;
	\item R1FA8.3;
	\item R1FA8.4;
	\item R1FA8.5;
	\item R1FA3.3;
	\item R1FA3.4;
	\item R1FA3.6;
	\item R1FA3.8;
	\item R1FA3.15;
	\item R1FA3.17.
\end{itemize}

\clearpage