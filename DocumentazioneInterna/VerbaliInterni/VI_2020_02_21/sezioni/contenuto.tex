\section{Informazioni Generali}
\begin{itemize}
\item \textbf{Luogo:} aula LU4, via Luigi Luzzatti.
\item \textbf{Data:} \Data.
\item \textbf{Ora:} 9:30 - 12:15.
\item \textbf{Partecipanti del gruppo:}
	\begin{itemize}
		\item \DF{};
		\item \MC{};
		\item \SE{}.
	\end{itemize} 
\item \textbf{Segretario:} \MC{}.
\end{itemize}

\section{Ordine del Giorno}
\begin{itemize}
	\item Discussione e creazione della struttura del \glo{Proof of Concept}.
\end{itemize}


\section{Resoconto}
\subsection{Analisi della struttura del Proof of Concept}
È stata analizzata e creata una struttura del \glo{Proof of Concept} basandosi sui casi d'uso presenti nel documento \AdR{}.

\subsection{Scelta delle funzionalità dell'applicazione}
Sono state scelte come funzionalità di base la rilevazione e trasmissione continua della posizione dell'utente che ha installato l'applicazione nel proprio dispositivo mobile. 
Esso dovrà anche accettare le condizioni d'uso al primo avvio dell'applicazione per sbloccare la rilevazione della posizione. L'applicazione deve ricevere in input (dal server) 
dei punti di coordinate geografiche che stabiliscono gli angoli della struttura dell'\glo{organizzazione} e da quelli sviluppare l'area dell'\glo{organizzazione}.
Se un utente generico entra nell'area dovrà ricevere una notifica che è entrato nell'\glo{organizzazione}.

\clearpage