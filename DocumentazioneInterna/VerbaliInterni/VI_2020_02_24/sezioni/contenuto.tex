\section{Informazioni Generali}
\begin{itemize}
\item \textbf{Luogo:} Discord.
\item \textbf{Data:} \Data.
\item \textbf{Ora:} 8:30 - 12:00.
\item \textbf{Partecipanti del gruppo:}
	\begin{itemize}
		\item \DF{};
		\item \MC{};
		\item \SE{};
		\item \BR{}.
	\end{itemize} 
\item \textbf{Segretario:} \MC{}.
\end{itemize}

\section{Ordine del Giorno}
\begin{itemize}
	\item Discussione generale lato applicazione;
	\item Organizzazione del lavoro da svolgere;
	\item Ricerca e testing di possibili tecniche o algoritmi da implementare nel \glo{Proof of Concept}.
\end{itemize}


\section{Resoconto}
\subsection{Approvazione tecnologie}
È stato approvato l'utilizzo del programma Android Studio come tecnologia standard per lo sviluppo del \glo{Proof of Concept} lato applicazione. L'applicazione sarà compatibile solo
con Android e con alcune versioni di esso.

\subsection{Organizzazione del lavoro}
Ogni membro che ha come compito lo sviluppo dell'applicazione dovrà imparare una base per poter iniziare ad utilizzare Android Studio, capire il corretto uso di alcune
librerie (come ad esempio la geolocalizzazione), come implementare ed adattare alcuni algoritmi ritenuti utili per \glo{Proof of Concept} e saper realizzare un'interfaccia
\glo{grafica} funzionante e responsiva.

\subsection{Ricerca e testing}
Ogni membro dovrà effettuare delle ricerche ed accertamenti sul funzionamento di alcune tecniche o algoritmi ritenuti utili da implementare nel codice del \glo{Proof of Concept}. 
Bisognerà anche verificare la compatibilità tra le versioni di Android e l'utilizzo di alcune tecniche integrabili. 

\clearpage