\section{Informazioni Generali}
\begin{itemize}
\item \textbf{Luogo:} aula 1BC45, presso Torre Archimede.
\item \textbf{Data:} \Data.
\item \textbf{Ora:} 13:00 - 13:30.
\item \textbf{Partecipanti del gruppo:}
	\begin{itemize}
		\item \AT{}; 
		\item \CE{}; 
		\item \DF{};
		\item \LD{};
		\item \MC{};
		\item \PF{};
		\item \SE{}.
	\end{itemize} 
\item \textbf{Segretario:} \SE{}.
\end{itemize}

\section{Ordine del Giorno}
\begin{itemize}
	\item Chiamata su Hangouts con Cardin per chiarire gli errori di alcuni casi d'uso.
\end{itemize}


\section{Resoconto}
\subsection{Spiegazione errori casi d'uso}
E' stata chiesta al professore Cardin, tramite videochiamata, una spiegazione di tutti gli errori riguardanti i casi d'uso segnati nella correzione dell'RR.\\
\subsection{Problemi riscontrati nei caso d'uso}
Di seguito vengono riportati tutti i problemi discussi con il professore Cardin durante la videochiamata:
\begin{itemize}
	\item In casi d’uso dove c’è bisogno che l’utente sia autentificato per poter poi compiere delle funzionalità, il professore ha consigliato di scriverle:
	" precondizioni: l’utente deve essere autenticato presso il sistema”
	UCA7 resta comunque un caso d’uso perché è una funzionalità che il nostro sistema fornisce;
	\item UCA3.3.1 e UCA3.3.2 sono mutualmente esclusivi quindi va bene che venga usata l’ereditarietà
	(dato che l’aggiornamento della lista delle organizzazioni viene fatto solo da una delle due tipologie, o viene fatta una o viene fatta l’altra).
	Stessa cosa per UCA3.4 (usare l’ereditarietà perché viene scelto SOLO UN metodo di visualizzazione della lista delle organizzazioni).
	Attenzione per i sotto casi d’uso di UCA3.4.3 e UCA3.4.4 (forse andrebbero tolti da UCA3.4 e messi come assestanti).
	Mentre per UCA3.5: una ricerca può essere fatta o per mutua esclusione (ereditarietà) o per composizione di filtri (che non è ereditarietà perché si possono effettuare tutte le varie ricerche o un sottoinsieme di esse [in questo caso: Per Nazione, Per Nome, Per Città]) non va usato per ereditarietà perché lo scenario principale accede a più sottoinsiemi;
	\item Nel caso d'uso UCA4, in UCA6.1 e UCA6.2 bisogna stare attenti perché l’attore primario non è l’utente ma è l’applicazione (che è il sistema), ed è essa che compie la funzionalità.
	Quindi Cardin, in minuto 13.27 circa, dice che toglierebbe UCA6 e porterebbe i suoi sotto casi, messi come inclusione, come casi d’uso propri.
	In 15:20 circa, Cardin dice che UCA6 è una funzionalità e quindi va bene che sia un caso d’uso, ma i suoi sotto casi non sono molto collegati con UCA4.
	Quindi riportare le inclusioni come propri casi d'uso;
	\item UCA4 e UCA5 le inclusioni non sono corrette perché stiamo usando i casi d’uso come dei diagrammi di attività. Invece dovrebbero essere usati ad esempio in precondizioni;
	\item UCA5.3 e UCA3.4 vanno tolti (viene detto al minuto 18:10 circa) e devono essere messe come precondizione;
	\item UCA5.1 ‘che cosa è stato visualizzato’ non bisogna scriverlo nelle POST ma sono suoi sotto casi d’uso per dire meglio questa funzionalità (vedere come gestire queste cose);
	\item Domanda 6. sulla fig.20: stiamo usando un linguaggio formale in un modo informale.
	Soluzione: spezzare e lasciare ogni caso d’uso con il suo diagramma senza raggruppare in panoramiche;
	\item UCS7.1: visualizzazione e ordinamento sono due funzionalità differenti, quindi vanno divisi dato che i sotto casi d’uso della visualizzazione possono specificare solo che cosa viene visualizzato
	Situazione già riscontrata nei filtri menzionati precedentemente.
\end{itemize}

\clearpage