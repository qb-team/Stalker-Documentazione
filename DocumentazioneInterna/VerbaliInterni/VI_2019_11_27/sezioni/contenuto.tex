\section{Informazioni Generali}
\begin{itemize}
\item \textbf{Luogo:} aula LUM250, presso Complesso Paolotti.
\item \textbf{Data:} \Data.
\item \textbf{Ora:} 16:00 - 17:15.
\item \textbf{Partecipanti del gruppo:}
	\begin{itemize}
		\item \AT{}; 
		\item \BR{};
		\item \CE{}; 
		\item \DF{};
		\item \LD{};
		\item \MC{};
		\item \PF{};
		\item \SE{}.
	\end{itemize} 
\item \textbf{Segretario:} \BR{}.
\end{itemize}


\section{Ordine del Giorno}
\begin{itemize}
	\item Scelta definitiva nome del gruppo;
	\item creazione indirizzo e-mail del gruppo;
	\item discussione sugli incontri tenuti dai proponenti delle aziende relativi ai capitolati C4 e C5;
	\item organizzazione\ap{G} degli incontri futuri;
	\item visione dei riassunti riguardanti i capitolati proposti dalle aziende.
\end{itemize}


\section{Resoconto}
\subsection{Nome del gruppo}
Si è deciso che il nome del gruppo sarà \Gruppo{}.

\subsection{Creazione documento Studio di Fattibilità}
\AT{} ed \CE{} sono stati incaricati di creare su Google Docs una prima versione dello studio di fattibilità.

\subsection{Logo del gruppo}
\DF{} è stato incaricato di trovare un sito internet adatto per la creazione di un logo per il gruppo.

\subsection{Indirizzo e-mail del gruppo}
\SE{} è stato incaricato di creare un indirizzo e-mail del gruppo per comunicare con il docente del corso e i proponenti dei capitolati.
E' stato scelto l'indirizzo e-mail: qbteamswe@gmail.com.

\subsection{Organizzazione dello studio autonomo}
È stato consigliato uno studio autonomo a tutti i componenti del gruppo che avessero poca esperienza in ambienti utili allo sviluppo software come Git, GitHub e \LaTeX.\\

\subsection{Scelta del capitolato}
Si è deciso, dopo un'attenta analisi critica dei vari capitolati, che il progetto verrà fatto sul capitolato C5: "Stalker".

\subsection{Prossimo incontro}
Il prossimo incontro tra i componenti del gruppo si terrà lunedì 2 Dicembre.
\clearpage