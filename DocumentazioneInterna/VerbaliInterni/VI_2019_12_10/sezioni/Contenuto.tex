\section{Informazioni Generali}
\begin{itemize}
\item \textbf{Luogo:} Aula P200, presso Complesso Paolotti.
\item \textbf{Data:} \Data.
\item \textbf{Ora:} 14:30 - 16:00.
\item \textbf{Partecipanti del gruppo:}
	\begin{itemize}
		\item \AT{}; 
		\item \BR{};
		\item \CE{}; 
		\item \DF{};
		\item \LD{};
		\item \MC{};
		\item \PF{};
		\item \SE{}.
	\end{itemize} 
\item \textbf{Segretario:} \DF{}.
\end{itemize}


\section{Ordine del Giorno}
\begin{itemize}
\item Risposta e-mail per \Proponente{};
\item Discussione casi d'uso;
\item Gestione della comunicazione.
\end{itemize}

\section{Resoconto}
\subsection{Risposta e-mail \Proponente{}}
In seguito alla risposta ricevuta si è deciso di organizzare un incontro con i proponenti.
È stato proposto agli altri gruppi che avevano scelto il capitolato "Stalker" di partecipare all'incontro. 

\subsection{Discussione casi d'uso}
La maggior parte della riunione è stata usata per discutere dei vari casi d'uso analizzandoli singolarmente.
Si è optato per una granularità di livello intermedio ed è stato deciso di separare i casi d'uso in due gruppi:
\begin{itemize}
\item Casi d'uso dell'applicazione (UCA);
\item Casi d'uso del server (UCS).
\end{itemize}


\subsection{Gestione della comunicazione}
I membri del gruppo hanno ritenuto l'uso di \textbf{WhatsApp} come metodo di comunicazione insufficiente per una buona \glo{organizzazione}, quindi si è deciso di iniziare ad usare \glo{Slack}.

\clearpage