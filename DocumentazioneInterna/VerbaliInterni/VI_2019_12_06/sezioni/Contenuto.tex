\section{Informazioni Generali}
\begin{itemize}
\item \textbf{Luogo:} Aula 1C150 presso Torre Archimede;
\item \textbf{Data:} \Data;
\item \textbf{Ora:} 08:30 - 10:30;
\item \textbf{Partecipanti del gruppo:}
	\begin{itemize}
	\item Azzalin Tommaso; 
	\item Cisotto Emanuele;
	\item Lazzaro Maria Davide;
	\item Mattei Christian;
	\item Perin Federico;
	\item Baratin Riccardo.
	\end{itemize} 
\item \textbf{Segretario:} Federico Perin.
\end{itemize}

\clearpage

\section{Ordine del Giorno}
\begin{itemize}
\item Redigere la mail da inviare a Davide Zanetti di Imola Informatica
\item Formulare delle domande da fare a Davide Zanetti sul capitolato 5
\item Continuazione del documento "Analisi dei Requisiti"
\end{itemize}

\section{Resoconto}
\subsection{Scrittura della lettera da inviare a Davide Zanetti}
Dopo aver scelto e analizzato il capitolato "Stalker", il gruppo ha deciso di proporre al proponente Davide Zanetti un incontro conoscitivo per poter chiarire tutti i dubbi nati durante l'analisi del documento. Inoltre, nella mail, abbiamo alcuni aspetti che avremmo voluto approfondire.
\\La lettera invita è la seguente:
\\\\Buongiorno,
\\siamo i ragazzi del gruppo "qbteam" nell'ambito del progetto del corso di Ingegneria del software del corso di Informatica dell'Università di Padova. Saremmo interessati al vostro capitolato e abbiamo già iniziato ad analizzare i primi aspetti del capitolato, prendendo come riferimento i vostri documenti di presentazione. Avremmo piacere di fissare un primo incontro conoscitivo. Inoltre vorremmo fin da subito discutere di alcune tematiche, in particolare di quella più importante, relativa al tema della geolocalizzazione. Visto che all'ultimo incontro, ci è stato detto che presso la vostra azienda vi è un esperto del settore, ci piacerebbe avere un incontro anche con lui. Di seguito elenchiamo una serie di aspetti che vorremmo approfondire:
\begin{itemize}
	\item quali sono i riferimento normativi sulla privacy da tenere in considerazione;
	\item  la licenza per lo sviluppo del progetto software;
	\item tecniche per il tracciamento della posizione degli utenti, utilizzando le varie tecnologie indicate nella presentazione del capitolato;
	\item tecniche relative alla scalabilità del server, autenticazione degli utenti nell'app e nel server, protocollo LDAP;
\end{itemize}
Inoltre, sappiamo che ci sono altri gruppi interessati al capitolato che avrebbero piacere di incontrarvi. Come gruppo, noi non avremmo alcun problema a fare un incontro condiviso.
In attesa di una vostra risposta, vi ringraziamo e vi auguriamo una buona giornata,
\\qbteam

\subsection{Domande sul capitolato 5 "Stalker"}
Il gruppo, dopo aver fatto un'analisi approfondita del documento "Stalker", ha stilato una serie di domande da chiedere a Davide Zanetti al momento dell'incontro.
Le seguenti domande sono:
\begin{itemize}
	\item qual'è la differenza tra anonimo nel caso d’uso "Azienda/Organizzazione" e l’anonimo del caso d’uso "fiera";
	\item se bisogna prevedere dei sistemi di sicurezza contro le applicazioni terze che modificano la posizione del GPS;
	\item come gestire il cambio di luogo per il tracciamento;
	\item come gestire il cambio password da parte dell’utente;
	\item se le credenziali da inserire nell’applicazione e nel server per gli amministratori sono uguali;
	\item come funziona la nomina di un amministratore da parte di un altro amministratore (per esempio se questo necessita di nuove credenziali e se restano uguali ma cambia il DB) 
	\item se i dati amministratori e utenti vengono inseriti nello stesso database o separati
	\item se gli amministratori promossi hanno meno potere e quindi ci saranno due livelli di amministratori
\end{itemize}

\subsection{Significato alla parola "anonimo" nel caso "fiera" e "azienda"}
Si è deciso di dare un significato alla parola "anonimo" nel caso "fiera" e nel caso "azienda" per non aver incomprensioni.
\\\\Per anonimo "azienda" si intende che l’utente vieni identificato (cioè so suo nome e cognome ecc) ma non so la sua posizione all’interno del luogo (esempio non so se l’utente e in bagno o in ufficio). ciò nonostante è permesso che l’utente possa essere tracciato nel caso in cui si tolga dalla modalità anonimo.
\\Per anonimo "fiera" si intende che l’utente non viene identificato (cioè non so chi è)
ma so dove si trova l’utente (esempio: so in che padiglione della fiera l’utente si trova).


\clearpage
