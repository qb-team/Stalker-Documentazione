  
\section{Riepilogo delle decisioni}
{
\rowcolors{2}{white}{grigetto}
\renewcommand{\arraystretch}{1.5}
\centering
\begin{longtable}{ >{\centering}p{0.20\textwidth} >{}p{0.70\textwidth}}

\caption{Decisioni della riunione interna del \Data}\\

\rowcolor{darkblue}

\textcolor{white}{\textbf{Codice}} & \textcolor{white}{\textbf{Decisione}} \\	
		
VI\_\Data.1 & Se i rischi in cui andiamo in contro sono asincroni(inaspettati) bisogna prenderli in modo manualistico per poi poterli risolvere ne modo più corretto. Rimane il fatto che bisogna fare un check periodico per intercettare correttamente i rischi attesi durante tutto il corso del progetto\\
		
VI\_\Data.2 & Lo sviluppo incrementale per essere considerato utile deve avere un periodo di attuazione relativamente breve, in modo da ottenere un feedback utile sull'andamento del prodotto.
Ciò comporta ad una pianificazione nel breve periodo ed incrementi veloci e rapidi. Uno degli scopi principali del feedback relativo agli incrementi è che ci permette di capire se abbiamo dimensionato bene o male l'incremento e se stiamo usando correttamente le tecnologie \\

VI\_\Data.3 & Mettere gli standard nei riferimenti normativi non va bene, è più ragionevole pensare ad ispirarsi ad essi e metterli nei riferimenti informativi \\

VI\_\Data.4 & Per quanto riguarda la giusta posizione di dove vanno collocate le varie metriche è bene capire il vero significato del \PdQ. Esso si divide in due parti. La prima decide quali valori andranno a cruscotto ovvero quali metriche abbiamo intenzione di usare con eventuali indicatori. La seconda parte, invece, manda i dati al cruscotto ovvero i valori delle metriche e valutazioni per migliorare  \\

VI\_\Data.5 & La descrizione delle metriche utilizzate non è necessaria nel \PdQ, nel cruscotto va inserito solamente a cosa serve tale metrica senza ulteriori dettagli. Quest'ultimi vanno inseriti nell'apposita sezione del documento \NdP \\

VI\_\Data.6 & Per quanto riguarda il manuale utente relativo al giusto utilizzo dell'applicazione va bene fornire un video "tutorial". Per il manuale sviluppatore pare saggia la decisione di presentare una documentazione delle API. Quest'ultima deve essere redatta con buon senso poichè potrebbe tornare utile a futuri sviluppatori qual'ora volessero riutilizzare il codice prodotto dal gruppo  \\


		
\end{longtable}
}