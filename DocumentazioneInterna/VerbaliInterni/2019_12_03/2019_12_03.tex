\documentclass[a4paper, oneside, dvipsnames, table]{article}
%openany
\usepackage{../../../template/Stiletemplate}
\usepackage{hyperref}
\usepackage{fancyhdr}
\usepackage[italian]{babel}


\newcommand{\Data}{2019-12-13}

\newcommand{\Titolo}{Verbale Riunione \Data}

\newcommand{\Redattori}{\MC{}}

\newcommand{\Verificatori}{\DF{}}

\newcommand{\Approvatore}{\AT{} \newline \SE{}}

\newcommand{\Distribuzione}{\VT{} \newline \CR{} \newline Gruppo \Gruppo{}}

\newcommand{\Uso}{Interno}

\newcommand{\DescrizioneDoc}{Questo documento si occupa di riportare quanto discusso nella riunione del \Data}

\newcommand{\pathimg}{../../../Utilita/Immagini/qbteam.png}

\newcommand{\Versionedoc}{1.0.0}

\fancyverbale
\begin{document}

\copertina{}
\newpage

\section*{Registro delle modifiche}
{
\rowcolors{2}{grigetto}{white}
\renewcommand{\arraystretch}{1.5}
\centering
\begin{longtable}{ c c  C{2.3cm} c C{3cm} C{3.2cm}}
\rowcolor{rossoep}
\textcolor{white}{\textbf{Versione}} & \textcolor{white}{\textbf{Data}} & \textcolor{white}{\textbf{Nominativo}} & \textcolor{white}{\textbf{Ruolo}} & 
\textcolor{white}{\textbf{Verificatore}}& \textcolor{white}{\textbf{Descrizione}}\\	


1.0.0 & \Data & \CE{} & Responsabili & \CE{} & Approvazione per il rilascio.  \\
		
0.0.1 & \Data & \DF{} & Analista & \AT{} & Stesura e verifica del documento.  \\
		
		
\end{longtable}
}

\clearpage
\tableofcontents
\clearpage



\section{Informazioni Generali}
\begin{itemize}
\item \textbf{Luogo:} Torre Archimede;
\item \textbf{Data:} \Data;
\item \textbf{Ora:} 13:00 - 15:30;
\item \textbf{Partecipanti del gruppo:}
	\begin{itemize}
	\item Azzalin Tommaso; 
	\item Cisotto Emanuele; 
	\item Drago Francesco;
	\item Lazzaro Davide;
	\item Perin Federico;
	\item Pit Enrico;
	\item Riccardo;
	\item Christian.
	\end{itemize} 
\item \textbf{Segretario:} Cognome nome.
\end{itemize}

\clearpage

\section{Ordine del Giorno}
\begin{itemize}
\item Piano di lavoro
\item Gestione delle repository;
\item Stesura email per Imola Informatica
\end{itemize}

\clearpage

\section{Resoconto}
\subsection{Piano di lavoro}
%L'argomento sul quale è stata spesa la maggior parte del tempo è stata la pianificazione del lavoro da svolgere.

È stata fissata come milestone il 7 Gennaio 2020 e si è preso nota delle indisponibilità dei membri del gruppo durante il periodo festivo. 
Definite le durate che ogni persona avrà nei diversi ruoli: ogni due settimane ci sarà una rotazione dei ruoli assegnati. \\

\subsection{Gestione delle repository}
È stato deciso di creare due repository, una pubblica per il codice(Stalker) e una privata per la documentazione (Stalker-Documentazione). \\

\subsection{Stesura email per Imola Informatica}
Nel corso della riunione è stata stilata una lista di quesiti da sottoporre al proponente nella mail. 
\begin{itemize}
\item Stabilire un incontro, possibilmente anche con l’esperto dei servizi di geo-localizzazione
\item Approfondire le normative sulla privacy da rispettare
\item Accordarsi sulla licenza per lo sviluppo dell'applicazione
\item Approfondire la tematica del tracciamento della posizione
\item Approfondire le tecniche di scalabilità del server, e autenticazione (LDAP)
\end{itemize} 

\clearpage

\section{Riepilogo delle decisioni}
{
\rowcolors{2}{white}{grigetto}
\renewcommand{\arraystretch}{1.5}
\centering
\begin{longtable}{ >{\centering}p{0.20\textwidth} >{}p{0.70\textwidth}}

\caption{Decisioni della riunione interna del \Data}\\

\rowcolor{rossoep}

	\textcolor{white}{\textbf{Codice}} 
&   \textcolor{white}{\textbf{Decisione}} \\	
		
VI\_1.1 & Tuttoappostoaferragosto \\
		
VI\_1.2 & Tuttoappostoaferragosto \\
		
\end{longtable}
}







\end{document}