\section{Riepilogo delle decisioni}
{
\rowcolors{2}{white}{grigetto}
\renewcommand{\arraystretch}{1.5}
\centering
\begin{longtable}{ >{\centering}p{0.20\textwidth} >{}p{0.70\textwidth}}

\caption{Decisioni della riunione interna del \Data}\\

\rowcolor{darkblue}

\textcolor{white}{\textbf{Codice}} & \textcolor{white}{\textbf{Decisione}} \\	
		
VI\_\Data.1 & Va scritto nelle norme di progetto tutto ciò che riguarda la divisione di progettazione \\
		
VI\_\Data.2 & Per quanto riguarda la gestione dei PoC (ad esempio se utilizziamo repository o no) e la loro pianificazione va normata di particolare interesse e sapere quali PoC sono da considerare usa e getta o invece incrementali, nel caso in cui i primi PoC siano usa e getta e poi successivamente si inizi a produrre PoC incrementali si deve indicare nella pianificazione quando questo avviene  \\

VI\_\Data.3 & E' ragionevole applicare le metriche solo a PoC incrementali \\

VI\_\Data.4 & PdP va tenuto aggiornato in corso d'opera  in base alle necessità della progettazione. \\

VI\_\Data.5 & Sarebbe ideale progettare tutto e fare poi il PoC che è ritenuto necessario ma è decisamente molto difficile, meglio concentrarsi sulla parte del PoC \\

VI\_\Data.6 & Per le API si ha l'adozione di specifica OpenAPI \\

VI\_\Data.7 & L'attività di codifica va certamente migliorata ora che si è iniziato a utilizzare le tecnologie di interesse \\

VI\_\Data.8 & Le metriche SFIN e SFOUT sono più precise della metrica CBO \\
		
\end{longtable}
}