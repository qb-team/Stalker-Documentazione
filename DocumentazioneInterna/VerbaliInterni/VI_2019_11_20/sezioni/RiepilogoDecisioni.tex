\section{Riepilogo delle decisioni}
{
\rowcolors{2}{white}{grigetto}
\renewcommand{\arraystretch}{1.5}
\centering
\begin{longtable}{ >{\centering}p{0.20\textwidth} >{}p{0.70\textwidth}}

\caption{Decisioni della riunione interna del \Data}\\

\rowcolor{rossoep}

	\textcolor{white}{\textbf{Codice}} 
&   \textcolor{white}{\textbf{Decisione}} \\	
		
VI\_\Data.1 & il successivo incontro deve essere comunicata la propria scelta fra i due nomi del gruppo considerati ammissibili: \textbf{qbteam} e \textbf{8bit} \\
		
VI\_\Data.2 & Francesco crea un repository su GitHub con il nome \textbf{SWE--Gruppo-5} in cui aggiunge tutti i membri del gruppo fra i collaboratori  \\

VI\_\Data.3 & Viene suddivisa l'analisi dei capitolati da esporre durante il successivo incontro come segue:
\begin{itemize}
	\item Federico si occupa del Capitolato 1 (Autonomous Highlights Platform);
	\item Enrico e Tommaso si occupano del Capitolato 2 (Etherless);
	\item Davide ed Emanuele si occupano del Capitolato 3 (NaturalAPI);
	\item Francesco si occupa del Capitolato 4 (Predire in Grafana);
	\item Riccardo si occupa del Capitolato 5 (Stalker);
	\item Christian si occupa del Capitolato 6 (ThiReMa - Things Relationship Management).
\end{itemize} \\
		
\end{longtable}
}

