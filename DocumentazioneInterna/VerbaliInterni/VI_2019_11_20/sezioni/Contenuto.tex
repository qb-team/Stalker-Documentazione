\section{Informazioni Generali}
\begin{itemize}
\item \textbf{Luogo:} Aula 1A150, presso Torre Archimede.
\item \textbf{Data:} \Data.
\item \textbf{Ora:} 17:30 - 18:30.
\item \textbf{Partecipanti del gruppo:}
	\begin{itemize}
		\item \AT{}; 
		\item \BR{};
		\item \CE{}; 
		\item \DF{};
		\item \LD{};
		\item \MC{};
		\item \PF{};
		\item \SE{}.
	\end{itemize} 
\item \textbf{Segretario:} \AT{}.
\end{itemize}


\section{Ordine del Giorno}
\begin{itemize}
	\item Creazione di un gruppo WhatsApp per le comunicazioni condivise; 
	\item Creazione di una cartella condivisa su Google Drive;
	\item Brain storming per scegliere il nome del gruppo;
	\item Creazione di un repository per il gruppo;
	\item Discussione sui capitolati.
\end{itemize}

\section{Resoconto}
\subsection{Creazione di un gruppo WhatsApp per le comunicazioni condivise}
È stato creato un gruppo su WhatsApp per poter comunicare fra tutti i membri del team.

\subsection{Creazione di una cartella condivisa su Google Drive}
È stata creata una cartella condivisa su Google Drive in cui scrivere i documenti per il lavoro condiviso e per segnare il calendario delle proprie disponibilità a incontrarsi (soprattutto nel breve periodo).
Viene scelto Google Drive e quindi Google Docs perché permette di lavorare contemporaneamente ad uno stesso documento e questo, in particolare durante gli incontri, è stato ritenuto molto comodo da tutto il gruppo.

\subsection{Brain storming per scegliere il nome del gruppo}
È stato proposto da ogni partecipante dell'incontro una o più idee per il nome da assegnare al gruppo. Alcune idee possibili proposte sono:
\begin{itemize}
	\item qb5;
	\item qbteam;
	\item tun4soft;
	\item 8bit;
	\item tecnotree.
\end{itemize}
Dopo una discussione si sono ritenuti come validi i seguenti nomi:
\begin{enumerate}
	\item qbteam;
	\item 8bit.
\end{enumerate}
Gli altri nomi che sono stati proposti vengono scartati. Fra i due nomi proposti come possibili, non viene effettuata una decisione finale condivisa. Viene posticipata la decisione finale al prossimo incontro, in cui verrà comunicata, da parte di ognuno, la propria scelta.

\subsection{Creazione di un repository per il gruppo}
Viene discussa la creazione di una repository per la documentazione e per i sorgenti sulla piattaforma online GitHub.
Non avendo ancora un nome per il gruppo, Francesco propone di creare, nel suo account, una repository pubblica che condividerà con i membri del gruppo dal nome \textbf{SWE--Gruppo-5}.

\subsection{Discussione sui capitolati}
Il gruppo, dopo aver assistito in data 2019-11-15 alla presentazione dei capitolati, espone rapidamente i propri pensieri sui capitolati ma comprende la necessità di dover fare un'analisi più approfondita.
Si decide quindi di suddividersi l'analisi dei documenti dei capitolati e di presentare per il successivo incontro un documento riassuntivo contenente i seguenti punti:
\begin{itemize}
	\item Breve riassunto delle richieste del capitolato, ovvero a cosa mira il progetto proposto;
	\item Tecnologie da usare per lo sviluppo del progetto;
	\item Prerequisiti (linguaggi, tecnologie, corsi universitari);
	\item Vantaggi e punti di forza del capitolato (per esempio, conoscenze acquisite a fine progetto, skill interessanti per la propria futura professione);
	\item Svantaggi e/o vincoli progettuali, problemi che si potrebbero riscontrare (per esempio, l'utilizzo necessario di un determinato linguaggio);
	\item Richieste specifiche del proponente (per esempio, voler vedere i diagrammi UML prima di ogni revisione di avanzamento);
	\item Servizi offerti dal proponente (per esempio, supporto nell'analisi dei requisiti, supporto remoto, incontri con l'azienda su richiesta).
\end{itemize}
Il lavoro viene così suddiviso:
\begin{itemize}
	\item Federico si occupa del Capitolato 1 (Autonomous Highlights Platform);
	\item Enrico e Tommaso si occupano del Capitolato 2 (Etherless);
	\item Davide ed Emanuele si occupano del Capitolato 3 (NaturalAPI);
	\item Francesco si occupa del Capitolato 4 (Predire in Grafana);
	\item Riccardo si occupa del Capitolato 5 (Stalker);
	\item Christian si occupa del Capitolato 6 (ThiReMa - Things Relationship Management).
\end{itemize}


\clearpage