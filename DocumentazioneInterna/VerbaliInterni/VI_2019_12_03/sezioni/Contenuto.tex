\section{Informazioni Generali}
\begin{itemize}
\item \textbf{Luogo:} Torre Archimede, aula 1A150;
\item \textbf{Data:} \Data;
\item \textbf{Ora:} 13:00 - 15:30;
\item \textbf{Partecipanti del gruppo:}
	\begin{itemize}
	\item Azzalin Tommaso; 
	\item Cisotto Emanuele; 
	\item Drago Francesco;
	\item Lazzaro Davide;
	\item Perin Federico;
	\item Salmaso Enrico;
	\item Baratin Riccardo;
	\item Mattei Christian.
	\end{itemize} 
\item \textbf{Segretario:} Davide Lazzaro.
\end{itemize}

\clearpage

\section{Ordine del Giorno}
\begin{itemize}
\item Piano di lavoro;
\item Gestione delle repository;
\item Stesura email per Imola Informatica;
\item Discussione sui ruoli da ricoprire nella fase di analisi e assegnazione dei suddetti;
\item Discussione sul capitolato C5;
\end{itemize}

\clearpage

\section{Resoconto}
\subsection{Piano di lavoro}
%L'argomento sul quale è stata spesa la maggior parte del tempo è stata la pianificazione del lavoro da svolgere.

Si è preso nota delle indisponibilità dei membri del gruppo durante il periodo festivo, Christian non sarà disponibile per incontrarsi fisicamente dal 19/12/2019 al 29/12/2019, gli altri membri del gruppo non hanno ancora fornito le loro indisponibilità. 
Definite le durate che ogni persona avrà nei diversi ruoli: i responsabili saranno permanenti nella fase di analisi mentre gli altri ruoli saranno a rotazione ceracndo di far ricoprire a tutti membri il ruolo di Amministratore, inoltre analista e verificatore saranno ricoperti da tutti i membri.  \\

\subsection{Gestione delle repository}
È stato deciso di creare due repository, una pubblica per il codice(Stalker) e una privata per la documentazione (Stalker-Documentazione), entrambe sulla piattaforma github, compito preso in carico da Tommaso. \\

\subsection{Stesura email per Imola Informatica}
Nel corso della riunione è stata stilata una lista di quesiti da sottoporre al proponente nella mail. 
Chiaramente tutti questi punti devono essere approfonditi da ogni membro del gruppo in modo tale da garantire la massima efficacia dell'incontro con il proponente.
\begin{itemize}
\item Stabilire un incontro, possibilmente anche con l’esperto dei servizi di geo-localizzazione
\item Approfondire le normative sulla privacy da rispettare
\item Accordarsi sulla licenza per lo sviluppo dell'applicazione
\item Approfondire la tematica del tracciamento\ap{G}della posizione
\item Approfondire le tecniche di scalabilità del server, e autenticazione\ap{G} (LDAP)
\end{itemize} 

\subsection{ Discussione sui ruoli da ricoprire nella fase di analisi e assegnazione dei suddetti}
I ruoli da assegnare sono stati così ripartiti: 
\begin{itemize}
\item Amministratore (una persona): Tommaso Azzalin
\item Responsabile (una persona): Enrico Salmaso
\item Analisti (4 persone): Davide Lazzaro, Riccardo Baratin, Federico Perin ed Emanuele Cisotto
\item Verificatori (2 persone): Francesco Drago, Christian Mattei.
\end{itemize}
Si è scelta questa ripartizione ( 4 analisti e 2 verificatori) per velocizzare una stesura in forma di 
bozza di studio di fattibilità e analisi dei requisiti mentre i due verificatori si adoperano a formare dei template
da fornire in seguito agli analisti.
\\
\subsubsection{Assegnazione di compiti supplementari}
\begin{itemize}
\item Ai verificatori è stato assgnato anche lo sviluppo dei template per i verbali in latex;
\item Visionare la documentazione per l'amministrazione sarà compito di Tommaso;
\item Informarsi anche tramite il prof. Marchiori su tecnologie inerenti al tracciamento\ap{G}della posizione è stato preso in carico da Riccardo;
\end{itemize}
\subsection{Discussione sul capitolato C5}
E' stato steso un documento cartaceo contentente i risultati della discussione avvenuta nella prima parte dell'incontro, servirà da base agli analisti per iniziare a 
scrivere gli UC a granularità grossa, il documento è stato condiviso nel gruppo whatsapp.
\clearpage