\section{Informazioni Generali}
\begin{itemize}
\item \textbf{Luogo:} aula 1A150, presso Torre Archimede.
\item \textbf{Data:} \Data.
\item \textbf{Ora:} 13:00 - 15:30.
\item \textbf{Partecipanti del gruppo:}
	\begin{itemize}
		\item \AT{}; 
		\item \BR{};
		\item \CE{}; 
		\item \DF{};
		\item \LD{};
		\item \MC{};
		\item \PF{};
		\item \SE{}.
	\end{itemize} 
\item \textbf{Segretario:} \LD{}.
\end{itemize}


\section{Ordine del Giorno}
\begin{itemize}
\item Piano di lavoro;
\item gestione delle repository;
\item stesura email per \Proponente{};
\item discussione sui ruoli da ricoprire nella fase di analisi e assegnazione dei suddetti;
\item discussione sul capitolato C5.
\end{itemize}



\section{Resoconto}
\subsection{Piano di lavoro}
Si è preso nota delle indisponibilità dei membri del gruppo durante il periodo festivo, Christian non sarà disponibile per incontrarsi fisicamente dal 19/12/2019 al 29/12/2019, gli altri membri del gruppo non hanno ancora fornito le loro indisponibilità. 
Definite le durate che ogni persona avrà nei diversi ruoli: i responsabili saranno permanenti nella fase di analisi mentre gli altri ruoli saranno a rotazione ceracndo di far ricoprire a tutti membri il ruolo di Amministratore, inoltre analista e verificatore saranno ricoperti da tutti i membri.  \\

\subsection{Gestione delle repository}
È stato deciso di scartare il precedente repository\ap{G} poiché è stato deciso di creare un profilo organizzazione su GitHub\ap{G}, il quale permette di creare un account condiviso in cui i membri del gruppo possono collaborare.
All'interno dell'organizzazione è stato deciso di creare due repository\ap{G} pubblici, uno per il codice sorgente (Stalker) e uno per la documentazione (Stalker-Documentazione).
Entrambe sulla piattaforma GitHub\ap{G}, il compito è stato preso in carico da Tommaso. 

\subsection{Stesura e-mail per \Proponente{}}
Nel corso della riunione è stata stilata una lista di quesiti da sottoporre al proponente nella e-mail. 
Chiaramente tutti questi punti devono essere approfonditi da ogni membro del gruppo in modo tale da garantire la massima efficacia dell'incontro con il proponente.
Verranno esposti i seguenti punti:
\begin{itemize}
\item stabilire un incontro, possibilmente anche con l'esperto dei servizi di geo-localizzazione\ap{G};
\item approfondire le normative sulla privacy da rispettare;
\item accordarsi sulla licenza per lo sviluppo dell'applicazione;
\item approfondire la tematica del tracciamento\ap{G} della posizione;
\item approfondire le tecniche di scalabilità del server, e autenticazione\ap{G} (LDAP).
\end{itemize} 

\subsection{Discussione sui ruoli da ricoprire nella fase di analisi e assegnazione dei suddetti}
I ruoli da assegnare sono stati così ripartiti: 
\begin{itemize}
\item Amministratore (una persona):\AT{};
\item Responsabile (una persona): \SE{};
\item Analisti (4 persone): \LD{}, \BR{}, \PF{} ed \CE{};
\item Verificatori (2 persone): \DF{}, \MC{}.
\end{itemize}
Si è scelta questa ripartizione (4 analisti e 2 verificatori) per velocizzare una stesura in forma di bozza di Studio di Fattibilità e Analisi dei Requisiti, mentre i due verificatori si adoperano a formare dei template da fornire in seguito agli analisti.

\subsubsection{Assegnazione di compiti supplementari}
\begin{itemize}
\item Ai verificatori è stato assegnato anche lo sviluppo dei template per i verbali in \LaTeX;
\item visionare la documentazione per l'amministrazione: sarà compito di Tommaso;
\item informarsi, anche tramite il prof. Marchiori, su tecnologie inerenti al tracciamento\ap{G} della posizione. Il compito è stato preso in carico da Riccardo;
\end{itemize}
\subsection{Discussione sul capitolato C5}
E' stato steso un documento cartaceo contenente i risultati della discussione avvenuta nella prima parte dell'incontro, servirà da base agli analisti per iniziare a scrivere gli UC a granularità grossa, il documento è stato condiviso nel gruppo WhatsApp.
\clearpage