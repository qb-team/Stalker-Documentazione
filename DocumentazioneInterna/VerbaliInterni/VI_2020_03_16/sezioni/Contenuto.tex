\section{Informazioni Generali}
\begin{itemize}
\item \textbf{Luogo:} Discord.
\item \textbf{Data:} \Data.
\item \textbf{Ora:} 14:00 - 15:30.
\item \textbf{Partecipanti del gruppo:}
	\begin{itemize}
		\item \AT{}; 
		\item \BR{};
		\item \CE{}; 
		\item \DF{};
		\item \LD{};
		\item \MC{};
		\item \PF{};
		\item \SE{}.
	\end{itemize} 
\item \textbf{Segretario:} \SE{}.
\end{itemize}

\section{Ordine del Giorno}
\begin{itemize}
	\item Gestione degli incrementi riportati nel Piano di Progetto;
	\item Assegnazione dei lavori;
	\item Creazione di issue in base ai requisiti obbligatori da soddisfare;
	\item Discutere di eventuali correzioni dei documenti;
	\item Ridefinizione dei ruoli principali tra i componenti del gruppo.
\end{itemize}


\section{Resoconto}

\subsection{Gestione degli incrementi}
Il gruppo durante la fase di Progettazione Architetturale è riuscito a portare a termine l'incremento 0 riportato nel documento Piano di Progetto
Nella fase che segue di Progettazione di Dettaglio e Codifica, dovranno essere soddisfatti gli incrementi 1,2,3,4,5.
La situazione è quindi la seguente:
\begin{itemize}
	\item Incremento 0: fatto;
	\item Incremento 1: da fare;
	\item Incremento 2: da fare;
	\item Incremento 3: da fare;
	\item Incremento 4: da fare;
	\item Incremento 5: da fare.
\end{itemize}


\subsection{Decisioni generali sulla suddivisione del lavoro}
In questo verbale, si è deciso che:
\begin{itemize}

\item Per un periodo che inizierà da oggi (2020-03-16) fino a giovedì (2020-03-19) ogni componente del gruppo studierà in maniera autonoma le slides del professore Cardin in modo da essere preparati per la successiva attività di progettazione. Successivamente si proseguirà con la progettazione che precede la codifica di quanto progettato.	
\item Durante la fase di Progettazione di Dettaglio e Codifica ci sarà la medesima suddivisione già formata nella fase di Progettazione Architetturale, suddividendo quindi il team in 3 sottogruppi che si occuperanno dell'applicazione mobile, applicazione web amministratori e server back-end;	
	\item Per la gestione dei test da effettuare nell'applicazione mobile e nell'applicazione web si è pensato inizialmente di utilizzare Travis CI;
	\item Sono stati creati due nuovi canali su Slack sulla creazione del manuale utente e del manuale manutentore.
\end{itemize} 

\subsection{Issue da aggiungere su GitHub}
Si è pensato di scrivere una nuova issue per ogni requisito di ciascun incremento. Per il loro svolgimento, verrà seguito l'ordine degli incrementi. Verranno quindi eseguite prima le issue sui requisiti dell'incremento 1; dopo di che verranno eseguite le issue appartenenti all'incremento 2 e cosi via.
Le issue verranno create nella repository della rispettiva sezione di appartenenza tra applicazione mobile, applicazione web amministratori e server back-end e applicazione mobile.

\subsection{Correzione dei documenti}
Per il miglioramento dei documenti consegnati nella revisione di Progettazione, si attende prima il suo esito.
Nel frattempo il gruppo rileggerà i documenti per la correzione di qualche errore grammaticale che potrebbe essere sfuggito durante la verifica dei verificatori.

\subsection{Ridefinizione dei ruoli principali tra i componenti del gruppo}
Ruoli assegnati da oggi (2020-03-16) al 2020-03-22:
\begin{itemize}
	\item \DF{}: \Responsabile{};
	\item \PF{}: \Amministratore{};
	\item Tutti: Progettista; 
	\item Tutti: Programmatore;
	\item Tutti: Verificatore.
\end{itemize}

Ruoli assegnati dal 2020-03-23 al 2020-04-19:
\begin{itemize}
	\item \AT{}: \Responsabile{}; 
	\item \MC{}: \Amministratore{};
	\item Tutti: Progettista; 
	\item Tutti: Programmatore;
	\item Tutti: Verificatore.
\end{itemize}


\clearpage