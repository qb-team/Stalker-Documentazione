\section{Introduzione}
\subsection{Scopo del Documento}
Questo documento contiene la stesura del studio di fattibilità riguardante i sei capitolati proposti, per ciascuno di essi verranno evidenziati i seguenti aspetti:
\begin{itemize}
\item Titolo del capitolato;
\item Descrizione generale;
\item Prerequisiti e tecnologie coinvolte;
\item Vincoli;
\item Aspetti positivi;
\item Aspetti critici.
\end{itemize}
Infine per ogni capitolo verranno esposte le motivazioni e le ragioni per cui il gruppo ha scelto come progetto il capitolato C5 \NomeProgetto{} a discapito degli altri cinque capitolati proposti.

\subsection{Glossario}
Al fine di evitare ambiguità fra i termini, e per avere chiare fra tutti gli stakeholder le terminologie utilizzate per la realizzazione del presente documento, il gruppo \Gruppo{} ha redatto un documento denominato “\Glossariov{1.0.0}”.
In tale documento, sono presenti tutti i termini tecnici, ambigui, specifici del progetto e scelti dai membri del gruppo con le loro relative definizioni.
Un termine presente nel \Glossariov{1.0.0} e utilizzato in questo documento viene indicato con un'apice \ap{G} alla fine della parola.
	
\subsection{Riferimenti}

\subsubsection{Normativi}
\begin{itemize}
\item \NdPv{1.0.0}.
\end{itemize}

\subsubsection{Informativi}

\begin{itemize}
\item \textbf {Capitolato d'appalto C1 - Autonomous Highlights Platform:}\\
\url{https://www.math.unipd.it/~tullio/IS-1/2019/Progetto/C1.pdf}
\item \textbf {Capitolato d'appalto C2 - Etherless:}\\
\url{https://www.math.unipd.it/~tullio/IS-1/2019/Progetto/C2.pdf}
\item \textbf {Capitolato d'appalto C3 - Natural API:}\\
\url{https://www.math.unipd.it/~tullio/IS-1/2019/Progetto/C3.pdf}
\item \textbf {Capitolato d'appalto C4 - Predire in Grafana:}\\
\url{https://www.math.unipd.it/~tullio/IS- 1/2019/Progetto/C4.pdf}
\item \textbf {Capitolato d'appalto C5 - Stalker:}\\
\url{https://www.math.unipd.it/~tullio/IS-1/2019/Progetto/C5.pdf}
\item \textbf {Capitolato d'appalto C6 - ThiReMa:}\\
\url{https://www.math.unipd.it/~tullio/IS-1/2019/Progetto/C6.pdf}

\end{itemize}
