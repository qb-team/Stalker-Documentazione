\section{Capitolato C6}
\subsection{Titolo del capitolato}
Il capitolato in questione si chiama \textit{"ThiReMa"}, il proponente \`e l'azienda \textit{Sanmarco Informatica} e i committenti sono \VT{} e \CR{}.

\subsection{Descrizione del capitolo}
L'azienda \`e dotata di un software che raccoglie dati da numerosi sensori eterogenei, di diversi tipi di apparecchi e dislocati geograficamente, in un database centralizzato. Questi dati vengono analizzati e vengono inoltrati, se necessario, messaggi di avviso per poter gestire gli apparecchi. Questi dati si suddividono in due categorie: dati operativi (provenienti dai sensori degli apparecchi) e fattori influenzanti (per esempio la temperatura o l'umidità).
L'obiettivo \`e realizzare una applicazione web che permetta di mettere in relazione i dati operativi con i fattori influenzanti, di analizzarli con uno o più algoritmi per poter predire l'andamento di questi ed offrire alcuni servizi, come ad esempio una manutenzione preventiva in caso di previsione di guasti. Inoltre, deve essere permesso a certi enti di poter monitorare i propri dati utili per un \glo{upselling} tecnologico.


\subsection{Prerequisiti e tecnologie coinvolte}
Prerequisiti:
\begin{itemize}
\item Java;	
\item Data base;
\item Web-application.
\end{itemize}
Tecnologie coinvolte:
\begin{itemize}
\item \textbf{\glo{Apache Kafka}}: Il cluster da cui provengono i dati da analizzare;
\item \textbf{Java 8}: È consigliato il suo utilizzo, in particolare per l'utilizzo delle \glo{API} \glo{Producer/Consumer}, Connect e \glo{Stream};
\item \textbf{\glo{Bootstrap}}: Framework per la realizzazione di interfacce web basato su \glo{HTML5}, \glo{CSS3}, \glo{JavaScript};
\item \textbf{\glo{Docker}}: Strumento per la creazione di container in cui istanziare i servizi per la gestione dell'architettura;
\item \textbf{\glo{TimescaleDB}, \glo{Clickhouse} o \glo{PostgreSQL}}: Sono DBMS (Database Management System), che hanno caratteristiche diverse e usi diversi. Possono essere usati assieme per tenere dati di diversa natura separati oppure \`e possibile raggrupparli in un unico database.
\end{itemize}

\subsection{Vincoli}
L'azienda richiede che prima di iniziare lo sviluppo del progetto siano consegnati:
\begin{itemize}
\item I diagrammi \glo{UML} relativi agli use case di progetto;
\item Lo schema design relativo alla base dati;
\item La documentazione delle \glo{API} che saranno realizzate. 
\end{itemize}
Materiale da consegnare a corredo del progetto: 
\begin{itemize}
\item Lista dei bug risolti durante le fasi di sviluppo;
\item Codice prodotto in formato sorgente utilizzando sistemi di versionamento del codice, quali \glo{GitHub} o \glo{BitBucket}.
\end{itemize}
L'aspettativa minima per la conclusione del progetto dovrà comprendere: 
\begin{itemize}
\item Codice sorgente di quanto realizzato;
\item \glo{Docker} file con la componente applicativa, se utilizzato \glo{Docker}.
\end{itemize}

\subsection{Aspetti positivi}
\begin{itemize}
\item Il collante fra le varie tecnologie e servizi è Java 8, conosciuto da tutti i membri del gruppo;
\item L'azienda si \`e mostrata molto disponibile al fornire materiale per il testing e figure professionali per il supporto agli studenti che operano per la realizzazione del prodotto commissionato;
\item Vengono acquisite competenze in ambito \glo{Internet of Things} e \glo{Big Data}.
\end{itemize}

\subsection{Aspetti critici}

\begin{itemize}
\item Le tecnologie richieste per la realizzazione del prodotto finale non sono interessanti poiché risultano essere abbastanza conosciute e già trattate nei vari corsi della laurea triennale;
\item Limitati posti per aggiudicarsi il capitolato.

\end{itemize}
\subsection{Conclusioni}
Il gruppo si \`e mostrato abbastanza interessato da ci\`o che \`e stato proposto dal capitolato ma dopo una discussione interna tra i membri pochi del team e emersa l'insoddisfazione da parte del team per via delle poche tecnologie nuove da apprendere e inoltre a causa del limitato numero di posti disponibili e la molta richiesta si \`e deciso infine di scartare questa proposta.