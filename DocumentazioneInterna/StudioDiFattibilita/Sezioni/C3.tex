\section{Capitolato C3}
\subsection{Titolo del capitolato}
Il capitolato in questione si chiama \textit{"Natural API"}, il proponente è l'azienda \textit{teal.blue} e i committenti sono \VT{} e \CR{}.

\subsection{Descrizione del capitolo}
Considerando i vari personaggi che prendono parte allo sviluppo di un progetto (clienti, project manager, developer, ecc.), è chiaro che non tutti si esprimano nello stesso linguaggio. Le difficoltà nel descrivere l'idea che si vuole trasmettere possono ripercuotersi sull'efficienza\ap{G}, l'efficacia\ap{G} e il costo per sviluppare il progetto. \textit{Natural API\ap{G}} mira a sviluppare un toolkit\ap{G} in grado di colmare la distanza tra le specifiche di progetto e le API\ap{G}. In pratica è composta da tre processi distinti:
\begin{enumerate}
\item \textbf{Discover}: Estrazione automatica di una lista di predicati, nomi e verbi da file testuali dati in input (per esempio guide, manuali ecc.), creando il Business Domain Language (BDL);
\item \textbf{Design}: Combinazione delle specifiche degli stakeholders in gherkin e del BDL per creare il Business Application Language (BAL), ovvero lo step intermedio tra linguaggio naturale e codice. Il BAL deve poter integrare requisiti addizionali e poter fondere o dividere azioni ed espandere nomi in oggetti;
\item \textbf{Develop}: Tramutare il BAL in un linguaggio di programmazione concreto, il quale deve essere interattivo in modo da specificare dei dettagli del linguaggio di programmazione.
\end{enumerate}

Il capitolato pone quindi come obiettivo l'implementazione di un sistema in grado di trasformare specifiche definite in linguaggio naturale in codice di un reale linguaggio di programmazione.

\subsection{Prerequisiti e tecnologie coinvolte}
Prerequisiti:
\begin{itemize}
\item \textbf{Gherkin\ap{G}}: \url{https://cucumber.io/docs/gherkin/}
\item \textbf{Natural language processing\ap{G}}: \url{https://en.wikipedia.org/wiki Natural_language_processing}
\item \textbf{API\ap{G} generation}: \url{https://github.com/OAI/OpenAPI-Specification}
\item \textbf{Code generation\ap{G}}: \url{https://swagger.io/}
\item \textbf{Behavior-driven development\ap{G}}: \url{https://dannorth.net/introducing-bdd/}
\item \textbf{Cucumber\ap{G}}: \url{https://cucumber.io/docs}
\item \textbf{Clean architecture\ap{G}}: \url{https://blog.cleancoder.com/uncle-bob/2012/08/13/the-clean-architecture.html}
\end{itemize}
Tecnologie consigliate:
\begin{itemize}
\item Qt\ap{G};
\item Python\ap{G};
\item React\ap{G};
\item Cucumber\ap{G}.
\end{itemize}

\subsection{Vincoli}
Sono imposti i seguenti vincoli:
\begin{itemize}
\item Ogni parte dell'applicazione deve soddisfare degli standard sull'output che produce almeno due modi di interagire con l'applicazione tra i seguenti:
\begin{itemize}
\item Linea di comando;
\item Interfaccia grafica\ap{G};
\item Interfaccia web\ap{G}.
\end{itemize}

\item Rispettare delle specifiche sui logic layer\ap{G} (si incoraggia l'applicazione del Joel Test). I logic layers\ap{G} dovranno essere rilasciati in uno dei seguenti metodi:
\begin{itemize}
\item Come una libreria (statica o dinamica);
\item Come parte di un eseguibile;
\item Come un processo/servizio indipendente, in locale o remoto.
\end{itemize}

\item Per la modalità di consegna sarà sufficiente l'accessibilità da almeno un sistema operativo tra:
\begin{itemize}
\item Linux;
\item Windows;
\item MacOS.
\end{itemize}

\item Input e output in codifica UTF-8\ap{G};
\item Licenza Open Source;
\item Il codice del progetto dovrà essere locato in un repository\ap{G} facilmente accessibile dal pubblico (ad esempio GitHub\ap{G}).
\end{itemize}

\subsection{Aspetti positivi}
È di ovvia utilità un sistema che sia in grado di soddisfare le esigenze sopracitate, poiché sarebbe applicabile in ogni azienda dove si sviluppi software.

\subsection{Aspetti critici}
\begin{itemize}
\item Risulta poco chiaro il passaggio dal BDL al BAL;
\item Il linguaggio utilizzato (gherkin) è di nicchia e la conversazione tra stakeholder e UML\ap{G} sono ancora lo standard.
\end{itemize}

\subsection{Conclusioni}
L'argomento trattato dal capitolato è risultato essere poco interessante per il gruppo, questo perché viene richiesto l'utilizzo di tecnologie poco conosciute e abbastanza di nicchia. Alla fine il prodotto risulterebbe poco soddisfacente per il gruppo dopo il lungo lavoro richiesto per realizzarlo.
Analizzando gli aspetti positivi e quelli critici non è emerso un forte interesse per il capitolato.