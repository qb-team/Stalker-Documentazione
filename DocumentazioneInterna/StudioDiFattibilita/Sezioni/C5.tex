\section{Capitolato C5}
\subsection{Titolo del capitolato}
Il capitolato in questione si chiama "Stalker", il proponente è l'azienda \Proponente{} e i committenti sono \VT{} e \CR{}.

\subsection{Descrizione del capitolo}
Stalker richiede la realizzazione di un sistema di monitoraggio delle posizioni delle persone (in versione anonima o non) grazie ad un'applicazione "complementare" installata sugli smartphone. La necessità del servizio sorge dal dover monitorare il numero e la posizione delle persone all'interno di fiere/musei/locali ai fini di sicurezza oppure per verificare la presenza e la locazione dei dipendenti di un'azienda. Nel primo caso si user\`a il \glo{tracciamento anonimo}, nel secondo caso invece il \glo{tracciamento} non anonimo. L'applicazione dovrà risalire alla posizione attuale grazie all'ausilio di infrastrutture e/o servizi resi disponibili dallo smartphone stesso, come per esempio \glo{beacon}, \glo{GPS}, dati del cellulare, ecc. Quando l'applicazione calcola la posizione aggiornata deve anche verificare se rientra in una zona da essere tracciata, in tal caso comunicherà col server remoto per autenticarsi (anonimamente o non) e risultare tracciabile dal sistema. 

\subsection{Prerequisiti e tecnologie coinvolte}
Prerequisiti:
\begin{itemize}
\item Java.
\end{itemize}
Tecnologie coinvolte:
\begin{itemize}
\item Utilizzo di Java (versione 8 o superiore), \glo{Python} o \glo{Node.js} per lo sviluppo del server \glo{back-end};
\item Utilizzo di protocolli asincroni per le comunicazioni app mobile-server;
\item Utilizzo del pattern di \glo{Publisher/Subscriber}, ovvero mittenti (Publisher) e destinatari (Subscriber) di messaggi dialogano attraverso un tramite (detto \glo{dispatcher});
\item Utilizzo dell'IAAS \glo{Kubernetes} o di un \glo{PAAS}, \glo{Openshift} o \glo{Rancher}, per il rilascio delle componenti del Server nonché per la gestione della \glo{scalabilità orizzontale};
\item API \glo{REST} attraverso le quali sia possibile utilizzare l'applicativo;
\item Utilizzo del \glo{GPS} o altre soluzioni per monitorare un utente;
\item Utilizzo di un \glo{IDE} per la creazione di applicazioni mobile (Android o iOS);
\item Utilizzo di \glo{LDAP} (Lightweight Directory Access Protocol) che \`e un protocollo standard per l'interrogazione e la modifica dei servizi di directory, come ad esempio un elenco aziendale di e-mail o una rubrica telefonica, o più in generale qualsiasi raggruppamento di informazioni che può essere espresso come record di dati organizzato in modo gerarchico.
\end{itemize}

\subsection{Vincoli}
\begin{itemize}
\item Il server deve essere in grado di scalare in base al numero di utilizzatori in modo dinamico sia in aggiunta che in riduzione;
\item Viene richiesto di garantire una precisione sufficiente che permetta di certificare la presenza della persona all'interno degli edifici;
\item Effettuare test di tipo \glo{end-to-end};
\item Creazione di un'applicazione Android/iOS con relativa interfaccia grafica.
\end{itemize}

\subsection{Aspetti positivi}
\begin{itemize}
\item Il prodotto richiesto risulta essere accattivante ed utile poiché ci sono applicazioni concrete su vasta gamma di contesti;
\item Essendo Android molto diffuso la documentazione necessaria per realizzare l'applicazione \`e di immediata disponibilità;
\item Utilizzo di Java che risulta essere una tecnologia conosciuta da tutti i membri del team;
\item L'azienda ha esposto in modo chiaro i vari vincoli, i vari casi d'uso presenti nel capitolato, e inoltre fornisce una sorta di glossario su alcuni termini utilizzati all'interno del documento di presentazione del capitolato;
\item L'azienda \`e disponibile a fornirci diversi strumenti per il testing e a tenere lezioni per spiegarne il funzionamento e l'utilizzo.
\end{itemize}
\subsection{Aspetti critici}
\begin{itemize}
\item Essendo su un dispositivo mobile, il servizio dell'applicazione che si connette al server per inviare la propria posizione deve essere molto efficiente in termine di consumo della batteria;
\item Non sono chiari alcuni aspetti della \glo{modalità di tracciamento anonima};
\item Bisogna rispettare le normative vigenti in tema privacy.
\end{itemize}
\subsection{Conclusioni}
La proposta del capitolato offerto dall'azienda \Proponente{} è stata accolta con grande interesse. Il gruppo è rimasto colpito e stimolato dalla possibilità di poter creare un prodotto che possa essere impiegato in molte realtà sia aziendali sia di eventi di varia natura e dimensione. 
Nonostante la tecnologia Android esista da molti anni è risultata particolarmente interessante da parte del gruppo, sia perché è supportata da un'ampia community di sviluppatori, sia perché per il gruppo è una tecnologia nuova che non viene trattata da nessun corso della laurea triennale. Dopo l'analisi del capitolato è emersa una unanime preferenza per il capitolato \NomeProgetto{}.