\section{Capitolato C5}
\subsection{Titolo del capitolato}
Il capitolato in questione si chiama \textit{"Stalker"}, il proponente \`e l'azienda \textit{Imola Informatica} e i committenti sono Prof. Tullio Vardanega e Prof. Riccardo Cardin.

\subsection{Descrizione del capitolo}
\textit{Stalker} richiede la realizzazione di un sistema di monitoraggio delle posizioni delle persone (in versione anonima o non) grazie ad un'applicazione "complementare" installata sugli smartphone. La necessità del servizio sorge dal dover monitorare il numero e la posizione delle persone all'interno di fiere/musei/locali ai fini di sicurezza oppure per verificare la presenza e la locazione dei dipendenti di un'azienda. Nel primo caso si user\`a il tracciamento\ap{G}anonimo, nel secondo caso invece il tracciamento\ap{G}non anonimo. L'applicazione dovr\`a risalire alla posizione attuale grazie all'ausilio di infrastrutture e/o servizi resi disponibili dallo smartphone stesso, come per esempio beacons\ap{G}, GPS\ap{G}, dati del cellulare, ecc. Quando l'applicazione calcola la posizione aggiornata deve anche verificare se rientra in una zona da essere tracciata, in tal caso comunicherà col server remoto per autenticarsi (anonimamente o non) e risultare tracciabile dal sistema.

\subsection{Prerequisiti e tecnologie coinvolte}
Prerequisiti:
\begin{itemize}
\item Java.
\end{itemize}
Tecnologie coinvolte:
\begin{itemize}
\item Utilizzo di Java (versione 8 o superiori), python\ap{G} o nodejs\ap{G} per lo sviluppo del server back-end\ap{G};
\item Utilizzo di protocolli asincroni per le comunicazioni app mobile-server;
\item Utilizzo del pattern di Publisher/Subscriber, ovvero mittenti(publisher) e destinatari (Subscriber) di messaggi dialogano attraverso un tramite(dispatcher\ap{G});
\item Utilizzo dell'IAAS Kubernetes\ap{G} o di un PAAS\ap{G}, Openshift\ap{G} o Rancher\ap{G}, per il rilascio delle componenti del Server nonch\'e per la gestione della scalabilit\`a orizzontale\ap{G};
\item API Rest\ap{G} attraverso le quali sia possibile utilizzare l'applicativo;
\item Utilizzo del GPS\ap{G} o altre soluzioni per monitorare un utente;
\item Utilizzo di un'IDE\ap{G} per la creazione di applicazioni mobile (Android o iOS);
\item Utilizzo di LDAP\ap{G} (Lightweight Directory Access Protocol) che \`e un protocollo standard per l'interrogazione e la modifica dei servizi di directory, come ad esempio un elenco aziendale di e-mail o una rubrica telefonica, o pi\`u in generale qualsiasi raggruppamento di informazioni che pu\`o essere espresso come record di dati organizzato in modo gerarchico.
\end{itemize}

\subsection{Vincoli}
\begin{itemize}
\item Il server deve essere in grado di scalare in base al numero di utilizzatori in modo dinamico sia in aggiunta che in riduzione;
\item Viene richiesto di garantire una precisione sufficiente che permetta di certificare la presenza della persona all'interno degli edifici;
\item Effettuare test di tipo end-to-end\ap{G};
\item Creazione di un'applicazione Android/iOS con relativa interfaccia grafica.
\end{itemize}

\subsection{Aspetti positivi}
\begin{itemize}
\item Il prodotto richiesto risulta essere accattivante ed utile poich\'e ci sono applicazioni concrete su vasta gamma di contesti;
\item Essendo Android molto diffuso la documentazione necessaria per realizzare l'applicazione \`e di immediata disponibilit\`a;
\item Utilizzo di Java che risulta essere una tecnologia conosciuta da tutti i membri del team;
\item L'azienda ha esposto in modo chiaro i vari vincoli, i vari casi d'uso presenti nel capitolato, e inoltre fornisce una sorta di glossario su alcuni termini utilizzati all'interno del documento di presentazione del capitolato;
\item L'azienda \`e disponibile a fornirci diversi strumenti per il testing e a tenere lezioni per spiegarne il funzionamento e l'utilizzo.
\end{itemize}
\subsection{Aspetti critici}
\begin{itemize}
\item Essendo su un dispositivo mobile, il servizio dell'applicazione che si connette al server per inviare la propria posizione deve essere molto efficiente in termine di consumo della batteria;
\item Non sono chiari alcuni aspetti della modalit\`a anonima\ap{G};
\item Bisogna rispettare le normative vigenti in tema privacy.
\end{itemize}
\subsection{Conclusioni}
La proposta del capitolato offerto dall'azienda \textit{Imola Informatica} \`e stata accolta con grande interesse. Il gruppo \`e rimasto colpito e stimolato dalla possibilit\`a di poter creare un prodotto che possa essere impiegato in molte realt\`a sia aziendali sia di eventi di varia natura e dimensione. 
Nonostante la tecnologia Android esista da molti anni è risultata particolarmente interessante da parte del gruppo, sia perché è supportata da un'ampia community di sviluppatori, sia perché per il gruppo è una tecnologia nuova che non viene trattata da nessun corso della laurea triennale. Dopo l'analisi del capitolato è emersa una unanime preferenza per il capitolato \textit{Stalker}.