\section{Capitolato C3}
\subsection{Titolo del capitolato}
Il capitolato in questione si chiama "Natural API", il proponente \`e l'azienda teal.blue e i committenti sono Prof. Tullio Vardanega e Prof. Riccardo Cardin.

\subsection{Descrizione del capitolo}
Considerando i vari personaggi che prendono parte allo sviluppo di un progetto (clienti, project managers, developers, ecc.), \`e chiaro che non tutti si esprimano nello stesso linguaggio. Difficolt\`a nel descrivere l'idea che si vuole trasmettere pu\`o ripercuotersi sull'efficienza\ap{G}, l'efficacia\ap{G} e il costo per sviluppare il progetto. Natural API\ap{G} mira a sviluppare un toolkit\ap{G} in grado di colmare la distanza tra le specifiche di progetto e le API. In pratica \`e composta da tre processi distinti: il primo (Discover) deve estrarre automaticamente una lista di predicati, nomi e verbi da file testuali dati in input (per esempio guide, manuali ecc.), creando il Business Domain Language (BDL).
Il secondo (Design) deve combinare le specifiche degli stakeholders in gherkin e il BDL per creare il business application language (BAL), ovvero lo step intermedio tra linguaggio naturale e codice. Il BAL deve poter integrare requisiti addizionali e poter fondere o dividere azioni ed espandere nomi in oggetti.
Il terzo (Develop) invece tramuta in codice il BAL in un linguaggio di programmazione concreto. Deve essere interattivo in modo da specificare dei dettagli del linguaggio di programmazione.
Il capitolato pone quindi come obiettivo l'implementazione di un sistema in grado di trasformare specifiche definite in linguaggio naturale in codice di un reale linguaggio di programmazione.

\subsection{Prerequisiti e tecnologie coinvolte}
Prerequisiti:
\begin{itemize}
\item gherkin\ap{G}: \url{https://cucumber.io/docs/gherkin/}
\item natural language processing\ap{G}: \url{https://en.wikipedia.org/wiki Natural_language_processing}
\item API\ap{G} generation: \url{https://github.com/OAI/OpenAPI-Specification}
\item code generation: \url{https://swagger.io/}
\item behavior-driven development\ap{G}: \url{https://dannorth.net/introducing-bdd/}
\item cucumber\ap{G}: \url{https://cucumber.io/docs}
\item clean architecture: \url{https://blog.cleancoder.com/uncle-bob/2012/08/13/the-clean-architecture.html}
\end{itemize}
Tecnologie consigliate:
\begin{itemize}
\item Qt\ap{G};
\item Python\ap{G};
\item React\ap{G};
\item cucumber\ap{G}.
\end{itemize}

\subsection{Vincoli}
Sono imposti i seguenti vincoli:
\begin{itemize}
\item Ogni parte dell'applicazione deve soddisfare degli standard sull'output che produce
almeno due modi di interagire con l'applicazione tra i seguenti:
\begin{itemize}
\item linea di comando;
\item interfaccia grafica\ap{G};
\item interfaccia web\ap{G}.
\end{itemize}

\item Rispettare delle specifiche sui logic layer\ap{G}. Si incoraggia l'applicazione del Joel Test I logic layers dovranno essere rilasciati in uno dei seguenti metodi:
\begin{itemize}
\item come una libreria (statica o dinamica);
\item come parte di un eseguibile;
\item come un processo/servizio indipendente, in locale o remoto.
\end{itemize}

\item Per la delivery mode sar\`a sufficiente l'accessibilit\`a da almeno un sistema operativo tra:
\begin{itemize}
\item Linux;
\item Window;
\item OS X.
\end{itemize}

\item input e output in codifica UTF-8\ap{G};
\item licenza Open Source;
\item Il codice del progetto dovr\`a essere locato in un repository\ap{G} facilmente accessibile dal pubblico (ad esempio GitHub\ap{G}).
\end{itemize}

\subsection{Aspetti positivi}
\`E di ovvia utilit\`a un sistema che sia in grado di soddisfare le esigenze sopracitate, poich\'e sarebbe applicabile in ogni azienda dove si sviluppi software.

\subsection{Aspetti critici}
\begin{itemize}
\item Risulta poco chiaro il passaggio dal BDL al BAL;
\item L'obiettivo finale del capitolato \`e prodotto per le aziende e il linguaggio utilizzato (gherkin) \`e di
nicchia e la conversazione tra stakeholder e UML\ap{G} sono ancora lo standard.
\end{itemize}

\subsection{Conclusioni}
L'argomento trattato dal capitolato \`e risultato essere poco interessante secondo il gruppo, questo perch\'e viene richiesto l'utilizzo di tecnologie poco conosciuta e di nicchia \`e il prodotto finale risulterebbe poco soddisfacente per il gruppo dopo il lungo lavoro richiesto per realizzarlo. Analizzando perci\`o gli aspetti positivi e quelli critici non \`e emerso un forte interesse per il capitolato.