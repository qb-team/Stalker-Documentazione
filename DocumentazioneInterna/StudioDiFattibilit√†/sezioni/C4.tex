\section{Capitolato C4}
\subsection{Titolo del capitolato}
Il capitolato in questione si chiama \textit{"Predire in Grafana"}, il proponente è l'azienda \textit{Zucchetti} e i committenti sono Prof. Tullio Vardanega e Prof. Riccardo Cardin.

\subsection{Descrizione del capitolo}
Si vuole realizzare un plug-in\ap{G} per lo strumento di monitoraggio Grafana\ap{G}. Il plug-in\ap{G} deve calcolare una predizione con l'applicazione del Support Vector Machine\ap{G} (SVM) e della regressione lineare\ap{G} (RL) al flusso di dati ricevuti da una data sorgente. La predizione sarà utile per decretare quando sia il caso di mandare dei segnali di allarme (in caso delle apparecchiature siano a rischio di sovraccarico) ai manutentori/sistemisti dei server per agire preventivamente.

\subsection{Prerequisiti e tecnologie coinvolte}
Prerequisiti:
\begin{itemize}
\item Cenni di calcolo numerico (RL\ap{G} e SVM\ap{G});
\item JavaScript\ap{G} per la realizzazione dei plug-in\ap{G};
\item Algoritmi di Machine Learning\ap{G} (l'azienda \`e disposta a fornire la formazione necessaria a riguardo).
\end{itemize}
Tecnologie da coinvolte:
\begin{itemize}
\item JavaScript\ap{G}, linguaggio di programmazione richiesto per costruire i plug-in di Grafana\ap{G} ;
\item Libreria in JavaScript\ap{G} per le SVM\ap{G};
\item Libreria in JavaScript\ap{G} per la RL\ap{G};
\item Libreria in JavaScript\ap{G} per le reti neurali;
\item Prodotto di monitoraggio Grafana\ap{G}, software open-source che, ricevuti i dati in input, consente di raccoglierli in un cruscotto, visualizzarli, analizzarli, misurarli e controllarli;
\item Orange Canvas\ap{G}, strumento per l'analisi dei dati.
\end{itemize}


\subsection{Vincoli}
\begin{itemize}
\item La realizzazione dei plug-in\ap{G} deve essere fatta attraverso JavaScript\ap{G};
\item il programma dovr\`a svolgere le seguenti compiti:
\begin{itemize}
\item Produrre un file json dai dati di addestramento con i parametri per le previsioni con Support Vector;
\item Machine (SVM\ap{G}) per le classificazioni o la Regressione Lineare\ap{G} (RL o LR da Linear Regression in inglese);
\item Leggere la definizione del predittore dal file in formato json\ap{G};
\item Associare i predittori letti dal file json al flusso di dati presente in Grafana\ap{G};
\item Applicare la previsione e fornire i nuovi dati ottenuti dalla previsione al sistema di Grafana;
\item Rendere disponibili i dati al sistema di creazione di grafici e dashboard\ap{G} per la loro visualizzazione.
\end{itemize}
\end{itemize}

\subsection{Aspetti positivi}
\begin{itemize}
\item La capacit\`a di analizzare i dati e fornire previsioni sono skill ampiamente richiesta tra le varie aziende;
\item Viene lasciata abbastanza libertà al gruppo nella scelta delle tecnologie da utilizzare per la realizzazione del prodotto richiesto, ciò nonostante l'azienda \textit{Zucchetti} suggerisce che tecnologie adottare;
\item L'azienda si presenta come la prima software house italiana, per storia e dimensione;
\item Poche tecnologie nuove da integrare tra le conoscenze per sviluppare il prodotto richiesto;
\item La \textit{Zucchetti} tra tutte le aziende coinvolte nei capitolati proposti risulta essere la pi\`u vicina tra i vari membri del gruppo, infatti tra le varie sedi presenti, l'azienda risulta essere presente sia a Padova sia a Treviso.

\end{itemize}

\subsection{Aspetti critici}
\begin{itemize}
\item Non è chiaro come le librerie fornite debbano essere implementate e con che flessibilità possa analizzare diversi tipi di dati in input;
\item La presentazione del capitolato \`e stata molto approssimativa;
\item L'utilizzo di conoscenze apprese durante il corso di calcolo numerico (RL\ap{G} e SVM\ap{G}) risulta essere poco stimolante.

\end{itemize}
\subsection{Conclusioni}
Nel capitolato viene trattato l'argomento del machine learning\ap{G},  purtroppo marginalmente e l'utilizzo del software Grafana non ha stimolato grande interesse all'interno del gruppo, perci\`o analizzando gli aspetti positivi e quelli critici e dopo una discussione fra i membri del gruppo, \`e emerso un discreto interesse per il capitolato ci\`o nonostante \`e stato preferito accantonare il capitolato .