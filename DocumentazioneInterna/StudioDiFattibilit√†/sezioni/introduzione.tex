\section{Introduzione}
\subsection{Scopo del Documento}
Questo documento contiene la stesura del studio di fattibilità riguardante i sei capitolati proposti, per ciascuno di essi verranno evidenziati i seguenti aspetti:
\begin{itemize}
\item Titolo del capitolato
\item Descrizione generale
\item Prerequisiti e tecnologie coinvolte
\item Vincoli
\item Aspetti positivi
\item Aspetti critici
\end{itemize}
Infine per ogni capitolo verranno esposte le motivazioni e le ragioni per cui il gruppo ha scelto come progetto il capitolato C5 Stalker a discapito degli altri cinque capitolati proposti.

\subsection{Glossario}
Si è scelto di redigere un documento di supporto "Glossario" al fine di rendere chiaro il documento e di evitare ogni possibile ambiguità dovuta al linguaggio utilizzato. In questo documento vengono raccolti vari termini che possono essere incontrati durante la lettura dei vari documenti prodotti dal gruppo, e per ognuno dei termini inseriti nel glossari  verrà esposta una definizione e una descrizione. Per facilitare la lettura i termini presenti nel glossario verranno indicati con il marcatore "G".
	
\subsection{Riferimenti}

\subsubsection{Normativi}
\begin{itemize}
\item Norme di Progetto v1.0.0.
\end{itemize}

\subsubsection{Informativi}

\begin{itemize}
\item \textbf {Capitolato d'appalto C1 -Autonomous Highlights Platform}\\
\url{https://www.math.unipd.it/~tullio/IS-1/2019/Progetto/C1.pdf}
\item \textbf {Capitolato d'appalto C2 -Etherless}\\
\url{https://www.math.unipd.it/~tullio/IS-1/2019/Progetto/C2.pdf}
\item \textbf {Capitolato d'appalto C3 -Natural API}\\
\url{https://www.math.unipd.it/~tullio/IS-1/2019/Progetto/C3.pdf}
\item \textbf {Capitolato d'appalto C4 -Predire in Grafana}\\
\url{https://www.math.unipd.it/~tullio/IS-1/2019/Progetto/C4.pdf}
\item \textbf {Capitolato d'appalto C5 -Stalker}\\
\url{https://www.math.unipd.it/~tullio/IS-1/2019/Progetto/C5.pdf}
\item \textbf {Capitolato d'appalto C6 -ThiReMa}\\
\url{https://www.math.unipd.it/~tullio/IS-1/2019/Progetto/C6.pdf}

\end{itemize}