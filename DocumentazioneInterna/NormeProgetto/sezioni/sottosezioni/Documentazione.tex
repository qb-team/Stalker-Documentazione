\subsection{Documentazione}
\subsubsection{Scopo}
Lo scopo di questa sezione è di redigere e standardizzare i documenti prodotti durante tutto il ciclo di vita del software. 
Di conseguenza ci si aspetta di avere:
\begin{itemize}
\item Una struttura ben organizzata e con una facile navigabilità;
\item Una serie di norme tipografiche da rispettare.
\end{itemize}
I documenti possono essere consultati nel seguente repository di \glo{GitHub}: \url{https://github.com/qb-team/Stalker-Documentazione}.

\subsubsection{Ciclo di vita}
Ogni documento prima di essere presentato deve passare per tre stati fondamentali:
\begin{enumerate}
\item \textbf{Stesura del documento}: Creazione del documento e stesura con il linguaggio \LaTeX;
\item \textbf{Verifica del Documento}: Il documento viene assegnato ad un verificatore, il cui lavoro è controllare che il documento rispetti gli standard definiti;
\item \textbf{Approvazione del documento}: In caso la verifica risulti positiva, il documento viene consegnato al \Responsabile{} per l'approvazione al rilascio.
\end{enumerate}

\subsubsection{Template}
È presente un template per i documenti, realizzato anch'esso in \LaTeX{}, per standardizzare e velocizzare la stesura della documentazione.
Ogni documento che viene redatto deve includere al suo interno il file di stile Stiletemplate.sty presente nella cartella Utilita/.
Il template fornisce i seguenti comandi:
\begin{itemize}
\item \textbf{\textbackslash copertina\{\}}: Da inserire all'inizio del documento, ne inserisce la prima pagina (o copertina);
\item \textbf{\textbackslash fancydoc\{\}}: Da inserire dopo l'indice, inserisce l'intestazione e il piè di pagina del documento. Va utilizzato in ogni documento, ad eccezione dei verbali;
\item \textbf{\textbackslash fancyverbale\{\}}: Da inserire dopo l'indice, inserisce l'intestazione e il piè di pagina del verbale. Da utilizzare solamente nei verbali.
\end{itemize}

\subsubsection{Struttura dei documenti}
Ogni documento è caratterizzato da una struttura che dovrà seguire obbligatoriamente.
Di seguito viene elencato, per ogni sezione del documento, le sue caratteristiche e la sua posizione.

\paragraph{Prima pagina - Copertina}\mbox{}
\begin{itemize}
\item \textbf{Logo del gruppo}: Il logo viene posizionato al centro della pagina, in alto;
\item \textbf{Titolo del documento}: Indica il nome del documento ed è posizionato sotto al logo del gruppo;
\item \textbf{Informazioni gruppo}: Nome del gruppo (\Gruppo{}) e del capitolato, seguiti subito dopo dall'e-mail del gruppo; 
\item \textbf{Informazioni documento}: Tabella posizionata al centro della pagina contenente le seguenti informazioni sul documento:
\begin{itemize}
\item Versione;
\item Approvatori;
\item Redattori;
\item Verificatori;
\item Uso;
\item Distribuzione.
\end{itemize}
\item \textbf{Descrizione documento}: Breve descrizione relativa al documento posizionata in fondo alla pagina, centrata.
\end{itemize}

\paragraph{Registro delle modifiche} \mbox{} \\
Tabella contenente diverse informazioni sul ciclo di vita del documento, composta come segue:
\begin{itemize}
\item Versione;
\item Data;
\item Nominativo;
\item Ruolo;
\item Descrizione.
\end{itemize}
Questa tabella non deve rispettare la norma definita in "Elementi grafici", poiché non deve presentare una didascalia.

\paragraph{Indice}\mbox{} \\ \\
Contiene i titoli di tutte le sezioni e sottosezioni del documento, rendendo più facile la navigazione.
Se sono presenti tabelle o immagini all'interno del documento, esse possono essere riepilogate in un indice separato, se ritenuto necessario.

\paragraph{Riferimenti}\mbox{} \\ \\
Tutti i riferimenti bibliografici, normativi ed informativi, vengono inseriti in ogni documento in cui siano necessari come ultima sezione del documento, sotto forma di elenco puntato.

\paragraph{Intestazione - Piè di pagina} \mbox{} \\ \\
Il contenuto del documento è posto tra intestazione e il piè di pagina:

\begin{itemize}
\item In alto a sinistra è presente il logo del gruppo \Gruppo{};
\item In alto una linea orizzontale separa l’intestazione dal contenuto;
\item In basso a sinistra è presente il titolo del documento;
\item In basso a destra è presente il numero della pagina;
\item In basso una linea orizzontale separa il piè di pagina dal contenuto.
\end{itemize}

\paragraph{Elementi grafici}\mbox{}
\begin{itemize}
\item \textbf{Tabelle}: Centrate, con la didascalia posizionata al di sopra di esse;
\item \textbf{Diagrammi}: Centrati, con la didascalia posizionata al di sotto di essi;
\item \textbf{Immagini}: Centrate, con la didascalia posizionata al di sotto di esse.
\end{itemize}

\paragraph{Stile del testo}\mbox{}
\begin{itemize}
    \item \textbf{Grassetto}:
    \begin{itemize}
        \item Utilizzato per evidenziare le voci di un elenco puntato, che vengono descritte in loco;
        \item Parole ritenute particolarmente importanti in un testo;
        \item Termini del documento \Glossario{};
        \item Codici identificativi, quando introdotti per la prima volta;
    \end{itemize}
    \item \textit{Corsivo}:
    \begin{itemize}
        \item Nome del capitolato;
        \item Nome del proponente;
        \item Nome del gruppo.
    \end{itemize}
    \item \textcolor{blue}{Link ipertestuali}: Tutti i link ipertestuali devono essere di colore blu.
\end{itemize}

\paragraph{Glossario}\mbox{} \\ \\
Ogni termine di un documento che contiene una definizione nel \Glossario{} viene identificato nel seguente modo:
\begin{center}
    \glo{termine}
\end{center}

\paragraph{Data}\mbox{} \\ \\
Si è deciso di seguire uno dei formati più diffusi per la rappresentazione della data:
\begin{center}
\textbf{YYYY-MM-DD}
\end{center}
in cui \textbf{YYYY} rappresenta l'anno, \textbf{MM} il mese e \textbf{DD} il giorno.

\paragraph{Elenchi puntati/numerati}\mbox{} \\ \\
Ogni voce di un elenco ordinato o numerato comincia con la lettera maiuscola e termina con punto e virgola (\textbf{";"}) , tranne per l'ultimo elemento dell'elenco che termina con un punto fermo (\textbf{"."}).
Nel caso di elenchi che definiscono un termine:
\begin{itemize}
    \item Il termine deve essere in grassetto;
    \item Il termine deve essere seguito da due punti (\textbf{":"});
    \item La definizione del termine inizia con la lettera maiuscola.
\end{itemize}

\paragraph{Nomenclatura dei documenti}\mbox{} \\ \\
I nomi di file (escludendo l'estensione) e cartelle sono scritti usando la convenzione "Camel Case\ap{G}".
I file della documentazione prodotti avranno i seguenti nomi:
\begin{itemize}
    \item \textbf{AnalisiDeiRequisiti.pdf}: Contenente il documento \AdR{};
    \item \textbf{PianoDiProgetto.pdf}: Contenente il documento \PdP{};
    \item \textbf{PianoDiQualifica.pdf}: Contenente il documento \PdQ{};
    \item \textbf{NormeDiProgetto.pdf}: Contenente il documento \NdP{};
    \item \textbf{StudioDiFattibilita.pdf}: Contenente il documento \SdF{}.
\end{itemize}
I file dei verbali, esterni e interni, avranno i seguenti nomi:
\begin{itemize}
    \item \textbf{VI\_[YYYY]\_[MM]\_[DD].pdf}: Contenente il verbale interno del [YYYY]-[MM]-[DD];
    \item \textbf{VE\_[YYYY]\_[MM]\_[DD].pdf}: Contenente il verbale esterno del [YYYY]-[MM]-[DD].
\end{itemize}
con:
\begin{itemize}
    \item \textbf{[YYYY]} l'anno in cui si è tenuto l'\glo{incontro formale};
    \item \textbf{[MM]} il mese in cui si è tenuto l'\glo{incontro formale};
    \item \textbf{[DD]} il giorno in cui si è tenuto l'\glo{incontro formale}.
\end{itemize}

\paragraph{Tabelle nei documenti}\mbox{} \\ \\
Le tabelle utilizzate nei documenti si differenziano sostanzialmente dalla loro struttura e dal colore.
Le tabelle con intestazione di colore:
\begin{itemize}
    \item \textcolor{rossoep}{\textbf{Rosso}}: Sono solamente le tabelle del "Registro delle modifiche";
    \item \textcolor{darkblue}{\textbf{Blu}}: Sono tutte le altre tabelle.
\end{itemize}
Le intestazioni delle tabelle devono ripetersi in ogni pagina, qualora si dovessero spezzare per mancanza di spazio.

\subsubsection{Linguaggi}
\paragraph{\LaTeX}\mbox{} \\ \\
Per quanto riguarda la stesura dei documenti il gruppo ha scelto di utilizzare il linguaggio \LaTeX{}, il quale consente una migliore qualità tipografica rispetto ai normali software di videoscrittura, ma soprattutto facilita il versionamento e la suddivisione in parti di un documento.

\paragraph{UML}\mbox{}\\ \\
Per la realizzazione dei diagrammi dei casi d'uso, delle classi, dei package, delle attività e delle sequenze viene utilizzato il linguaggio di modellazione \textbf{UML}, in particolare la versione v2.0 dello standard.

\subsubsection{Strumenti}
\paragraph{\LaTeX}\mbox{} \\ \\
Per quanto riguarda gli editor per l'utilizzo e la scrittura di \LaTeX{} vengono scelti come ufficiali:
\begin{itemize}
	\item \href{https://www.xm1math.net/texmaker/}{Texmaker}, che include editor e supporto alla correzione ortografica;
	\item \href{https://code.visualstudio.com/}{Visual Studio Code}, insieme ai plugin \href{https://github.com/James-Yu/LaTeX-Workshop}{LaTeX Workshop} (per il supporto a \LaTeX) e \href{https://github.com/bartosz-antosik/vscode-spellright}{Spell Right} (per il supporto alla correzione ortografica).
\end{itemize}
I file di \LaTeX{} (file \textbf{.tex}), per generare file PDF distribuibili, devono essere compilati.
Per \LaTeX{} sono disponibili parecchi compilatori, ma avendo definito come ufficiali i soli editor TexStudio e Visual Studio Code,
i compilatori e strumenti necessari per la compilazione sono:
\begin{itemize}
    \item \textbf{MiKTeX}: Disponibile a \href{https://miktex.org/}{questo indirizzo}, se si sceglie l'utilizzo dell'editor Texmaker;
    \item \textbf{TeX Live}: Disponibile a \href{https://www.tug.org/texlive/}{questo indirizzo}, se si sceglie l'utilizzo dell'editor Visual Studio Code (nota: è una distribuzione completa di \LaTeX, quindi offre anche strumenti che possono non essere utili ai fini del progetto);
    \item \textbf{Perl}: Disponibile a \href{http://strawberryperl.com/}{questo indirizzo} per sistemi Windows, mentre, seguendo la guida di \href{https://learn.perl.org/installing/unix_linux.html}{quest'altro} per sistemi Unix/Linux, se si sceglie l'utilizzo dell'editor Visual Studio Code.
\end{itemize}
Per compilare un file \textbf{.tex} e generare un file PDF corretto deve essere invocato tre volte il comando \textbf{pdflatex $nomeFile$}, dove $nomeFile$ è il percorso ad un file con estensione \textbf{.tex} contenente al suo interno le istruzioni:
\begin{itemize}
    \item \textbackslash begin\{document\};
    \item \textbackslash end\{document\}.
\end{itemize}
Questi file hanno sempre il nome del documento che producono, con le convenzioni di nomenclatura indicate nelle \NdP{}.

\paragraph{UML}\mbox{}\\ \\
Per realizzare tutti i diagrammi in UML viene utilizzato il software \href{https://draw.io}{Draw.io}, che è disponibile sia come \glo{applicazione we} accessibile da browser, che come applicazione desktop.
Draw.io permette sia di salvare i file di lavoro per poterli modificare, sia di esportarli.
Il salvataggio dei file di lavoro può avvenire in file in formato:
\begin{itemize}
    \item \textbf{XML}: Il file ha estensione \textbf{.xml};
    \item \textbf{Draw.io}: Il file ha estensione \textbf{.drawio}.
\end{itemize}
Il salvataggio delle immagini da includere nei documenti deve essere in formato \textbf{PNG}, con estensione \textbf{.png}, ed esportata con i seguenti parametri:
\begin{itemize}
    \item \textbf{Spessore bordo}: 20;
    \item \textbf{DPI}: 400dpi;
    \item \textbf{Sfondo}: Non trasparente (spunta rimossa).
\end{itemize}
Le seguenti impostazioni di esportazioni si trovano in File $\rightarrow$ Esporta come $\rightarrow$ Avanzate\dots; il nome del file e lo zoom, così come profondità e altezza, sono indifferenti.
