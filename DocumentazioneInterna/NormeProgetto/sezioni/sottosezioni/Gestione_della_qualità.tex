%scritto da Federico Perin
\subsection{Gestione della qualità}
\subsubsection{Obiettivo}
Il gruppo \Gruppo{} ha come obiettivo prefissato di essere sistematico\ap{G}, disciplinato\ap{G} e quantificabile\ap{G}, ai fini di 
\begin{itemize}
    \item garantire la qualità nel prodotto software da realizzare;
    \item soddisfare le richieste del proponente e del committente;
    \item migliorare le proprie capacità di gestione di un progetto software.
\end{itemize}

\subsubsection{Piano di Qualifica}
Nel documento \textbf{Piano di Qualifica} il gruppo illustra come intende gestire la qualità di processo e di prodotto, elenca le metriche definite per aderire alle definizioni dello standard e i test per verificare la soddisfazione dei requisiti del prodotto software.
La qualità di processo e di prodotto sono due aspetti chiaramente coordinati, ma vengono gestiti separatamente e soprattutto si basano su due standard diversi:
\begin{itemize}
    \item la qualità di processo si basa sullo standard ISO/IEC 12207;
    \item la qualità di prodotto si basa sullo standard ISO/IEC 9126.
\end{itemize}
Il documento si occupa poi di indicare come poter calcolare i valori per le metriche indicate e gli strumenti per farlo.