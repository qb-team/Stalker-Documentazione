\subsection{Documentazione}
\subsubsection{Scopo}
Lo scopo di questa sezione è di redigere e standardizzare i documenti prodotti durante tutto il ciclo di vita del software. 
Di conseguenza ci si aspetta di avere:
\begin{itemize}
\item Una struttura ben organizzata e con una facile navigabilità;
\item Un serie di norme tipografiche da rispettare.
\end{itemize}
I documenti possono essere consultati nella seguente repository di GitHub\ap{G}: \url{https://github.com/qb-team/Stalker-Documentazione}.

\subsubsection{Ciclo di vita}
Ogni documento prima di essere presentato deve passare per 3 stati fondamentali:
\begin{enumerate}
\item \textbf{Stesura del documento}: Creazione del documento e stesura in latex;
\item \textbf{Verifica del Documento}: Il documento viene assegnato ad un verificatore, nella quale controllerà se il documento rispetti determinati standard;
\item \textbf{Approvazione del documento}: In caso la verifica risulti positiva, il documento sarà consegnato al responsabile il quale lo approverà per il rilascio.
\end{enumerate}

\subsubsection{Template}
È stato creato un template in Latex per standardizzare e velocizzare la stesura dei documenti.
Ogni membro deve includere nel proprio documento il file di style Stiletemplate.sty presente nella cartella template, così facendo il suo lavoro sarà semplificato grazie all'uso di semplici comandi, dei quali i più ricorrenti:
\begin{itemize}
\item \textbf{\textbackslash copertina{}}: Da inserire all'inizio del documento, inserisce la copertina del documento;
\item \textbf{\textbackslash fancy"nome del progetto"{}}: Da inserire dopo l'indice, inserisce l'intestazione e il piè di pagina del documento.
\end{itemize}

\subsubsection{Struttura dei documenti}
Ogni documento è caratterizzato da una struttura che dovrà seguire obbligatoriamente.
Di seguito sarà elencato, per ogni elemento, le sue caratteristiche e la sua posiziona nel foglio.

\paragraph{Prima pagina - Frontespizio}\mbox{}
\begin{itemize}
\item \textbf{Logo del gruppo}: Il logo è in posizione alta ed accentrata;
\item \textbf{Titolo del documento}: Indica la tipologia del documento ed è posizionato sotto al logo;
\item \textbf{Informazioni qbteam}: Nome del gruppo \Gruppo e del capitolato, seguiti subito sotto dall'email del gruppo \Gruppo; 
\item \textbf{Informazioni documento}: Tabella accentrata contenente le informazioni sul documento:
\begin{itemize}
\item Versione;
\item Approvatore;
\item Redattori;
\item Verificatori;
\item Uso;
\item Distribuzione.
\end{itemize}
\item \textbf{Descrizione documento}: Breve descrizione relativa al documento posizionata in centro e nel basso della pagina.
\end{itemize}

\paragraph{Registro delle modifiche} \mbox{} \\
Tabella contenente diverse informazioni sul ciclo di vita del documento, composta come segue:
\begin{itemize}
\item Versione;
\item Data;
\item Nominativo;
\item Ruolo;
\item Descrizione.
\end{itemize}

\paragraph{Indice}\mbox{} \\
Contiene i titoli di tutte le sezioni e sottosezioni del documento rendendo più facile la navigazione.
Se sono presenti tabelle o immagini all'interno del documento saranno riepilogate in un indice separato.

\paragraph{Intestazione - Piè di pagina} \mbox{} \\
Il contenuto del documento è posto tra intestazione e il piè di pagina:

\begin{itemize}
\item In alto a sinistra è presente il logo del gruppo \Gruppo;
\item In alto una linea orizzontale separa l’intestazione dal contenuto;
\item In basso a sinistra è presente il titolo del documento;
\item In basso a destra è presente il numero della pagina;
\item In basso una linea orizzontale separa il piè di pagina dal contenuto.
\end{itemize}
In caso del verbale, sarà presente anche la data in cui è stato steso in basso a destra.

\paragraph{Elementi grafici}\mbox{}
\begin{itemize}
\item Tabelle - Diagrammi: Centrate con la didascalia posizionata al di sopra di esse;
\item Immagini: Centrate con la didascalia posizionata al di sotto di esse.
\end{itemize}

\subsubsection{Norme tipografiche}
I nomi di file(escludendo l'estensione) e cartelle sono scritti usando la convenzione "CamelCase\ap{G}". Inoltre il gruppo \Gruppo ha scelto di adottare delle regole aggiuntive, quali:
\begin{itemize}
\item I nomi dei file iniziano con la lettera maiuscola;
\item I nomi dei file composti da più parole vengono scritti tutti attaccati e ogni parola inizia con la lettera maiuscola;
\item Le preposizioni non di omettono;

\paragraph{Stile del testo}\mbox{}
\begin{itemize}
\item Grassetto:
\begin{itemize}
\item Utilizzato per evidenziare le voci di un elenco puntato;
\item Titoli delle sezioni e sottosezioni;
\item Parole ritenute importanti in un testo.
\end{itemize}
\item Corsivo:
\begin{itemize}
\item Nome del capitolato;
\item Nome dell'azienda proponente;
\item Nome del gruppo.
\end{itemize}
\item Link: tutti i collegamenti URL devono essere di colore blu.
\end{itemize}

\paragraph{Glossario}\mbox{} \\
Ogni termine del glossario sarà identificato nel seguente modo:
\begin{center}
termine\ap{G}
\end{center}

\paragraph{Data}\mbox{} \\
Si è deciso di seguire uno dei formati più diffusi per la rappresentazione della data:
\begin{center}
\textbf{YYYY-MM-DD}
\end{center}
in cui \textbf{YYYY} rappresenta l'anno, \textbf{MM} il mese e \textbf{DD} il giorno.

\paragraph{Elenchi puntati/numerati}\mbox{} \\
Ogni voce di un elenco comincia con la lettera maiuscola e termina con \textbf{";"} , tranne per l'ultimo elemento che terminerà con un \textbf{"."} . 
Nel caso di elenchi che definiscono un termine, esso sarà in grassetto seguito da \textbf{":"} e la lettera maiuscola per la descrizione del termine.

\paragraph{Nomenclatura dei documenti}\mbox{} \\
DA DEFINIRE

\subsubsection{Strumenti}

\paragraph{\LaTeX}\mbox{} \\
La scrittura dei documenti dovrà essere realizzata con \LaTeX ,  linguaggio di markup,  particolarmente  indicato  per  l'elaborazione  di  documenti.
Gli editor utilizzati sono \TeX maker e \TeX studio.

\paragraph{Draw.io}\mbox{} \\
Una piattaforma che permette di creare diagrammi UML direttamente dal proprio browser Web.


