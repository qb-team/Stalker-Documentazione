%scritto da Federico Perin
\subsection{Gestione della qualità}
\subsubsection{Obiettivo}
Il gruppo \Gruppo{} ha come obiettivo prefissato di essere sistematico, disciplinato e quantificabile, ai fini di garantire la qualità nel prodotto software da realizzare al fine di soddisfare le richieste del proponente e del committente e per migliorare le proprie capacità di gestione di un progetto software.

\subsubsection{Piano di Qualifica}
Nel documento \textbf{Piano di Qualifica} il gruppo illustra come intende gestire la qualità di processo e di prodotto, elenca le metriche definite per aderire alle definizioni dello standard e i test per verificare la soddisfazione dei requisiti del prodotto software.
La qualità di processo e di prodotto sono due aspetti chiaramente coordinati, ma vengono gestiti separatamente e soprattutto si basano su due standard diversi:
\begin{itemize}
    \item la qualità di processo si basa sullo standard ISO/IEC 12207;
    \item la qualità di prodotto si basa sullo standard ISO/IEC 9126.
\end{itemize}
Il documento si occupa poi di indicare come poter calcolare i valori per le metriche indicate e gli strumenti per farlo.

\setlength{\parindent}{-0.1em}
\paragraph{Metriche di qualità}
bla bla bla
\subparagraph{Codici metriche}
bla bla bla 2
\subparagraph{Struttura descrittiva metriche}
bla bla bla 3
\subparagraph{Tabella riassuntiva metriche}
bla bla bla 4

\subsubsection{Strumenti per il controllo di qualità}
bla bla bla 5