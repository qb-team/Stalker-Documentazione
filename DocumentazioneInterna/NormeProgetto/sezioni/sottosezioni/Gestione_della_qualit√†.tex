%scritto da Federico Perin
\subsection{Gestione della qualità}
\subsubsection{Obiettivo}
Il gruppo \Gruppo ha come obiettivo prefissato di essere \glo{sistematico}, \glo{disciplinato} e \glo{quantificabile}, ai fini di:
\begin{itemize}
    \item garantire la qualità nel prodotto software da realizzare;
    \item soddisfare le richieste del proponente e del committente;
    \item migliorare le proprie capacità di gestione di un progetto software.
\end{itemize}

\subsubsection{Piano di Qualifica}
Nel documento \textbf{Piano di Qualifica} il gruppo \Gruppo illustra come intende gestire la \glo{qualità di processo} e di \glo{qualità di prodotto}, elenca le varie metriche definite per aderire alle definizioni dello standard e i test per verificare la soddisfazione dei requisiti del prodotto software.
La \glo{qualità di processo} e di \glo{qualità di prodotto} sono due aspetti chiaramente coordinati, ma vengono gestiti separatamente. \\ \\
I capitoli principali del documento sono i seguenti:
\begin{itemize}
    \item \textbf{Qualità di processo:} sezione dove vengono elencate le metriche inerenti al processo;
    \item \textbf{Qualità di prodotto:} sezione dove vengono elencate le metriche inerenti al prodotto;
    \item \textbf{Strategia di testing:} sezione dove viene elencato il piano di testing delle componenti e del sistema software nel suo complesso;
    \item \textbf{Standard di qualità adottati:} sezione dove vengono spiegati gli standard adottati.
\end{itemize}

\paragraph{Metriche di qualità}\mbox{}\\ \\
\textbf{Metriche di processo:}\\
Vengono utilizzate delle metriche che servono per monitorare lo stato dei processi scelti nello standard ISO/IEC 12207. Il Responsabile, grazie ai valori ricavati dalle metriche, è facilitato nel
valutare il \glo{processo} e di eseguire modifiche alla pianificazione, se necessario.\\

\textbf{Metriche di prodotto:}\\
Il modello di qualità del software descritto dalle norme ISO/IEC 9126 definisce le caratteristiche e attributi del software, ciascuna misurabile da metriche interne o esterne.
Una volta specificati i requisiti di qualità del prodotto software, si identificano le caratteristiche e attributi di qualità che più contribuiscono ad indirizzare i requisiti elencati.
Il gruppo \Gruppo{} si è impegnato a scegliere le metriche interne che maggiormente influenzano le caratteristiche esterne del prodotto finale, in modo che esse possano predire quanto più possibile il risultato finale.

\subparagraph{Codici metriche}\mbox{}\\ \\
Ogni metrica di \glo{processo} ha un codice univoco ed è strutturato in questo formato \textbf{MPC-00}:
\begin{itemize}
    \item \textbf{M}: metrica;
    \item \textbf{PC}: \glo{processo};
    \item \textbf{-} : trattino separatore;
    \item \textbf{00}: numero incrementale.
\end{itemize}
Ogni metrica di prodotto ha un codice univoco ed è strutturato in questo formato \textbf{MPD-00}:
\begin{itemize}
    \item \textbf{M}: metrica;
    \item \textbf{PD}: prodotto;
    \item \textbf{-} : trattino separatore;
    \item \textbf{00}: numero incrementale.
\end{itemize}

\subparagraph{Struttura descrittiva metriche}\mbox{}\\ \\
Ogni caratteristica di ogni metrica viene elencata in una lista, mentre il nome della metrica rappresenta il titolo di questo elenco ed è visibile nell'indice del documento. La struttura è la seguente:
\begin{itemize}
    \item \textbf{Codice:} codice univoco;
    \item \textbf{Descrizione:} breve descrizione della metrica e del contesto applicativo;
    \item \textbf{Processo di riferimento:} questa parte riguarda esclusivamente le metriche di \glo{processo} e viene spiegato in quale attività di \glo{processo} viene applicata tale metrica (riferimento allo standard ISO/IEC 12207);
    \item \textbf{Attributo di riferimento:} questa parte riguarda esclusivamente le metriche di prodotto e viene spiegato in quale attributo della caratteristica di prodotto viene applicata tale metrica (riferimento allo standard ISO/IEC 9126);
    \item \textbf{Sigla:} nome della metrica in formato sigla, utilizzato principalmente nelle formule matematiche e nella rappresentazione dei range delle metriche.;
    \item \textbf{Formula:} sezione opzionale nel caso che la metrica preveda una formula matematica per essere calcolata;
    \item \textbf{Range di valori che può assumere:} sezione dove sotto ad essa vengono elencati i range accettabili ed ottimali della metrica;
    \begin{itemize}
        \item \textbf{Accettabile:} valore della metrica ritenuto accettabile per garantire la qualità;
        \item \textbf{Ottimale:} valore della metrica ritenuto ottimale per garantire la qualità.
    \end{itemize}
\end{itemize} 

\subparagraph{Tabella riassuntiva metriche}\mbox{}\\ \\
Per riassumere tutte le metriche e le loro caratteristiche sono rappresentate in delle tabelle situate alla fine del capitolo di appartenenza. Le tabelle hanno questo formato:

{
\rowcolors{2}{grigetto}{white}
\renewcommand{\arraystretch}{1.5}
\begin{longtable}{ c C{4cm} c c c}
\caption{Tabella metriche dei processi/prodotti}\\
\rowcolor{darkblue}
\textcolor{white}{\textbf{Metrica}} & \textcolor{white}{\textbf{Nome}} & \textcolor{white}{\textbf{Sigla}} & \textcolor{white}{\textbf{Valore Accettabile}} & \textcolor{white}{\textbf{Valore Ottimale}}\\
\end{longtable}
}

\subsubsection{Strumenti per il controllo di qualità}
\paragraph{Metrica - Indice di Gulpease}\mbox{}\\ \\
È l'indice di leggibilità di un determinato testo. Calcola la lunghezza delle parole e delle frasi rispetto al numero totale delle lettere. Per poter calcolare tale indice bisogna utilizzare la seguente formula:
$$IG = 89 + {{300 \; \cdot \; (numero\; delle \; frasi) \; - \; 10 \; \cdot \; (numero \; delle \; lettere)}\over numero \; delle \; parole}$$
Basandosi su questa formula il gruppo \Gruppo ha sviluppato un algoritmo capace di ottenere l'indice di leggibilità di Gulpease dandogli in pasto come input un qualsiasi pdf 
documentato. Grazie a questo strumento i tempi di verifica sono \glo{efficienti} mentre per la sua \glo{efficacia} la si ottimizzerà nella prossima consegna.