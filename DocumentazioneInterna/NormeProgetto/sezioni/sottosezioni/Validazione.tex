%scritto da Federico Perin
\subsection{Validazione}
\subsubsection{Obiettivo}
L’obiettivo della validazione è quello di confermare che i requisiti e il prodotto software realizzato siano soddisfatti cioè il cliente ha ottenuto ciò che aveva chiesto. La validazione viene fatta alla fine quando ho il prodotto completo e dopo un processo ben strutturato di verifica. Per dimostrare la validità del prodotto al committente si portano prove evidenti che il prodotto rispetta tutto ciò che è stato richiesto, tale azione avviene con il test d’accettazione (o anche detto collaudo).

\subsubsection{Test d’accettazione} 
Il test d’accettazione è un test che avviene con il committente, vengono perciò verificati se ci sono i vari use case e i requisiti richiesti dal committente. Una volta che viene superato questo test cioè si è raggiunto il soddisfacimento del committente, il prodotto software e pronto per il rilascio.