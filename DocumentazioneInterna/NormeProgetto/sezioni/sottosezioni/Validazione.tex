%scritto da Federico Perin
\subsection{Validazione}
\subsubsection{Obiettivo}
L’obiettivo della Validazione è confermare che i requisiti concordati con il committente siano soddisfatti e che il prodotto software realizzato dal gruppo sia concorde alle aspettative.
Il committente del prodotto software è soddisfatto quando ha la dimostrazione che il prodotto commissionato rispetti al minimo i requisiti obbligatori, con efficacia ed efficienza.
Per accertarsi di questo fattore, quando il gruppo ritiene di avere un prodotto conforme alle aspettative deve effettuare con il committente il collaudo del prodotto software che, come detto in precedenza, altro non è che la ripetizione dei test svolti fino a quel momento (che devono necessariamente eseguire con successo).