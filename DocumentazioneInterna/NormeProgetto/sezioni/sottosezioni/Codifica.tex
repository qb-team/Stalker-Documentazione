\documentclass[a4paper, oneside, dvipsnames, table]{article}
%openany
\usepackage{hyperref}
\usepackage{fancyhdr}
\usepackage[italian]{babel}
\usepackage[raggedright]{titlesec}
\usepackage{blindtext}
\titleformat{\paragraph}[hang]{\normalfont\normalsize\bfseries}{\theparagraph}{1em}{}
\titlespacing*{\paragraph}{0pt}{3.25ex plus 1ex minus .2ex}{0.5em}


\begin{document}

\subsection{Codifica}

Questa sezione è stata fatta durante la fase iniziale del progetto e di conseguenza verrà ampliata in seguito durante la fase di \textbf{Programmazione}.
%%SCRIVERE MEGLIO
Lo scopo di questa sezione è di descrivere le norme che i programmatori dovranno rispettare durante tutto il ciclo di vita del software. 

\subsubsection{Commenti}
I commenti devono essere i più sintetici meno invasivi possibili.
Ogni volta che verrà scritto un nuovo frammento di codice i commenti dovranno essere usati per:
\begin{itemize}
\item Identificare chi ha scritto il frammento di codice;
\item Descrivere cosa fa il frammento di codice.
\end{itemize}

\subsubsection{Nomi}
\begin{itemize}
\item I nomi devono essere univoci;
\item I nomi devono essere chiari e descrittivi;
\item I nomi non devono essere simili fra di loro, per evitare confusione;
\item I nomi formati da più parole si devono scrivere usando l'underscore come separatore o, in alternativa, separare i termini con una lettera maiuscola.
\end{itemize}

\end{document}