\subsection{Codifica}
Lo scopo di questa sezione è descrivere le norme che i Programmatori devono rispettare durante tutto il ciclo di vita del software.

\subsection{Note sulla sezione di Codifica}
Questa sezione è viene redatta durante la fase iniziale del progetto e il suo contenuto è limitato alle disponibilità di conoscenza attuali.
Verrà ampliata in seguito per l'attività di \textbf{Codifica}.

\subsubsection{Commenti}
Il codice sorgente deve essere prontamente commentato, sia per chi lo scrive, sia per chi lo deve leggere in futuro.
È quindi necessario che ogni porzione di codice significativa sia dotata di commenti esplicativi, in particolar modo se le istruzioni usate non sono comuni.
I commenti devono essere chiari ma al contempo sintetici, per non distrarre dal contesto.
Ogni volta che viene scritto un nuovo frammento di codice, questo deve avere una sezione di codice di documentazione (in molti linguaggi indicato con /** */ per un commento a più righe e /// per un commento a singola riga).
Questa sezione deve saper dire:
\begin{itemize}
    \item chi ha scritto il codice;
    \item chi lo ha modificato l'ultima volta;
    \item in che contesto viene utilizzato il codice che segue;
    \item se è un metodo/funzione, che argomenti richiede in input, che valori restituisce e se può lanciare eccezioni.
\end{itemize}

\subsubsection{Nomi dei file e delle variabili}
I nomi dei file:
\begin{itemize}
    \item devono essere univoci;
    \item devono rispettare le condizioni del linguaggio che contengono, se presenti.
\end{itemize}
I nomi delle variabili:
\begin{itemize}
    \item devono essere univoci nel file in cui vengono usati;
    \item devono essere chiari e descrittivi (è vietato usare variabili temp, tmp, x, ecc.) e possibilmente brevi;
    \item non devono essere simili fra di loro, per evitare confusione;
    \item se formati da più parole si devono scrivere usando l'underscore come separatore (\glo{Snake Case}) oppure separando i termini con una lettera maiuscola ad ogni inizio parola (\glo{Camel Case}).
    \item devono rispettare le condizioni del linguaggio in cui vengono usate, se presenti.
\end{itemize}