
	\subsubsection{Progettazione}
	\paragraph{Scopo}
	La progettazione, svolta dai progettisti, ha lo scopo di soddisfare i requisiti stabiliti nel documento Analisi dei Requisiti per trovare una soluzione accettabile per tutti gli stakeholder.
	Per fare ciò, si cerca di seguire un approccio sintetico dove si pensa prima all’architettura del prodotto e poi al codice.
	\paragraph{Architettura}	I progettisti dovranno definire l’architettura logica del prodotto utilizzando parti con specifiche chiare, coese e realizzabili con risorse sostenibili e mantenibili. Essa dovrà avere determinate caratteristiche per:
	\begin{itemize}
		\item	soddisfare tutti requisiti degli stakeholder;
		\item	riuscire a gestire la nascita di eventuali errori;
		\item	garantire che venga eseguito il suo compito nel modo corretto quando utilizzata;
		\item	cercare di ridurre i tempi di manutenzione;
		\item   avere componenti semplici, coesi, incapsulati e di basso accoppiamento tra di loro;
	\end{itemize}
	La realizzazione dell’architettura del prodotto è divisa in due parti:
	\begin{itemize}
		\item Technology Baseline;
		\item Product Baseline;
	\end{itemize}
	\subparagraph{Technology Baseline}\mbox{}\\
	La Technology baseline dimostrerà l’adeguatezza dell’architettura tramite un Proof of Concept (PoC) che rappresenterà la baseline per lo sviluppo. 
	La Technology baseline dovrà includere:
	\begin{itemize}
		\item le tecnologie;
		\item i framework;
		\item le librerire utilizzate nel Proof of Concept;
		\item diagrammi UML con le seguenti rappresentazioni:
		\begin{itemize}
			\item 	diagrammi dei casi d’uso; 
			\item 	diagrammi delle classi; 
			\item 	diagrammi dei package;
			\item 	diagrammi di sequenza; 
			\item 	diagrammi di attività;
		\end{itemize}
	\end{itemize}
	\subparagraph{Product Baseline}\mbox{}\\
	La Product baseline illustrerà la baseline architetturale del prodotto, in coerenza con la Technology Baseline.
	Essa dovrà includere un allegato che conterrà:
	\begin{itemize}
		\item	diagrammi delle classi;
		\item	diagrammi di sequenza;
		\item	contestualizzazione dei design pattern adottati;	
	\end{itemize}

\paragraph{Obiettivi}
\begin{itemize}
	\item controllare la complessità del prodotto suddividendo il sistema in parti di complessità trattabile;
	\item soddisfare i requisiti garantendo qualità;
	\item definire un’architettura logica del prodotto che dovrà avere determinate caratteristiche;
	\item avere una progettazione dettagliata con la consapevolezza di fermarsi quando la suddivisione porterà più svantaggi che benefici;
\end{itemize}
