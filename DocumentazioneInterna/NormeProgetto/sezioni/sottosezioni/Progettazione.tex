\subsubsection{Progettazione}
\paragraph{Scopo}\mbox{}\\ \\
La progettazione, svolta dai Progettisti, ha lo scopo di soddisfare i requisiti stabiliti nel documento \AdR{} per trovare una soluzione accettabile per tutti gli stakeholder.
Per fare ciò, si cerca di seguire un approccio sintetico dove si pensa prima all’architettura del prodotto e poi al codice.
\paragraph{Descrizione}\mbox{}\\ \\
La progettazione consiste nei seguenti compiti:
\begin{itemize}
	\item Controllare la complessità del prodotto suddividendo il sistema in parti di complessità trattabile;
	\item Soddisfare i requisiti garantendo qualità;
	\item Definire un’architettura logica del prodotto che dovrà avere determinate caratteristiche;
	\item Avere una progettazione dettagliata con la consapevolezza di fermarsi quando la suddivisione porterà più svantaggi che benefici.
\end{itemize}

\paragraph{Architettura}\mbox{}\\ \\
I progettisti devono definire l’architettura logica del prodotto creando parti con specifiche chiare, coese e realizzabili con risorse sostenibili e mantenibili. L'architettura deve avere determinate caratteristiche per:
\begin{itemize}
	\item Soddisfare tutti requisiti degli stakeholder;
	\item Riuscire a gestire gli errori quando presenti;
	\item Garantire che venga eseguito il suo compito nel modo corretto;
	\item Cercare di ridurre i tempi di manutenzione;
	\item Avere componenti semplici, coesi, incapsulati e di basso accoppiamento tra di loro.
\end{itemize}

La realizzazione dell’architettura del prodotto è divisa in due parti:
\begin{itemize}
	\item Technology Baseline\ap{G};
	\item Product Baseline\ap{G}.
\end{itemize}

\subparagraph{Technology Baseline}\mbox{}\\ \\
La Technology Baseline deve dimostrare l’adeguatezza dell’architettura tramite un Proof of Concept (PoC) che rappresenta la baseline per lo sviluppo. 
La Technology baseline deve includere:
\begin{itemize}
	\item Le tecnologie;
	\item I framework;
	\item Le librerie utilizzate nel Proof of Concept;
	\item Diagrammi UML con le seguenti rappresentazioni:
	\begin{itemize}
		\item Diagrammi dei casi d'uso; 
		\item Diagrammi delle classi; 
		\item Diagrammi dei package;
		\item Diagrammi di sequenza; 
		\item Diagrammi di attività.
	\end{itemize}
\end{itemize}

\subparagraph{Product Baseline}\mbox{}\\ \\
La Product Baseline illustrerà la baseline architetturale del prodotto, in coerenza con la Technology Baseline.
Essa deve includere un allegato che contenga:
\begin{itemize}
	\item Diagrammi delle classi;
	\item Diagrammi di sequenza;
	\item Contestualizzazione dei design pattern adottati.	
\end{itemize}