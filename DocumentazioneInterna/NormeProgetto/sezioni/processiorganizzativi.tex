\section{Processi organizzativi}
In questa sezione vengono definite le norme che regolano le comunicazioni tra:
\begin{enumerate}
	\item I membri del gruppo, dette anche “comunicazioni interne”.
	\item Il gruppo e soggetti esterni, dette anche “comunicazioni esterne”.
	
	\end{enumerate}
	
\subsection{Comunicazioni interne}
Per le comunicazione interne viene utilizzato un workspace\ap{G} di Slack\ap{G},
strumento di collaborazione molto utile per inviare messaggi ai membri del proprio gruppo.
Grazie a questa piattaforma virtuale sono stati creati dei canali\ap{G} per organizzare al meglio il lavoro tra i membri del gruppo.
I canali sono:
\begin{itemize}
	\item \textbf{Analisi dei requisiti}:  contiene le discussioni per la stesura del documento “Analisi dei Requisiti”;
	\item \textbf{Calendario}: contiene le discussioni per l’organizzazione\ap{G} di luoghi e orari degli incontri;
	\item \textbf{Documentazione progetto}: contiene le discussioni che trattano la struttura dello scheletro di ogni documento che dovrà essere scritto durante la fase di progetto;
	\item \textbf{General}: contiene tutte le discussioni non per forza inerenti al progetto;
	\item \textbf{Norme di progetto}: contiene le discussioni per la stesura del documento “Norme di progetto”;
        \item \textbf{Piano di progetto}: contiene le discussioni per la stesura del documento “Piano di progetto”;
	\item \textbf{Piano di qualifica}: contiene le discussioni per la stesura del documento “Piano di qualifica”;
	\item \textbf{Studio di fattibilità}: contiene le discussioni per la stesura del documento “Studio di fattibilità”;
	\item \textbf{Verbali}: contiene le discussioni per la stesura dei verbali relativi ai vari incontri tenuti dal gruppo Qbteam;
	\item \textbf{Source code management}: contiene tutte le discussioni relative a Git\ap{G}, gestione del codice sorgente, GitHub\ap{G};
\end{itemize}

\subsection{Comunicazioni esterne}

In questa sezione vengono definite le norme che regolano le comunicazioni tra il gruppo e soggetti esterni, in particolare:
\begin{enumerate}
	\item Il proponente Imola informatica, con referenti Davide Zanetti e Tommaso Cardona.
	\item Prof. Tullio Vardanega, Prof. Riccardo Cardin, ai quali verrà fornita tutta la documentazione richiesta in ciascuna revisione di avanzamento.
	Le comunicazioni esterne avvengono esclusivamente via mail attraverso l’indirizzo di posta elettronica del gruppo:
	\url{qbteamswe@gmail.com} \\
	Ogni membro del gruppo possiede le credenziali per poter accedere all’indirizzo e-mail.

	\end{enumerate}
	
\subsection{Gestione delle riunioni}
Le riunioni possono essere interne o esterne. All’inizio di ogni riunione il Responsabile di progetto nomina un segretario che si occupa di prendere nota di tutto ciò che viene discusso durante l’incontro. Quest’ultimo, oltre ad avere l’onere di far rispettare l’ordine del giorno dovrà anche redigere il verbale dell’incontro.

\subsubsection{Riunioni interne}
Alle riunioni interne, che avvengono principalmente di persona in luoghi prefissati, potranno parteciparvi solo i membri del gruppo QbTeam.
Il Responsabile deve organizzare preventivamente tutti gli argomenti da trattare presenti nell’ordine del giorno, fissare la prossima data di incontro e approvare il verbale redatto dal segretario.
Tutti i membri del gruppo sono tenuti a presentarsi in orario segnalando eventuali ritardi o assenze.
\subsubsection{Riunioni esterne}
Nelle riunioni esterne sono coinvolti i membri del gruppo QbTeam e uno o più membri del proponente Imola Informatica.
Le riunioni si possono svolgere:
\begin{itemize}
	\item Nella sede della proponente;
	\item Presso l’ateneo dell’università di Padova
	\item Tramite piattaforme virtuali di chiamata remota quali: \url{www.skype.com}, \url{hangouts.google.com} etc..

	\end{itemize}
	
Le varie comunicazioni per stabilire gli incontri tra il gruppo e il proponente avverranno via indirizzo e-mail con relativo margine di anticipo.
\subsubsection{Verbali delle riunioni}
Al termine di ogni riunione il segretario dovrà redigere il relativo verbale, rispettando il seguente
schema:
\begin{enumerate}
	\item \textbf{Informazioni generali}:
		\begin{itemize}
			\item Luogo incontro;
			\item Ora incontro;
			\item Data incontro;
			\item Nome partecipanti;
			\item Nome Segretario;
		\end{itemize}
	\item \textbf{Ordine del giorno}: argomenti trattati durante la riunione. Quest’ultimi vengono decisi dal Responsabile del progetto e possono essere consultati in qualsiasi momento da ogni membro del gruppo;
	\item \textbf{Resoconto}: riassunto dell’incontro che verrà redatto dal segretario in un documento apposito seguendo i punti dell’ordine del giorno.
\end{enumerate}
\subsubsection{Nomenclatura}
Ogni verbale eretto dopo gli incontri dovrà avere la seguente nomenclatura:\\ \\
\centerline {\textbf{VERBALE-TIPO-DATA}}
dove \\
\begin{itemize}
\item \textbf{VER}: sta per verbale;
\item \textbf{TIPO}: indica la tipologia del verbale, che può essere:
	\begin{itemize}
  		\item \textbf{I}: verbale della riunione interna;
 		 \item \textbf{E}: verbale della riunione esterna;
	\end{itemize}
\item \textbf{DATA}: sta per data del verbale;
\end{itemize}

\subsection{ Processi di pianificazione}
In questa sezione sono presentati tutti i sistemi e le metodologie utilizzate dal gruppo per organizzare un corretto svolgimento del progetto.
Ogni  membro e  tenuto  a  consultare  e utilizzare  attivamente  questi strumenti,  in  modo  da  tener traccia del lavoro svolto.
\subsubsection{Ruoli di progetto}
Ogni membro del gruppo dovrà, a rotazione, ricoprire diversi ruoli per tutta la durata del progetto. Ciò comporta ad una corretta continuità delle attività da svolgere e la piena consapevolezza dell’elaborato finale.
I ruoli sono i seguenti:
\begin{itemize}
\item \textbf{Responsabile:} ha l'incarico di pianificare, motivare, coordinare e controllare i membri del gruppo.
Il suo compito prevede inoltre l'approvazione dell'emissione di documenti e l'emanazione di piani e scadenze.
Ha l'onere di rappresentante di progetto presso l'azienda proponente.
\item \textbf{Amministratore:} ha l'incarico di controllare l'efficienza dell'ambiente di lavoro e di gestire tutti i documenti relativi al progetto. Si occupa, inoltre, della configurazione e versionamento del prodotto.
\item \textbf{Progettista:} ha l'incarico di studiare la fattibilità del prodotto e di costruirne l'architettura in termini di efficienza ed efficacia. Segue lo sviluppo e non la manutenzione del prodotto.
\item \textbf{Programmatore:}; partecipa sia alla realizzazione che alla manutenzione del prodotto. E' competente nella codifica e nella realizzazione di componenti necessarie all’esecuzione delle prove di verifica e validazione. Il codice prodotto dal programmatore deve essere mantenibile nel tempo.
\item \textbf{Analista:} segue il progetto dall'inizio fino alla fine e redige  i documenti relativi allo Studio di Fattibilità e all'Analisi dei Requisiti. Il suo lavoro si basa nel conoscere a fondo il problema e definire i requisiti espliciti ed impliciti.
\item \textbf{Verificatore:} ha l'incarico, per l'intero ciclo di vita del progetto, di svolgere le attività di verifica e validazione. Si occupa, inoltre, di redigere il documento relativo al Piano di Qualifica che conterrà gli esiti delle verifiche e delle prove effettuate.
\end{itemize}
\subsubsection{Pianificazione}
Per quanto riguarda la pianificazione del lavoro da svolgere, il gruppo qbteam ha scelto di utilizzare 
il servizio di gestione delle issues\ap{G} fornito da GitHub. Quest'ultimo permette di:
\begin{itemize}
\item \textbf{Inserire issues};
\item \textbf{Attribuire un titolo e una descrizione};
\item \textbf{Inserire delle milestone};
\item \textbf{Assegnare issues a uno o più membri del gruppo};
\item \textbf{Chiudere le issues una volta risolte};
\item \textbf{Etichettare le issues};
\item \textbf{Commentare le issues};
\end{itemize}
Le issues vengono create dal responsabile di progetto e sono accompagnate da un'etichetta che ne identifica la priorità, il tipo di task da svolgere e la data ultima per il completamento. Le issues verranno poi assegnate ad uno o più componenti del gruppo.\\
E' stato creato un canale apposito su Slack per segnalare la creazione e la chiusura di una issues in modo da consapevolizzare tutti i membri del gruppo riguardo l'andamento del progetto.
%\subsubsection{Coordinamento} come abbiamo deciso di dividerci i ruoli?
\subsubsection{Documentazione}
Per quanto riguarda la stesura dei documenti il gruppo ha scelto di utilizzare LateX, il quale consente una migliore qualità tipografica rispetto ai normali programmi per la stesura del testo.
Ogni membro del gruppo ha scelto l'editor, per il linguaggio LaTeX, che preferiva.
\subsubsection{Formazione dei membri del gruppo}
Ogni membro del gruppo ha il compito di formarsi in modo autonomo per poter padroneggiare al meglio tutte le tecnologie che verranno utilizzate nel corso del progetto.
E' necessaria, inoltre, la completa disponibilità da parte di tutti i membri del gruppo di condividere le conoscenze già possedute o realizzate durante tutta la fase del progetto.

