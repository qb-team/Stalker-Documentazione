\section{Processi organizzativi}
In questa sezione vengono definite le norme che regolano le comunicazioni tra:
\begin{enumerate}
	\item i membri del gruppo \Gruppo{}, dette anche "comunicazioni interne";
	\item il gruppo e soggetti esterni, dette anche "comunicazioni esterne".
\end{enumerate}
	
\subsection{Comunicazioni interne}
Per le comunicazione interne viene utilizzato un \glo{workspace} di \glo{Slack}, strumento di collaborazione molto utile per inviare messaggi ai membri del proprio gruppo.
Grazie a questa piattaforma virtuale sono stati creati dei \glo{canali} per organizzare al meglio la suddivisione del lavoro e la collaborazione tra i membri del gruppo.
I canali sono:
\begin{itemize}
	\item \textbf{\#analisi-dei-requisiti}: Contiene le discussioni per la stesura del documento \AdR{};
	\item \textbf{\#calendario}: Contiene le discussioni per l’organizzazione di luoghi e orari degli incontri;
	\item \textbf{\#documentazione-progetto}: Contiene le discussioni che trattano in generale della documentazione del progetto;
	\item \textbf{\#general}: Contiene discussioni inerenti al progetto a carattere generale, non di uno specifico tema;
	\item \textbf{\#norme-di-progetto}: Contiene le discussioni per la stesura del documento \NdP{};
    \item \textbf{\#piano-di-progetto}: Contiene le discussioni per la stesura del documento \PdP{};
	\item \textbf{\#piano-di-qualifica}: Contiene le discussioni per la stesura del documento \PdQ{};
	\item \textbf{\#studio-di-fattibilita}: Contiene le discussioni per la stesura del documento \SdF{};
	\item \textbf{\#verbali}: Contiene le discussioni per la stesura dei verbali relativi ai vari incontri tenuti dal gruppo \Gruppo{};
	\item \textbf{\#source-code-management}: contiene tutte le discussioni relative a \glo{Git}, \glo{GitHub}, quindi sulla gestione del codice sorgente e gestione delle attività da svolgere dell'\glo{ITS} di \glo{GitHub}.
\end{itemize}

In aggiunta a \glo{Slack}, per poter comunicare oralmente da remoto, viene utilizzato il software \glo{Discord}, un software che permette di effettuare chiamate vocali grazie ad una connessione ad Internet.
All'interno di questa piattaforma, sono stati creati dei \glo{canali} per poter discutere di diversi documenti in contemporanea senza disturbi.
I canali sono:
\begin{itemize}
	\item \textbf{Analisi dei Requisiti}: Come per il canale \#analisi-dei-requisiti di \glo{Slack};
	\item \textbf{Calendario}: Come per il canale \#calendario di \glo{Slack};
	\item \textbf{Documentazione Progetto}: Come per il canale \#documentazione-progetto di \glo{Slack};
	\item \textbf{Generale}: Come per il canale \#general di \glo{Slack};
	\item \textbf{Norme di Progetto}: Come per il canale \#norme-di-progetto di \glo{Slack};
    \item \textbf{Piano di Progetto}: Come per il canale \#piano-di-progetto di \glo{Slack};
	\item \textbf{Piano di Qualifica}: Come per il canale \#piano-di-qualifica di \glo{Slack};
	\item \textbf{Studio di Fattibilità}: Come per il canale \#studio-di-fattibilita di \glo{Slack};
	\item \textbf{Verbali}: Come per il canale \#verbali di \glo{Slack};
	\item \textbf{Source Code Management}: Come per il canale \#source-code-management di \glo{Slack};
	\item \textbf{\#solo-emergenze}: L'unico canale testuale di \glo{Discord}, da usare solo nel caso in cui \glo{Slack} non sia disponibile per l'utilizzo.
\end{itemize}

\subsection{Comunicazioni esterne}
In questa sezione vengono definite le norme che regolano le comunicazioni tra il gruppo e soggetti esterni, in particolare:
\begin{itemize}
	\item Il proponente \Proponente{}, con referenti \ZD{}{} e \CT{}.
	\item \VT{}, \CR{}, ai quali verrà fornita tutta la documentazione richiesta in ciascuna revisione di avanzamento.
	Le comunicazioni esterne avvengono esclusivamente via mail attraverso l’indirizzo di posta elettronica del gruppo:
	\url{qbteamswe@gmail.com}. \\
	Ogni membro del gruppo possiede le credenziali per poter accedere all’indirizzo e-mail.
\end{itemize}
	
\subsection{Gestione degli incontri formali}
Gli \glo{incontri formali} fra i membri del gruppo possono essere interni o esterni.
All’inizio di ogni riunione il Responsabile di Progetto nomina un segretario che si occupa di prendere nota di tutto ciò che viene discusso durante l’incontro.
Quest’ultimo, oltre ad avere l’onere di far rispettare l’ordine del giorno dovrà anche redigere il verbale dell’incontro.

\subsubsection{Incontri formali interni}
Agli incontri formali interni, che avvengono principalmente di persona in luoghi prefissati, potranno parteciparvi solo i membri del gruppo \Gruppo{}.
Il Responsabile di Progetto deve organizzare preventivamente tutti gli argomenti da trattare presenti nell’ordine del giorno e approvare il verbale redatto dal segretario.
Tutti i membri del gruppo sono tenuti a presentarsi in orario segnalando eventuali ritardi o assenze.

\subsubsection{Incontri formali esterni}
Agli incontri formali esterni, sono coinvolti i membri del gruppo \Gruppo{} e uno o più membri dell'azienda proponente \Proponente{}.
Le riunioni si possono svolgere:
\begin{itemize}
	\item Nella sede del proponente;
	\item Presso l’ateneo dell’Università di Padova;
	\item Tramite piattaforme virtuali di chiamata remota quali: \href{https://www.skype.com/it/}{Skype}, \href{https://hangouts.google.com/}{Hangouts}.
\end{itemize}

Le varie comunicazioni per stabilire gli incontri tra il gruppo e il proponente avverranno tramite posta elettronica con relativo margine di anticipo.
\subsubsection{Verbali delle riunioni}
Al termine di ogni riunione il segretario dovrà redigere il relativo verbale, rispettando il seguente
schema:
\begin{enumerate}
	\item \textbf{Informazioni generali}:
		\begin{itemize}
			\item Luogo dell'incontro;
			\item Data dell'incontro;
			\item Orario di inizio e di fine dell'incontro;
			\item Lista dei partecipanti all'incontro;
			\item Segretario dell'incontro.
		\end{itemize}
	\item \textbf{Ordine del Giorno}: Argomenti trattati durante la riunione. Quest’ultimi vengono decisi dal Responsabile di Progetto e possono essere consultati in qualsiasi momento da ogni membro del gruppo;
	\item \textbf{Resoconto}: riassunto delle discussioni svolte durante l'incontro, redatto dal segretario, seguendo i punti dell’ordine del giorno;
	\item \textbf{Riepilogo decisioni}: Tabella in cui vengono indicate le decisioni prese durante l'incontro.
	A ogni decisione corrisponde un codice univoco identificativo, che può essere utilizzato per il tracciamento.
	Tale codice ha la seguente forma:
	\begin{center}
		V[Destinazione]\_[YYYY]-[MM]-[DD].X	
	\end{center}
	con:
	\begin{itemize}
		\item \textbf{[Destinazione]}:
		\begin{itemize}
			\item \textbf{I}, se il verbale è interno;
			\item \textbf{E}, se il verbale è esterno;
		\end{itemize}
		\item \textbf{[YYYY]-[MM]-[DD]}: data in formato anno-mese-giorno, separati da un trattino;
		\item \textbf{X}: Dato X $\in \mathbb{N}$, X è un numero progressivo per indicare la decisione.
	\end{itemize}
\end{enumerate}

\subsection{Gestione di progetto}
In questa sezione vengono presentati tutti i sistemi e le metodologie utilizzate dal gruppo per una corretta organizzazione e collaborazione.
\subsubsection{Ruoli di progetto}
Ogni membro del gruppo deve, a rotazione, ricoprire almeno una volta ciascun ruolo di progetto.
I ruoli sono i seguenti:
\begin{itemize}
\item \textbf{Responsabile}: Ha l'incarico di pianificare, motivare, coordinare e controllare i membri del gruppo \Gruppo{}.
Il suo compito prevede inoltre l'approvazione dei documenti e l'emanazione di piani e scadenze.
Ha l'onere di rappresentante il gruppo presso il proponente \Proponente{};
\item \textbf{Amministratore}: Ha l'incarico di controllare l'efficienza dell'ambiente di lavoro e di gestire tutti i documenti relativi al progetto.
Si occupa, inoltre, della configurazione e del versionamento del prodotto;
\item \textbf{Progettista}: Ha l'incarico di definire l'architettura alla base del sistema del prodotto software.
Segue lo sviluppo e non la manutenzione del prodotto;
\item \textbf{Programmatore}: Partecipa sia alla realizzazione che alla manutenzione del prodotto.
È competente nella codifica e nella realizzazione di componenti necessarie all’esecuzione delle prove di verifica e validazione.
Il codice prodotto dal programmatore deve essere mantenibile nel tempo;
\item \textbf{Analista}: Segue il progetto dall'inizio fino alla fine e redige i documenti relativi allo \SdF{} e all'\AdR{}.
Il suo lavoro si basa nel conoscere a fondo il problema e definire i requisiti espliciti ed impliciti;
\item \textbf{Verificatore}: Ha l'incarico, per l'intero ciclo di vita del progetto, di svolgere le attività di verifica e validazione.
Si occupa, inoltre, di redigere il documento \PdQ{} che conterrà gli esiti delle verifiche e delle prove effettuate.
\end{itemize}

\subsubsection{Pianificazione}
Per quanto riguarda la pianificazione del lavoro da svolgere, il gruppo \Gruppo{} ha scelto di utilizzare l'\glo{Issue Tracking System (ITS)} fornito da \glo{GitHub}.
Le operazioni permesse dall'\glo{ITS} di \glo{GitHub} sono:
\begin{itemize}
	\item Creazione di \glo{bacheche} (chiamate "Projects") in cui avere un'istantanea delle attività, che possono ritrovarsi in uno e un solo dei seguenti stati: da fare, in corso e completate. Queste attività corrispondo alle "issue";
	\item Creazione di attività da svolgere, dette \glo{issue}, dotate di:
	\begin{itemize}
		\item Titolo;
		\item Descrizione dei compiti da svolgere;
		\item Numero progressivo identificativo (fondamentale per il tracciamento delle attività);
		\item I membri del gruppo assegnati per lo svolgimento;
		\item Etichette (chiamate "label") per favorire il filtraggio per argomento;
		\item Milestone di riferimento;
		\item Bacheca di riferimento (ovvero in quale \glo{bacheca} viene visualizzata l'issue).
	\end{itemize}
	\item Creazione di una milestone, con titolo, descrizione e data di scadenza;
	\item Creazione di etichette per le issue;
	\item Creazione di pull request, in cui si richiede di effettuare il merge di un \glo{branch} all'interno di un altro, verificare l'assenza di conflitti di \glo{merge} e discutere con i membri del gruppo delle modifiche apportate da unire;
	\item Modifica di issue, milestone, bacheche e label;
	\item Visualizzazione delle attività svolte dai membri del gruppo relative ad un'issue grazie al tracciamento e all'integrazione con il \glo{SCM} \glo{Git}.
\end{itemize}
Oltre a queste funzionalità, l'\glo{ITS} di \glo{GitHub} fornisce altro, ma il gruppo \Gruppo{} ritiene opportuno l'utilizzo solamente di queste elencate.
È stato creato un canale apposito su \glo{Slack} per segnalare la creazione e la chiusura delle issues in modo da consapevolizzare tutti i membri del gruppo riguardo l'andamento del progetto.

\subsubsection{Formazione dei membri del gruppo}
Ogni membro del gruppo ha il compito di formarsi in modo autonomo per poter padroneggiare al meglio tutte le tecnologie che verranno utilizzate nel corso del progetto.
È necessaria, inoltre, la completa disponibilità da parte di tutti i membri del gruppo di condividere le conoscenze già possedute o realizzate durante tutta la fase del progetto.