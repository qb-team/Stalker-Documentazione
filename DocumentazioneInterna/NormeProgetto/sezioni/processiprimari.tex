\section{Processi primari}
\subsection{Fornitura}
\subsubsection{Studio di fattibilità}

Il responsabile del progetto ha il compito di convocare tutti i membri del team per discutere le varie tematiche riguardo ai capitolati presi in considerazione.
Lo studio di fattibilità viene poi realizzato dagli analisti per ogni capitolato trattato, tenendo anche conto di quello che è stato detto nelle riunioni precedenti. Ogni appalto avrà un documento che conterrà i seguenti punti:

\begin{itemize}
\item Titolo del capitolato:
	\begin{itemize}
	\item Nome capitolato;
	\item Azienda proponente;
	\item Committenti.
	\end{itemize}
\item Descrizione del capitolato:
	\begin{itemize}
	\item Un riassunto del prodotto da realizzare secondo le specifiche richieste.
	\end{itemize}
\item Prerequisiti e tecnologie coinvolte:
	\begin{itemize}
	\item Elenco delle tecnologie da utilizzare con eventuali link o spiegazioni del contesto applicativo;
	\item In alcuni casi l'azienda proponente consiglia l'uso di certe tecnologie.
	\end{itemize}
\item Vincoli:
	\begin{itemize}
	\item Richieste: generali, tecniche, o esplicite da parte dell'azienda proponente.
	\end{itemize}
\item Aspetti positivi:
	\begin{itemize}
	\item Vengono descritti i punti positivi che il gruppo ha analizzato se si facesse quel specifico capitolato.
	\end{itemize}
\item Aspetti critici:
	\begin{itemize}
	\item Vengono descritti i punti critici che il gruppo ha analizzato se si facesse quel specifico capitolato.
	\end{itemize}
\item Conclusioni:
	\begin{itemize}
	\item È una valutazione finale complessiva motivata dai membri del gruppo dove sono state esposte le ragioni di interesse, disinteresse ed eventuale scelta collettiva del capitolato.
	\end{itemize}
\end{itemize}

Tutte queste informazioni appena elencate devono essere raccolte in un unico documento chiamato StudioDiFattibilità che verrà sottoposto a verifiche da parte dei verificatori e svariati versionamenti e modifiche fatti dagli analisti nel corso del lavoro svolto.

\subsection{Sviluppo}
\subsubsection{Analisi dei requisiti}
\textbf{Scopo} \\ \\
Il documento Analisi dei requisiti, redatto dagli analisti, ha come scopo i seguenti punti:

\begin{itemize}
\item Definire lo scopo del prodotto da realizzare;
\item Fissare le funzionalità del progetto concordate col cliente;
\item Fornire ai progettisti precisi ed affidabili riferimenti;
\item Definire una base a cui integrare raffinamenti per permettere un miglioramento continuo del prodotto e del processo di sviluppo;
\item Fornire ai verificatori dei riferimenti per l’attività di controllo;
\item Fornire una stima del quantitativo di lavoro per tracciare una stima dei costi. 
\end{itemize}

\textbf{Aspettative} \\ \\
Ci si pone come obiettivo dell’attività la creazione di un documento formale contenente tutti i requisiti richiesti e concordati col proponente. Sarà possibile fare riferimento a quanto redatto nel documento Analisi dei requisiti qualora sorgessero incomprensioni e dubbi al momento del collaudo del prodotto.

\textbf{Descrizione} \\ \\
I requisiti verranno raccolti dalle seguenti fonti:
\begin{itemize}
\item Capitolato, attraverso la sua lettura, analisi ed approfondimento;
\item Confronti tra i membri del gruppo, qualora si ritenga sensato aggiungere determinati requisiti;
\item Confronto con il proponente, concordando requisiti aggiuntivi o scartandone altri;
\item Dall’analisi di uno o più casi d’uso.
\end{itemize}