\section{Processi primari}
\subsection{Fornitura}
\subsubsection{Studio di fattibilità}

Il responsabile del progetto ha il compito di convocare tutti i membri del team per discutere le varie tematiche riguardo ai capitolati presi in considerazione.
Lo studio di fattibilità viene poi realizzato dagli analisti per ogni capitolato trattato, tenendo anche conto di quello che è stato detto nelle riunioni precedenti. Ogni appalto avrà un documento che conterrà i seguenti punti:

\begin{itemize}
\item Titolo del capitolato:
	\begin{itemize}
	\item Nome capitolato;
	\item Azienda proponente;
	\item Committenti.
	\end{itemize}
\item Descrizione del capitolato:
	\begin{itemize}
	\item Un riassunto del prodotto da realizzare secondo le specifiche richieste.
	\end{itemize}
\item Prerequisiti e tecnologie coinvolte:
	\begin{itemize}
	\item Elenco delle tecnologie da utilizzare con eventuali link o spiegazioni del contesto applicativo;
	\item In alcuni casi l'azienda proponente consiglia l'uso di certe tecnologie.
	\end{itemize}
\item Vincoli:
	\begin{itemize}
	\item Richieste: generali, tecniche, o esplicite da parte dell'azienda proponente.
	\end{itemize}
\item Aspetti positivi:
	\begin{itemize}
	\item Vengono descritti i punti positivi che il gruppo ha analizzato se si facesse quel specifico capitolato.
	\end{itemize}
\item Aspetti critici:
	\begin{itemize}
	\item Vengono descritti i punti critici che il gruppo ha analizzato se si facesse quel specifico capitolato.
	\end{itemize}
\item Conclusioni:
	\begin{itemize}
	\item È una valutazione finale complessiva motivata dai membri del gruppo dove sono state esposte le ragioni di interesse, disinteresse ed eventuale scelta collettiva del capitolato.
	\end{itemize}
\end{itemize}

Tutte queste informazioni appena elencate devono essere raccolte in un unico documento chiamato StudioDiFattibilità che verrà sottoposto a verifiche da parte dei verificatori e svariati versionamenti e modifiche fatti dagli analisti nel corso del lavoro svolto.

\subsection{Processo di sviluppo}
Il processo di sviluppo formata da attività e compiti è una parte del ciclo di vita del software essa, dove si deve produrre il software che soddisfi tutti i requisiti indicati nelle specifiche dal proponente. Perciò i vari membri del gruppo hanno come obbiettivo sviluppare il software richiesto dal proponente. Deve essere perciò fissati i seguenti punti per una corretta implementazione del software:
\begin{itemize}
	\item Fissare gli obiettivi di sviluppo;
	\item Fissare i vincoli tecnologici e di design;
	\item Realizzare un prodotto finale che soddisfa tutti i test di verifica e validazione i quali devono rispettare quanto definito dal piano di qualità;
	\item Realizzare un prodotto finale conforme ai requisiti richiesti dal proponente.
\end{itemize}		
Il processo di sviluppo come detto in precedenza è formato da attività, esso sono:
\begin{itemize}
	\item \textbf{Analisi dei requisiti}: attività in cui si stabilisce che cosa c’è da fare cioè capire appieno il problema posto dal cliente da cui alla fine se ne ricava i requisiti (obbiettivi da raggiungere) che devono essere rispettati dal software che verrà prodotto;
	\item \textbf{Progettazione}: attività in cui in base a ciò che si è ricevuto dalla attività di Analisi dei requisiti si stabilisce come vanno fatti i vari requisiti per sodisfare i bisogni del cliente;
	\item \textbf{Codifica}: attività in cui si realizza concretamente quello che nel progetto è stato precedentemente ideato.
	\item \textbf{Test}: attività in cui in base a ciò che si è implementato nella attività di codifica, si verifica se non ci sono errori e conforme alle aspettative ciò se si sta rispettando la way of working stabilito;
	\item \textbf{Collaudo}: attività in cui si esegue il test finale insieme al proponente in cui si dimostra che il prodotto fatto soddisfa i requisiti richiesti dal proponente.
\end{itemize}
\subsubsection{Analisi dei requisiti}
\paragraph{Scopo}\mbox{}\\

Il documento Analisi dei requisiti, redatto dagli analisti, ha come scopo i seguenti punti:

\begin{itemize}
\item Definire lo scopo del prodotto da realizzare;
\item Fissare le funzionalità del progetto concordate col cliente;
\item Fornire ai progettisti precisi ed affidabili riferimenti;
\item Definire una base a cui integrare raffinamenti per permettere un miglioramento continuo del prodotto e del processo di sviluppo;
\item Fornire ai verificatori dei riferimenti per l’attività di controllo;
\item Fornire una stima del quantitativo di lavoro per tracciare una stima dei costi. 
\end{itemize}

\paragraph{Aspettative}\mbox{}\\
Ci si pone come obiettivo dell’attività la creazione di un documento formale contenente tutti i requisiti richiesti e concordati col proponente. Sarà possibile fare riferimento a quanto redatto nel documento Analisi dei requisiti qualora sorgessero incomprensioni e dubbi al momento del collaudo del prodotto.

\paragraph{Descrizione}\mbox{}\\
I requisiti verranno raccolti dalle seguenti fonti:
\begin{itemize}
\item Capitolato, attraverso la sua lettura, analisi ed approfondimento;
\item Confronti tra i membri del gruppo, qualora si ritenga sensato aggiungere determinati requisiti;
\item Confronto con il proponente, concordando requisiti aggiuntivi o scartandone altri;
\item Dall’analisi di uno o più casi d’uso.
\end{itemize}

\paragraph{Casi d'uso}\mbox{}\\
Definiscono uno scenario in cui uno o più attori interagiscono con il sistema. Saranno così strutturati:
\begin{itemize}
\item Codice identificativo
\item Titolo
\item Diagramma UML (solo se apporta valore aggiunto)
\item Attori primari
\item Attori secondari (se presenti)
\item Precondizioni
\item Postcondizioni
\item Scenario principale (nei casi d’uso a grana più fine non è strettamente necessario)
\item Scenario alternativi (se presenti)
\item Inclusioni (se presenti)
\item Estensioni (se presenti)
\item Specializzazioni (se presenti)
\end{itemize}
\paragraph{Codice idefinticativo dei casi d'uso}\mbox{}\\
Il codice identificativo di ciascun caso d’uso, univoco nel suo complesso, sarà conforme alla sintassi indicata di seguito:
\begin{center}
	\textbf{UC[Destinazione] X.Y.Z}
\end{center}
Dove:
\begin{itemize}
	\item \textbf{Destinazione} può essere:
	\begin{itemize}
		\item \textbf{A}: il caso d’uso descrive un comportamento o una funzione che l’utente ha a disposizione nella parte del sistema corrispondente all’app
		\item \textbf{B}: se il caso d’uso descrive un comportamento o una funzione che l’utente ha a disposizione nella parte del sistema corrispondente al server
	\end{itemize}
	\item \textbf{X}, \textbf{Y}, \textbf{Z} sono numeri naturali
	\item \textbf{X} indica un caso d’uso kite-level\ap{G}
	\item \textbf{Y} indica un caso d’uso sea-level\ap{G}
	\item \textbf{Z} indica un caso d’uso fish-level\ap{G}
\end{itemize}
La descrizione di un caso d’uso con codice \textbf{UC X} viene strutturata come segue:
\begin{itemize}
	\item Titolo
	\item Nome
	\item Attori primari
	\item Attori secondari (opzionale)
	\item Precondizioni
	\item Postcondizioni
	\item Scenario principale
	\item Flusso di eventi (opzionale, sotto forma di elenco numerato)	
\end{itemize}
La descrizione di un caso d’uso con codice \textbf{UC X.Y} viene strutturata come segue:
\begin{itemize}
	\item Titolo
	\item Nome
	\item Attori primari
	\item Attori secondari (opzionale)
	\item Precondizioni
	\item Postcondizioni
	\item Scenario principale
	\item Flusso di eventi (sotto forma di elenco numerato)	
	\item Estensioni (riportare solo il codice identificativo del caso d’uso)
	\item Inclusioni (riportare solo il codice identificativo del caso d’uso)	
\end{itemize}
La descrizione di un caso d’uso con codice \textbf{UC X.Y.Z} viene strutturata come segue:
\begin{itemize}
	\item Titolo
	\item Nome
	\item Attori primari
	\item Attori secondari (opzionale)
	\item Precondizioni
	\item Postcondizioni
	\item Flusso di eventi (opzionale)	
\end{itemize}
\paragraph{Requisiti}\mbox{}\\
Rappresentano dei requisiti che deve soddisfare il prodotto che si vuole realizzare.\\
I requisiti saranno organizzati in forma tabellare.\\
La tabella avrà le seguenti tre colonne:
\begin{itemize}
	\item Codice idefinticativo
	\item Classificazione
	\item Descrizione
	\item Fonti
\end{itemize}
\paragraph{Codice idefinticativo dei requisiti}\mbox{}\\
Ogni requisito sarà strutturato come segue:
\begin{center}
	\textbf{R[Importanza][Tipologia][Codice]}
\end{center}
Dove:
\begin{itemize}
		\item Importanza:
		\begin{itemize}
			\item \textbf{1}: requisito obbligatorio, ovvero irrinunciabile per almeno uno degli stakeholder
			\item \textbf{2}: requisito desiderabile, ovvero non strettamente necessario ma che porta valore aggiunto riconoscibile
			\item \textbf{3}: requisito opzionale, ovvero relativamente utile oppure contrattabile più avanti nel progetto
		\end{itemize}
		\item Tipologia:
		\begin{itemize}
			\item \textbf{F}: funzionale, definisce una funzione di un sistema di uno o più dei suoi componenti
			\item \textbf{Q}: qualitativo, definisce un requisito per garantire la qualità per un certo aspetto del progetto
			\item \textbf{V}: vincolo, definisce un requisito che è volto a far rispettare un dato vincolo
		\end{itemize}
	\item \textbf{Classificazione}: informazione ridondante, sotto forma testuale, dell’importanza del requisito. Facilita la lettura dei requisiti.
	\item \textbf{Descrizione}: descrizione sintetica ma al contempo esaustiva del requisito.
	\item \textbf{Fonti}: definisce da dove deriva il requisito. I requisiti vengono raccolti da una o più fonti tra quelle citate di seguito:
	\begin{itemize}
		\item \textbf{Capitolato}: requisito individuato dalla lettura e/o analisi del capitolato
		\item \textbf{Interno}: requisito che gli analisti hanno ritenuto opportuno aggiungere
		\item \textbf{Caso d’uso}: il requisito è derivato da uno o più casi d’uso. Riportare anche il/i codice/i identificativo/i del/i caso/i d’uso.
		\item \textbf{Verbale}: requisito derivato in seguito ad una richiesta di chiarimento con il committente. Riportare il nome del documento del verbale da cui deriva il requisito in questione.		
	\end{itemize}
\end{itemize}
\paragraph{UML}\mbox{}\\
I diagrammi UML devono essere realizzati usando la versione v2.0 del linguaggio UML.