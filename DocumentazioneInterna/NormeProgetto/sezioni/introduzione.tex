\section{Introduzione}
\subsection{Scopo del documento}
Questo documento ha lo scopo di essere utilizzato come linea guida per svolgere le attività nell'intero ciclo di vita del progetto.
Al suo interno vengono quindi dichiarate le norme, le tecnologie e gli strumenti che il gruppo \Gruppo{} intende utilizzare.
Ogni membro del gruppo è obbligato a tenere in considerazione questo documento al fine di garantire la massima coerenza del materiale prodotto.
	
\subsection{Scopo generale del prodotto}
L'obbiettivo del prodotto \NomeProgetto{} di \Proponente{} è la creazione di un sistema software composto di un applicativo per cellulare e di un server, con cui interagire tramite un'interfaccia utente. La necessità nasce dal bisogno di adempiere alle normative vigenti in tema di sicurezza.
Le due componenti del sistema software, applicativo e server, devono soddisfare i seguenti obiettivi rispettivamente di:
Tracciare e registrare i \glo{movimenti} di un utente in un \glo{luogo di tracciamento} di un'\glo{organizzazione}, siano essi autenticati da credenziali di un'\glo{organizzazione} oppure visitatori anonimi, il tutto nel rispetto della normativa sulla privacy;
Poter visionare gli accessi degli utenti autenticati e visionare il numero di visitatori anonimi all'interno di un luogo.

\subsection{Glossario}
Al fine di evitare ambiguità fra i termini, e per avere chiare fra tutti gli stakeholder le terminologie utilizzate per la realizzazione del presente documento, il gruppo \Gruppo{} ha redatto un documento denominato “\Glossariov{1.0.0}”.
In tale documento, sono presenti tutti i termini tecnici, ambigui, specifici del progetto e scelti dai membri del gruppo con le loro relative definizioni.
Un termine presente nel \Glossariov{1.0.0} e utilizzato in questo documento viene indicato con un'apice \ap{G} alla fine della parola.


