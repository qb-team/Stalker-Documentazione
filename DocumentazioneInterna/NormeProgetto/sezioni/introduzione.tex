\section{Introduzione}
\subsection{Scopo del documento}
Questo documento ha lo scopo di essere utilizzato come linea guida per svolgere le attività nell'intero ciclo di vita del progetto.
Al suo interno vengono quindi dichiarate le norme, le tecnologie e gli strumenti che il gruppo \Gruppo{} intende utilizzare.
Ogni membro del gruppo è obbligato a tenere in considerazione questo documento al fine di garantire la massima coerenza del materiale prodotto.
	
\subsection{Scopo del prodotto}
Si tratta di realizzare un sistema composto da un applicativo per cellulari e un server che adempiano lo scopo di tracciare la posizione in tempo reale, in maniera autenticata o anonima, dei possessori della sopracitata applicazione. Tale tracciamento può avvenire all’interno di strutture private (come Imola Informatica) oppure pubbliche/aperte al pubblico (come la fiera di Verona) in base al contesto di utilizzo.
	 
\subsection{Glossario}
Composto in un file separato, contiene tutti i termini tecnici, acronimi, protocolli e tecnologie menzionate in questo documento, le parole inserite saranno indicate con una G a pedice.
	
\subsection{Riferimenti} 
\subsubsection{Normativi}
\begin{itemize}
	\item \href{https://www.math.unipd.it/~tullio/IS-1/2019/Progetto/C5.pdf}{Capitolato d'appalto C5 - Stalker}
\end{itemize}

\subsubsection{Informativi}
\begin{itemize}
	\item \PdPv{1.0.0}
	\item \PdQv{1.0.0}
	\item \href{https://www.math.unipd.it/~tullio/IS-1/2019/Dispense/L05.pdf}{Slide L05 del corso Ingegneria del Software - Ciclo di vita del software}
	\item \href{https://www.latex-project.org/help/documentation/}{\LaTeX}
	\item \href{https://www.texstudio.org/}{TeXStudio}
\end{itemize}

