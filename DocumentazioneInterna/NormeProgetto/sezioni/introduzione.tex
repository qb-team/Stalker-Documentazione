\section{Introduzione}
\subsection{Scopo del documento}
	Questo documento viene utilizzato come linee guida per le tutte le attività presenti nel ciclo di vita del progetto.
	Al suo interno vengono quindi dichiarate tutte le norme, tecnologie e strumenti che ciascun membro di qbteam dovrà visionare e utilizzare.
	In questo modo verrà garantita univocità in tutti i documenti del progetto facilitando la cooperazione e comunicazione tra i vari componenti del gruppo.
	E' organizzato per processi (primari, di supporto e organizzativi).
	
\subsection{Scopo del prodotto}
	Si tratta di realizzare un sistema composto da un applicativo per cellulari e un server che adempiano lo scopo di tracciare la posizione in tempo reale, in maniera autenticata o anonima, dei possessori della sopracitata applicazione all’interno di strutture private (come Imola Informatica),  pubbliche o aperte al pubblico (come la fiera di Verona).
	 
\subsection{Glossario}
	Composto in un file separato, contiene tutti i termini tecnici, acronimi, protocolli e tecnologie menzionate in questo documento, le parole inserite saranno indicate con una G a pedice.
	
\subsection{Riferimenti} 
\subsubsection{Normativi}
\begin{itemize}
	\item Capitolato d'appalto C5 - Stalker 
	\\ \url{https://www.math.unipd.it/~tullio/IS-1/2019/Progetto/C5.pdf}
\end{itemize}

\subsubsection{Informativi}
\begin{itemize}
	\item Piano di Progetto: Piano di Progetto v1.0.0
	\item Piano di Qualifica: Piano di Qualifica v1.0.0 
	\item Slide L05 del corso Ingegneria del Software - Ciclo di vita del software
	\\ \url{https://www.math.unipd.it/~tullio/IS-1/2019/Dispense/L05.pdf}
	\item Latex
	\\ \url{https://www.latex-project.org/help/documentation/}
	\item TextStudio
	\\ \url{https://www.texstudio.org/}
	
\end{itemize}

