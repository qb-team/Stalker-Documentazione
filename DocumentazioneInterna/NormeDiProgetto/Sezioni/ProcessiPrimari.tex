\section{Processi primari}
\subsection{Processo di fornitura}
\subsubsection{Scopo}
Lo scopo della seguente sezione è formalizzare le norme e le procedure che il gruppo \Gruppo{} deve applicare al fine di diventare fornitore del proponente \Proponente{} e dei committenti \VT{} e \CR{}.

\subsubsection{Prospettiva}
Il gruppo si impegnerà a mantenere un rapporto costante con il proponente per poter riuscire ad applicare le seguenti procedure:
\begin{itemize}
	\item Comprendere e redigere i requisiti richiesti dal proponente;
	\item Chiarire qualsiasi domanda del proponente;
	\item Fornire eventuali spiegazioni del costo per la realizzazione del progetto \NomeProgetto{};
	\item Approssimare le tempistiche di lavoro per tenere aggiornato continuamente il proponente sullo stato del prodotto; 
	\item Far capire al proponente ciò che è possibile implementare.
\end{itemize}

\subsubsection{Descrizione} 
Il processo primario di fornitura ha l'obiettivo di capire e comprendere tutte le richieste del proponente, redigendo uno studio di fattibilità. 
Il gruppo \Gruppo{} ha il compito di valutare tali richieste e tenere in considerazione le risorse necessarie ai fini del progetto, esponendo al proponente \Proponente{} le attività che intende svolgere per realizzare il prodotto richiesto.
Tutte queste informazioni vengono inserite nel \PdP{}, il quale deve accompagnare il gruppo per tutta la durata del progetto.
Il gruppo \Gruppo{} deve cercare di essere il più possibile in contatto con il proponente per poter comprendere i suoi bisogni.\\

\subsubsection{Attività} 
Il processo di Fornitura comprende le seguenti attività:
\begin{itemize}
	\item Avvio;
	\item Preparazione della proposta;
	\item Contrattazione;
	\item Pianificazione;
	\item Esecuzione e controllo;
	\item Revisione e valutazione;
	\item Consegna e completamento.
\end{itemize}
In particolare nel nostro caso associamo alle attività i seguenti compiti:
{
\rowcolors{2}{grigetto}{white}
\renewcommand{\arraystretch}{2}
\begin{longtable}{ C{4cm} C{12cm}}
\rowcolor{darkblue}
\textcolor{white}{\textbf{Attività}} & \textcolor{white}{\textbf{Compiti}}
\endfirsthead
\rowcolor{darkblue}
\textcolor{white}{\textbf{Attività}} & \textcolor{white}{\textbf{Compiti}}
\endhead

	Avvio & \begin{itemize} \item Valutazione dei capitolati proposti; \item Scelta di un capitolato. \end{itemize}\\
	Preparazione della proposta & \begin{itemize} \item Formulazione della nostra proposta da consegnare ai committenti. \end{itemize}\\
	Contrattazione & Il gruppo entrerà in contatto col il proponente per: \begin{itemize} \item Chiarire dubbie emersi durante la stesura della proposta; \item Approfondire gli aspetti chiave per soddisfare i bisogni del proponente; \item Proporre le nostre soluzioni alle loro richieste per assicurarsi siano in sintonia. \end{itemize}\\
	Pianificazione & \begin{itemize} \item Scelta del modello di sviluppo e conseguente specifica delle attività e compiti, specificati nel \PdP{}; \item Stilare rigide regole per garantire la qualità durante tutto lo sviluppo del prodotto; \item Identificazione e valutazione dei rischi che si possono presentare in determinate attività dello sviluppo del progetto. \end{itemize}\\
	Esecuzione e controllo & \begin{itemize} \item Attuazione della pianificazione; \item Monitoraggio dei costi, tempistiche e riscontro dei rischi; \item Sulla base del capitolato d'appalto lo sviluppo deve essere monitorato per garantire che il prodotto rispetti tutte le richieste del proponente e dei committenti.  \end{itemize}\\
	Revisione e valutazione & \begin{itemize} \item Chiedere attivamente la valutazione del lavoro svolto al proponente; \item Rispettare i dettagli sul testing del prodotto come specificato nel capitolato d'appalto. \end{itemize}\\	
	Consegna e completamento & \begin{itemize} \item Consegna del prodotto ai committenti nelle modalità da loro specificate e conforme alle richieste del proponente. \end{itemize}\\
\end{longtable}
}
\paragraph{Studio di fattibilità} \mbox{}
Il \Responsabile{} ha il compito di convocare tutti i membri del gruppo \Gruppo{} per discutere le varie tematiche riguardanti i capitolati d'appalto disponibili.
Lo \SdF{} viene redatto dagli analisti, i quali devono analizzare il materiale disponibile e inoltre tenere in considerazione anche ciò che è stato discusso nelle riunioni sul tema.\\
Il documento è strutturato in più sezioni, ognuna riguardante un capitolato d'appalto.
Per ogni capitolato verranno trattati i seguenti punti:
\begin{itemize}
\item Titolo del capitolato:
	\begin{itemize}
	\item Nome del capitolato;
	\item Azienda proponente;
	\item Committenti.
	\end{itemize}
\item Descrizione del capitolato:
	\begin{itemize}
	\item Breve riassunto del prodotto da realizzare, secondo le specifiche richieste dal proponente.
	\end{itemize}
\item Prerequisiti e tecnologie coinvolte:
	\begin{itemize}
	\item Elenco delle tecnologie da utilizzare, con eventuali riferimenti per ulteriori approfondimenti o spiegazioni del contesto applicativo;
	\item In alcuni casi l'azienda proponente consiglia l'utilizzo di certe tecnologie.
	\end{itemize}
\item Vincoli:
	\begin{itemize}
	\item Richieste generali, tecniche e/o organizzative da parte dell'azienda proponente.
	\end{itemize}
\item Aspetti positivi:
	\begin{itemize}
	\item Vengono descritti gli aspetti ritenuti positivi dai membri del gruppo \Gruppo{} del capitolato.
	Possono essere considerati aspetti positivi, ad esempio, l'apprendimento di nuove tecnologie e/o linguaggi, disponibilità di contatto e collaborazione offerta dal proponente e documentazione disponibile online riguardante le tecnologie coinvolte.
	\end{itemize}
\item Aspetti critici:
	\begin{itemize}
	\item Vengono descritti gli aspetti ritenuti critici dai membri del gruppo \Gruppo{} del capitolato.
	Possono essere considerati aspetti critici, ad esempio, l'eccessiva mole di tecnologie da apprendere, i requisiti di vincolo da tenere in considerazione e la scarsa presenza di documentazione online riguardante le tecnologie coinvolte.
	\end{itemize}
\item Conclusioni:
	\begin{itemize}
	\item Valutazione finale motivata dai membri del gruppo \Gruppo{} nella quale vengono esposte le ragioni di interesse o disinteresse nella scelta o meno del capitolato.
	\end{itemize}
\end{itemize}
Tutte queste informazioni appena elencate vengono raccolte nel documento interno \glo{\SdF{}}, sottoposto al processo di \glo{verifica{}} da parte dei verificatori.
\documentclass[a4paper, oneside, dvipsnames, table]{article}
\usepackage{../../Utilita/Stiletemplate}
\usepackage{hyperref}
\usepackage{fancyhdr}
\usepackage[italian]{babel}
\usepackage{pdflscape}
\usepackage[raggedright]{titlesec}
\usepackage{blindtext}
\titleformat{\paragraph}[hang]{\normalfont\normalsize\bfseries}{\theparagraph}{1em}{}
\titlespacing*{\paragraph}{0pt}{3.25ex plus 1ex minus .2ex}{0.5em}

\newcommand{\Data}{2020-05-25}

\newcommand{\Titolo}{Verbale Riunione \Data}

\newcommand{\Redattori}{\PF{}}

\newcommand{\Verificatori}{\AT{}}

\newcommand{\Approvatore}{\CE{}}

\newcommand{\Distribuzione}{\VT{} \newline \CR{} \newline Gruppo \Gruppo{}}

\newcommand{\Uso}{Interno}

\newcommand{\DescrizioneDoc}{Questo documento si occupa di riportare quanto discusso nella riunione del \Data}

\newcommand{\pathimg}{../../../Utilita/Immagini/qbteam.png}

\newcommand{\Versionedoc}{1.0.0}
% scritto da \DF{},\AT{}

% info generali 
\newcommand{\NomeProgetto}{\textit{Stalker}}

% fornitore
\newcommand{\Gruppo}{\textit{qbteam}}
\newcommand{\Mail}{qbteamswe@gmail.com}
% \newcommand{\pathimg}{Immagini/qbteam.png}

% committenti
\newcommand{\Committente}{\VT \newline \CR}
\newcommand{\VT}{Prof. Vardanega Tullio}
\newcommand{\CR}{Prof. Cardin Riccardo}

% proponenti
\newcommand{\Proponente}{\textit{Imola Informatica}}
\newcommand{\ZD}{Zanetti Davide}
\newcommand{\CT}{Cardona Tommaso}

% qbteam
\newcommand{\AT}{Azzalin Tommaso}
\newcommand{\DF}{Drago Francesco}
\newcommand{\BR}{Baratin Riccardo}
\newcommand{\MC}{Mattei Christian}
\newcommand{\PF}{Perin Federico}
\newcommand{\CE}{Cisotto Emanuele}
\newcommand{\SE}{Salmaso Enrico}
\newcommand{\LD}{Lazzaro Davide}

% ruoli
\newcommand{\Responsabile}{Responsabile di Progetto}
\newcommand{\Amministratore}{Amministratore di Progetto}

% documenti

\newcommand{\SdF}{Studio di Fattibilità}
\newcommand{\SdFv}[1]{\textit{Studio di Fattibilità {#1}}}
\newcommand{\PdQ}{Piano di Qualifica}
\newcommand{\PdQv}[1]{\textit{Piano di Qualifica {#1}}}
\newcommand{\PdP}{Piano di Progetto}
\newcommand{\PdPv}[1]{\textit{Piano di Progetto {#1}}}
\newcommand{\NdP}{Norme di Progetto}
\newcommand{\NdPv}[1]{\textit{Norme di Progetto {#1}}}
\newcommand{\AdR}{Analisi dei Requisiti}
\newcommand{\AdRv}[1]{\textit{Analisi dei Requisiti {#1}}}
\newcommand{\Glossario}{Glossario}
\newcommand{\Glossariov}[1]{\textit{Glossario {#1}}}
\newcommand{\MM}{Manuale Manutentore}
\newcommand{\MMv}[1]{\textit{Manuale Manutentore {#1}}}
\newcommand{\MU}{Manuale Utente}
\newcommand{\MUv}[1]{\textit{Manuale Utente {#1}}}

% comandi generali
\newcommand{\glo}[1]{#1\ap{G}}

\setlength{\parindent}{-0.1em}


\begin{document}

\copertina{}
\newpage


\fancyPdP{}

\section*{Registro delle modifiche}
{
\rowcolors{2}{grigetto}{white}
\renewcommand{\arraystretch}{1.5}
\centering
\begin{longtable}{C{2cm} C{2cm}  C{3cm}  C{3cm} C{4.5cm}}
\rowcolor{rossoep}
\textcolor{white}{\textbf{Versione}} & \textcolor{white}{\textbf{Data}} & \textcolor{white}{\textbf{Nominativo}} & \textcolor{white}{\textbf{Ruolo}} & \textcolor{white}{\textbf{Descrizione}}\\	
\endhead
1.2.0 & 2020-03-07 & \CE{} & Verificatore & Verifica del documento. \\

1.1.4 & 2020-02-16 & \SE{} & Amministratore & Aggiornati 4.3.2 e 4.3.3 VERIFICATO DA \LD{}\\

1.1.3 & 2020-02-15 & \SE{} & Amministratore & Aggiornato 4.2.2 VERIFICATO DA \BR{}\\

1.1.5 & 2020-02-16 & \SE{} & Amministratore & Aggiunto 2.2.5.5 VERIFICATO DA \BR{}. \\

1.1.4 & 2020-02-15 & \SE{} & Amministratore & Aggiornato 4.2.2 VERIFICATO DA \BR{}. \\

1.1.3 & 2020-02-14 & \SE{} & Amministratore & Aggiunta 2.2.6 VERIFICATO DA \LD{}. \\

1.1.2 & 2020-02-12 & \SE{} & Amministratore & Aggiornamento 4.2.4 VERIFICATO DA \LD{}. \\ 

1.1.1 & 2020-02-12 & \BR{} & Amministratore & Aggiornamento 3.1 VERIFICATO DA \LD{}. \\ 

1.1.0 & 2020-02-12 & \LD{} & Verificatore & Verifica VERIFICATO DA \LD{}.  \\ 

1.0.3 & 2020-02-12 & \BR{} & Amministratore & Aggiunto e verificato 3.4.2 VERIFICATO DA \LD{}. \\ 

1.0.2 & 2020-02-12 & \SE{} & Amministratore & Aggiunte e verificate metriche SFIN e SFOUT VERIFICATO DA \LD{}. \\ 

1.0.1 & 2020-02-12 & \SE{} & Amministratore & Modificati e verificati i paragrafi 2.2.4.1 e 2.2.4.2 VERIFICATO DA \LD{}. \\ 

1.0.0 & 2020-01-13 & \AT{} & Amministratore & Approvazione per il rilascio.  \\

0.2.0 & 2020-01-13 & \PF{}, \CE{} & Verificatori & Verifica documento.  \\ 

0.1.9 & 2020-01-13 & \CE{} & Amministratore & Aggiunta del template dei digrammi UML dei casi d'uso. \\

0.1.8 & 2020-01-13 & \BR{} & Amministratore & Modifica dei casi d'uso d'errore, test di sistema. \\

0.1.7 & 2020-01-13 & \AT{} & Amministratore & Revisione Introduzione, Processo di fornitura, sviluppo, attività di codifica e di progettazione, processi organizzativi, documentazione. \\

0.1.6 & 2020-01-12 & \MC{} & Amministratore & Revisione e modifica strutturale dei capitoli del documento. \\

0.1.5 & 2020-01-12 & \AT{} & Amministratore & Modifica Processi Primari. \\

0.1.4 & 2020-01-11 & \MC{} & Amministratore & Stesura capitolo gestione della qualità. \\

0.1.3 & 2020-01-10 & \MC{} & Amministratore & Revisione documentazione nei Processi di supporto. \\

0.1.2 & 2020-01-06 & \AT{} & Amministratore & Modifica del processo di verifica, validazione, piano di qualifica. \\

0.1.1 & 2020-01-06 & \AT{} & Amministratore & Modifica del processo di verifica. \\

0.1.0 & 2019-12-23 & \PF{}, \CE{} & Verificatori & Verifica del documento. \\

0.0.11 & 2019-12-22 & \PF{} & Amministratore & Stesura delle sottosezioni introduzione e scopo della sezione processi di sviluppo. \\

0.0.10 & 2019-12-22 & \PF{}  & Amministratore & Stesura Sviluppo dei processi primari. \\

0.0.9 & 2019-12-21 & \PF{} & Amministratore & Stesura delle sottosezioni Gestione della qualità, Verifica e Validazione della sezione processi di supporto. \\

0.0.8 & 2019-12-20 & \MC{} & Amministratore & Modifica descrizione repository, Studio di fattibilità. \\

0.0.7 & 2019-12-19 & \SE{} & Amministratore & Revisione del documento fino ad ora redatto. \\

0.0.6 & 2019-12-19 & \CE{} & Amministratore & Modificata la sezione dei casi d’uso con le decisioni prese per la loro nomenclatura il 2019-12-18. \\

0.0.5 & 2019-12-15 & \SE{} & Amministratore & Aggiunta parte di progettazione. \\

0.0.4 & 2019-12-15 & \BR{}, \PF{}  & Amministratori & Aggiunta parte processi organizzativi, gestione delle risorse prodotte. \\

0.0.3 & 2019-12-15 & \MC{} & Amministratore & Stesura studio di fattibilità. \\

0.0.2 & 2019-12-14 & \CE{} & Amministratore & Aggiunta parte relativa all’Analisi dei requisiti. \\

0.0.1 & 2019-12-14 & \CE{} & Amministratore & Creato il documento. \\
		
\end{longtable}
}


\clearpage
\tableofcontents
\clearpage

\renewcommand{\figurename}{Diagramma}

\renewcommand{\listfigurename}{Elenco diagrammi}

\renewcommand{\listtablename}{Elenco tabelle}

\listoffigures
\clearpage
\listoftables
\clearpage

\section{Introduzione}
\subsection{Scopo del documento}
Lo scopo del documento è quello di descrivere in maniera dettagliata i requisiti e i casi d'uso che sono stati individuati durante lo studio del progetto Stalker.

\subsection{Scopo generale del prodotto}
L'obiettivo del prodotto \NomeProgetto{} di \Proponente{} è la creazione di un sistema software composto di un applicativo per cellulare e di un server, con cui interagire tramite un'interfaccia utente. La necessità nasce dal bisogno di adempiere alle normative vigenti in tema di sicurezza.
Le due componenti del sistema software, applicativo e server, devono soddisfare i seguenti obiettivi rispettivamente di:
\begin{itemize}
\item Tracciare e registrare i \glo{movimenti} di un utente in un \glo{luogo di tracciamento} di un'\glo{organizzazione}, siano essi autenticati da credenziali di un'\glo{organizzazione} oppure visitatori anonimi, il tutto nel rispetto della normativa sulla privacy;
\item Poter visionare gli accessi degli utenti autenticati e visionare il numero di visitatori anonimi all'interno di un luogo.
\end{itemize}

\subsection{Glossario}
Al fine di evitare ambiguità fra i termini, e per avere chiare fra tutti gli stakeholder le terminologie utilizzate per la realizzazione del presente documento, il gruppo \Gruppo{} ha redatto un documento denominato \Glossariov{1.0.0}.
In tale documento, sono presenti tutti i termini tecnici, ambigui, specifici del progetto e scelti dai membri del gruppo con le loro relative definizioni.
Un termine presente nel \Glossariov{1.0.0} e utilizzato in questo documento viene indicato con un apice \ap{G} alla fine della parola.

\subsection{Riferimenti}

\subsubsection{Normativi}
\begin{itemize}
\item \NdPv{1.0.0};
\item \textit{VE\_2019\_12\_13}.
\end{itemize}

\subsubsection{Informativi}
\begin{itemize}
\item \SdFv{1.0.0};
\item \textbf{Slide del capitolato C5 - Stalker}: \\ \url{https://www.math.unipd.it/~tullio/IS-1/2019/Progetto/C5.pdf}
\item \textbf{Guide to the Software Engineering Body of Knowledge};
\item \textbf{Software Engineering (10th edition) - Ian Sommerville}.
\end{itemize}
\clearpage
\section{Analisi dei Rischi}
La gestione dei rischi è un processo al quale il gruppo \Gruppo{} dà molto importanza. Questo perché incorrere in rischi potrebbe equivalere al danneggiamento del progetto, sia nella sua \glo{organizzazione} e sia nella sua qualità.
Si cerca quindi di fare una previsione dei problemi che si potrebbero verificare durante l'intero corso del progetto e, per ogni rischio identificato, si cerca una soluzione per poterlo evitare.

\subsection{Fasi della gestione dei rischi}
Il gruppo intende seguire i seguenti step nel processo di gestione dei rischi:
\begin{itemize}
	\item \textbf{Identificazione del rischio}: Questo è il primo step del processo e ci serve per identificare i rischi che potrebbero portare a dei problemi durante l'avanzamento del progetto; 
	\item \textbf{Analisi dei rischi}: Dopo aver individuato i rischi nello step precedente, per ognuno di essi viene valutata la probabilità che si verifichi e le conseguenze negative che potrebbe portare;
	\item \textbf{Pianificazione del rischio}: Nella pianificazione del rischio si sviluppano dei piani per sapere quali rimedi vanno intrapresi nel momento in cui i rischi si verificano. In tale maniera si riuscirà a risolvere i problemi prima che essi si aggravino;
	\item \textbf{Monitoraggio del rischio}: Nell'ultimo step della gestione del rischio viene verificato che le ipotesi relative ai rischi non abbiano subito delle variazioni. Quindi si cerca di valutare periodicamente la probabilità che il rischio si verifichi e i suoi possibili effetti, migliorando le strategie adottate per la loro risoluzione.
\end{itemize}

\subsection{Tipologia del rischio}
Ci sono 5 tipi di rischi che il gruppo \Gruppo{} terrà in considerazione. 
\\Ad ogni rischio verrà assegnato un codice identificativo:
\begin{itemize}
	\item Rischi Tecnologici [RT];
	\item Rischi Organizzativi [RO];
	\item Rischi Personali [RP];
	\item Rischi dei Requisiti [RR];
	\item Rischi di Stima [RS].
\end{itemize}

\subsection{Tabella dei rischi}
Nella seguente tabella vengono elencati i rischi che il gruppo \Gruppo{} potrebbe incontrare durante l'intero ciclo di vita del progetto.
Ogni riga della tabella corrisponde ad un rischio ed è composta da:
\begin{itemize}
	\item Codice [codice del tipo + numero sequenziale] e Nome del rischio;
	\item Descrizione;
	\item Rilevamento;
	\item Piano di Contingenza.
\end{itemize}

\input{Sezioni/TabellaRischi.tex}
{
\rowcolors{2}{grigetto}{white}
\renewcommand{\arraystretch}{2}
\centering
\begin{longtable}{ C{2cm} C{4.5cm} C{4.5cm} C{4.5cm}}
\rowcolor{rossoep}
\textcolor{white}{\textbf{Codice Nome}} & \textcolor{white}{\textbf{Descrizione}} & \textcolor{white}{\textbf{Rilevamento}} &  \textcolor{white}{\textbf{Piano di Contingenza}}\\	

RT1 Inesperienza con le tecnologie & Il gruppo dovrà relazionarsi con tecnologie mai utilizzate precedentemente e quindi servirà del tempo per poterle utilizzare nel modo corretto & Ogni componente del gruppo sarà consapevole di saper usare o no una determinata tecnologia & Ogni componente del gruppo che ha acquisito una certa dimestichezza nell'utilizzo di una tecnologia cercherà di aiutare i componenti del gruppo che hanno più difficoltà con essa \\

RP1 Attriti Interni & Durante i verbali o incontri interni, qualche componente del gruppo potrebbe essere indisponibile & Ciascun componente del gruppo comunicherà la sua assenza nel giorno dei verbali o incontri & Si aggiornerà continuamente un calendario condiviso permettendo al responsabile di fissare gli incontri in giorni e in orari in cui tutti i componenti di qbteam (o la maggior parte di essi) siano disponibili \\ 

RP2 Comunicazione Esterna & Si potrebbero avere delle difficoltà nel comunicare con il proponente esterno & Il proponente non risponderà alle mail del responsabile di qbteam in tempi brevi & Si cercherà di far presente al proponente Davide Zanetti che la comunicazione tra fornitore e cliente è molto importante per ridurre i tempi e quindi i costi \\

RR1 Disattenzione nella definizione dei requisiti & I componenti del gruppo potrebbero interpretare male qualche requisito & I verificatori si accorgono che un requisito non è stato definito nel modo corretto & Si cercherà di condurre una precisa analisi dei requisiti chiarendo ogni dubbio di ciascuno dei componenti del gruppo \\

RS1 Stime errate delle attività & Si potrebbero fare delle stime sbagliate sui costi, tempi e risorse utilizzate delle attività & Ciascun componente comunicherà al responsabile se non avrà rispettato una delle stime di qualche attività & Si cercherà di condurre una pianificazione e un preventivo attento per essere più coerenti possibili \\

% Ecco basta che riempi i successivi Rischi, ovviamente rinominando il nome
% Te ne ho messi 5 intanto, poi se son di più o di meno non importa

RO1 Non rispetto delle milestone imposte & Potrebbe accadere che per impegni personali o mancanza delle conoscenze qualche membro del gruppo impieghi più tempo del previsto non riuscendo a portare a termine il compito all'interno della scadenza assegnatagli portando il team a sforare una milestone concordata precedentemente & Il componente che si trova in difficoltà avrà il compito di comunicarlo al gruppo, inoltre anche gli altri membri possono valutare se un componente procede a rilento e comunicare il problema al responsabile & Tutto il gruppo dovrà agire per offrire supporto in modo tale da terminare il lavoro entro la scadenza\\

RO2 Eccesso o difetto nell'assegnazione delle scadenze & Data la presenza di numerosi scenari nei quali abbiamo poca esperienza potrebbe accadere una errata assegnazione di scadenze che si rivelano sovra-stimate o sotto-stimate in relazione alla difficoltà del problema da risolvere & I componenti a cui è assegnato lo svolgimento di un compito devono riferire se la scadenza a loro imposta sia ragionevole dopo aver approfondito e compreso a fondo la difficoltà del lavoro che devono portare a termine & Se i membri del gruppo assegnati a un compito riferiscono al responsabile l'errore nella valutazione delle tempistiche si procede immediatamente ad una nuova pianificazione alla luce delle affermazioni dei membri coinvolti nel compito.\\

RO3 Assenza di comunicazione gruppo-proponente & Potrebbe accadere che presi dal lavoro si trascuri la comunicazione con Imola Informatica & Tutti i membri devono ricordarsi di mantenere un dialogo con l'azienda cercando di raccogliere dubbi e portando i nostri avanzamenti & In caso di assenza di comunicazione gruppo-azienda il responsabile deve fissare una data per un incontro, prima della prima scadenza di revisione, entro la quale il team deve impegnarsi a raccogliere domande, se presenti, e proporre tutti i progressi che sono stati fatti per ricevere feedback essenziali per la corretta riuscita del prodotto. \\

RO4 Impossibilità di stabilire un incontro tra i membri del gruppo & Potrebbe accadere che dato il numero di componenti del gruppo stabilire un incontro in cui tutti siano presenti risulti difficile e in caso si riesca non abbastanza immediato come servirebbe & Il gruppo tiene una tabella con le proprie disponibilità in settimana e si possono vedere gli orari in cui tutti possono essere presenti, oppure si può notare che in nessun giorno tutti sono liberi & Per discutere si possono usare servizi di chiamata come Hangouts, inoltre molto raramente sarà strettamente necessaria la presenza di tutti i membri del gruppo quindi diventa molto più facile organizzarsi a sotto-gruppi e stabilire un incontro.\\


\end{longtable}
}

\subsubsection{Tabella del Grado del Rischio}
Come descritto nelle fasi della gestione del rischio, è importante valutare il grado del rischio, ovvero stabilire la probabilità e la gravità che il rischio potrebbe avere durante il progetto.
//Ogni colonna riporterà il codice di ciascuno dei rischi analizzati nella tabella precedente e sarà composta da:
\begin{itemize}
	\item Codice del Rischio;
	\item Frequenza;
	\item Gravità;
\end{itemize}

{
	\rowcolors{2}{grigetto}{white}
	\renewcommand{\arraystretch}{2}
	\centering
	\begin{longtable}{ C{2cm} C{3cm} C{3cm}}
		\rowcolor{rossoep}
		\textcolor{white}{\textbf{Codice}} & \textcolor{white}{\textbf{Frequenza}} & \textcolor{white}{\textbf{Gravità}}\\	
		
		RT1 & Alta & Media\\
		
		RP1 & Media & Media\\
		
		RP2 & Media & Alta\\
		
		RR1 & Alta & Alta \\
		
		RS1 & Media & Bassa \\
		
		RO1 & Media & Alta \\
		
		RO2 & Media & Media \\
		
		RO3 & Bassa & Media \\
		
		RO4 & Alta & Bassa \\
		
	\end{longtable}
}
\clearpage
\section{Modello di Sviluppo}
Come modello di sviluppo il gruppo \Gruppo{} ha deciso di adottare il \textbf{modello incrementale}.
\subsection{Descrizione}
Nel modello incrementale il prodotto viene sviluppato tramite rilasci successivi. Questi rilasci hanno l'obiettivo di aggiungere funzionalità separate e accessorie a un sistema stabile in cui sono presenti requisiti di base.
Nel caso in cui un rilascio sia fallace è molto facile tornare allo stato funzionante precedente.\\
Il modello incrementale richiede, dunque, una suddivisione preliminare dei requisiti atta ad identificare quelli da sviluppare per primi e quali aggiungere al sistema stabile per incrementi. \\
Inoltre, una volta implementate le caratteristiche base del sistema lo si può sottoporre al committente e al proponente per assicurarsi di star procedendo nella giusta direzione.
In caso negativo, non è troppo tardi per cambiare la struttura del prodotto corrente. \\
Infine, non è particolarmente dispendioso riformulare degli incrementi previsti ma che devono ancora essere implementati. 

\subsection{Motivazioni}
Il gruppo ha scelto questo modello di sviluppo perché si adatta bene alle specifiche del progetto \NomeProgetto{} del proponente \Proponente{}.
Nella fattispecie, è stato facile identificare i requisiti minimi e separare molti requisiti accessori perfetti per essere implementati tramite rilasci incrementali su di un sistema stabile.\\
Inoltre, data la nostra inesperienza, il modello scelto permette a eventuali cambiamenti in corso d'opera di essere poco dispendiosi dal punto di vista sia del tempo di codifica (se circoscritti a singoli rilasci), sia del lavoro di cambiamento della documentazione. \\
In aggiunta a ciò, i rilasci successivi di funzionalità permettono di poter stabilire un confronto migliore con il proponente, riuscendo a sottoporre al suo giudizio un prodotto che sia sempre funzionante e col tempo sempre più completo e conforme alle sue aspettative. \\
Abbiamo inoltre valutato che i principali difetti del modello incrementale, quali la degradazione della struttura causata dall'aggiunta di incrementi e l'invisibilità del processo al manager, 
non influenzano il gruppo data la dimensione ridotta, relativamente ad ambienti aziendali dove i modelli di sviluppo sono sfruttati a pieno, del progetto che stiamo affrontando.


\subsection{Individuazione degli incrementi}
In seguito è riportata una tabella con indicati i requisiti che vengono sviluppati in ciascun incremento, sia dell'applicazione che del server.
I requisiti sono identificati dal loro codice identificativo e sono reperibili nel documento \AdR{}.\\
I codici dei requisiti enfatizzati in grassetto sono obbligatori, quelli non evidenziati sono requisiti desiderabili oppure opzionali.
La scelta di enfatizzare quelli grafici è puramente per una maggior comodità di consultazione.

\input{Sezioni/Sottosezioni/Incrementi.tex}
\clearpage
\section{Pianificazione}
Nella pianificazione, il \Responsabile{} suddivide il lavoro in attività e le assegna a ciascun membro del team.
Lo scopo è dimostrare come deve venire svolto il lavoro, valutare i progressi nel progetto e anticipare i problemi che potrebbero sorgere preparando delle soluzioni a tali problemi.\\
La pianificazione di progetto viene organizzata seguendo le scadenze presentate nella sezione §8.3.
Lo sviluppo del progetto viene suddiviso nelle seguenti quattro fasi: 
\begin{itemize}
	\item Analisi;
	\item Progettazione Architetturale;
	\item Progettazione di Dettaglio e Codifica;
	\item Validazione e Collaudo.
\end{itemize}
Ogni fase è suddivisa in periodi più brevi all'interno dei quali vengono elencate le diverse attività che il gruppo \Gruppo{} deve svolgere e gli incrementi previsti.


\subsection{Analisi}
Periodo: dal 2019-11-15 al 2020-01-20\\
Inizia con la formazione del gruppo e finisce con la data di consegna della Revisione dei Requisiti.\\
In questa fase viene definito il gruppo, la normazione (\glo{way of working}) e la garanzia di qualità che vuole fornire, oltre alla definizione dei requisiti del capitolato che viene scelto.
\subsubsection{Periodo 1} 
Dal 2019-11-15 al 2019-11-29\\
In questo periodo, che parte dalla formazione del gruppo e termina con la scelta del capitolato C5 \NomeProgetto{}, il gruppo ha affrontato le seguenti tematiche al fine di porre le basi per il lavoro che andava affrontato:
\begin{itemize}
	\item \textbf{Discussione capitolati}: Ogni membro del gruppo ha studiato individualmente e in seguito discusso durante gli incontri tutti i capitolati proposti, ponendo le basi per la stesura del documento \SdF{} e ha indirizzato verso la scelta del capitolato scelto;
	\item \textbf{Assegnazione e studio dei ruoli di progetto}: Ad ogni membro del gruppo è stato assegnato il ruolo principale da ricoprire nella fase di Analisi;
	\item \textbf{Definizione degli strumenti}: Vengono discusse e definite le tecnologie da usare per affrontare la fase di Analisi;
	\item \textbf{Pianificazione milestone fase di Analisi}: Vengono discusse e fissate delle \glo{milestone} intermedie da rispettare per completare la fase di Analisi entro le scadenze imposteci.
\end{itemize}
\subsubsection{Periodo 2} 
Dal 2019-11-30 al 2019-12-31\\
Questo periodo inizia con la scelta definitiva del capitolato C5 \NomeProgetto{}.\\
Dopo la scelta, sono state focalizzate le risorse del gruppo nei seguenti punti:
\begin{itemize}
	\item \textbf{Normazione}: Vengono definite le regole per la stesura dei documenti e per l'utilizzo delle tecnologie identificate in precedenza;
	\item \textbf{Approfondimento capitolati}: Vengono ulteriormente discussi tutti i capitolati in modo da terminare lo studio di fattibilità e focalizzare la nostra analisi sul capitolato scelto in modo da predisporre le basi per l'analisi dei requisiti;
	\item \textbf{Prima definizione dei casi d'uso};
	\item \textbf{Determinazione standard di qualità}: Abbiamo definito le nostre strategie per garantire la qualità di processo e la qualità di prodotto;
	\item \textbf{Verifica}: Verifica dell'andamento del gruppo in relazione alle tempistiche e allo svolgimento dei compiti assegnati.
\end{itemize}
\subsubsection{Periodo 3}
 Dal 2020-01-01 al 2020-01-14\\
 Questo periodo si estende fino alla data ultima di consegna per affrontare la Revisione dei Requisiti a cui il nostro gruppo ha deciso di partecipare.\\
 \begin{itemize}
	\item \textbf{Normazione}: Ulteriori approfondimenti alle regole per la stesura dei documenti e per l'utilizzo delle tecnologie;
	\item \textbf{Approfondimento delle tecnologie}: Vengono ampliate le conoscenze sulle tecnologie richieste dal capitolato per essere svolto;
	\item \textbf{Analisi dei requisiti}: Studio dei requisiti e raffinamento dei casi d'uso;
	\item \textbf{Pianificazione attività}: Pianificazione del lavoro da svolgere nelle fasi successive a quella di Analisi;
	\item \textbf{Verifica}: Verifica dell'andamento del team in relazione alle tempistiche e allo svolgimento dei compiti assegnati.

 \end{itemize}
\subsubsection{Periodo 4} 
Dal 2020-01-15 al 2020-01-20\\
In questo periodo, che ha inizio con la consegna dei documenti per la Revisione dei Requisiti alla presentazione pubblica della proposta, il gruppo consolida il lavoro svolto in vista delle successive fasi e della discussione per la quale serve una presentazione;
\begin{itemize}
	\item \textbf{Consolidamento}: Ogni membro del gruppo si prende del tempo per ripassare tutto il lavoro svolto e per studiare il necessario per affrontare al meglio le fasi successive;
	\item \textbf{Preparazione per la Revisione dei Requisiti}: Il gruppo produce il materiale necessario da esporre alla presentazione pubblica della nostra proposta.
\end{itemize}

\newpage
% Inizia la pagina orientata orizzontalmente
\begin{landscape}
% Ora la pagina e' in orizzontale!
\subsubsection{Diagramma di Gantt delle attività della fase di Analisi}
\pagestyle{empty}
\begin{figure}[h]
	\centering	
	\includegraphics[scale=0.455]{Sezioni/DiagrammiGantt/Analisi.png}
	\caption{Diagramma di Gantt delle attività della fase di Analisi}
\end{figure}
\end{landscape}
\clearpage

\subsection{Progettazione Architetturale}
Periodo: dal 2020-01-22 al 2020-03-15\\
Inizia al termine della fase di Analisi e finisce con la data di consegna della Revisione di Progettazione.\\
In questa fase viene definita una soluzione architetturale in modo da soddisfare i requisiti individuati nella fase di Analisi.

\subsubsection{Periodo 1} 
Dal 2020-01-22 al 2020-02-11
\begin{itemize}
	\item \textbf{Normazione}: Standardizzazione e correzione di alcune parti dei documenti che non aderiscono completamente alle \NdP{};
	\item \textbf{Analisi dei requisiti}: Correzione e modifica dei casi d'uso segnalati;
	\item \textbf{Assegnazione dei ruoli di progetto}: Assegnazione dei ruoli di ciascun membro del gruppo in base alla suddivisione oraria indicata in §5.2.1;
	\item \textbf{Pianificazione attività}: Le attività da svolgere devono essere prima pianificate e discusse dal gruppo per garantire il \glo{way of working} sancito nelle \NdP{};
	\item \textbf{Approfondimento delle tecnologie}: Ricerca di documentazione e materiali utili per l'apprendimento delle nuove tecnologie da utilizzare per la realizzazione del prodotto finale;
	\item \textbf{Verifica}: Verifica dell'andamento del team in relazione alle tempistiche e allo svolgimento dei compiti assegnati.
\end{itemize}
\subsubsection{Periodo 2} 
Dal 2020-02-12 al 2020-03-08
\begin{itemize}
<<<<<<< HEAD
	\item \textbf{Studio delle tecnologie:} IAAS \glo{Kubernetes} o \glo{PaaS}, \glo{Openshift} o \glo{Rancher}, \glo{LDAP} e \glo{GPS};
	\item \textbf{Normazione:} Decisioni ed inserimento delle nuove regole da adottare per le prossime condizioni dello sviluppo;
	\item \textbf{Miglioramento standard di qualità:} Aggiunzione, rimozione o modifica di alcuni standard per garantire qualità nei processi e prodotti software;
	\item \textbf{Technology Baseline:} Redazione della \glo{Technology Baseline}, cioè un allegato tecnico nella quale vengono stesi i design pattern che verranno utilizzati durante lo sviluppo;
	\item \textbf{Proof of Concept:} Rappresentazione della \glo{Baseline};
	\item \textbf{Codifica:} Viene codificato il \glo{Proof of Concept}, nella quale viene condiviso tramite \glo{repository} al committente e proponente in una data da definire;
	\item \textbf{Verifica:} Verifica dell'andamento del team in relazione alle tempistiche e allo svolgimento dei compiti assegnati.
=======
	\item \textbf{Studio delle tecnologie}: l'\glo{IaaS} \glo{Kubernetes} o i \glo{PaaS} \glo{Openshift} o \glo{Rancher}, \glo{LDAP} e \glo{GPS};
	\item \textbf{Normazione}: Decisioni ed inserimento delle nuove regole da adottare per le prossime condizioni dello sviluppo;
	\item \textbf{Miglioramento standard di qualità}: Aggiunta, rimozione o modifica di alcune metriche per garantire le qualità di processo e di prodotto affermate nel \PdQ{};
	\item \textbf{Technology Baseline}: Redazione della \glo{Technology Baseline}, cioè un allegato tecnico nel quale vengono indicate le tecnologie e i design pattern che vengono utilizzati durante lo sviluppo del prodotto;
	\item \textbf{Proof of Concept}: Creazione di un eseguibile che permetta di dimostrare la validità del prodotto che si vuole fornire, concretizzando la \glo{Technology Baseline};
	\item \textbf{Codifica}: Viene codificato il \glo{Proof of Concept} e successivamente condiviso tramite i \glo{repository} del gruppo al committente e al proponente in una data da definire;
	\item \textbf{Verifica}: Verifica dell'andamento del team in relazione alle tempistiche e allo svolgimento dei compiti assegnati.
>>>>>>> b3dd61d2b76f06616fbece5634acabe2aacf85cc
\end{itemize}
\subsubsection{Periodo 3} 
Dal 2020-03-09 al 2020-03-15
\begin{itemize}
	\item \textbf{Consolidamento}: Ogni membro si prende del tempo per ripassare tutto il lavoro svolto e per studiare il necessario per affrontare al meglio le fasi successive;
	\item \textbf{Preparazione per la Revisione di Progettazione}: Il gruppo produce il materiale necessario da esporre alla presentazione pubblica della nostra proposta.
\end{itemize}

\newpage
% Inizia la pagina orientata orizzontalmente
\begin{landscape}
% Ora la pagina e' in orizzontale!
\subsubsection{Diagramma di Gantt delle attività della fase di Progettazione Architetturale}
\pagestyle{empty}
\begin{figure}[h]
	\centering
	\includegraphics[scale=1.48]{Sezioni/DiagrammiGantt/ProgettazioneArchitetturale.png}
	\caption{Diagramma di Gantt delle attività della fase di Progettazione Architetturale}	
\end{figure}
\end{landscape}

\subsection{Progettazione di Dettaglio e Codifica}
Dal 2020-03-16 al 2020-04-19\\
Inizia al termine della Progettazione Architetturale e finisce con la data di consegna della Revisione di Qualifica.\\
In questa fase si definisce nel dettaglio e si implementa l'architettura logica costruita nella fase di Progettazione Architetturale.\\


\subsubsection{Periodo 1} 
Dal 2020-03-16 al 2020-03-27\\
\begin{itemize}
	\item \textbf{Approfondimento delle tecnologie:} Ricerca documentazione e materiali utili per l'apprendimento delle nuove tecnologie da utilizzare per la realizzazione del prodotto finale;
	\item \textbf{Normazione:} Standardizzazione e correzione di alcune parti dei documenti che non aderiscono completamente alle norme;
	\item \textbf{Assegnazione dei ruoli di progetto:} Ad ogni membro del gruppo viene assegnato il ruolo principale da ricoprire nella fase di progettazione di dettaglio e codifica;
	\item \textbf{Pianificazione delle attività:} Le attività da svolgere devono essere prima pianificate e discusse dal gruppo per garantire successivamente un buon \glo{way of working};
	\item \textbf{Progettazione:} Ricerca di una soluzione soddisfacente per tutti gli \glo{stakeholder}, descrive l'architettura del prodotto prima di pensare al codice ed attua un approccio sintetico;
	\item \textbf{Codifica:} Implementazione dei requisiti di base identificati per ottenere un sistema stabile;
	\item \textbf{Manuali:} Stesura manuale utente e manuale manutentore in relazione alle funzionalità di base del sistema.
\end{itemize}
\subsubsection{Periodo 2} 
Dal 2020-03-28 al 2020-04-08\\
\begin{itemize}
	\item \textbf{Implementazione della Product Baseline:} Seguendo le specifiche della \glo{Technology Baseline};
	\item \textbf{Codifica incrementale:} Aggiunta di requisiti al sistema tramite incrementi;
	\item \textbf{Incremento e verifica:} Incrementi e verifiche con eventuali aggiunte al lavoro svolto in precedenza;
	\item \textbf{Manuali:} Aggiunta nel manuale utente e nel manuale manutentore delle funzionalità inserite incrementalmente nel sistema;
\end{itemize}
\subsubsection{Periodo 3}
Dal 2020-04-09 al 2020-04-12\\
\begin{itemize}
	\item \textbf{Primo rilascio del prodotto} Pubblicazione del prodotto in un apposito \glo{repository} condiviso dai membri del gruppo;
	\item \textbf{Verifica:} Verifica dell'andamento del team in relazione alle tempistiche e allo svolgimento dei compiti assegnati;
\end{itemize}
\subsubsection{Periodo 4} 
Dal 2020-04-13 al 2020-04-19\\
\begin{itemize}
	\item \textbf{Consolidamento:} Ogni membro si prende del tempo per ripassare tutto il lavoro svolto e per studiare il necessario per affrontare al meglio le fasi successive;
	\item \textbf{Preparazione per la Revisione di Qualifica:} Il gruppo produce il materiale necessario da esporre alla presentazione pubblica della nostra proposta.
\end{itemize}

\newpage
% Inizia la pagina orientata orizzontalmente
\begin{landscape}
	% Ora la pagina e' in orizzontale!
	\subsubsection{Diagramma di Gantt delle attività}
	\pagestyle{empty}
	\begin{figure}[h]
			
		\begin{center}	
				\includegraphics[scale=0.5]{Sezioni/DiagrammiGantt/ProgettazioneDiDettaglio.png}	
		\end{center}
	\caption{Diagramma di Gantt delle attività di Progettazione di Dettaglio e Codifica}	
	\end{figure}
\end{landscape}

\subsection{Validazione e Collaudo}
Inizia al termine della progettazione di dettaglio e codifica e finisce con la data di consegna della revisione di accettazione.
\\In questo fase vengono definite le attività che servono per verificare che il prodotto corrisponde a quello desiderato dal cliente.
\subsubsection{Periodo 1} 
Dal 2020-04-21 al 2020-04-28
\begin{itemize}
	\item \textbf{Normazione:} Standardizzazione e correzione di alcune parti dei documenti che non aderiscono completamente alle norme;
	\item \textbf{Assegnazione dei ruoli di progetto:} Ad ogni membro del gruppo viene assegnato il ruolo principale da ricoprire nella fase di progettazione di validazione e collaudo;
	\item \textbf{Soddisfazione dei requisiti:} Controllo che i requisiti siano soddisfatti;
	\item \textbf{Pianificazione attività:} Le attività da svolgere devono essere prima pianificate e discusse dal gruppo per garantire successivamente un buon \glo{way of working};
	\item \textbf{Verifica} Verifica dell'andamento del team in relazione alle tempistiche e allo svolgimento dei compiti assegnati.
\end{itemize}
\subsubsection{Periodo 2} 
Dal 2020-04-29 al 2020-05-10
\begin{itemize}
	\item \textbf{Codifica:} Esecuzione dell'ultimo versionamento del prodotto;
	\item \textbf{Verifica:} Accertamento che le esecuzioni delle attività siano esenti da errori;
	\item \textbf{Validazione:} Verifica se il prodotto realizzato sia conforme alle attese, e validazione finale in caso di esito positivo;
	\item \textbf{Scrittura dei manuali:} Esecuzione del secondo versionamento del manuale utente e del manuale manutentore;
	\item \textbf{Collaudo:} Vengono eseguiti gli ultimi test sul prodotto per verificare se le funzionalità rispettano i risultati attesi.
\end{itemize}
\subsubsection{Periodo 3} 
Dal 2020-05-11 al 2020-05-17
\begin{itemize}
	\item \textbf{Preparazione per la Revisione di Accettazione:} Il gruppo produce il materiale necessario da esporre alla presentazione pubblica della nostra proposta.
\end{itemize}


\newpage
% Inizia la pagina orientata orizzontalmente
\begin{landscape}
	% Ora la pagina e' in orizzontale!
	\subsubsection{Diagramma di Gantt delle attività}
	\pagestyle{empty}
	\begin{figure}[h]
		
		\begin{center}	
			\includegraphics[scale=1.6]{Sezioni/DiagrammiGantt/Validazione.png}
		\end{center}
	\caption{Diagramma di Gantt delle attività di Validazione e Collaudo}	
	\end{figure}
\end{landscape}
\clearpage
\section{Preventivo}
Sigle identificative per i ruoli indicati nelle tabelle e nei grafici
\begin{itemize}
    \item RE: Responsabile;
    \item AM: Amministratore;
    \item AN: Analista;
    \item PT: Progettista;
    \item PR: Programmatore;
    \item VE: Verificatore.
\end{itemize}
{
	
	\rowcolors{2}{grigetto}{white}
	\renewcommand{\arraystretch}{2}
	\begin{table}[h!]
		\centering
		\begin{longtable}{ C{2cm} C{3cm}}
			\rowcolor{rossoep}
			\textcolor{white}{\textbf{Ruolo}} & \textcolor{white}{\textbf{Costo per ora espresso in euro}}\\	
			
			Responsabile & 30\\
			Amministratore & 20\\
			Analista & 25\\
			Progettista & 22\\
			Programmatore & 15\\
			Verificatore & 15\\
			
		\end{longtable}
		\caption{  \ref{table:5} rappresenta il costo orario associato a ciascun ruolo}
		\label{table:5}
	\end{table}
}

\subsection{Fase di Analisi}

\subsubsection{Divisione Oraria}
La seguente tabella §5.1.1 rappresenta la distribuzione oraria dei ruoli per ogni componente del gruppo.
{
	\rowcolors{2}{grigetto}{white}
	\renewcommand{\arraystretch}{2}
	\centering
	\begin{longtable}{ C{5cm} C{1cm} C{1cm} C{1cm} C{1cm} C{1cm} C{1cm} C{3cm}}
		\rowcolor{rossoep}
		\textcolor{white}{\textbf{Nome membro del gruppo}} & \textcolor{white}{\textbf{RE}} & \textcolor{white}{\textbf{AM}} & \textcolor{white}{\textbf{AN}} & \textcolor{white}{\textbf{PT}} & \textcolor{white}{\textbf{PR}} & \textcolor{white}{\textbf{VE}} & \textcolor{white}{\textbf{Ore complessive}}\\	
        
        
        Christian Mattei     &  & 7 & 12 &  & & 11 & 30 \\
		Davide Lazzaro       &  & 5 & 16 &  &  & 9 & 30 \\
        Emanuele Cisotto     &  &  & 21 &  &  & 9 & 30 \\
        Enrico Salmaso       & 15 & 2 & 8  &  &  & 5 & 30 \\
        Federico Perin       &  &  & 21 &  &  & 9 & 30 \\
        Francesco Drago      &  & 7 & 16 &  &  & 7 & 30 \\
        Riccardo Baratin     &  & 5 & 11 &  &  & 14 & 30 \\
        Tommaso Azzalin      & 4 & 12 & 9  &  &  & 5 & 30 \\
        \textbf{Ore totali ruolo} & 19 & 38 & 114 &  &  & 69 & 240 \\
		
	\end{longtable}
}

La quantità di ore che ciascun componente del gruppo ha svolto per ogni ruolo viene rappresentata nel seguente istogramma:

\begin{figure}[h]
	\centering
	\includegraphics[scale=2]{sezioni/Istogrammi/IstogrammaAnalisi.png}
	\caption{Istogramma della disposizione ore per ruolo di ciascun componente del fase di analisi}
\end{figure}

\subsubsection{Costo Risultate}
La seguente tabella §5.1.2 rappresenta, per ruolo, le ore totali investite e il corrispondente costo in euro.
{
	\rowcolors{2}{grigetto}{white}
	\renewcommand{\arraystretch}{2}
	\centering
	\begin{longtable}{ C{3cm} C{2cm} C{4cm}}
		\rowcolor{rossoep}
		\textcolor{white}{\textbf{Ruolo}} & \textcolor{white}{\textbf{Totale ore}} & \textcolor{white}{\textbf{Costo Ruolo in euro}}\\	
        
        Responsabile & 19 & 570\\
        Amministratore & 38 & 760\\
        Analista & 114 & 2850 \\
        Progettista & 0 & 0 \\
        Programmatore & 0 & 0 \\
        Verificatore & 69 & 1035 \\
        \textbf{Totale} & 240 & 5215 \\
		
	\end{longtable}
}

La quantità di ore totali di ciascun ruolo viene rappresentata nel seguente aerogramma:

\begin{figure}[h]
	\centering
	\includegraphics[scale=2]{sezioni/Aerogrammi/AerogrammaAnalisi.png}
	\caption{Suddivisione ore per ruolo della fase di analisi}
\end{figure}

\subsection{Progettazione Architetturale}

\subsubsection{Divisione Oraria}
La seguente tabella §5.2.1 rappresenta la distribuzione oraria dei ruoli per ogni componente del gruppo.
{
	\rowcolors{2}{grigetto}{white}
	\renewcommand{\arraystretch}{2}
	\centering
	\begin{longtable}{ C{5cm} C{1cm} C{1cm} C{1cm} C{1cm} C{1cm} C{1cm} C{3cm}}
		\rowcolor{rossoep}
		\textcolor{white}{\textbf{Nome membro del gruppo}} & \textcolor{white}{\textbf{RE}} & \textcolor{white}{\textbf{AM}} & \textcolor{white}{\textbf{AN}} & \textcolor{white}{\textbf{PT}} & \textcolor{white}{\textbf{PR}} & \textcolor{white}{\textbf{VE}} & \textcolor{white}{\textbf{Ore complessive}}\\	
        
        Christian Mattei & 5 & & & 14 & 10 & 4 & 33\\
        Davide Lazzaro & 8 & & & 18 & & 7 & 33 \\
        Emanuele Cisotto & & 8 & & 10 & 5 & 6 & 29 \\
        Enrico Salmaso & & 10 & 6 & 10 & & 6 & 32 \\
        Federico Perin & & 6 & & 5 & 7 & 13 &  31\\
        Francesco Drago & & & 13 & 5 & 7 & 6 & 31 \\
        Riccardo Baratin & 6 & & & & 17 & 8 & 31\\
        Tommaso Azzalin & & & 8 & 7 & & 16 & 31\\
        \textbf{Ore totali ruolo} & 19 & 24 & 27 & 69 & 46 & 66 & 251\\
		
	\end{longtable}
}

La suddivisione di quante ore ha svolto ciascun componente del gruppo per ogni ruolo viene rappresentata nel seguente istogramma:

\begin{figure}[h]
	\centering
	\includegraphics[scale=3]{sezioni/Istogrammi/IstogrammaProgettArchitetturale.png}
	\caption{Istogramma della disposizione ore per ruolo di ciascun componente della fase di Progettazione Architetturale}
\end{figure}

\subsubsection{Costo Risultate}
La seguente tabella §5.2.2 rappresenta, per ruolo, le ore totali investite e il corrispondente costo in euro.
{
	\rowcolors{2}{grigetto}{white}
	\renewcommand{\arraystretch}{2}
	\centering
	\begin{longtable}{ C{3cm} C{2cm} C{4cm}}
		\rowcolor{rossoep}
		\textcolor{white}{\textbf{Ruolo}} & \textcolor{white}{\textbf{Totale ore}} & \textcolor{white}{\textbf{Costo Ruolo in euro}}\\	
        
        Responsabile & 19 & 570 \\
        Amministratore & 24 & 480 \\
        Analista & 27 & 675 \\
        Progettista & 69 & 1518 \\
        Programmatore & 46 & 690 \\
        Verificatore & 66 & 990 \\
        \textbf{Totale} & 251 & 4923 \\
		
	\end{longtable}
}

La suddivisione della quantità di ore totali di ciascun ruolo viene rappresentata nel seguente aerogramma:

\begin{figure}[h]
	\centering
	\includegraphics[scale=2]{sezioni/Aerogrammi/AerogrammaProgettArchitetturale.png}
	\caption{Suddivisione ore per ruolo della fase di Progettazione Architetturale}
\end{figure}

\subsection{Progettazione di dettaglio e codifica}

\subsubsection{Divisione Oraria}
La seguente tabella §5.3.1 rappresenta la distribuzione oraria dei ruoli per ogni componente del gruppo.
{
	\rowcolors{2}{grigetto}{white}
	\renewcommand{\arraystretch}{2}
	\centering
	\begin{longtable}{ C{5cm} C{1cm} C{1cm} C{1cm} C{1cm} C{1cm} C{1cm} C{3cm}}
		\rowcolor{rossoep}
		\textcolor{white}{\textbf{Nome membro del gruppo}} & \textcolor{white}{\textbf{RE}} & \textcolor{white}{\textbf{AM}} & \textcolor{white}{\textbf{AN}} & \textcolor{white}{\textbf{PT}} & \textcolor{white}{\textbf{PR}} & \textcolor{white}{\textbf{VE}} & \textcolor{white}{\textbf{Ore complessive}}\\	
        
        Christian Mattei & & 10 & & 12 & 15 & 10 & 47\\
        Davide Lazzaro & 6 & & & 10 & 15 & 16 & 47\\
        Emanuele Cisotto & & 5 & & 13 & 16 & 13 & 47 \\
        Enrico Salmaso & & 4 & & 11 & 18 & 14 & 47\\
        Federico Perin & & 8 & & 8 & 19 & 12 & 47\\
        Francesco Drago & 10 & & & 7 & 17 & 13 & 47\\
        Riccardo Baratin & & 3 & & 12 & 19 & 13 & 47\\
        Tommaso Azzalin & 10 & & & 10 & 17 & 10 & 47 \\
        \textbf{Ore totali ruolo} & 26 & 30 & & 83 & 136 & 101 & 376\\
		
	\end{longtable}
}

La suddivisione di quante ore ha svolto ciascun componente del gruppo per ogni ruolo viene rappresentata nel seguente istogramma:

\begin{figure}[h]
	\centering
	\includegraphics[scale=3]{sezioni/Istogrammi/IstogrammaDiDettaglio.png}
	\caption{Istogramma della disposizione ore per ruolo di ciascun componente della fase di progettazione di dettaglio e codifica}
\end{figure}

\subsubsection{Costo Risultate}
La seguente tabella §5.3.2 rappresenta, per ruolo, le ore totali investite e il corrispondente costo in euro.
{
	\rowcolors{2}{grigetto}{white}
	\renewcommand{\arraystretch}{2}
	\centering
	\begin{longtable}{ C{3cm} C{2cm} C{4cm}}
		\rowcolor{rossoep}
		\textcolor{white}{\textbf{Ruolo}} & \textcolor{white}{\textbf{Totale ore}} & \textcolor{white}{\textbf{Costo Ruolo in euro}}\\	
        
        Responsabile & 26 & 780 \\
        Amministratore & 30 & 600 \\
        Analista & 0 & 0 \\
        Progettista & 83 & 1826 \\
        Programmatore & 136 & 2040 \\
        Verificatore & 101 & 1515\\
        \textbf{Totale} & 376 & 6761 \\
		
	\end{longtable}
}

La suddivisione della quantità di ore totali di ciascun ruolo viene rappresentata nel seguente aerogramma:

\begin{figure}[h]
	\centering
	\includegraphics[scale=2]{sezioni/Aerogrammi/AerogrammaDiDettaglio.png}
	\caption{Suddivisione ore per ruolo della fase di Progettazione di dettaglio e codifica}
\end{figure}

\subsection{Validazione e Collaudo}

\subsubsection{Divisione Oraria}
La seguente tabella §5.4.1 rappresenta la distribuzione oraria dei ruoli per ogni componente del gruppo.
{
	\rowcolors{2}{grigetto}{white}
	\renewcommand{\arraystretch}{2}
	\centering
	\begin{longtable}{ C{5cm} C{1cm} C{1cm} C{1cm} C{1cm} C{1cm} C{1cm} C{3cm}}
		\rowcolor{rossoep}
		\textcolor{white}{\textbf{Nome membro del gruppo}} & \textcolor{white}{\textbf{RE}} & \textcolor{white}{\textbf{AM}} & \textcolor{white}{\textbf{AN}} & \textcolor{white}{\textbf{PT}} & \textcolor{white}{\textbf{PR}} & \textcolor{white}{\textbf{VE}} & \textcolor{white}{\textbf{Ore complessive}}\\	
        
        Christian Mattei & & & 7 & & 4 & 8 & 19\\
        Davide Lazzaro & & 4 & & 5 & & 10 & 19\\
        Emanuele Cisotto & 5 & & & & 6 & 12 & 23\\ 
        Enrico Salmaso & & & & 6 & 5 & 9 & 20 \\
        Federico Perin & 6 & & & & 7 & 8 & 21\\
        Francesco Drago & & 6 & & & 9 & 6 & 21\\
        Riccardo Baratin & & 5 & & & 6 & 10 & 21\\
        Tommaso Azzalin & & & 4 & 10 & & 7 & 21\\
        \textbf{Ore totali ruolo} & 11 & 15 & 11 & 21 & 37 & 70 & 165\\
		
	\end{longtable}
}


La suddivisione di quante ore ha svolto ciascun componente del gruppo per ogni ruolo viene rappresentata nel seguente istogramma:

\begin{figure}[h]
	\centering
	\includegraphics[scale=2.5]{sezioni/Istogrammi/IstogrammaValidazione.png}
	\caption{Istogramma della disposizione ore per ruolo di ciascun componente della fase di Validazione e Collaudo}
\end{figure}

\subsubsection{Costo Risultate}
La seguente tabella §5.4.2 rappresenta, per ruolo, le ore totali investite e il corrispondente costo in euro.
{
	\rowcolors{2}{grigetto}{white}
	\renewcommand{\arraystretch}{2}
	\centering
	\begin{longtable}{ C{3cm} C{2cm} C{4cm}}
		\rowcolor{rossoep}
		\textcolor{white}{\textbf{Ruolo}} & \textcolor{white}{\textbf{Totale ore}} & \textcolor{white}{\textbf{Costo Ruolo in euro}}\\	
        
        Responsabile & 11 & 330\\
        Amministratore & 15 & 300 \\
        Analista & 11 & 275\\
        Progettista & 21 & 462\\
        Programmatore & 37 & 555\\
        Verificatore & 70 & 1050\\
        \textbf{Totale} & 169 & 2972\
		
	\end{longtable}
}

La suddivisione della quantità di ore totali di ciascun ruolo viene rappresentata nel seguente aerogramma:

\begin{figure}[h]
	\centering
	\includegraphics[scale=2.5]{sezioni/Aerogrammi/AerogrammaValidazione.png}
	\caption{Suddivisione ore per ruolo della fase di Validazione e Collaudo}
\end{figure}


\clearpage
\subsection{Preventivo finale} 
Nel preventivo riportiamo la spesa totale che il committente dovrà affrontare, derivata dal totale delle ore rendicontate e preventivate nelle fasi di Progettazione Architetturale, Progettazione di Dettaglio e Codifica, Validazione e Collaudo.\\

\subsubsection{Divisione oraria complessiva} 
Nella seguente tabella viene mostrata la distribuzione oraria dei ruoli per ogni componente del gruppo.\\
{
	\rowcolors{2}{grigetto}{white}
	\renewcommand{\arraystretch}{2}
	\centering
	\begin{longtable}{ C{5cm} C{1cm} C{1cm} C{1cm} C{1cm} C{1cm} C{1cm} C{3cm}}
		\rowcolor{rossoep}
		\textcolor{white}{\textbf{Nome membro del gruppo}} & \textcolor{white}{\textbf{RE}} & \textcolor{white}{\textbf{AM}} & \textcolor{white}{\textbf{AN}} & \textcolor{white}{\textbf{PT}} & \textcolor{white}{\textbf{PR}} & \textcolor{white}{\textbf{VE}} & \textcolor{white}{\textbf{Ore complessive}}\\	
        
        Christian Mattei & 5 & 10 & 7 & 26 & 29 & 22 & 99 \\
        Davide Lazzaro & 14 & 4 & & 33 & 15 & 33 & 99\\
        Emanuele Cisotto & 5 & 13 & & 23 & 27 & 31 & 99 \\
        Enrico Salmaso & & 14 & 6 & 27 & 23 & 29 & 99\\
        Federico Perin & 6 & 14 & & 13 & 33 & 33 & 99\\
        Francesco Drago & 10 & 6 & 13 & 12 & 33 & 25 & 99 \\
        Riccardo Baratin & 6 & 8 & & 12 & 42 & 31 & 99 \\
        Tommaso Azzalin & 10 & & 12 & 27 & 17 & 33 & 99 \\
        \textbf{Ore totali ruolo} & 56 & 69 & 38 & 173 & 219 & 236 &  792 \\

	\end{longtable}
}
\subsubsection{Costo complessivo per ruolo}
Nella seguente tabella viene illustrato il monte ore risultante per ogni ruolo con il costo ad esso associato.\\
{
	\rowcolors{2}{grigetto}{white}
	\renewcommand{\arraystretch}{2}
	\centering
	\begin{longtable}{ C{3cm} C{2cm} C{4cm}}
		\rowcolor{rossoep}
		\textcolor{white}{\textbf{Ruolo}} & \textcolor{white}{\textbf{Totale ore}} & \textcolor{white}{\textbf{Costo Ruolo in euro}}\\	
        
        Responsabile & 56 &  1680\\
        Amministratore & 69 & 1380 \\
        Analista & 38 & 950 \\
        Progettista & 173 & 3806 \\
        Programmatore & 219 & 3285 \\
        Verificatore & 237 & 3555 \\
		
	\end{longtable}
}

\subsubsection{Costo complessivo}
Nella seguente tabella vengono riportati i costi complessivi delle varie fasi e infine l'importo proposto da qbTeam per la realizzazione del progetto Stalker.\\
{
	\rowcolors{2}{grigetto}{white}
	\renewcommand{\arraystretch}{2}
	\centering
	\begin{longtable}{ C{5cm} C{5cm}}
		\rowcolor{rossoep}
		\textcolor{white}{\textbf{Fase}} & \textcolor{white}{\textbf{Costo Fase}}\\	
		
		Progettazione Architetturale & 4923 \\
		Progettazione di Dettaglio e Codifica & 6761 \\
		Validazione e Collaudo & 2972 \\
		\textbf{Totale} & 14656\\
		
	\end{longtable}
}




\clearpage
\section{Consuntivo}

Nel Consuntivo approfondiamo il bilancio effettivo della fase di Analisi, ovvero la differenza tra le ore preventivate e quelle effettive.

\subsection{Analisi dei Requisiti}

Il bilancio della fase di Analisi è negativo, ovvero il gruppo ha impiegato più ore di quelle anticipate, quindi il costo finale della fase di Analisi supera l'aspettaviva calcolata in precedenza.\\
La seguente tabella illustra la differenza oraria ed economica rilevata a posteriori.

\rowcolors{2}{grigetto}{white}
\renewcommand{\arraystretch}{2}
\centering
\begin{longtable}{ C{5cm} C{1cm} C{1cm} C{1cm} C{1cm} C{1cm} C{1cm} C{3cm}}
	\rowcolor{rossoep}
	\textcolor{white}{\textbf{Ruolo}} & \textcolor{white}{\textbf{Ore preventivate}} & \textcolor{white}{\textbf{Ore supplementari}} & \textcolor{white}{\textbf{Costo preventivato}} & \textcolor{white}{\textbf{Costo effettivo}} & \textcolor{white}{\textbf{Differenza di costo}}\\	
	
	Responsabile & & & & & \\
	Amministratore & & & & & \\
	Analista & & & & & \\
	Progettista & & & & & \\
	Programmatore & & & & & \\
	Verificatore & & & & & \\
	\textbf{Totale} & & & & & & \\
	
\end{longtable}
\clearpage
\section{Organigramma}
\subsection{Redazione}
{
	\rowcolors{2}{grigetto}{white}
	\renewcommand{\arraystretch}{2}
	\begin{longtable}{ C{5cm} C{4cm} C{5cm} }
		\rowcolor{darkblue}
		\textcolor{white}{\textbf{Nome}} & \textcolor{white}{\textbf{Data di redazione}} & \textcolor{white}{\textbf{Firma}}\\	\endhead
        
        \LD{} & 2020-01-07 &
        \includegraphics[scale=0.60]{sezioni/Firme/Davide.png}  \\
        \SE{} & 2020-01-07 &
        \includegraphics[scale=0.70]{sezioni/Firme/Enrico.png}  \\
        		
	\end{longtable}
}

\subsection{Approvazione}
{
	\rowcolors{2}{grigetto}{white}
	\renewcommand{\arraystretch}{2}
	\centering
	\begin{longtable}{ C{5cm} C{4cm} C{5cm} }
		\rowcolor{darkblue}
		\textcolor{white}{\textbf{Nome}} & \textcolor{white}{\textbf{Data di Approvazione}} & \textcolor{white}{\textbf{Firma}}\\	\endhead
		
		
		\SE{} & 2020-01-13 &  
		\includegraphics[scale=0.70]{sezioni/Firme/Enrico.png}\\
		\VT{} &  & \\
		\CR{} & &  \\
		
	\end{longtable}
}

\subsection{Accettazione dei componenti}
{
	\rowcolors{2}{grigetto}{white}
	\renewcommand{\arraystretch}{2}
	\begin{longtable}{ C{5cm} C{4cm} C{5cm} }
		\rowcolor{darkblue}
		\textcolor{white}{\textbf{Nome}} & \textcolor{white}{\textbf{Data di Accettazione}} & \textcolor{white}{\textbf{Firma}}\\	\endhead
		
		
		\MC{} & 2020-01-10 & \includegraphics[scale=0.70]{sezioni/Firme/Christian.png}\\
		\LD{} & 2020-01-10 & \includegraphics[scale=0.60]{sezioni/Firme/Davide.png}\\
		\CE{} & 2020-01-10 & \includegraphics[scale=0.70]{sezioni/Firme/Emanuele.png} \\
		\SE{} & 2020-01-10 & \includegraphics[scale=0.70]{sezioni/Firme/Enrico.png}\\
		\PF{} & 2020-01-10 & \includegraphics[scale=0.50]{sezioni/Firme/Federico.png}\\
		\DF{} & 2020-01-10 & \includegraphics[scale=0.70]{sezioni/Firme/Francesco.png} \\
		\BR{} & 2020-01-10& \includegraphics[scale=0.70]{sezioni/Firme/Riccardo.png} \\
		\AT{} & 2020-01-10 & \includegraphics[scale=0.70]{sezioni/Firme/Tommaso.png} \\
		
		
	\end{longtable}
}

\clearpage
\subsection{Componenti}
{
	\rowcolors{2}{grigetto}{white}
	\renewcommand{\arraystretch}{2}
	\begin{longtable}{ C{4cm} C{2cm} C{8cm} }
		\rowcolor{darkblue}
		\textcolor{white}{\textbf{Nome}} & \textcolor{white}{\textbf{Matricola}} & \textcolor{white}{\textbf{Indirizzo di posta elettronica}}\\\endhead	
		
		\MC{} & 1121305 & christian.mattei@studenti.unipd.it \\
		\LD{} & 1162190 & davidemaria.lazzaro@studenti.unipd.it\\
		\CE{} & 1161514 & emanuele.cisotto@studenti.unipd.it\\
		\SE{} & 1166175 & enrico.salmaso.2@studenti.unipd.it \\
		\PF{} & 1170747 & federico.perin.1@studenti.unipd.it \\
		\DF{} & 1146928 & francesco.drago.1@studenti.unipd.it \\
		\BR{} & 1148142 & riccardo.baratin@studenti.unipd.it \\
		\AT{} & 1169740 & tommaso.azzalin@studenti.unipd.it \\
		
		
	\end{longtable}
}
\clearpage


\end{document}
Nel documento \PdQ{} verrà descritta la strategia utilizzata dai verificatori per effettuare nel miglior modo possibile la verifica e la validazione di tutti i documenti prodotti da \Gruppo{}.

Lo scopo nel dirigere il \PdQ{} è quello di:
\begin{itemize}
	\item Illustrare come si intende gestire la qualità di processo e di prodotto;
	\item Elencare le varie metriche definite per aderire alle definizioni degli standard;
	\item Elencare i test per verificare la corretta soddisfazione dei requisiti del prodotto software.
\end{itemize}

La qualità di processo e la qualità di prodotto sono due aspetti chiaramente coordinati, ma vengono gestiti separatamente. \\ \\
Le sezioni principali del documento sono le seguenti:
\begin{itemize}
    \item \textbf{Qualità di processo:} Sezione dove vengono elencate le metriche inerenti ai \glo{processi};
    \item \textbf{Qualità di prodotto:} Sezione dove vengono elencate le metriche inerenti al prodotto;
    \item \textbf{Strategia di testing:} Sezione dove viene elencato il piano di testing delle componenti e del sistema software nel suo complesso;
\end{itemize}


%inizio parte scritta da \PF{}
\subsection{Processo di sviluppo}
\subsubsection{Scopo}
Lo scopo del processo di sviluppo è quello di contenere compiti e attività da svolgere, in modo tale da poter fornire al proponente il prodotto finale da lui richiesto.\\
Per una corretta implementazione del prodotto software devono essere svolti i seguenti punti: 
\begin{itemize}
	\item Fissare gli obiettivi di sviluppo;
	\item Fissare i vincoli tecnologici e di design;
	\item Realizzare un prodotto finale che soddisfi tutti i test di \glo{verifica} e \glo{validazione} i quali devono rispettare quanto definito dal piano di qualità;
	\item Realizzare un prodotto finale conforme ai \glo{requisiti} richiesti dal proponente.
\end{itemize}
\subsubsection{Descrizione}
Il processo di sviluppo è formato da attività e compiti che fanno parte del ciclo di vita del prodotto software, il quale deve soddisfare tutti i \glo{requisiti} indicati nelle specifiche dal proponente per poter essere accettato.		
Il \glo{processo} di sviluppo come detto in precedenza è formato dalle seguenti attività:
\begin{itemize}
	\item \textbf{Analisi dei Requisiti}: Attività in cui si cerca di capire appieno il problema presentato dal cliente ricavando i \glo{requisiti} da soddisfare nella realizzazione del prodotto;
	\item \textbf{Progettazione}: Attività che segue l'analisi dei requisiti nella quale si stabilisce come devono essere realizzati i \glo{requisiti} imposti cliente;
	\item \textbf{Codifica}: Attività in cui si realizza concretamente quello che nel progetto è stato  studiato precedentemente;
	\item \textbf{Test}: Attività in cui si verifica che ogni pezzo di codice scritto durante l'attività di codifica sia privo di errori. Il tutto deve risultare conforme alle aspettative prefissate precedentemente nella way of working;
	\item \textbf{Collaudo}: Attività svolta in concomitanza con il proponente in cui si esegue un test finale che dimostri che il prodotto software realizzato soddisfa tutti i \glo{requisiti} presenti nel capitolato.
\end{itemize}
%fine parte scritta da \PF{}

\documentclass[a4paper, oneside, dvipsnames, table]{article}
\usepackage{../../Utilita/Stiletemplate}
\usepackage{hyperref}
\usepackage{fancyhdr}
\usepackage[italian]{babel}
\usepackage[raggedright]{titlesec}
\usepackage{blindtext}
\usepackage[export]{adjustbox}
\titleformat{\paragraph}[hang]{\normalfont\normalsize\bfseries}{\theparagraph}{1em}{}
\titlespacing*{\paragraph}{0pt}{3.25ex plus 1ex minus .2ex}{0.5em}

\newcommand{\Data}{2020-05-25}

\newcommand{\Titolo}{Verbale Riunione \Data}

\newcommand{\Redattori}{\PF{}}

\newcommand{\Verificatori}{\AT{}}

\newcommand{\Approvatore}{\CE{}}

\newcommand{\Distribuzione}{\VT{} \newline \CR{} \newline Gruppo \Gruppo{}}

\newcommand{\Uso}{Interno}

\newcommand{\DescrizioneDoc}{Questo documento si occupa di riportare quanto discusso nella riunione del \Data}

\newcommand{\pathimg}{../../../Utilita/Immagini/qbteam.png}

\newcommand{\Versionedoc}{1.0.0}
% scritto da \DF{},\AT{}

% info generali 
\newcommand{\NomeProgetto}{\textit{Stalker}}

% fornitore
\newcommand{\Gruppo}{\textit{qbteam}}
\newcommand{\Mail}{qbteamswe@gmail.com}
% \newcommand{\pathimg}{Immagini/qbteam.png}

% committenti
\newcommand{\Committente}{\VT \newline \CR}
\newcommand{\VT}{Prof. Vardanega Tullio}
\newcommand{\CR}{Prof. Cardin Riccardo}

% proponenti
\newcommand{\Proponente}{\textit{Imola Informatica}}
\newcommand{\ZD}{Zanetti Davide}
\newcommand{\CT}{Cardona Tommaso}

% qbteam
\newcommand{\AT}{Azzalin Tommaso}
\newcommand{\DF}{Drago Francesco}
\newcommand{\BR}{Baratin Riccardo}
\newcommand{\MC}{Mattei Christian}
\newcommand{\PF}{Perin Federico}
\newcommand{\CE}{Cisotto Emanuele}
\newcommand{\SE}{Salmaso Enrico}
\newcommand{\LD}{Lazzaro Davide}

% ruoli
\newcommand{\Responsabile}{Responsabile di Progetto}
\newcommand{\Amministratore}{Amministratore di Progetto}

% documenti

\newcommand{\SdF}{Studio di Fattibilità}
\newcommand{\SdFv}[1]{\textit{Studio di Fattibilità {#1}}}
\newcommand{\PdQ}{Piano di Qualifica}
\newcommand{\PdQv}[1]{\textit{Piano di Qualifica {#1}}}
\newcommand{\PdP}{Piano di Progetto}
\newcommand{\PdPv}[1]{\textit{Piano di Progetto {#1}}}
\newcommand{\NdP}{Norme di Progetto}
\newcommand{\NdPv}[1]{\textit{Norme di Progetto {#1}}}
\newcommand{\AdR}{Analisi dei Requisiti}
\newcommand{\AdRv}[1]{\textit{Analisi dei Requisiti {#1}}}
\newcommand{\Glossario}{Glossario}
\newcommand{\Glossariov}[1]{\textit{Glossario {#1}}}
\newcommand{\MM}{Manuale Manutentore}
\newcommand{\MMv}[1]{\textit{Manuale Manutentore {#1}}}
\newcommand{\MU}{Manuale Utente}
\newcommand{\MUv}[1]{\textit{Manuale Utente {#1}}}

% comandi generali
\newcommand{\glo}[1]{#1\ap{G}}

\setlength{\parindent}{-0.1em}


\begin{document}

\copertina{}
\newpage


\fancydoc{}

\section*{Registro delle modifiche}
{
\rowcolors{2}{grigetto}{white}
\renewcommand{\arraystretch}{1.5}
\centering
\begin{longtable}{C{2cm} C{2cm}  C{3cm}  C{3cm} C{4.5cm}}
\rowcolor{rossoep}
\textcolor{white}{\textbf{Versione}} & \textcolor{white}{\textbf{Data}} & \textcolor{white}{\textbf{Nominativo}} & \textcolor{white}{\textbf{Ruolo}} & \textcolor{white}{\textbf{Descrizione}}\\	
\endhead
1.2.0 & 2020-03-07 & \CE{} & Verificatore & Verifica del documento. \\

1.1.4 & 2020-02-16 & \SE{} & Amministratore & Aggiornati 4.3.2 e 4.3.3 VERIFICATO DA \LD{}\\

1.1.3 & 2020-02-15 & \SE{} & Amministratore & Aggiornato 4.2.2 VERIFICATO DA \BR{}\\

1.1.5 & 2020-02-16 & \SE{} & Amministratore & Aggiunto 2.2.5.5 VERIFICATO DA \BR{}. \\

1.1.4 & 2020-02-15 & \SE{} & Amministratore & Aggiornato 4.2.2 VERIFICATO DA \BR{}. \\

1.1.3 & 2020-02-14 & \SE{} & Amministratore & Aggiunta 2.2.6 VERIFICATO DA \LD{}. \\

1.1.2 & 2020-02-12 & \SE{} & Amministratore & Aggiornamento 4.2.4 VERIFICATO DA \LD{}. \\ 

1.1.1 & 2020-02-12 & \BR{} & Amministratore & Aggiornamento 3.1 VERIFICATO DA \LD{}. \\ 

1.1.0 & 2020-02-12 & \LD{} & Verificatore & Verifica VERIFICATO DA \LD{}.  \\ 

1.0.3 & 2020-02-12 & \BR{} & Amministratore & Aggiunto e verificato 3.4.2 VERIFICATO DA \LD{}. \\ 

1.0.2 & 2020-02-12 & \SE{} & Amministratore & Aggiunte e verificate metriche SFIN e SFOUT VERIFICATO DA \LD{}. \\ 

1.0.1 & 2020-02-12 & \SE{} & Amministratore & Modificati e verificati i paragrafi 2.2.4.1 e 2.2.4.2 VERIFICATO DA \LD{}. \\ 

1.0.0 & 2020-01-13 & \AT{} & Amministratore & Approvazione per il rilascio.  \\

0.2.0 & 2020-01-13 & \PF{}, \CE{} & Verificatori & Verifica documento.  \\ 

0.1.9 & 2020-01-13 & \CE{} & Amministratore & Aggiunta del template dei digrammi UML dei casi d'uso. \\

0.1.8 & 2020-01-13 & \BR{} & Amministratore & Modifica dei casi d'uso d'errore, test di sistema. \\

0.1.7 & 2020-01-13 & \AT{} & Amministratore & Revisione Introduzione, Processo di fornitura, sviluppo, attività di codifica e di progettazione, processi organizzativi, documentazione. \\

0.1.6 & 2020-01-12 & \MC{} & Amministratore & Revisione e modifica strutturale dei capitoli del documento. \\

0.1.5 & 2020-01-12 & \AT{} & Amministratore & Modifica Processi Primari. \\

0.1.4 & 2020-01-11 & \MC{} & Amministratore & Stesura capitolo gestione della qualità. \\

0.1.3 & 2020-01-10 & \MC{} & Amministratore & Revisione documentazione nei Processi di supporto. \\

0.1.2 & 2020-01-06 & \AT{} & Amministratore & Modifica del processo di verifica, validazione, piano di qualifica. \\

0.1.1 & 2020-01-06 & \AT{} & Amministratore & Modifica del processo di verifica. \\

0.1.0 & 2019-12-23 & \PF{}, \CE{} & Verificatori & Verifica del documento. \\

0.0.11 & 2019-12-22 & \PF{} & Amministratore & Stesura delle sottosezioni introduzione e scopo della sezione processi di sviluppo. \\

0.0.10 & 2019-12-22 & \PF{}  & Amministratore & Stesura Sviluppo dei processi primari. \\

0.0.9 & 2019-12-21 & \PF{} & Amministratore & Stesura delle sottosezioni Gestione della qualità, Verifica e Validazione della sezione processi di supporto. \\

0.0.8 & 2019-12-20 & \MC{} & Amministratore & Modifica descrizione repository, Studio di fattibilità. \\

0.0.7 & 2019-12-19 & \SE{} & Amministratore & Revisione del documento fino ad ora redatto. \\

0.0.6 & 2019-12-19 & \CE{} & Amministratore & Modificata la sezione dei casi d’uso con le decisioni prese per la loro nomenclatura il 2019-12-18. \\

0.0.5 & 2019-12-15 & \SE{} & Amministratore & Aggiunta parte di progettazione. \\

0.0.4 & 2019-12-15 & \BR{}, \PF{}  & Amministratori & Aggiunta parte processi organizzativi, gestione delle risorse prodotte. \\

0.0.3 & 2019-12-15 & \MC{} & Amministratore & Stesura studio di fattibilità. \\

0.0.2 & 2019-12-14 & \CE{} & Amministratore & Aggiunta parte relativa all’Analisi dei requisiti. \\

0.0.1 & 2019-12-14 & \CE{} & Amministratore & Creato il documento. \\
		
\end{longtable}
}

\clearpage
\tableofcontents
\clearpage

\setcounter{table}{0}

\renewcommand{\listtablename}{Elenco tabelle}

\listoffigures
\clearpage
\listoftables
\clearpage


\section{Introduzione}
\subsection{Scopo del documento}
Lo scopo del documento è quello di descrivere in maniera dettagliata i requisiti e i casi d'uso che sono stati individuati durante lo studio del progetto Stalker.

\subsection{Scopo generale del prodotto}
L'obiettivo del prodotto \NomeProgetto{} di \Proponente{} è la creazione di un sistema software composto di un applicativo per cellulare e di un server, con cui interagire tramite un'interfaccia utente. La necessità nasce dal bisogno di adempiere alle normative vigenti in tema di sicurezza.
Le due componenti del sistema software, applicativo e server, devono soddisfare i seguenti obiettivi rispettivamente di:
\begin{itemize}
\item Tracciare e registrare i \glo{movimenti} di un utente in un \glo{luogo di tracciamento} di un'\glo{organizzazione}, siano essi autenticati da credenziali di un'\glo{organizzazione} oppure visitatori anonimi, il tutto nel rispetto della normativa sulla privacy;
\item Poter visionare gli accessi degli utenti autenticati e visionare il numero di visitatori anonimi all'interno di un luogo.
\end{itemize}

\subsection{Glossario}
Al fine di evitare ambiguità fra i termini, e per avere chiare fra tutti gli stakeholder le terminologie utilizzate per la realizzazione del presente documento, il gruppo \Gruppo{} ha redatto un documento denominato \Glossariov{1.0.0}.
In tale documento, sono presenti tutti i termini tecnici, ambigui, specifici del progetto e scelti dai membri del gruppo con le loro relative definizioni.
Un termine presente nel \Glossariov{1.0.0} e utilizzato in questo documento viene indicato con un apice \ap{G} alla fine della parola.

\subsection{Riferimenti}

\subsubsection{Normativi}
\begin{itemize}
\item \NdPv{1.0.0};
\item \textit{VE\_2019\_12\_13}.
\end{itemize}

\subsubsection{Informativi}
\begin{itemize}
\item \SdFv{1.0.0};
\item \textbf{Slide del capitolato C5 - Stalker}: \\ \url{https://www.math.unipd.it/~tullio/IS-1/2019/Progetto/C5.pdf}
\item \textbf{Guide to the Software Engineering Body of Knowledge};
\item \textbf{Software Engineering (10th edition) - Ian Sommerville}.
\end{itemize}

\section{Descrizione generale}
\subsection{Contesto d'uso del prodotto}
Il prodotto è orientato ai seguenti utenti: proprietari di \glo{organizzazioni} per la sezione della web-app, mentre l'app sarà orientata a visitatori e clienti di \glo{organizzazioni} pubbliche e dipendenti di \glo{organizzazioni} private.
Il prodotto servirà per tracciare gli utenti dell'applicazione al fine di rispettare le norme vigenti sulla sicurezza nei luoghi pubblici oppure per agevolare la gestione e la tracciabilità dei dipendenti dell'azienda.
Alcune funzionalità del prodotto, come la creazione di un'\glo{organizzazione} e la sua eliminazione non verranno eseguite dagli utenti a cui è rivolto lo stesso. Sarà compito degli amministratori del sistema \NomeProgetto{} occuparsene, che non rientrano tra gli attori del prodotto.

\subsection{Funzioni del prodotto}
Il prodotto garantirà le seguenti funzionalità:
\begin{itemize}
    \item \textbf{Amministratori:} gli amministratori dovranno essere in grado, attraverso la web-app, di gestire la propria \glo{organizzazione}, visualizzare gli accessi dei dipendenti e nominare altri amministratori per assisterli nella gestione e monitoraggio. \\
        L'amministratore avrà accesso alle seguenti funzionalità:
        \begin{itemize}
            \item \textbf{Modifica ai parametri dell'\glo{organizzazione}}: L'amministratore può ridefinire il nome, la descrizione, l'immagine e l'indirizzo dell'\glo{organizzazione} selezionata;
            \item \textbf{Modifica delle superfici geografiche di \glo{tracciamento} dell'\glo{organizzazione}}: Può modificare il perimetro di \glo{tracciamento} dell'\glo{organizzazione} e quello degli specifici \glo{luoghi}, inserendo un numero a piacere di coordinate per delimitarne la superficie di \glo{tracciamento} (manualmente o tramite Google Maps API);
            \item \textbf{Gestione degli amministratori}: È possibile nominare e/o eliminare amministratori e modificarne i privilegi;
            \item \textbf{Monitoraggio degli utenti tracciati}: L'amministratore può sapere, in tempo reale, quanti utenti si trovano all'interno dei vari \glo{luoghi} dell'\glo{organizzazione}, o dell'organizzazione in generale. Qualora l'\glo{organizzazione} monitorata fosse gestita con tracciamento riconosciuto, l'amministratore è anche in grado di sapere l'identità dei vari utenti tracciati;
            \item \textbf{Visualizzazione degli accessi effettuati}: L'amministratore ha la possibilità di visualizzare lo storico degli accessi degli utenti che hanno effettuato l'accesso all'\glo{organizzazione}, qualora quest'ultima fosse monitorata con tracciamento riconosciuto. Per ogni accesso di uno specifico utente viene mostrato: il timestamp di ingresso, quello di uscita e il suo tempo di permanenza presso l'organizzazione.
            \item \textbf{Estrapolazione di report tabellari riguardanti gli accessi dei dipendenti e gli accesi ai vari luoghi dell'\glo{organizzazione}}: L'amministratore può ricavare tabelle dei seguenti tipi:
            \begin{itemize}
                \item ore di entrata e uscita da un luogo per uno specifico utente;
                \item totale di ore spese in ogni luogo per uno specifico utente;
                \item il numero di dipendenti e il totale delle ore da loro trascorse in ogni luogo dell'\glo{organizzazione}.
            \end{itemize}
        \end{itemize}
    \item \textbf{Utenti:} gli utenti necessiteranno della possibilità, con l'applicazione, di registrarsi e autenticarsi nell'app, di venire tracciati nelle \glo{organizzazioni} e autenticarsi presso le \glo{organizzazioni} che lo richiedono. Agli utenti saranno fornite le seguenti funzionalità:
    \begin{itemize}
        \item \textbf{Funzionalità di registrazione e \glo{autenticazione}}: L'utente può registrarsi con delle nuove credenziali o, alternativamente, effettuare l'accesso con un account già registrato nel sistema. Qualora l'utente avesse smarrito la password, avrebbe comunque la possibilità di effettuarne il reset;
        \item \textbf{Possibilità di scaricare e aggiornare la lista delle \glo{organizzazioni}}: L'utente ha la possibilità di scaricare la lista delle organizzazioni, sia quelle con \glo{tracciamento} autenticato che quelle senza. Può inoltre effettuare l'aggiornamento della lista in maniera manuale, tramite un pulsante o temporizzata;
        \item \textbf{Venire tracciati nelle \glo{organizzazioni} desiderate}: L'utente verrà tracciato qualora effettuasse un \glo{movimento} all'interno dell'\glo{organizzazione};
        \item \textbf{Gestione delle \glo{organizzazioni} preferite}: L'utente può selezionare un'organizzazione e renderla preferita, così facendo essa sarà visualizzata nelle prime righe della lista delle organizzazioni;
        \item \textbf{Visualizzare gli accessi effettuati presso le varie \glo{organizzazioni} e i relativi luoghi}: L'utente ha a disposizione un registro degli accessi in cui sarà visualizzato l'orario di entrata e uscita da una determinata organizzazione, o luogo, e il tempo trascorso al suo interno;
        \item \textbf{Passare in tracciabilità anonima e non presso \glo{organizzazioni} private}: Un utente riconosciuto potrà decidere di passare all'anonimato, cioè di diventare un utente anonimo, selezionando l'apposita funzionalità.
    \end{itemize}
\end{itemize}
\subsection{Vincoli generali}
Per gli amministratori è sufficiente un browser (su di un computer con connessione ad internet); per gli utenti dell'applicazione un dispositivo con SO Android, una connessione a internet e/o un modulo GPS.

\section{Casi d'uso}
\subsection{Attori}
\subsubsection{Attori generici}
\begin{figure}[h]
  \centering
    \includegraphics[scale=0.8]{Sezioni/UseCase/Immagini/Generici.png}
    \caption{Gerarchia dei generici}
\end{figure}

\paragraph{Utente non autenticato}
Utente non ancora autenticato all'applicazione o all'interfaccia per amministratori che può avere o non avere le credenziali per autenticarsi.

\subsubsection{Utenti}
\begin{figure}[h]
  \centering
    \includegraphics[scale=0.8]{Sezioni/UseCase/Immagini/Utenti.png}
    \caption{Gerarchia degli utenti}
\end{figure}

\paragraph{Utente anonimo}
L'attore Utente autenticato che può venire tracciato all'interno di una \glo{organizzazione} senza fornire dettagli sulla propria identità.
\paragraph{Utente riconosciuto}
L'attore Utente autenticato che è attualmente tracciato all'interno di una precisa \glo{organizzazione} fornendo dettagli sulla propria identità.
L'utente si è precedentemente autenticato presso l'\glo{organizzazione} tramite LDAP.

\subsubsection{Amministratori}
\begin{figure}[h]
  \centering
    \includegraphics[scale=0.8]{Sezioni/UseCase/Immagini/Amministratori.png}
    \caption{Gerarchia degli amministratori}
\end{figure}

\paragraph{Amministratore proprietario}
L'attore è un Amministratore autenticatosi con il ruolo di proprietario dell'\glo{organizzazione}.
Si trova al gradino più alto della gerarchia degli amministratori e dispone delle seguenti funzioni:
\begin{itemize}
\item Gestire l'\glo{organizzazione}, ovvero modificarne i dati (come nome, descrizione, ecc.) e le superfici geografiche per il \glo{tracciamento} degli utenti;
\item Gestire gli amministratori, cioè la loro nomina, rimozione e modifica dei privilegi;
\item Monitorare gli accessi;
\item Eliminare l'\glo{organizzazione}.
\end{itemize}
Il proprietario dell'\glo{organizzazione} può nominare altri amministratori.
\paragraph{Amministratore gestore}
L'attore è un Amministratore autenticatosi con il ruolo di gestore. 
Si trova al secondo gradino della gerarchia degli amministratori e dispone delle seguenti funzioni:
\begin{itemize}
\item Gestire l'\glo{organizzazione};
\item Monitorare gli accessi.
\end{itemize}
\paragraph{Amministratore visualizzatore}
L'attore è un Amministratore autenticatosi con il ruolo di visualizzatore.
Si trova all'ultimo gradino della gerarchia degli amministratori e può compiere solo la seguente funzione:
\begin{itemize}
\item Monitorare gli accessi.
\end{itemize}

%Alternativa in caso esista una gerarchia a 3 livelli




\clearpage
\subsection{Elenco casi d'uso}
\setcounter{secnumdepth}{0}
\subsubsection{UCA 1 - Accesso all'applicazione}%kite level

\begin{figure}[h]
  \centering
    \includegraphics[scale=0.5]{Sezioni/UseCase/Immagini/UCA1.png}
  \caption{UCA 1 - Accesso all'applicazione}
\end{figure}

\begin{itemize}
\item \textbf{Attori primari:} Utente non autenticato
\item \textbf{Precondizione:} L'utente non è autenticato.
\item \textbf{Postcondizione:} L'utente viene autenticato all'interno del sistema tramite registrazione [UCA 1.2] oppure tramite \glo{autenticazione} [UCA 1.1].
\item \textbf{Scenario principale:} L'utente non identificato può scegliere se registrarsi [UCA 1.2] nel sistema oppure, se possiede già un account, accedere [UCA 1.1] all'applicazione. %cosa potrebbe fare l'utente con il UC, descrizione
\end{itemize}

\subsubsection{UCA 1.1 - Autenticazione con credenziali Stalker}%sea level

\begin{figure}[h]
  \centering
    \includegraphics[scale=0.5, center]{Sezioni/UseCase/Immagini/UCA1.1.png}
  \caption{UCA 1.1 - \glo{Autenticazione} con credenziali Stalker}
\end{figure}

\begin{itemize}
\item \textbf{Attori primari:} Utente non autenticato
\item \textbf{Precondizione:} L'utente non è autenticato.
\item \textbf{Postcondizione:} L'utente viene autenticato all'interno del sistema e accede all'applicazione.
\item \textbf{Scenario principale:} L'utente non autenticato inserisce l'indirizzo e-mail e la password per autenticarsi attraverso il modulo di \glo{login} presente.
\item \textbf{Scenario alternativo 1:} L'utente tenta di accedere con delle credenziali errate [UCA 8.1.4].
%\item \textbf{Scenario alternativo 2:} Se l'utente non dovesse ricordarsi la password ha la possibilità di selezionare la funzionalità password dimenticata [UCA 1.3] per reimpostarla.
\item \textbf{Flusso di eventi:}
  \begin{enumerate}
        \item Inserimento indirizzo e-mail [UCA 1.1.1];
        \item Inserimento password [UCA 1.1.2].
    \end{enumerate}
\item \textbf{Estensioni:}
	\begin{itemize}
		\item UCA 8.1.4 - Visualizzazione messaggio di errore credenziali errate;
	\end{itemize}
%\item \textbf{Inclusioni:}
\end{itemize}

\subsubsection{UCA 1.1.1 - Inserimento indirizzo e-mail}%fish level
\begin{itemize}
\item \textbf{Attori primari:} Utente non autenticato
\item \textbf{Precondizione:} L'utente non è autenticato.
\item \textbf{Postcondizione:} L'utente ha inserito il proprio indirizzo e-mail.
\item \textbf{Scenario principale:} Inserimento indirizzo e-mail per l'autenticazione.
\end{itemize}

\subsubsection{UCA 1.1.2 - Inserimento password}%fish level
\begin{itemize}
\item \textbf{Attori primari:} Utente non autenticato
\item \textbf{Precondizione:} L'utente non è autenticato.
\item \textbf{Postcondizione:} L'utente ha inserito la propria password.
\item \textbf{Scenario principale:} Inserimento password per l'autenticazione.
\end{itemize}

\subsubsection{UCA 1.2 - Registrazione di account in Stalker}%sea level

\begin{figure}[h]
  \centering
    \includegraphics[scale=0.4, center]{Sezioni/UseCase/Immagini/UCA1.2.png}
  \caption{UCA 1.2 -  Registrazione di account in Stalker}
\end{figure}

\begin{itemize}
\item \textbf{Attori primari:} Utente non autenticato
\item \textbf{Precondizione:} L'utente non è ancora registrato nel sistema.
\item \textbf{Postcondizione:} L'utente si è registrato e accede al sistema.
\item \textbf{Scenario principale:} L'utente non registrato compila il modulo di registrazione al fine di poter accedere al sistema.
\item \textbf{Flusso di eventi:}
  \begin{enumerate}
        \item Inserimento indirizzo e-mail [UCA 1.2.1];
        \item Inserimento password [UCA 1.2.2];
        \item Inserimento conferma password [UCA 1.2.3];
        \item Accettazione delle condizioni generali d'uso [UCA 1.2.4];
    \end{enumerate}
\end{itemize}

\subsubsection{UCA 1.2.1 - Inserimento indirizzo e-mail}%fish level

\begin{itemize}
\item \textbf{Attori primari:} Utente non autenticato
\item \textbf{Precondizione:} L'utente non è registrato.
\item \textbf{Postcondizione:} L'utente ha inserito il proprio indirizzo e-mail, di cui è stata verificata la non esistenza in associazione ad altri account nel sistema.
\item \textbf{Scenario principale:} Inserimento indirizzo e-mail per la registrazione.
\item \textbf{Estensioni:}
	\begin{itemize}
		\item UCA 8.1.1 - Visualizzazione messaggio di errore in caso di account con l'e-mail inserita già presente.
	\end{itemize}
\end{itemize}

\subsubsection{UCA 1.2.2 - Inserimento password}%fish level
\begin{itemize}
\item \textbf{Attori primari:} Utente non autenticato
\item \textbf{Precondizione:} L'utente non è registrato.
\item \textbf{Postcondizione:} L'utente ha inserito una password che rispetta i criteri di complessità definiti dal sistema.
\item \textbf{Scenario principale:} Inserimento password per la registrazione.
\item \textbf{Estensioni:}
	\begin{itemize}
		\item UCA 8.1.2 - Visualizzazione messaggio di errore in caso di password troppo debole.
	\end{itemize}
\end{itemize}

\subsubsection{UCA 1.2.3 - Inserimento conferma password}%fish level
\begin{itemize}
\item \textbf{Attori primari:} Utente non autenticato
\item \textbf{Precondizione:} L'utente non è registrato.
\item \textbf{Postcondizione:} L'utente ha confermato la password inserendola nuovamente (è uguale a quella inserita durante l'inserimento della password [UCA 1.2.2]).
\item \textbf{Scenario principale:} Inserimento conferma password.
\item \textbf{Estensioni:}
	\begin{itemize}
		\item UCA 8.1.3 - Visualizzazione messaggio di errore in caso di password e conferma password diverse.
	\end{itemize}
\end{itemize}

\subsubsection{UCA 1.2.4 - Accettazione delle condizioni generali d'uso}%fish level
\begin{itemize}
\item \textbf{Attori primari:} Utente non autenticato
\item \textbf{Precondizione:} L'utente non è registrato.
\item \textbf{Postcondizione:} L'utente ha accettato le condizioni generali sull'uso.
\item \textbf{Scenario principale:} Accettazione delle condizioni generali d'uso.
\item \textbf{Scenario alternativo:} Se l'utente dovesse rifiutare le condizioni generali sull'uso allora la registrazione verrebbe interrotta e l'applicazione chiusa.
\end{itemize}

\begin{figure}[h]
	\centering
	\includegraphics[scale=0.6]{Sezioni/UseCase/Immagini/UCA1.3.png}
	\caption{UCA 1.3 - Password dimenticata}
\end{figure}

\subsubsection{UCA 1.3 - Password dimenticata}%fish level
\begin{itemize}
\item \textbf{Attori primari:} Utente non autenticato
\item \textbf{Precondizione:}  L'utente non è autenticato.
\item \textbf{Postcondizione:} L'utente ha fatto richiesta per avviare la procedura di password dimenticata.
\item \textbf{Scenario principale:} La password è stata dimenticata.
\item \textbf{Flusso di eventi:}
  \begin{enumerate}
        \item Inserimento dell'indirizzo e-mail a cui inviare la e-mail per il cambio password [UCA 1.3.1];
        \item Viene inviata la e-mail per il cambio password [UCA 1.3.2];
        \item Inserimento della nuova password [UCA 1.3.3];
        \item Inserimento della conferma della nuova password [UCA 1.3.4].
    \end{enumerate}
\end{itemize}

\subsubsection{UCA 1.3.1 - Inserimento e-mail}
\begin{itemize}
\item \textbf{Attori primari:} Utente non autenticato
\item \textbf{Precondizione:} L'utente non è autenticato.
\item \textbf{Postcondizione:} L'utente ha inserito l'e-mail.
\item \textbf{Scenario principale:} Inserimento indirizzo e-mail per cambiare password.
\end{itemize}


\subsubsection{UCA 1.3.3 - Nuova password}
\begin{itemize}
\item \textbf{Attori primari:} Utente non autenticato
\item \textbf{Precondizione:}  L'utente non è autenticato.
\item \textbf{Postcondizione:} L'utente inserisce la nuova password.
\item \textbf{Scenario principale:} Inserimento della nuova password.
\end{itemize}

\subsubsection{UCA 1.3.4 - Conferma nuova password}
\begin{itemize}
\item \textbf{Attori primari:} Utente non autenticato
\item \textbf{Precondizione:} L'utente non è autenticato.
\item \textbf{Postcondizione:} L'utente reinserisce la nuova password per confermarla.
\item \textbf{Scenario principale:} Inserimento della conferma della nuova password.
\end{itemize}

% UC... specificare App o Server -> UCA / UCS

%\centering\includegraphics[scale=0.8]{sezioni/UseCase/Immagini/Panoramica.png}
\section{UCS ?}%kite level
\begin{itemize}
    \item \textbf{Nome:} Modifica parametri dell'organizzazione\\
    \item \textbf{Attori primari:} Amministratore gestore\\
    \item \textbf{Precondizione:} L’amministratore dispone di almeno un'organizzazione.\\
    \item \textbf{Postcondizione:} L’amministratore ha modificato i parametri desiderati dell'organizzazione e le modifiche ono state salvate nel sistema.\\
    \item \textbf{Scenario principale:} L'amministratore deve scegliere l'organizzazione che vuole modificare, selezionare la funzionalità di modifica dell'organizzazione e quindi procedere con il cambiamento dei dati.'\\
    \item \textbf{Inclusioni:} UCS Selezione organizzazione
    \item \textbf{Flusso di eventi:}
    \begin{enumerate}
        \item UCS Selezione organizzazione;
        \item L'amministratore ha la possibilità di modificare i campi delle informazioni dell’organizzazione (UCS )
        \item L’amministratore ha la possibilità di modificare il luogo di tracciamento dell’organizzazione [UCS 3.5]
    \end{enumerate}
\end{itemize}
\clearpage

\setcounter{secnumdepth}{3}
\section{Requisiti}
\renewcommand{\o}{Obbligatorio}
\renewcommand{\d}{Desiderabile}
\newcommand{\op}{Opzionale}

Rappresentano dei requisiti che deve soddisfare il prodotto che si vuole realizzare.\\
I requisiti saranno organizzati in forma tabellare.\\
La tabella avrà le seguenti tre colonne:
\begin{itemize}
	\item Codice identificativo
	\item Classificazione
	\item Descrizione
	\item Fonti
\end{itemize}
\paragraph{Codice identificativo dei requisiti}\mbox{}\\
Ogni requisito sarà strutturato come segue:
\begin{itemize}
	\item Identificativo: \textbf{R[Importanza][Tipologia][Codice]}\\
Dove:
\begin{itemize}
		\item \textbf{Importanza:}
		\begin{itemize}
			\item \textbf{1}: Requisito obbligatorio, ovvero irrinunciabile per almeno uno degli stakeholder
			\item \textbf{2}: Requisito desiderabile, ovvero non strettamente necessario ma che porta valore aggiunto riconoscibile
			\item \textbf{3}: Requisito opzionale, ovvero relativamente utile oppure contrattabile più avanti nel progetto
		\end{itemize}
		\item \textbf{Tipologia:}
		\begin{itemize}
			\item \textbf{F}: Funzionale, definisce una funzione di un sistema di uno o più dei suoi componenti
			\item \textbf{Q}: Qualitativo, definisce un requisito per garantire la qualità per un certo aspetto del progetto
			\item \textbf{P}: Prestazionale, definisce un requisito che garantisce efficienza prestazionale nel prodotto
			\item \textbf{V}: Vincolo, definisce un requisito volto a far rispettare un dato vincolo
		\end{itemize}
		\item \textbf{Codice:}\\
		Composto da:
		\begin{itemize}
			\item A*\ap{I} se il requisito proviene da un caso d'uso dell'applicazione, dove *\ap{I} sarà composto da: il numero del caso d'uso (lato app) kite-level di provenienza, un punto e il numero progressivo univoco
			\item S*\ap{II} se il requisito proviene da un caso d'uso del server, dove *\ap{II} sarà composto da: il numero del caso d'uso (lato server) kite-level di provenienza, un punto e il numero progressivo univoco
			\item C*\ap{III} se il requisito proviene dal capitolato, dove *\ap{III} sarà il numero progressivo univoco
			\item V*\ap{IV} se il requisito proviene da un verbale, dove *\ap{IV} inizierà con il numero che indica il verbale da cui proviene il requisito; il numero si calcola a partire da 1, che sarà associato al primo verbale redatto (in ordine temporale). Infine vi sarà un punto seguito dal numero progressivo
			\item I*\ap{V} se il requisito proviene da una decisione presa internamente al gruppo, dove *\ap{V} sarà un numero progressivo
		\end{itemize}
\end{itemize}
	\item \textbf{Descrizione}: Descrizione sintetica ma al contempo esaustiva del requisito.
	\item \textbf{Classificazione}: Informazione ridondante, sotto forma testuale, dell’importanza del requisito. Facilita la lettura dei requisiti.
	\item \textbf{Fonti}: Definisce da dove deriva il requisito. I requisiti vengono raccolti da una o più fonti tra quelle citate di seguito:
	\begin{itemize}
		\item \textbf{Capitolato}: Requisito individuato dalla lettura e/o analisi del capitolato
		\item \textbf{Interno}: Requisito che gli analisti hanno ritenuto opportuno aggiungere
		\item \textbf{Caso d’uso}: Il requisito è derivato da uno o più casi d’uso. Riportare anche il/i codice/i identificativo/i del/i caso/i d’uso.
		\item \textbf{Verbale}: Requisito derivato in seguito ad una richiesta di chiarimento con il committente. Riportare il nome del documento del verbale da cui deriva il requisito in questione.		
	\end{itemize}
\end{itemize}

\subsection{Requisiti funzionali}
{
\rowcolors{2}{grigetto}{white}
\renewcommand{\arraystretch}{1.5}
\centering
\begin{longtable}{ c C{6.5cm} c c}
\caption{Tabella dei Requisiti funzionali}\\
\rowcolor{darkblue}
\textcolor{white}{\textbf{Identificativo}} & \textcolor{white}{\textbf{Descrizione}} & \textcolor{white}{\textbf{Classificazione}} & \textcolor{white}{\textbf{Fonti}}\\	
\endfirsthead
\rowcolor{darkblue}
\textcolor{white}{\textbf{Identificativo}} & \textcolor{white}{\textbf{Descrizione}} & \textcolor{white}{\textbf{Classificazione}} & \textcolor{white}{\textbf{Fonti}}\\
\endhead

R1FI1 & Un utente non autenticato non può effettuare alcuna azione a meno di autenticazione e registrazione. & \o & Interno\\

R1FA1.1 & L'autenticazione da parte di un utente non autenticato necessita di e-mail. & \o & UCA 1.1.1 Interno\\

R1FA1.2 & L'autenticazione da parte di un utente non autenticato necessita di password. & \o & UCA 1.1.2 Interno\\

R1FA8.1 & L'autenticazione viene negata qualora l'utente non autenticato tenti di autenticarsi con delle credenziali errate. Viene inoltre visualizzato un messaggio d'errore. & \o & UCA 8.1.4 Interno\\

R1FA1.3 & La registrazione da parte di un utente non autenticato necessita di e-mail. & \o & UCA 1.2.1 Interno\\

R1FA8.2 & Il processo di registrazione dell'utente non autenticato viene negato qualora l'e-mail inserita fosse già registrata nel sistema. Viene visualizzato inoltre un messaggio d'errore. & \o & UCA 8.1.1 Interno\\

R1FA1.4 & La registrazione da parte di un utente non autenticato necessita di password. & \o & UCA 1.2.2 Interno\\

R1FA1.5 & La registrazione da parte di un utente non autenticato necessita di conferma della password. & \o & UCA 1.2.3 Interno\\

R1FA1.6 & Durante la registrazione viene chiesto all'utente non autenticato di accettare le condizioni generali d'uso. & \o & UCA 1.2.4 Interno\\

R1FA1.7 & Qualora l'utente non autenticato dovesse rifiutare le condizioni generali d'uso allora la registrazione non può essere terminata correttamente.  & \o & UCA 1.2.4 Interno\\

R2FA1.8 & L'utente non autenticato deve essere in grado di effettuare il reset della password qualora se la fosse dimenticata. & \d & UCA 1.3 Interno\\

R2FA1.9 & Il reset della password da parte dell'utente non autenticato richiede l'e-mail del proprio account. & \d & UCA 1.3.1 Interno\\

R2FA1.11 & Il reset della password da parte dell'utente non autenticato richiede la nuova password. & \d & UCA 1.3.3 Interno\\

R2FA1.12 & Il reset della password da parte dell'utente non autenticato richiede la conferma della nuova password. & \d & UCA 1.3.4 Interno\\

R1FA8.3 & Il processo di autenticazione viene negato qualora la password inserita non sia abbastanza sicura. Viene visualizzato inoltre un messaggio d'errore. & \o & UCA 8.1.2 Interno\\

R1FA8.4 & Il processo di registrazione viene negato se password e conferma password inserita non combaciano. Viene visualizzato inoltre un messaggio d'errore. & \o & UCA 8.1.3 Interno\\


%PERIN

R1FA2.1 & L'utente anonimo e riconosciuto deve essere in grado di effettuare il logout. & \o & UCA 2 Interno\\

R1FA3.1 & L'utente anonimo può gestire la propria lista delle organizzazioni. & \o & UCA 3 Interno\\

R1FA3.2 & L'utente anonimo deve poter essere in grado di scaricare la lista di tutte le organizzazioni. & \o & UCA 3.1 Capitolato \\

R1FA8.5 & Qualora fallisca lo scaricamento della lista delle organizzazioni deve venire visualizzato un messaggio d'errore che lo informa di tale evento. & \o & UCA 8.3.1 Interno \\

R1FA3.3 & L’utente anonimo deve poter essere in grado di gestire la propria lista delle organizzazioni preferite. & \o & UCA 3.2 Interno \\

R1FA3.4 & L’utente anonimo può inserire una organizzazione presente nella lista delle organizzazioni, nella propria lista delle organizzazioni preferite. & \o & UCA 3.2.1 Interno \\

R1FA3.5 & Qualora l’utente anonimo inserisca un'organizzazione nella propria lista delle organizzazioni preferite che richiede autenticazione con credenziali LDAP, deve autenticarsi con credenziali LDAP. & \o & UCA 3.2.2 Capitolato\\

R1FA3.6 & L’utente anonimo può rimuovere una organizzazione presente nella propria lista delle organizzazioni preferite. & \o & UCA 3.2.3 Interno \\

R1FA8.6 & Qualora non sia memorizzata nessuna lista delle organizzazioni nel dispositivo, viene informato l’utente di questo fatto. & \o & UCA 8.3.2 Interno \\

R1FA3.7 & L’utente anonimo ha la possibilità di aggiornare la lista delle organizzazioni. & \o & UCA 3.3 Interno \\

R1FA3.8 & L’utente anonimo può aggiornare la lista delle organizzazioni tramite refresh manuale. & \o & UCA 3.3.1 Interno \\

R1FA3.9 & L’utente  anonimo può aggiornare la lista delle organizzazioni tramite temporizzazione. & \o & UCA 3.3.2 Interno \\

R1FA3.10 & L’utente anonimo può visualizzare la lista delle organizzazioni. & \o & UCA 3.4 Interno \\

R2FA3.11 & L’utente anonimo ha la possibilità di visualizzare la lista delle organizzazioni ordinate alfabeticamente, dalla A alla Z. & \d & UCA 3.4.1 Interno \\

R2FA3.12 & L’utente anonimo ha la possibilità di visualizzare la lista delle organizzazioni ordinate secondo politica \glo{FIFO}. & \d & UCA 3.4.2 Interno \\

R3FA3.13 & L’utente anonimo può ricercare organizzazioni presenti nella lista delle organizzazioni che permettono solamente il tracciamento anonimo. & \op & UCA 3.5.4 Interno \\

R3FA3.14 & L’utente anonimo può ricercare organizzazioni presenti nella lista delle organizzazioni che permettono il \glo{tracciamento autenticato}. & \op & UCA 3.5.5 Interno \\

R1FA3.15 & L’utente anonimo può effettuare ricerche personalizzate per cercare le organizzazioni presenti nella lista delle organizzazioni. & \o & UCA 3.5 Interno\\

R2FA3.16 & L’utente anonimo può ricercare organizzazioni presenti nella lista delle organizzazioni appartenenti alla nazione indicata dall’utente. & \d & UCA 3.5.1 Interno \\

R1FA3.17 & L’utente anonimo può ricercare organizzazioni presenti nella lista delle organizzazioni che hanno nel nome una sotto-stringa scelta dall'utente. & \o & UCA 3.5.2 Interno \\

R2FA3.18 & L’utente anonimo può ricercare organizzazioni presenti nella lista delle organizzazioni appartenenti alla città indicata dall’utente. & \d & UCA 3.5.3 Interno \\

R1FA4.1 & L’utente riconosciuto deve poter inserire la modalità di tracciamento che preferisce. & \o & UCA 4 Capitolato \\

R1FA4.2 & L’utente riconosciuto può selezionare la modalità di tracciamento anonimo. & \o & UCA 4.1 Capitolato \\

R1FA4.3 & L’utente riconosciuto può selezionare la modalità di \glo{tracciamento autenticato}. & \o & UCA 4.2 Capitolato \\

R2FA5.1 & L’utente anonimo ha la possibilità di visualizzare il proprio storico degli accessi. & \d & UCA 5 Capitolato \\

R2FA5.2 & L’utente anonimo ha la possibilità di visualizzare il proprio storico degli accessi presso una organizzazione. & \d & UCA 5.1 Capitolato \\

R2FA5.3 & L'utente anonimo nella visualizzazione del proprio storico degli accessi nell'organizzazione visualizza la data per ogni accesso di quando è stato fatto. & \d & UCA 5.1.4 \\

R2FA5.4 & L'utente anonimo nella visualizzazione del proprio storico degli accessi nell'organizzazione visualizza il luogo per ogni accesso di quando è stato fatto. & \d & UCA 5.1.4 \\

R2FA5.5 & L'utente anonimo nella visualizzazione del proprio storico degli accessi nell'organizzazione visualizza il tempo trascorso per ogni accesso. & \d & UCA 5.1.4 \\

R2FA5.6 & L’utente anonimo ha la possibilità di visualizzare il proprio storico degli accessi presso un luogo dell’organizzazione. & \d & UCA 5.2 Capitolato\\

R2FA5.7 & L'utente anonimo nella visualizzazione del proprio storico degli accessi nel luogo dell'organizzazione visualizza la data per ogni accesso di quando è stato fatto. & \d &  UCA 5.2.4 \\

R2FA5.8 & L'utente anonimo nella visualizzazione del proprio storico degli accessi nel luogo dell'organizzazione visualizza il luogo per ogni accesso di quando è stato fatto. & \d &  UCA 5.2.4 \\

R2FA5.9 & L'utente anonimo nella visualizzazione del proprio storico degli accessi nel luogo dell'organizzazione visualizza il tempo trascorso per ogni accesso. & \d &  UCA 5.2.4 \\

R2FA5.10 & L’utente anonimo può visualizzare la propria lista degli accessi in una organizzazione ordinata per data decrescente. & \d & UCA 5.4.1 \\

R2FA5.11 & L’utente anonimo può visualizzare la propria lista degli accessi in una organizzazione ordinata per data crescente. & \d & UCA 5.4.2 \\

R3FA5.12 & L’utente anonimo può effettuare una ricerca degli accessi presso un'organizzazione in un giorno specifico. & \op & UCA 5.5 \\

R2FA5.13 & L’utente anonimo può visualizzare la propria lista degli accessi presso un luogo dell’organizzazione ordinata per data decrescente. & \d & UCA 5.4.1 \\

R2FA5.14 & L’utente anonimo può visualizzare la propria lista degli accessi presso un luogo dell’organizzazione ordinata per data crescente. & \d & UCA 5.4.2 \\

R3FA5.15 & L’utente anonimo può effettuare una ricerca degli accessi presso un luogo dell’organizzazione in un giorno specifico. & \op & UCA 5.5 \\

R2FA5.16 & L’utente anonimo se si trova all’interno dell’organizzazione ha la possibilità di visualizzare il tempo passato all’interno dall'ultimo ingresso effettuato. & \d & UCA 5.1.5 Capitolato \\

R2FA5.17 & L’utente anonimo se si trova all’interno dell’luogo dell’organizzazione ha la possibilità di visualizzare il tempo passato all’interno dall'ultimo ingresso effettuato. & \d & UCA 5.2.5 Capitolato \\

R2FA8.5 & Qualora non ci sono accessi effettuati presso l'organizzazione selezionata, l'utente anonimo deve essere informato di ciò. & \d & UCA 8.5.1 Interno \\

R2FA8.6 & Qualora non ci sono accessi effettuati presso il luogo selezionato, l'utente anonimo deve essere informato di ciò. & \d & UCA 8.5.2 Interno \\

R1FA6.1 & L’utente che effettua un movimento nell’organizzazione deve essere notificato della sua azione e il movimento deve essere memorizzato nel sistema. & \o & UCA 6.1 Capitolato \\

R1FA6.2 & L’utente che effettua un movimento nell’organizzazione, deve essere notificato della sua azione. & \o & UCA 6.1.1 \\

R1FA6.3 & L’utente che effettua un movimento in un luogo all'interno di una organizzazione, deve essere notificato della sua azione. & \o & UCA 6.2.1 \\

R1FA6.4 & L’utente che effettua un movimento in un luogo di un'organizzazione deve essere notificato della sua azione e il movimento deve essere memorizzato nel sistema. & \o & UCA 6.2 Capitolato \\

R1FA8.7 & Qualora non vengano memorizzate le informazioni necessarie per la registrazione del movimento effettuato dall’utente, deve essere notificato tale evento all’utente. & \o & UCA 8.6.1 \\

%Cisotto

R1FA7.1 & L'utente anonimo deve avere la possibilità di autenticarsi con le credenziali aziendali in un'organizzazione che richiede il tracciamento riconosciuto. & \o & UCA 7 Capitolato \\

R1FA8.8 & Qualora le credenziali LDAP aziendali inserite dall'utente non fossero riconosciute dal server aziendale associato viene mostrato un messaggio d'errore. & \o & UCA 8.7.1 Interno \\

R1FA7.2 & L'utente anonimo deve avere la possibilità di inserire il nome utente durante l'autenticazione con le credenziali LDAP aziendali. & \o & UCA 7.1.1 Interno \\

R1FA7.3 & L'utente anonimo deve avere la possibilità di inserire la password durante l'autenticazione con le credenziali LDAP aziendali. & \o & UCA 7.1.2 Interno \\

R1FI2 & Un utente non autenticato non può effettuare alcuna azione a meno di autenticazione. & \o & Interno \\

R1FS1.1 & L’autenticazione da parte di un amministratore necessita di e-mail. & \o & UCS 1.1.1 Interno\\

R1FS10.1 & L’autenticazione viene negata qualora l'amministratore tenti di autenticarsi con delle credenziali errate. & \o & UCS 10.1.1 Interno \\

R1FS10.2 & Qualora l'amministratore tenti di autenticarsi con le credenziali errate viene visualizzato un messaggio d’errore. & \o & UCS 10.1.1 Interno \\

R1FS1.2 & L’autenticazione da parte di un amministratore necessita di password. & \o & UCS 1.1.2 Interno\\

R2FS1.3 & L'utente non autenticato deve essere in grado di effettuare il reset della password qualora se la fosse dimenticata. & \d & UCS 1.3 Interno\\

R2FS1.4 & Il reset della password dell'utente non autenticato richiede l'e-mail. & \d & UCS 1.3.1 Interno \\

R2FS1.6 & Il reset della password dell'utente non autenticato richiede la nuova password. & \d & UCS 1.3.3 Interno \\

R2FS1.7 & Il reset della password dell'utente non autenticato richiede la conferma della password. & \d & UCS 1.3.4 Interno \\

R1FS2.1 & L'amministratore deve essere in grado di effettuare il logout. & \o & UCS 2 Interno\\

R1FC3 & L'amministratore visualizzatore deve poter visualizzare le organizzazioni disponibili. & \o & Capitolato\\

R1FI3 & Deve venire mostrato il nome dell'organizzazione durante la sua visualizzazione da parte di un amministratore. & \o & Interno\\

R2FI4 & Deve venire mostrata l'immagine dell'organizzazione durante la sua visualizzazione da parte di un amministratore. & \d & Interno\\

R1FS3.1 & L'amministratore visualizzatore deve poter selezionare un'organizzazione tra quelle da lui visualizzate. & \o & UCS 3 Interno\\

R1FI5 & Deve venire mostrato il nome dell'organizzazione selezionata durante la sua visualizzazione da parte di un amministratore. & \o & Interno\\

R2FI6 & Deve venire mostrata l'immagine dell'organizzazione selezionata durante la sua visualizzazione da parte di un amministratore. & \d & Interno\\

R2FI7 & Deve venire mostrata la descrizione dell'organizzazione selezionata durante la sua visualizzazione da parte di un amministratore. & \d & Interno\\

R1FI8 & Deve venire mostrato l'indirizzo dell'organizzazione selezionata durante la sua visualizzazione da parte di un amministratore. & \o & Interno\\


R1FS4.1 & L'amministratore gestore deve poter modificare il nome dell'organizzazione. & \o & UCS 4.1.1 Interno\\

R2FS4.2 & L'amministratore gestore deve poter modificare l'immagine dell'organizzazione. & \d & UCS 4.1.2 Interno\\

R2FS4.3 & L'amministratore gestore deve poter modificare la descrizione dell'organizzazione. & \d & UCS 4.1.3 Interno\\

R1FS4.4 & L'amministratore gestore deve poter modificare l'indirizzo dell'organizzazione. & \o & UCS 4.1.4 Interno\\

R1FS4.5 & L'amministratore gestore deve poter modificare l'indirizzo IP dell'organizzazione. & \o & UCS 4.1.5 Interno\\

R1FS10.3 & Se il nome dell'organizzazione inserito dall'amministratore non rispetta i vincoli imposti viene mostrato un messaggio d'errore. & \o & UCS 10.4.2\\

R1FS10.4 & Se il nome dell'organizzazione inserito dall'amministratore dovesse essere già presente nel sistema e associato ad un'altra organizzazione viene mostrato un messaggio d'errore. & \o & UCS 10.4.3\\

R2FS10.5 & Se l'immagine dell'organizzazione selezionata dall'amministratore non rispetta i vincoli imposti viene mostrato un messaggio d'errore. & \d & UCS 10.4.4\\

R2FS10.6 & Se la descrizione dell'organizzazione inserita dall'amministratore non rispetta i vincoli imposti viene mostrato un messaggio d'errore. & \d & UCS 10.4.5\\

R1FS10.7 & Se l'indirizzo dell'organizzazione inserito dall'amministratore non rispetta i vincoli imposti viene mostrato un messaggio d'errore. & \o & UCS 10.4.6\\

R1FS10.8 & Se l'indirizzo URL non è valido viene mostrato un messaggio d'errore. & \o & UCS 10.4.7\\

R1FS4.6 & L'amministratore proprietario deve avere la possibilità di inviare la richiesta di eliminazione per un'organizzazione. & \o & UCS 4.2 Capitolato\\

R3FS4.7 & L'amministratore proprietario deve poter inserire una motivazione per la richiesta di eliminazione dell'organizzazione. & \op & UCS 4.2.1 Interno \\

R1FS4.8 & L'amministratore gestore deve poter annullare le modifiche che sta apportando. & \o & UCS 4.3 Interno\\



% Drago

R1FS5.1 & L'amministratore gestore deve poter modificare il perimetro di \glo{tracciamento} dell'\glo{organizzazione}. & \o & UCS 5.1 Capitolato\\

R1FS10.9 & La modifica del perimetro dell'organizzazione viene negata qualora l'amministratore selezioni un area che non rispetta i vincoli imposti. Viene visualizzato un messaggio di errore. & \o & UCS 10.5.1 Capitolato \\

R1FS5.2 & L'amministratore gestore deve essere in grado di creare dei nuovi luoghi di tracciamento nell'organizzazione. & \o & UCS 5.2 Capitolato\\

R1FS5.3 & L'amministratore gestore deve essere in grado di modificare i luoghi di tracciamento dell'organizzazione. & \o & UCS 5.3 Capitolato\\

R1FS10.10 & La creazione di nuovi luoghi e la modifica dell'area di tracciamento di essi vengono negati qualora l'amministratore selezioni un area che fuoriesce dal perimetro. Viene visualizzato un messaggio di errore. & \o & UCS 10.5.2 Capitolato \\

R1FS5.4 & L'amministratore gestore deve essere in grado di eliminare i luoghi di tracciamento dell'organizzazione. & \o & UCS 5.4 Capitolato\\

R1FS5.5 & L'amministratore gestore deve essere in grado di selezionare un'area geografica per il tracciamento del luogo scelto. & \o & UCS 5.5 Capitolato\\

R1FS5.6 & L'amministratore gestore può scegliere l'area di tracciamento tramite l'inserimento delle coordinate geografiche. & \o & UCS 5.5.1 Capitolato\\

R2FS5.7 & L'amministratore gestore può scegliere l'area di tracciamento tramite Google Maps API. & \d & UCS 5.5.2 Interno\\

R1FS5.9 & L'amministratore deve poter inserire un nome per il nuovo luogo di tracciamento. & \o & UCS 5.2.1 Interno\\

R1FS6.1 & L'amministratore può monitorare l'organizzazione visualizzando il numero di utenti anonimi presenti nell'organizzazione. & \o & UCS 6 Capitolato\\

R1FS6.2 & L'amministratore visualizzatore può monitorare l'organizzazione visualizzando il numero di utenti anonimi presenti in un specifico luogo dell'organizzazione. & \o & UCS 6.1 Capitolato\\

R1FS6.3 & L'amministratore visualizzatore ha la possibilità di ritornare al monitoraggio dell'organizzazione in generale dal monitoraggio di un luogo specifico. & \o & UCS 6.1.1 Interno\\

R1FS7.1 & L'amministratore visualizzatore può monitorare gli accessi effettuati dagli utenti riconosciuti. & \o & UCS 7 Capitolato\\

R1FS7.2 & L'amministratore visualizzatore può monitorare gli accessi effettuati presso una organizzazione da un specifico utente riconosciuto visualizzandone il nome, il cognome, l'orario di accesso, di uscita e il tempo di permanenza. & \o & UCS 7.1.4 Capitolato\\

R2FS7.3 & L’amministratore visualizzatore può filtrare la lista degli accessi presso una organizzazione di un utente riconosciuto per data decrescente. & \d & UCS 7.4.1 Interno \\

R2FS7.4 & L’amministratore visualizzatore può filtrare la lista degli accessi presso una organizzazione  di un utente riconosciuto per data crescente. & \d & UCS 7.4.2 Interno \\

R2FS7.5 & L'amministratore visualizzatore può monitorare gli accessi presso una organizzazione filtrandoli in base a una data precisa. & \d & UCS 7.5 Interno\\

R1FS7.6 & L'amministratore visualizzatore può monitorare gli accessi effettuati presso un luogo all'interno di una organizzazione da un specifico utente riconosciuto visualizzandone il nome, il cognome e l'orario di accesso. & \o & UCS 7.2.4 Capitolato\\

R2FS7.7 & L’amministratore visualizzatore può filtrare la lista degli accessi presso un luogo di un'organizzazione di un utente riconosciuto per data decrescente. & \d & UCS 7.4.1 Interno \\

R2FS7.8 & L’amministratore visualizzatore può filtrare la lista degli accessi presso un luogo di un'organizzazione di un utente riconosciuto per data crescente. & \d & UCS 7.4.2 Interno \\

R2FS7.9 & L'amministratore visualizzatore può monitorare gli accessi presso un luogo di un'organizzazione filtrandoli in base a una data precisa. & \d & UCS 7.5 Interno\\

R2FS8.1 & L'amministratore visualizzatore può ricevere un report tabellare degli accessi ai luoghi dell'organizzazione. & \o & UCS 8 Capitolato\\

R2FS8.2 &  Tabella delle entrate e uscite degli utenti nei luoghi dell'organizzazione generabile dall'amministratore visualizzatore di un organizzazione a \glo{tracciamento autenticato}. & \o & UCS 8.1.1 Capitolato\\

R2FS8.3 & Tabella delle ore spese dagli utenti nei luoghi dell'organizzazione generabile dall'amministratore visualizzatore di un organizzazione a \glo{tracciamento autenticato}. & \o & UCS 8.1.2 Capitolato\\

R2FS8.4 & Tabella contenente il numero degli utenti e il totale delle ore passate da essi nei luoghi dell'organizzazione generabile dall'amministratore visualizzatore di un organizzazione a \glo{tracciamento autenticato} o anonimo. & \o & UCS 8.1.3 Capitolato\\

%Cisotto

R1FS9.1 & L'amministratore proprietario ha la possibilità di entrare nella sezione di gestione degli amministratori (per la nomina, eliminazione e modifica dei privilegi ad altri amministratori). & \o & UCS 9 Capitolato \\

R1FI9 & L'amministratore proprietario ha la possibilità di visualizzare gli amministratori da esso nominati una volta entrato nella gestione degli amministratori. & \o & Interno \\

R1FI10 & La visualizzazione di un amministratore deve mostrare la sua e-mail. & \o & Interno \\

R1FI11 & La visualizzazione di un amministratore deve mostrare i suoi privilegi. & \o & Interno \\

R1FS9.2 & L'amministratore proprietario ha la possibilità di nominare un nuovo amministratore. & \o & UCS 9.1 Capitolato\\

R1FS9.3 & L'amministratore proprietario deve poter inserire un'e-mail per il nuovo amministratore da nominare. & \o & UCS 9.1.1 Interno\\

R1FS10.11 & Viene mostrato un messaggio d'errore qualora l'e-mail del nuovo amministratore da nominare risulti già registrata nel sistema. & \o & UCS 10.9.1 Interno\\

R1FS10.12 & Viene mostrato un messaggio d'errore qualora la password del nuovo amministratore da nominare risulti troppo debole. & \o & UCS 10.9.2 Interno\\

R1FS10.13 & Viene mostrato un messaggio d'errore qualora la password non combaci con la conferma password del nuovo amministratore da nominare. & \o & UCS 10.9.3 Interno\\

R1FS9.4 & L'amministratore proprietario deve poter inserire una nuova password per il nuovo amministratore da nominare. & \o & UCS 9.1.2 Interno\\

R1FS9.5 & L'amministratore proprietario deve poter inserire la conferma della nuova password per il nuovo amministratore da nominare. & \o & UCS 9.1.3 Interno\\

R1FS9.6 & L'amministratore proprietario deve poter selezionare i privilegi per il nuovo amministratore da nominare. & \o & UCS 9.1.4 Interno\\

R1FS9.7 & L'amministratore proprietario ha la possibilità di eliminare un amministratore togliendogli i permessi di amministrazione dell'organizzazione. & \o & UCS 9.2 Capitolato\\

R1FS9.8 & L'amministratore proprietario ha la possibilità di inserire la e-mail dell'account amministratore da eliminare. & \o & UCS 9.2.1 Interno\\

R1FS10.14 & Viene mostrato un messaggio d'errore qualora l'e-mail dell'amministratore da eliminare inserita non risulti registrata nel sistema. & \o & UCS 10.9.4 Interno\\

R1FS9.9 & L'amministratore proprietario ha la possibilità di modificare i privilegi di un amministratore. & \o & UCS 9.3 Interno\\

R1FS9.10 & L'amministratore proprietario ha la possibilità di inserire la e-mail dell'account amministratore a cui desidera modificare i privilegi. & \o & UCS 9.3.1 Interno\\

R1FS10.15 & Viene mostrato un messaggio d'errore qualora l'e-mail dell'amministratore a cui si vuole modificare i privilegi non risulti registrata nel sistema. & \o & UCS 10.9.4 Interno\\

R1FS9.11 & L'amministratore proprietario ha la possibilità annullare le modifiche che sta apportando agli amministratori. & \o & UCS 9.4 Interno\\

R1FS9.12 & L'amministratore proprietario ha la possibilità di nominare un account già presente nel sistema come amministratore della propria organizzazione & \o & UCS 9.5 Interno\\

R1FS9.13 & L'amministratore proprietario deve poter inserire un'e-mail per il nuovo amministratore da nominare. & \o & UCS 9.5.1 Interno\\

R1FS9.14 & L'amministratore proprietario deve poter selezionare i privilegi per il nuovo amministratore da nominare. & \o & UCS 9.5.2 Interno\\

R1FS10.16 & L'amministratore gestore deve poter annullare le modifiche che stava apportando all'organizzazione. & \o & UCS 10.4.1\\

R1FS10.17 & L'amministratore gestore deve poter annullare la selezione della nuova area geografica per il tracciamento di un luogo o del perimetro dell'organizzazione. & \o & UCS 10.5.3 Interno\\

R2FS10.18 & Deve venir mostrato un errore qualora avvenisse un errore da parte del server sulla generazione del report tabellare. & \d & UCS 10.6.1 Interno\\

\end{longtable}
}
\clearpage
\subsection{Requisiti prestazionali}
{
\rowcolors{2}{grigetto}{white}
\renewcommand{\arraystretch}{1.5}
\centering
\begin{longtable}{ c C{4cm} c c}
\rowcolor{rossoep}
\textcolor{white}{\textbf{Identificativo}} & \textcolor{white}{\textbf{Descrizione}} & \textcolor{white}{\textbf{Classificazione}} & \textcolor{white}{\textbf{Fonti}}\\	

R1PC1 & L'applicazione deve consumare la minor quantità possibile di batteria & Obbligatorio & Capitolato\\

R1PC2 & Il rintracciamento deve essere il più preciso possibile & Obbligatorio & Capitolato\\


\subsection{Requisiti qualitativi}
{
\rowcolors{2}{grigetto}{white}
\renewcommand{\arraystretch}{1.5}
\centering
\begin{longtable}{ c C{8cm} c c}
\caption{Tabella dei Requisiti qualitativi}\\
\rowcolor{darkblue}
\rowcolor{darkblue}
\textcolor{white}{\textbf{Identificativo}} & \textcolor{white}{\textbf{Descrizione}} & \textcolor{white}{\textbf{Classificazione}} & \textcolor{white}{\textbf{Fonti}}\\	
\endfirsthead
\rowcolor{darkblue}
\textcolor{white}{\textbf{Identificativo}} & \textcolor{white}{\textbf{Descrizione}} & \textcolor{white}{\textbf{Classificazione}} & \textcolor{white}{\textbf{Fonti}}\\
\endhead

R1QI1 & Le norme e le metriche definite nei documenti \textit{NormeDiProgetto} e \textit{PianoDiQualifica} sono state rispettate. & Obbligatorio & Interno\\

R1QI2 & Deve essere prodotto un manuale utente. & Obbligatorio & Interno\\

R1QI3 & Deve essere prodotto un manuale amministratore. & Obbligatorio & Interno\\

R1QI4 & Il codice sorgente deve essere caricato in una repository su GitHub. & Obbligatorio & Interno\\

R1QI5 & La documentazione riguardante il software deve essere disponibile in lingua italiana. & Obbligatorio & Interno\\

R1QV1 & Si deve scegliere una licenza tra GNU\ap{G}, GPL\ap{G}, LGPL\ap{G} e MIT\ap{G}. & Obbligatorio & VE\_2019\_12\_16 \\

\end{longtable}
}
\renewcommand{\o}{Obbligatorio}
\renewcommand{\d}{Desiderabile}
\subsection{Requisiti di vincolo}
{
\rowcolors{2}{grigetto}{white}
\renewcommand{\arraystretch}{2}
\centering
\begin{longtable}{ c C{6.5cm} c c}
\caption{Tabella dei Requisiti di vincolo}\\
\rowcolor{darkblue}
\textcolor{white}{\textbf{Identificativo}} & \textcolor{white}{\textbf{Descrizione}} & \textcolor{white}{\textbf{Classificazione}} & \textcolor{white}{\textbf{Fonti}}\\	
\endfirsthead
\rowcolor{darkblue}
\textcolor{white}{\textbf{Identificativo}} & \textcolor{white}{\textbf{Descrizione}} & \textcolor{white}{\textbf{Classificazione}} & \textcolor{white}{\textbf{Fonti}}\\
\endhead

R1VC1.1 & Deve essere sviluppato un server back-end. & \o & Capitolato \\
R1VC1.2 & Il server deve essere correlato di una UI\ap{G} per la gestione delle funzioni implementate. & \o & Capitolato \\
R1VC1.3 & Deve essere sviluppata una applicazione mobile per sistemi operativi Android o iOS. & \o & Capitolato \\
R1VC2.1 & Le comunicazioni di tracciamento tra applicazione cellulare e server devono avvenire solo al momento d’ingresso ed uscita dai luoghi designati. & \o & Capitolato \\
R2VC3.1 & Possibile utilizzo di Java\ap{G} (versione 8 o superiore) per sviluppo server back-end. & \d & Capitolato \\
R2VC3.2 & Possibile utilizzo di Python\ap{G} per sviluppo server back-end. & \d & Capitolato \\
R2VC3.3 & Possibile utilizzo di Node.js\ap{G}per sviluppo server back-end. & \d & Capitolato \\
R2VC4.1 & Possibile utilizzo di protocolli asincroni\ap{G} per le comunicazioni app mobile-server. & Desiderato & Capitolato \\
R2VC5.1 & Possibile utilizzo del pattern di Publisher/Subscriber\ap{G}. & \d & Capitolato \\
R2VC6.1 & Possibile utilizzo dell’IAAS Kubernetes\ap{G} per la gestione della scalabilità orizzontale\ap{G}. & \d & Capitolato \\
R2VC6.2 & Possibile utilizzo di PAAS\ap{G} per la gestione della scalabilità orizzontale\ap{G}. & \d & Capitolato \\
R2VC6.3 & Possibile utilizzo di Openshift\ap{G} per la gestione della scalabilità orizzontale\ap{G}. & \d & Capitolato \\
R2VC6.4 & Possibile utilizzo di Rancher\ap{G} per la gestione della scalabilità orizzontale\ap{G}. & \d & Capitolato \\
R1VC7.1 & Il server deve esporre oltre a eventuali protocolli richiesti per l’interazione con il servizio, delle API REST\ap{G} necessarie per permettere l’utilizzo applicativo, come alternativa alle API REST\ap{G} è possibile utilizzare gRPC\ap{G}. & \o & Capitolato \\
R2VC8.1 & Utilizzo di tecnologie network-GPS\ap{G} per il tracciamento della posizione. & \d & Capitolato \\
R1VC8.2 & Si deve fornire un resoconto sulle scelte fatte e sui test effettuati relativi al tracciamento della posizione. & \o & Capitolato \\
R1VC9.1 & Si deve correlare a tutte le componenti applicative dei test di unità e d’integrazione. & \o & Capitolato \\
R1VC9.2 & Si deve testare tramite test end to end il sistema nella sua interezza. & \o & Capitolato \\
R1VC9.3 & Devono essere effettuati test di carico per testare il corretto funzionamento della scalabilità. & \o & Capitolato \\
R1VC9.4 & Si deve avere una copertura dei test maggiore o uguale al 80\%. & \o & Capitolato \\
R1VC9.5 & Tutti i test effettuati devono essere correlati da un report. & \o & Capitolato \\
R1VC10.1 & Deve essere scritta la documentazione relativa alle scelte implementative e progettuali con annessa motivazione. & \o & Capitolato \\
R1VC10.2 & Deve essere scritta la documentazione relativa ai problemi aperti e eventuali soluzioni proposte da esplorare. & \o & Capitolato \\
R2VC11.1 & Cifratura di tutte le comunicazioni fra App e Server. & \d & Capitolato  \\
R1VC12.1 & Si deve utilizzare il protocollo LDAP per autenticare i singoli dipendenti e gli amministratori di una organizzazione per poter effettuare il tracciamento autenticato della posizione. & \o & Capitolato \\	
R1VV1.1 & Deve essere garantita la privacy e quindi verificare che tutte le informazioni raccolte rispettino le GDPR\ap{G}. & \o & VE\_2019\_12\_16 \\
R1VV1.2 & Le credenziali per accedere alla applicazione e le credenziali per accedere e autenticarsi in una organizzazione che richiede tracciamento autenticato devono essere diverse. & \o & VE\_2019\_12\_16 \\
R1VV1.3 & I server LDAP delle organizzazioni che richiedono autenticazione devono essere esterni al sistema.  & \o & VE\_2019\_12\_16 \\
R1VV1.4 & Per gli utenti che non devono essere autenticati, si deve assegnare a loro un codice che ne certifichi la validità dell’ingresso o dell'uscita da un luogo in modo tale da non conoscere chi lo ha generato. & \o & VE\_2019\_12\_16 \\
\end{longtable}
}

\clearpage
\section{Tracciamento}
\subsection{Requisiti qualitativi}
{
\rowcolors{2}{grigetto}{white}
\renewcommand{\arraystretch}{1.5}
\centering
\begin{longtable}{ c C{4cm} c c}
\rowcolor{rossoep}
\textcolor{white}{\textbf{Fonte}} & \textcolor{white}{\textbf{Requisito}}\\	

Interno & R1FI1\\

Interno & R1FI2\\

\end{document}

\subsubsection{Progettazione}
\paragraph{Scopo}\mbox{}\\ \\
La progettazione, svolta dai Progettisti, ha lo scopo di soddisfare i requisiti stabiliti nel documento \AdR{} per trovare una soluzione accettabile per tutti gli stakeholder.
Per fare ciò, si cerca di seguire un approccio sintetico dove si pensa prima all’architettura del prodotto e poi al codice.
\paragraph{Descrizione}\mbox{}\\ \\
La progettazione consiste nei seguenti compiti:
\begin{itemize}
	\item Controllare la complessità del prodotto suddividendo il sistema in parti di complessità trattabile;
	\item Soddisfare i requisiti garantendo qualità;
	\item Definire un’architettura logica del prodotto che dovrà avere determinate caratteristiche;
	\item Avere una progettazione dettagliata con la consapevolezza di fermarsi quando la suddivisione porterà più svantaggi che benefici.
\end{itemize}

\paragraph{Architettura}\mbox{}\\ \\
I progettisti devono definire l’architettura logica del prodotto creando parti con specifiche chiare, coese e realizzabili con risorse sostenibili e mantenibili. L'architettura deve avere determinate caratteristiche per:
\begin{itemize}
	\item Soddisfare tutti requisiti degli stakeholder;
	\item Riuscire a gestire gli errori quando presenti;
	\item Garantire che venga eseguito il suo compito nel modo corretto;
	\item Cercare di ridurre i tempi di manutenzione;
	\item Avere componenti semplici, coesi, incapsulati e di basso accoppiamento tra di loro.
\end{itemize}

La realizzazione dell’architettura del prodotto è divisa in due parti:
\begin{itemize}
	\item Technology Baseline\ap{G};
	\item Product Baseline\ap{G}.
\end{itemize}

\subparagraph{Technology Baseline}\mbox{}\\ \\
La Technology Baseline deve dimostrare l’adeguatezza dell’architettura tramite un Proof of Concept (PoC) che rappresenta la baseline per lo sviluppo. 
La Technology baseline deve includere:
\begin{itemize}
	\item Le tecnologie;
	\item I framework;
	\item Le librerie utilizzate nel Proof of Concept;
	\item Diagrammi UML con le seguenti rappresentazioni:
	\begin{itemize}
		\item Diagrammi dei casi d'uso; 
		\item Diagrammi delle classi; 
		\item Diagrammi dei package;
		\item Diagrammi di sequenza; 
		\item Diagrammi di attività.
	\end{itemize}
\end{itemize}

\subparagraph{Product Baseline}\mbox{}\\ \\
La Product Baseline illustrerà la baseline architetturale del prodotto, in coerenza con la Technology Baseline.
Essa deve includere un allegato che contenga:
\begin{itemize}
	\item Diagrammi delle classi;
	\item Diagrammi di sequenza;
	\item Contestualizzazione dei design pattern adottati.	
\end{itemize}

\subsection{Codifica}
\subsubsection{Scopo}
Lo scopo della codifica consiste nell'implementare le specifiche del prodotto definite nelle attività precedenti.

\subsection*{Note sulla sezione di Codifica}
Questa sezione viene redatta durante la fase iniziale del progetto e il suo contenuto è limitato alle disponibilità di conoscenza attuali.
Verrà ampliata in seguito per l'attività di \textbf{Codifica}.

\subsubsection{Commenti}
Il codice sorgente deve essere adeguatamente commentato, sia per chi lo scrive, sia per chi lo deve leggere in futuro.
È quindi necessario che ogni porzione di codice significativa sia dotata di commenti esplicativi, in particolar modo se le istruzioni usate non sono comuni.
I commenti devono essere chiari ma al contempo sintetici, per non distrarre dal contesto.
Ogni volta che viene scritto un nuovo frammento di codice, questo deve avere una sezione di codice di documentazione (in molti linguaggi indicato con /** */ per un commento a più righe e /// per un commento a singola riga).
Questa sezione deve saper dire:
\begin{itemize}
    \item chi ha scritto il codice;
    \item chi lo ha modificato l'ultima volta;
    \item in che contesto viene utilizzato il codice che segue;
    \item se è un metodo/funzione, che argomenti richiede in input, che valori restituisce e se può lanciare eccezioni.
\end{itemize}
I commenti devono essere mantenuti.

\subsubsection{Nomi dei file e delle variabili}
I nomi dei file:
\begin{itemize}
    \item devono essere univoci;
    \item devono rispettare le condizioni del linguaggio che contengono, se presenti.
\end{itemize}
I nomi delle variabili:
\begin{itemize}
    \item devono essere chiari e descrittivi (è vietato usare variabili temp, tmp, x, ecc.) e possibilmente brevi;
    \item non devono essere simili fra di loro, per evitare confusione;
    \item se formati da più parole si devono scrivere usando l'underscore come separatore (\glo{Snake Case}) oppure separando i termini con una lettera maiuscola ad ogni inizio parola (\glo{Camel Case});
    \item devono rispettare le condizioni del linguaggio in cui vengono usate, se presenti.
\end{itemize}