Nel documento \PdQ{} verrà descritta la strategia utilizzata dai verificatori per effettuare nel miglior modo possibile la verifica e la validazione di tutti i documenti prodotti da \Gruppo{}.

Lo scopo nel dirigere il \PdQ{} è quello di:
\begin{itemize}
	\item Illustrare come si intende gestire la qualità di processo e di prodotto;
	\item Elencare le varie metriche definite per aderire alle definizioni degli standard;
	\item Elencare i test per verificare la corretta soddisfazione dei requisiti del prodotto software.
\end{itemize}

La qualità di processo e la qualità di prodotto sono due aspetti chiaramente coordinati, ma vengono gestiti separatamente. \\ \\
Le sezioni principali del documento sono le seguenti:
\begin{itemize}
    \item \textbf{Qualità di processo:} Sezione dove vengono elencate le metriche inerenti ai \glo{processi};
    \item \textbf{Qualità di prodotto:} Sezione dove vengono elencate le metriche inerenti al prodotto;
    \item \textbf{Strategia di testing:} Sezione dove viene elencato il piano di testing delle componenti e del sistema software nel suo complesso;
\end{itemize}
