\subsubsection{Studio di fattibilità}
Il \Responsabile{} ha il compito di convocare tutti i membri del gruppo \Gruppo{} per discutere le varie tematiche riguardanti i capitolati d'appalto disponibili.
Lo \SdF{} viene redatto dagli analisti, i quali devono analizzare il materiale disponibile e inoltre tenere in considerazione anche ciò che è stato discusso nelle riunioni sul tema.\\
Il documento è strutturato in più sezioni, ognuna riguardante un capitolato d'appalto.
Per ogni capitolato verranno trattati i seguenti punti:
\begin{itemize}
\item Titolo del capitolato:
	\begin{itemize}
	\item Nome del capitolato;
	\item Azienda proponente;
	\item Committenti.
	\end{itemize}
\item Descrizione del capitolato:
	\begin{itemize}
	\item Breve riassunto del prodotto da realizzare, secondo le specifiche richieste dal proponente.
	\end{itemize}
\item Prerequisiti e tecnologie coinvolte:
	\begin{itemize}
	\item Elenco delle tecnologie da utilizzare, con eventuali riferimenti per ulteriori approfondimenti o spiegazioni del contesto applicativo;
	\item In alcuni casi l'azienda proponente consiglia l'utilizzo di certe tecnologie.
	\end{itemize}
\item Vincoli:
	\begin{itemize}
	\item Richieste generali, tecniche e/o organizzative da parte dell'azienda proponente.
	\end{itemize}
\item Aspetti positivi:
	\begin{itemize}
	\item Vengono descritti gli aspetti ritenuti positivi dai membri del gruppo \Gruppo{} del capitolato.
	Possono essere considerati aspetti positivi, ad esempio, l'apprendimento di nuove tecnologie e/o linguaggi, disponibilità di contatto e collaborazione offerta dal proponente e documentazione disponibile online riguardante le tecnologie coinvolte.
	\end{itemize}
\item Aspetti critici:
	\begin{itemize}
	\item Vengono descritti gli aspetti ritenuti critici dai membri del gruppo \Gruppo{} del capitolato.
	Possono essere considerati aspetti critici, ad esempio, l'eccessiva mole di tecnologie da apprendere, i requisiti di vincolo da tenere in considerazione e la scarsa presenza di documentazione online riguardante le tecnologie coinvolte.
	\end{itemize}
\item Conclusioni:
	\begin{itemize}
	\item Valutazione finale motivata dai membri del gruppo \Gruppo{} nella quale vengono esposte le ragioni di interesse o disinteresse nella scelta o meno del capitolato.
	\end{itemize}
\end{itemize}
Tutte queste informazioni appena elencate vengono raccolte nel documento interno \glo{\SdF{}}, sottoposto al processo di \glo{verifica{}} da parte dei verificatori.