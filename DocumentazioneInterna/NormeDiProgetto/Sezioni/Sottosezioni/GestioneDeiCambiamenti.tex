\subsection{Gestione dei cambiamenti}
\subsubsection{Scopo}
La gestione dei cambiamenti ha lo scopo di affrontare tutte le modifiche che il gruppo può apportare al prodotto in seguito a problemi o indicazione del proponente in ogni fase dello sviluppo del prodotto.
La causa predominante di cambiamenti sarà il riscontro di errori quindi bisogna istanziare due attività: 
\begin{itemize}
    \item Implementazione del processo;
    \item Risoluzione del problema.
\end{itemize}
\subsubsection{Implementazione del processo}
Il procedimento a cui fa riferimento è il processo di risoluzione del problema.
Infatti, innanzitutto, il gruppo deve impegnarsi a segnalare e analizzare ogni problema in cui si incorre nella maniera più rapida possibile; dopodichè i seguenti punti devono essere seguiti prima di arrivare alla risoluzione vera e propria del problema.
\begin{itemize}
    \item Seguendo uno schema il probelma deve essere categorizzato secondo la sua appartenenza e la priorità per la risoluzione;
    \item Bisogna operare un'analisi del problema per verificare che non sia già stato riscontrato o per ricavare informazioni per la ricerca di altri problemi;
    \item L'attuazione della risoluzione deve essere valutata per verificare che il problema sia stato correttamente risolto.
\end{itemize}
In tutto questo processo è essenziale, inoltre, tenere traccia dello stato del problema per non dimenticarsene e lasciarlo irrisolto.

\subsubsection{Risoluzione del problema}
In questa attività avviene il cambiamento effettivo del prodotto, che sia esso causato da un problema o da una richiesta del proponente.\\
Si compone un report del problema partendo dal tracciamento fatto dal momento della rilevazione arrivando fino alla risoluzione e alla verifica della risoluzione.\\
L'analisi di tutti i problemi risolti consente di scovare trend e anticipare la comparsa di altri problemi, uno strumento importatte che deve essere utilizzato.
