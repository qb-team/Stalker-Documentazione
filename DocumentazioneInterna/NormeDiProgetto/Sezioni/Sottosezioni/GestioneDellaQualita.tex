%scritto da \PF{}
\subsection{Gestione della qualità}
\subsubsection{Obiettivo}
Il gruppo \Gruppo{} ha come obiettivo prefissato di essere \glo{sistematico}, \glo{disciplinato} e \glo{quantificabile}, ai fini di:
\begin{itemize}
    \item Garantire la qualità nel prodotto software da realizzare;
    \item Soddisfare le richieste del proponente e del committente;
    \item Migliorare le proprie capacità di gestione di un progetto software.
\end{itemize}

\subsubsection{Piano di Qualifica}
Nel documento \PdQ{} il gruppo \Gruppo{} illustra come intende gestire la qualità di processo e di qualità di prodotto, elenca le varie metriche definite per aderire alle definizioni degli standard e i test per verificare la corretta soddisfazione dei requisiti del prodotto software.
La qualità di processo e la qualità di prodotto sono due aspetti chiaramente coordinati, ma vengono gestiti separatamente. \\ \\
Le sezioni principali del documento sono le seguenti:
\begin{itemize}
    \item \textbf{Qualità di processo:} Sezione dove vengono elencate le metriche inerenti ai \glo{processi};
    \item \textbf{Qualità di prodotto:} Sezione dove vengono elencate le metriche inerenti al prodotto;
    \item \textbf{Strategia di testing:} Sezione dove viene elencato il piano di testing delle componenti e del sistema software nel suo complesso;
    \item \textbf{Standard di qualità adottati:} Sezione dove vengono spiegati gli standard adottati.
\end{itemize}

\subsubsection{Metriche di qualità}\mbox{}\\ \\
\paragraph{Codici metriche}\mbox{}\\ \\
Ogni metrica di \glo{processo} ha un codice univoco ed è strutturato in questo formato:
\begin{center}
    M[Destinazione][Numero progressivo]
\end{center}
con:
\begin{itemize}  
    \item \textbf{[Destinazione]}:
    \begin{itemize}
        \item \textbf{PC}: Se la metrica fa riferimento al \glo{processo};
        \item \textbf{PD}: Se la metrica fa riferimento al prodotto.
    \end{itemize}
    \item \textbf{[Numero progressivo]}: Il numero della metrica in relazione alla sua destinazione, progressivo perché diverso per ogni metrica e in serie. Il conteggio parte da 1.
\end{itemize}

\paragraph{Struttura descrittiva metriche}\mbox{}\\ \\
La seguente è una struttura ad elenco che descrive una metrica di processo o di prodotto. \\
I punti dell'elenco racchiusi fra parentesi quadre indicano che tale punto è opzionale e va inserito solo se necessario.\\
Inoltre, \textbf{Processo di riferimento} deve essere presente solo nelle metriche della qualità di processo, mentre \textbf{Attributo di riferimento} solo per la qualità di prodotto.
elencata in una lista, mentre il nome della metrica rappresenta il titolo di questo elenco ed è visibile nell'indice del documento. La struttura è la seguente:
\begin{itemize}
    \item \textbf{Codice}: Codice univoco;
    \item \textbf{Descrizione}: Breve descrizione della metrica e del contesto applicativo;
    \item \textbf{Processo di riferimento}: Viene indicata in quale \glo{processo} viene applicata tale metrica (riferendosi allo standard);
    \item \textbf{Attributo di riferimento}: Viene indicato in quale attributo della caratteristica di prodotto viene applicata tale metrica (riferendosi allo standard);
    \item \textbf{Sigla}: nome della metrica sotto forma di acronimo, utilizzato principalmente nelle formule matematiche e nei range di accettazione;
    \item \textbf{[Formula:} formula matematica per poter calcolare il valore della metrica];
    \item \textbf{Range di valori che può assumere}: sezione in cui sono descritti i range di accettazione per i valori delle metriche.
    \begin{itemize}
        \item \textbf{Accettabile}: range in cui i valori della metrica possono essere ritenuti accettabili per garantire la qualità;
        \item \textbf{Ottimale}: range in cui i valori della metrica possono essere ritenuti ottimali per garantire la qualità.
    \end{itemize}
\end{itemize}

\paragraph{Metriche di processo}\mbox{}\\ \\
Per monitorare l'aderenza ai processi dallo standard ISO/IEC 12207 istanziati, vengono utilizzate delle metriche. Il Responsabile, grazie ai valori ricavati dalle metriche, è facilitato nel
valutare il \glo{processo} e di effettuare, se necessario, modifiche alla pianificazione.\\

\paragraph{Metriche di prodotto}\mbox{}\\ \\
Il modello di qualità del software descritto dallo standard ISO/IEC 9126 definisce le caratteristiche e gli attributi del prodotto software, ciascuna misurabile da metriche interne (che richiedono la disponibilità del codice sorgente - white box) o esterne (che richiedono il prodotto software in esecuzione - black box).
Una volta specificati i requisiti di qualità del prodotto software, si identificano le caratteristiche e attributi di qualità che più contribuiscono a verificare l'aderenza allo standard e ai requisiti.
Il gruppo \Gruppo{} si è impegnato a scegliere le metriche interne che maggiormente influenzano (in positivo) le caratteristiche esterne del prodotto finale, in modo che esse possano predire quanto più possibile la qualità del risultato finale. 

\paragraph{Tabella riassuntiva metriche}\mbox{}\\ \\
Per riassumere tutte le metriche e le loro caratteristiche descritte all'interno del \PdQ{}, alla fine della sezione in cui vengono illustrate vi sono delle tabelle riassuntive che hanno questa struttura:
{
\rowcolors{2}{grigetto}{white}
\renewcommand{\arraystretch}{1.5}
\begin{longtable}{ c C{4cm} c c c}
\caption{Tabella metriche dei processi/prodotti}\\
\rowcolor{darkblue}
\textcolor{white}{\textbf{Metrica}} & \textcolor{white}{\textbf{Nome}} & \textcolor{white}{\textbf{Sigla}} & \textcolor{white}{\textbf{Range Accettabile}} & \textcolor{white}{\textbf{Range Ottimale}}\\
Codice & Nome della metrica & Sigla & Range Accettabile & Range Ottimale \\
\end{longtable}
}

\subsubsection{Lista delle metriche di processo}
\paragraph{Processi primari}\mbox{}\\ \\
\subparagraph{Analisi dei requisiti}\mbox{}\\ \\
    \subsubparagraph{Metrica - Percentuale requisiti obbligatori soddisfatti}\mbox{}\\ \\
    \begin{itemize}
        \item \textbf{Codice:} MPC1
        \item \textbf{Descrizione:} È la percentuale dei requisiti che devono essere soddisfatti;
        \item \textbf{Processo di riferimento:} Sviluppo;
        \item \textbf{Attività di riferimento:} Analisi dei requisiti;
        \item \textbf{Sigla:} $PRS$
        \item \textbf{Formula:} $$PRS = {|requisiti \; soddisfatti| \over |requisiti \; totali|}\; \cdot \; 100$$
        \item \textbf{Strumenti utilizzati:}
    \end{itemize}
\subparagraph{Progettazione di dettaglio}\mbox{}\\
    \subsubparagraph{Metrica - Incapsulamento CBO}\mbox{}\\
    \begin{itemize}
        \item \textbf{Codice:} MPC2
        \item \textbf{Descrizione:} Il "Coupling Between Objects" misura il numero delle classi correlate ad una classe in esame al di fuori dalla gerarchia di ereditarietà. Più è alto il grado di coupling della classe in esame e più il sistema è difficile da mantenere;
        \item \textbf{Processo di riferimento:} Sviluppo;
        \item \textbf{Sigla:} $CBO$
        \item \textbf{Formula:} $$CBO = {\sum_{i=1}^{N} C_i}$$
        con:
        \begin{itemize}
            \item $N$ = numero classi non appartenenti alla gerarchia di ereditarietà della classe in esame;
            \item $C_i$ =
            \begin{math} {
                \begin{cases}
                    1, & la \; $i$-esima \; classe \; \grave{e} \; correlata \; a \; quella \; in \; esame \\
                    0, & la \; $i$-esima \; classe \; non \; \grave{e} \; correlata \; a \; quella \; in \; esame
                \end{cases}
            }
            \end{math}
        \end{itemize}
        \item \textbf{Strumenti utilizzati:}
    \end{itemize}

    \subsubparagraph{Metrica - Livello profondità gerarchia}\mbox{}\\ \\
    \begin{itemize}
        \item \textbf{Codice:} MPC3
        \item \textbf{Descrizione:} È il valore intero che indica la profondità della gerarchia formata tra classi. Se una gerarchia è formata da una classe allora il valore è uguale a 1;
        \item \textbf{Processo di riferimento:} Sviluppo;
        \item \textbf{Sigla:} $LPG$
        \item \textbf{Strumenti utilizzati:}
    \end{itemize}

\subparagraph{Codifica}\mbox{}\\ \\
    \subsubparagraph{Metrica - Numero di parametri per metodo}\mbox{}\\ \\
    \begin{itemize}
        \item \textbf{Codice:} MPC4
        \item \textbf{Descrizione:} Un numero elevato di parametri per metodo può indicare il bisogno di ridurre funzionalità associate a tale metodo. Più è grande questo valore e più la possibilità aumenta nel commettere errori progettuali;
        \item \textbf{Processo di riferimento:} Sviluppo;
        \item \textbf{Sigla:} $NPM$
        \item \textbf{Strumenti utilizzati:}
    \end{itemize}

    \subsubparagraph{Metrica - Linee di commento per linee di codice}\mbox{}\\ \\
    \begin{itemize}
        \item \textbf{Codice:} MPC5
        \item \textbf{Descrizione:} È il rapporto tra linee di commento e linee di codice. Per le linee di codice si intende Logical SLOC il numero di linee di codice effettive che corrispondono al numero di statement;
        \item \textbf{Processo di riferimento:} Sviluppo;
        \item \textbf{Sigla:} $LCLC$
        \item \textbf{Formula:}$$LCLC = {|linee \; di \; commento| \over |linee \; di \; codice|}$$
        \item \textbf{Strumenti utilizzati:}
    \end{itemize}
\paragraph{Processi di supporto}\mbox{}\\ \\
\subparagraph{Implementazione}\mbox{}\\ \\
\subsubparagraph{Metrica - Indice di Gulpease}
\begin{itemize}
	\item \textbf{Codice:} MPC6
	\item \textbf{Descrizione:} È l'indice di leggibilità di un determinato testo. Calcola la lunghezza delle parole e delle frasi rispetto al numero totale delle lettere. Il valore è un intero da 0 a 100; se esso è inferiore a 80 sarà difficile da leggere per chi ha la licenza elementare, mentre se è inferiore a 40 sarà difficili da leggere per chi ha un diploma superiore;
	\item \textbf{Processo di riferimento:} Documentazione;
	\item \textbf{Sigla:} $IG$
	\item \textbf{Formula:} $$IG = 89 + {{300 \; \cdot \; |frasi| \; - \; 10 \; \cdot \; |lettere|}\over |parole|}$$
	\item \textbf{Strumenti utilizzati:}
\end{itemize}

\subparagraph{Verifica}\mbox{}\\ \\
\subsubparagraph{Metrica - Code coverage}\mbox{}\\ \\
\begin{itemize}
	\item \textbf{Codice:} MPC7
	\item \textbf{Descrizione:} È la percentuale di copertura del codice attraversato dai test rispetto al totale del codice di base. Per dare una misurazione in termini di grandezza si adoperano le linee di codice come riferimento;
	\item \textbf{Processo di riferimento:} \glo{Processi} di verifica;
	\item \textbf{Sigla:} $CC$
	\item \textbf{Formula:} $$CC = {|linee \; di \; codice \; percorse \; dai  \; test| \over |linee \; di \; codice \; totali|} \; \cdot \; 100$$
	\item \textbf{Strumenti utilizzati:}
\end{itemize}
    \paragraph{Processi organizzativi}\mbox{}\\ \\

\subparagraph{Pianificazione}\mbox{}\\ \\
    \subsubparagraph{Metrica - Schedule variance}\mbox{}\\ \\
    \begin{itemize}
        \item \textbf{Codice:} MPC8
        \item \textbf{Descrizione:} È il valore che indica se si è in linea ($=0$), in anticipo ($>0$) oppure in ritardo ($<0$) rispetto alla schedulazione delle attività di progetto pianificate nella \glo{baseline};
        \item \textbf{Processo di riferimento:} Gestione;
        \item \textbf{Sigla:} $SV$
        \item \textbf{Formula:} $$SV = {BCWP \; - \; BCWS}$$
        con:
        \begin{itemize}
            \item $BCWP$ = Budgeted Cost of Work Performed (valore delle attività eseguite nella data corrente);
            \item $BCWS$ = Budgeted Cost of Work Scheduled (costo pianificato per la realizzazione delle attività di progetto alla data corrente);
        \end{itemize}
        \item \textbf{Range di valori che può assumere:}
        \begin{itemize}
            \item \textbf{Accettabile:} $SV = 0$
            \item \textbf{Ottimale:} $SV > 0$
        \end{itemize}
    \end{itemize}

    \subsubparagraph{Metrica - Budget variance}\mbox{}\\ \\
        \begin{itemize}
            \item \textbf{Codice:} MPC9
            \item \textbf{Descrizione:} È il valore che indica se alla data corrente si è speso di più ($>0$) o di meno ($<0$) rispetto a quanto pianificato dal budget totale $B_{tot}$;
            \item \textbf{Processo di riferimento:} Gestione;
            \item \textbf{Sigla:} $BV$
            \item \textbf{Formula:} $$BV = {BCWS \; - \; ACWP}$$
            con:
            \begin{itemize}
                \item $BCWS$ = Budgeted Cost of Work Scheduled (costo pianificato per la realizzazione delle attività di progetto alla data corrente);
                \item $ACWP$ = Actual Cost of Work Performed (costo effettivamente sostenuto alla data corrente);
                \item $B_{tot}$ = Budget totale.
            \end{itemize}
            \item \textbf{Range di valori che può assumere:}
            \begin{itemize}
                \item \textbf{Accettabile:} $0 \leq BV < ACWP$
                \item \textbf{Ottimale:} $0 \leq BV \leq B_{tot}$
            \end{itemize}
        \end{itemize}
\newpage
\subsubsection{Lista delle metriche di prodotto}
\paragraph{Metriche interne}\mbox{}\\ \\
Le metriche della qualità "interne" del software sono utilizzate durante la fase di sviluppo e permettono di valutare il comportamento del software dal punto di vista degli sviluppatori e di predire quello che sarà il punto di vista esterno degli utenti.

\subparagraph{Funzionalità}\mbox{}\\ \\
Capacità del prodotto software di soddisfare i requisiti funzionali e le necessità degli utenti.
\subsubparagraph{Metrica - Aderenza agli Standard di Funzioni o Interfacce} 
\begin{itemize}
    \item \textbf{Codice:} MPD1;
    \item \textbf{Descrizione:} Misura in percentuale il livello di aderenza delle funzioni e delle interfacce sviluppate rispetto agli standard, alle normative e alla regolamentazioni;
    \item \textbf{Attributo di riferimento:} Aderenza alle funzionalità;
    \item \textbf{Sigla:} $ASFI$ 
    \item \textbf{Formula:} $$ASFI = {|funzioni \; o \; interfacce \; aderenti \; a \; standard| \over |funzioni \; o \; interfacce \; che \; devono \; aderire \; a \; standard|} \cdot 100 $$ 
    \item \textbf{Strumenti utilizzati:}
\end{itemize}
              
\subparagraph{Affidabilità} \mbox{}\\ \\
Capacità di predire se il prodotto software in questione potrà soddisfare i requisiti prescritti per l'affidabilità dal punto di vista degli sviluppatori.

\subsubparagraph{Metrica - Rilevamento dei Difetti} 
\begin{itemize}
    \item \textbf{Codice:} MPD2
    \item \textbf{Descrizione:} Misurare in percentuale l'efficacia nel rilevare i difetti presenti nel software durante lo sviluppo del prodotto;
    \item \textbf{Attributo di riferimento:} \glo{Maturità}
    \item \textbf{Sigla:} $RD$
    \item \textbf{Formula:} $$RD = {|difetti \; rilevati \; nei \; test \; del \; prodotto| \over |difetti \; previsti \; durante \; lo \; sviluppo|} \cdot 100 $$
    \item \textbf{Strumenti utilizzati:}
\end{itemize}

\subparagraph{Usabilità} \mbox{}\\ \\
Capacità del prodotto software di essere comprensibile, di poter essere usato e compreso facilmente, in ogni sua parte, da qualsiasi utente che lo voglia usare.\\
\subsubparagraph{Metrica - Validità dei Dati d'Input} 
\begin{itemize}
    \item \textbf{Codice:} MPD3
    \item \textbf{Descrizione:} Misurare in percentuale la correttezza dei dati forniti in input all'applicazione;
    \item \textbf{Attributo di riferimento:} \glo{Operabilità};
    \item \textbf{Sigla:} $VDI$
    \item \textbf{Formula:} $$VDI = {|dati \; di \; input \; controllati \; e \; validitati| \over |dati \; di \; input \; previsti|} \cdot 100$$  
    \item \textbf{Strumenti utilizzati:}
\end{itemize}

\subsubparagraph{Metrica - Attrattività della User Interface (UI)} 
\begin{itemize}
    \item \textbf{Codice: } MPD4
    \item \textbf{Descrizione:} Misurare quanto attrattive risultino le interfacce agli utenti dal punto di vista grafico.
    Gli utenti dovranno poi compilare un questionario in base all'esperienza che hanno fatto;
    Il valore medio di tale valutazione è ritenuto valido se almeno tre utenti hanno compilato il questionario. 
    Viene utilizzata una scala a quattro valori: Molto attrattivo, Attrattivo, Poco attrattivo, Non Attrattivo;
    \item \textbf{Attributo di riferimento:} \glo{Attrattività};
    \item \textbf{Sigla:} $AUI$
    \item \textbf{Formula:}$$AUI = V(q) $$
    con:
        \begin{itemize}
        \item $V$ = Valore medio dei risultati;
        \item $q$ = Questionario compilato;
        \end{itemize}
    \item \textbf{Strumenti utilizzati:}
\end{itemize}

\subparagraph{Manutenibilità} \mbox{}\\ \\
Capacità di predire il livello di impegno richiesto per modificare il prodotto software dal punto di vista degli sviluppatori.           
\subsubparagraph{Metrica - Complessità Ciclomatica del Software} 
    \begin{itemize}
    \item \textbf{Codice:} MPD5
    \item \textbf{Descrizione:} Misurare la complessità ciclomatica dei singoli moduli sviluppati;
    \item \textbf{Attributo di riferimento:} \glo{Modificabilità};
    \item \textbf{Sigla:} $CCS$
    \item \textbf{Formula:} $$CCS = e - n + p$$
    con:
    \begin{itemize}
        \item $G$ = grafo del modulo;
        \item $e$ = numero di congiunzioni tra statement (corrispondenti agli archi di un grafo);
        \item $n$ = numero di statement (nodi presenti nel grafo);
        \item $p$ = numero delle componenti connesse da ogni nodo (per esecuzione sequenziale: p=2, essendovi un predecessore e un successore);
    \end{itemize}

    \item \textbf{Strumenti utilizzati:}
\end{itemize}

\subsubparagraph{Metrica - Unità Documentate} 
\begin{itemize}
    \item \textbf{Codice:} MPD6
    \item \textbf{Descrizione:} Misurare in percentuale il numero di unità di codice con documentazione tecnica;
    \item \textbf{Attributo di riferimento:} \glo{Modificabilità};
    \item \textbf{Sigla:} $UD$
    \item \textbf{Formula:} $$UD = {|unit\grave{a} \; di \; codice \; con \; documentazione \; tecnica| \over |unit\grave{a} \; di \; codice|} \cdot 100$$
    \item \textbf{Strumenti utilizzati:}
\end{itemize}
              
       
\paragraph{Metriche esterne}\mbox{}\\ \\
Le metriche relative alla qualità "esterna" indirizzano le caratteristiche esteriori del software, cioè quelle rilevabili direttamente dagli utenti e dagli operatori.

\subparagraph{Affidabilità}\mbox{}\\ \\
Capacità del prodotto software di dimostrare un adeguato livello di affidabilità quando opererà nel sistema in cui è previsto debba operare.
    
\subsubparagraph{Metrica - Maturità dei Test} 
\begin{itemize}
    \item \textbf{Codice:} MPD7
    \item \textbf{Descrizione:} Misurare la percentuale di casi di test eseguiti con successo rispetto al numero totale previsto per garantire piena copertura dei requisiti sia funzionali che qualitativi(usabilità, affidabilità, efficienza);
    \item \textbf{Attributo di riferimento:} \glo{Maturità};
    \item \textbf{Sigla:} $MT$
    \item \textbf{Formula:} $$MT = {|casi \; di \; test \; eseguiti \; con \; successo| \over |casi \; di \; test \; previsti|} \cdot 100$$
    \item \textbf{Strumenti utilizzati:}
\end{itemize}
       
\subparagraph{Usabilità}\mbox{}\\ \\
Capacità del prodotto software di essere facilmente comprensibile, apprendibile ed operabile per ogni utente intenzionato a usarlo.

\subsubparagraph{Metrica - Profondità Strutturale dell'Interfaccia}
\begin{itemize}
    \item \textbf{Codice:} MPD8
    \item \textbf{Descrizione:} È sconsigliato per le interfacce utente l'utilizzo di una struttura troppo profonda per questioni di usabilità. L'utente non deve fare troppi passaggi per raggiungere la funzionalità desiderata;
    \item \textbf{Attributo di riferimento:} \glo{Operabilità};
    \item \textbf{Sigla:} $PSI$
    \item \textbf{Strumenti utilizzati:}
\end{itemize}
% <commento ai commenti> gli strumenti per il controllo di qualità verranno integrati nelle metriche nell'apposito campo
%\subsubsection{Strumenti per il controllo di qualità}
%In questa sezione vengono indicati gli strumenti per valutare le metriche di processo e di prodotto.

%\paragraph*{Note sugli strumenti per il controllo di qualità}\mbox{}\\ \\
%La seguente sezione, relativa agli strumenti, non è completa in quanto il suo contenuto dipende in larga parte dalle decisioni progettuali che devono essere prese con l'avanzamento del progetto.


%\paragraph{Metrica - Indice di Gulpease}\mbox{}\\ \\
%Per poter calcolare questo indice, descritto nel \PdQ{} viene utilizzato come strumento il "Calcolatore dell'Indice Gulpease" ospitato a \href{https://farfalla-project.org/readability_static/}{questo indirizzo}.
%Non vengono contati per l'indice di Gulpease il testo presente nelle seguenti parti o sezioni del documento:
%\begin{itemize}
%    \item Copertina;
%    \item Indice;
%    \item Registro delle modifiche;
%    \item Riferimenti.
%\end{itemize}