%scritto da \PF{}
\subsection{Gestione della qualità}
\subsubsection{Obiettivo}
Il gruppo \Gruppo{} ha come obiettivo prefissato di essere \glo{sistematico}, \glo{disciplinato} e \glo{quantificabile}, ai fini di:
\begin{itemize}
    \item Garantire la qualità nel prodotto software da realizzare;
    \item Soddisfare le richieste del proponente e del committente;
    \item Migliorare le proprie capacità di gestione di un progetto software.
\end{itemize}

\subsubsection{Piano di Qualifica}
Nel documento \PdQ{} il gruppo \Gruppo{} illustra come intende gestire la qualità di processo e di qualità di prodotto, elenca le varie metriche definite per aderire alle definizioni degli standard e i test per verificare la corretta soddisfazione dei requisiti del prodotto software.
La qualità di processo e la qualità di prodotto sono due aspetti chiaramente coordinati, ma vengono gestiti separatamente. \\ \\
Le sezioni principali del documento sono le seguenti:
\begin{itemize}
    \item \textbf{Qualità di processo:} Sezione dove vengono elencate le metriche inerenti ai \glo{processi};
    \item \textbf{Qualità di prodotto:} Sezione dove vengono elencate le metriche inerenti al prodotto;
    \item \textbf{Strategia di testing:} Sezione dove viene elencato il piano di testing delle componenti e del sistema software nel suo complesso;
    \item \textbf{Standard di qualità adottati:} Sezione dove vengono spiegati gli standard adottati.
\end{itemize}

\paragraph{Metriche di qualità}\mbox{}\\ \\
\subparagraph{Metriche di processo}\mbox{}\\ \\
Per monitorare l'aderenza ai processi dallo standard ISO/IEC 12207 istanziati, vengono utilizzate delle metriche. Il Responsabile, grazie ai valori ricavati dalle metriche, è facilitato nel
valutare il \glo{processo} e di effettuare, se necessario, modifiche alla pianificazione.\\

\subparagraph{Metriche di prodotto}\mbox{}\\ \\
Il modello di qualità del software descritto dallo standard ISO/IEC 9126 definisce le caratteristiche e gli attributi del prodotto software, ciascuna misurabile da metriche interne (che richiedono la disponibilità del codice sorgente - white box) o esterne (che richiedono il prodotto software in esecuzione - black box).
Una volta specificati i requisiti di qualità del prodotto software, si identificano le caratteristiche e attributi di qualità che più contribuiscono a verificare l'aderenza allo standard e ai requisiti.
Il gruppo \Gruppo{} si è impegnato a scegliere le metriche interne che maggiormente influenzano (in positivo) le caratteristiche esterne del prodotto finale, in modo che esse possano predire quanto più possibile la qualità del risultato finale.

\subparagraph{Codici metriche}\mbox{}\\ \\
Ogni metrica di \glo{processo} ha un codice univoco ed è strutturato in questo formato:
\begin{center}
    M[Destinazione][Numero progressivo]
\end{center}
con:
\begin{itemize}  
    \item \textbf{[Destinazione]}:
    \begin{itemize}
        \item \textbf{PC}: Se la metrica fa riferimento al \glo{processo};
        \item \textbf{PD}: Se la metrica fa riferimento al prodotto.
    \end{itemize}
    \item \textbf{[Numero progressivo]}: Il numero della metrica in relazione alla sua destinazione, progressivo perché diverso per ogni metrica e in serie. Il conteggio parte da 1.
\end{itemize}

\subparagraph{Struttura descrittiva metriche}\mbox{}\\ \\
La seguente è una struttura ad elenco che descrive una metrica di processo o di prodotto. \\
I punti dell'elenco racchiusi fra parentesi quadre indicano che tale punto è opzionale e va inserito solo se necessario.\\
Inoltre, \textbf{Processo di riferimento} deve essere presente solo nelle metriche della qualità di processo, mentre \textbf{Attributo di riferimento} solo per la qualità di prodotto.
elencata in una lista, mentre il nome della metrica rappresenta il titolo di questo elenco ed è visibile nell'indice del documento. La struttura è la seguente:
\begin{itemize}
    \item \textbf{Codice}: Codice univoco;
    \item \textbf{Descrizione}: Breve descrizione della metrica e del contesto applicativo;
    \item \textbf{Processo di riferimento}: Viene indicata in quale \glo{processo} viene applicata tale metrica (riferendosi allo standard);
    \item \textbf{Attributo di riferimento}: Viene indicato in quale attributo della caratteristica di prodotto viene applicata tale metrica (riferendosi allo standard);
    \item \textbf{Sigla}: nome della metrica sotto forma di acronimo, utilizzato principalmente nelle formule matematiche e nei range di accettazione;
    \item \textbf{[Formula:} formula matematica per poter calcolare il valore della metrica];
    \item \textbf{Range di valori che può assumere}: sezione in cui sono descritti i range di accettazione per i valori delle metriche.
    \begin{itemize}
        \item \textbf{Accettabile}: range in cui i valori della metrica possono essere ritenuti accettabili per garantire la qualità;
        \item \textbf{Ottimale}: range in cui i valori della metrica possono essere ritenuti ottimali per garantire la qualità.
    \end{itemize}
\end{itemize} 

\subparagraph{Tabella riassuntiva metriche}\mbox{}\\ \\
Per riassumere tutte le metriche e le loro caratteristiche descritte all'interno del \PdQ{}, alla fine della sezione in cui vengono illustrate vi sono delle tabelle riassuntive che hanno questa struttura:
{
\rowcolors{2}{grigetto}{white}
\renewcommand{\arraystretch}{1.5}
\begin{longtable}{ c C{4cm} c c c}
\caption{Tabella metriche dei processi/prodotti}\\
\rowcolor{darkblue}
\textcolor{white}{\textbf{Metrica}} & \textcolor{white}{\textbf{Nome}} & \textcolor{white}{\textbf{Sigla}} & \textcolor{white}{\textbf{Range Accettabile}} & \textcolor{white}{\textbf{Range Ottimale}}\\
Codice & Nome della metrica & Sigla & Range Accettabile & Range Ottimale \\
\end{longtable}
}

\subsubsection{Strumenti per il controllo di qualità}
In questa sezione vengono indicati gli strumenti per valutare le metriche di processo e di prodotto.

\paragraph*{Note sugli strumenti per il controllo di qualità}\mbox{}\\ \\
La seguente sezione, relativa agli strumenti, non è completa in quanto il suo contenuto dipende in larga parte dalle decisioni progettuali che devono essere prese con l'avanzamento del progetto.

\paragraph{Metrica - Indice di Gulpease}\mbox{}\\ \\
Per poter calcolare questo indice, descritto nel \PdQ{} viene utilizzato come strumento il "Calcolatore dell'Indice Gulpease" ospitato a \href{https://farfalla-project.org/readability_static/}{questo indirizzo}.
Non vengono contati per l'indice di Gulpease il testo presente nelle seguenti parti o sezioni del documento:
\begin{itemize}
    \item Copertina;
    \item Indice;
    \item Registro delle modifiche;
    \item Riferimenti.
\end{itemize}