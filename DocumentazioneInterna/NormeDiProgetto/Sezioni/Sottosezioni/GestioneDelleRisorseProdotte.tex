%scritto da \PF{}
\subsection{Gestione della configurazione}
\subsubsection{Obiettivo}
Si vuole garantire che tutto ciò che viene prodotto dal gruppo \Gruppo{}, dalla documentazione al prodotto software, sia facilmente reperibile all’interno del \glo{repository}.
Ogni elaborato viene etichettato con un particolare codice di versione e, ai fini della manutenibilità, vengono riportate le pratiche che devono essere adottate nelle operazioni di modifica e versionamento.

\subsubsection{Versionamento}
Allo scopo di tracciare gli incrementi del prodotto, ogni sua versione è accompagnata da un particolare codice di versione a tre numeri. Dati X, Y, Z $\in \mathbb{N}$, la forma del codice di versione è la seguente:
\begin{center}
	\textbf{X.Y.Z}
\end{center}
Il gruppo ha scelto di scindere il versionamento del prodotto software dal versionamento della documentazione al fine di garantire una definizione chiara e univoca del numero di versione.\\
Questa scelta nasce dalla distanza che intercorre tra documentazione e prodotto nell'ambito del versionamento, ovvero aumentare il numero di versione nella documentazione non è correlato con un aumento del numero di versione del prodotto software.
I soli documenti che avranno lo stesso numero di versione del prodotto software sono: 
\begin{itemize}
	\item Manuale utente;
	\item Manuale manutentore.
\end{itemize}  
Perché saranno prodotti in contemporanea allo sviluppo delle funzionalità del prodotto e quindi le modifiche e aggiunte a quest'ultimo sono direttamente collegate ai due manuali.\\

Numero versione per la documentazione:
\begin{itemize}
	\item \textbf{X}: Detto indice major, rappresenta il numero di una versione stabile, quando X $> 0$.
	Viene incrementato ogni volta che il \Responsabile{} approva il documento, successivamente allo svolgimento del processo di verifica da parte di un verificatore;
	\item \textbf{Y}: Detto indice minor, rappresenta una versione del documento a cui sono state aggiunte sezioni, sottosezioni o paragrafi, la cui correttezza è garantita dalla verifica da parte di un verificatore. 
	Viene azzerato ad ogni incremento dell'indice X;
	\item \textbf{Z}: Detto indice patch, rappresenta una versione del documento in cui è stata effettuata un'aggiunta o una modifica, questa aggiunta deve essere verificata da parte di un verificatore.
	Quindi, l'indice Z viene incrementato ad ogni modifica e correzione, di qualsiasi dimensione, previa verifica.
	Viene azzerato quando l'indice X o l'indice Y vengono incrementati.
\end{itemize}

Numero versione per il prodotto software e il manuale utente e manuale manutentore:
\begin{itemize}
	\item \textbf{X}: Detto indice major, rappresenta il numero di una versione stabile, quando X $> 0$.
	Viene incrementato quando tutti i requisiti obbligatori sono implementati e il prodotto supera la verifica da parte dei verificatori e viene approvato dal \Responsabile{}.
	\item \textbf{Y}: Detto indice minor, rappresenta una versione del prodotto a cui sono state aggiunte tutte le funzionalità atte a soddisfare un requisito, la correttezza di tali funzionalità deve essere garantita dai verificatori.  
	Viene azzerato ad ogni incremento dell'indice X;
	\item \textbf{Z}: Detto indice patch, rappresenta una versione del prodotto in cui è stata effettuata un'aggiunta o una correzione, che superi la verifica da parte di un verificatore
	Quindi, l'indice Z viene incrementato ad ogni modifica, di qualsiasi dimensione, previa verifica.
	Viene azzerato quando l'indice X o l'indice Y vengono incrementati.
\end{itemize}
Seguendo lo stesso numero di versione, il prodotto e i manuali incrementalo il loro contenuto ogni volta che viene aggiunta una funzionalità al prodotto software.

\subsubsection{Servizi di supporto per il versionamento} 
Il gruppo ha deciso di adottare \glo{Git}, che è un software per la gestione del codice sorgente (\glo{SCM}) e controllo di versione (\glo{VCS}).
Git è un VCS distribuito: ogni membro del gruppo \Gruppo{} ha quindi a disposizione una copia del repository e svolge il suo lavoro principalmente in locale.
Per condividere il proprio lavoro con gli altri membri del gruppo, effettua un'operazione di \textbf{push} in remoto.
Per ricevere il lavoro condiviso dagli altri membri del gruppo, effettua un'operazione di \textbf{pull} da remoto.
Il repository remoto a cui tutto il gruppo fa riferimento è ospitato sulla piattaforma \glo{GitHub}.
Il gruppo \Gruppo{} potrà interagire con il \glo{VCS} sia tramite riga di comando (git) che tramite software come \glo{GitKraken}.

\subsubsection{Repository creati}
Sono stati creati due \glo{repository} per supportare il lavoro del gruppo, quali:
\begin{itemize}
	\item \textbf{Stalker}: Viene usato per versionare il codice sorgente prodotto dal gruppo.
	Al momento tale repository è vuoto e verrà usato dopo la RR, organizzandoli nella maniera che verrà ritenuta più appropriata;
	\item \textbf{Stalker-Documentazione}: Viene usato per versionare la documentazione prodotta dal gruppo.
	\item \textbf{Stalker-App-PoC-Tracciamento}: Viene usato per versionare il codice sorgente relativo dell'applicazione mobile per lo sviluppo del \glo{PoC}.
	\item \textbf{Stalker-Backend}: Viene usato per lo sviluppo dell'applicativo lato server.
	\item \textbf{Stalker-Admin-PoC-Auth}: Viene usato per lo sviluppo dell'interfaccia web dedicata all'amministratore.
\end{itemize}

\subsubsection{Organizzazione del repository Stalker-Documentazione}
Il \glo{repository} per la documentazione ha la seguente \glo{organizzazione} di cartelle:
\begin{itemize}
	\item \textbf{DocumentazioneEsterna/}: Dentro a questa cartella vi è la documentazione da fornire ai committenti e al proponente, oltre che i verbali redatti durante gli incontri con quest'ultimi;
	\begin{itemize}
		\item \textbf{VerbaliEsterni/}: All'interno di questa cartella sono presenti i file sorgente in \LaTeX{} per la generazione dei verbali degli \glo{incontri formali} fra i membri del gruppo e il proponente del progetto.
		Per ogni incontro viene redatto un verbale che è presente in un'unica cartella;
		\item \textbf{AnalisiDeiRequisiti/}: In questa cartella sono presenti i file sorgente in \LaTeX{} per la generazione dell'\AdR{};
		\item \textbf{PianoDiProgetto/}: Dentro a questa cartella sono presenti i file sorgente in \LaTeX{} per la generazione dell'\PdP{};
		\item \textbf{PianoDiQualifica/}: All'interno di questa cartella sono presenti i file sorgente in \LaTeX{} per la generazione dell'\PdQ{}.
	\end{itemize}
	\item \textbf{DocumentazioneInterna/}: Dentro a questa cartella vi è la documentazione ad uso e consumo da parte dei membri del gruppo;
	\begin{itemize}
		\item \textbf{VerbaliInterni/}: In questa cartella sono presenti i file sorgente in \LaTeX{} per la generazione dei verbali degli \glo{incontri formali} fra i membri del gruppo.
		Per ogni incontro viene redatto un verbale che è presente in un'unica cartella;
		\item \textbf{StudioDiFattibilita/}: All'interno di questa cartella sono presenti i file sorgente in \LaTeX{} per la generazione dell'\SdF{};
		\item \textbf{NormeDiProgetto/}: Dentro a questa cartella sono presenti i file sorgente in \LaTeX{} per la generazione dell'\NdP{}.
	\end{itemize}	
	\item \textbf{Glossario/}: In questa cartella sono presenti i file sorgente in \LaTeX{} per la generazione dell'\Glossario{};
	\item \textbf{Utilita/}: Dentro a questa cartella vi sono file che permettono una scrittura più rapida della documentazione.
	Vi sono i file e le immagini comuni contente le componenti, gli stili e i comandi comuni a tutti i documenti da realizzare in \LaTeX{};
	\item \textbf{.gitignore}: File utilizzato per garantire che non vengano versionati certi tipi di file non utili alla compilazione.
\end{itemize}

\subsubsection{Procedura di lavoro collaborativo}
Viene scelto di utilizzare il modello \glo{Git Flow} per ogni repository gestito dal gruppo.
Con Git Flow, lo sviluppo viene diviso in più rami, detti \glo{branch}:
\begin{itemize}
	\item \textbf{master}: Contenente il codice sorgente (indipendentemente dal fatto che sia per la documentazione che per il prodotto software) del prodotto rilasciato.
	Ad ogni \glo{merge} in questo branch effettuato tramite il processo di \textbf{release} corrisponde un etichetta, che è il modo in cui \glo{Git} gestisce le versioni stabili di un prodotto;
	\item \textbf{develop}: Contenente codice sorgente corretto ma non completo in tutte le parti, non pronto per essere rilasciato;
	\item \textbf{feature/[FEATURE]}: Contenente codice sorgente altamente soggetto a modifiche perché riguardante una funzionalità in fase di sviluppo.
	\textbf{[FEATURE]} corrisponde al nome della funzionalità in sviluppo, ed è decisa dal creatore del \glo{branch};
	\item \textbf{release/[RELEASE]}: Contenente codice sorgente stabile e pronto per il rilascio;
	\item \textbf{bugfix/[BUGFIX]} e \textbf{hotfix/[HOTFIX]}: Contenente codice sorgente in cui sono presenti funzionalità o sezioni contenenti bug, i quali vanno risolti dai membri del gruppo.
	Un \glo{branch} di bugfix o hotfix viene creato a partire dal \glo{branch} master.
\end{itemize}

Oltre all'utilizzo di Git Flow, devono venire rispettate le seguenti regole:
\begin{itemize}
	\item Non è permesso effettuare \glo{commit} nel ramo master se non attraverso un \glo{pull request} che deve essere accettato dall’\Amministratore{};
	\item Per ogni \glo{commit} è opportuno inserire nel messaggio una descrizione del lavoro svolto, anche riferendosi alle issue presenti nell'\glo{Issue Tracking System} di \glo{GitHub} indicando il numero di queste con "\#[NUM]", in cui [NUM] è un codice di issue.
\end{itemize}
