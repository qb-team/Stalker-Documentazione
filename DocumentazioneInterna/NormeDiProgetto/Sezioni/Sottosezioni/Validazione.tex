%scritto da \PF{}
\subsection{Validazione}
\subsubsection{Scopo}
L’obiettivo della Validazione è confermare che i requisiti concordati con il committente siano soddisfatti e che il prodotto software realizzato dal gruppo sia concorde alle aspettative.
Il committente del prodotto software è soddisfatto quando ha la dimostrazione che il prodotto commissionato rispetti al minimo i requisiti obbligatori, con efficacia ed efficienza.
\subsubsection{Aspettative}
Per assicurarsi che venga raggiunto lo scopo, nel processo di validazione vengono testate le attività confrontando i risultati ottenuti con quelli attesi.
\subsubsection{Descrizione}
Nel processo di validazione viene eseguito il test completo sul sistema, in modo tale da accertarsi che il prodotto realizzato sia conforme alle aspettative e alle necessità del committente. 
Per essere certi che il test dia il risultato atteso occorre aver già svolto correttamente il processo di verifica, in modo che tutte le unità del sistema permettano di effettuare un test completo.
\subsubsection{Attività}
\paragraph{Implementazione del processo}

\subsubsection{Metriche}
Al momento, non sono ancora state individuate delle metriche per quanto riguarda il processo di validazione.
\subsubsection{Strumenti}
Al momento, non sono ancora stati individuati degli strumenti per quanto riguarda il processo di validazione.
