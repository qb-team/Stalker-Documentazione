\subsubsection{Codifica}
\paragraph{Scopo}\mbox{}\\ \\
Lo scopo della codifica consiste nell'implementare le specifiche del prodotto definite nelle attività precedenti.

\paragraph{Note sulla sezione di Codifica}\mbox{}\\ \\
Questa sezione viene redatta durante la fase iniziale del progetto e il suo contenuto è limitato alle disponibilità di conoscenza attuali.
Verrà ampliata in seguito per l'attività di \textbf{Codifica}.

%\paragraph{Intestazione dei file}\mbox{}\\ \\
%Questa sezione viene redatta durante la fase iniziale del progetto e il suo contenuto è limitato alle disponibilità di conoscenza attuali.
%Verrà ampliata in seguito per l'attività di \textbf{Codifica}.

\paragraph{Commenti}\mbox{}\\ \\
Il codice sorgente deve essere adeguatamente commentato, sia per chi lo scrive, sia per chi lo deve leggere in futuro.
È quindi necessario che ogni porzione di codice significativa sia dotata di commenti esplicativi, in particolar modo se le istruzioni usate non sono comuni.
I commenti devono essere chiari ma al contempo sintetici, per non distrarre dal contesto.
Ogni volta che viene scritto un nuovo frammento di codice, questo deve avere una sezione di codice di documentazione (in molti linguaggi indicato con /** */ per un commento a più righe e /// per un commento a singola riga).
Questa sezione deve saper dire:
\begin{itemize}
    \item Chi ha scritto il codice;
    \item Chi lo ha modificato l'ultima volta;
    \item In che contesto viene utilizzato il codice che segue;
    \item Se è un metodo/funzione, che argomenti richiede in input, che valori restituisce e se può lanciare eccezioni.
\end{itemize}
I commenti devono essere mantenuti.

\paragraph{Nomi dei file e delle variabili}\mbox{}\\ \\
I nomi dei file:
\begin{itemize}
    \item Devono essere univoci;
    \item Devono rispettare le condizioni del linguaggio che contengono, se presenti.
\end{itemize}
I nomi delle variabili:
\begin{itemize}
    \item Devono essere chiari e descrittivi (è vietato usare variabili temp, tmp, x, ecc.) e possibilmente brevi;
    \item Non devono essere simili fra di loro, per evitare confusione;
    \item Se formati da più parole si devono scrivere usando l'underscore come separatore (\glo{Snake Case}) oppure separando i termini con una lettera maiuscola ad ogni inizio parola (\glo{Camel Case});
    \item Devono rispettare le condizioni del linguaggio in cui vengono usate, se presenti.
\end{itemize}

\paragraph{Stile di codifica}\mbox{}\\ \\
Il gruppo qbteam al fine di garantire una maggiore leggibilità, uniformità e manuntenibilità del codice prodotto, ha deciso di utilizzare uno standard di codifica. Per quanto riguarda la parte di applicazione mobile si è utilizzato il plugin checkstyle \href{https://checkstyle.sourceforge.io}{https://checkstyle.sourceforge.io} per \glo{maven} e per \glo{gradle}. Invece per la parte di interfaccia web il gruppo ha fatto riferimento al plugin tslint \href{https://palantir.github.io/tslint/}{https://palantir.github.io/tslint/}. 
Per emtrambi i plugin, si è deciso di configurare lo stile di codifica seguendo le seguenti regole:
\begin{itemize}
	\item AbstractClassName;
	\item BooleanExpressionComplexity;
	\item ConstantName;
	\item IllegalType;
	\item Indentation;
	\item LineLength;
	\item MethodName;
	\item TypeName.
\end{itemize}

Per quanto riguarda l'intestazione dei file, ogni file deve rispettare le seguenti regole: 
\begin{itemize}
	\item nome e cognome di chi ha prodotto il file;
	\item data di creazione;
	\item data ultima modifica;
	\item breve descrizione del contenuto del file.
\end{itemize} 

Il codice e i commenti devono essere scritti in lingua inglese.

