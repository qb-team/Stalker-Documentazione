\paragraph{Progettazione}
\subparagraph*{Scopo}
La progettazione, svolta dai Progettisti, ha lo scopo di soddisfare i requisiti stabiliti nel documento \AdR{} per trovare una soluzione accettabile per tutti gli stakeholder.
Per fare ciò, si cerca di seguire un approccio sintetico dove si pensa prima all’architettura del prodotto e poi al codice. 
Inoltre, per la \glo{Technology Baseline} è prevista la realizzazione di un \glo{Proof of Concept}.

\subparagraph*{Aspettative}
L'aspettativa di questa attività è di definire l’architettura logica del prodotto creando parti con specifiche chiare, coese e realizzabili con risorse sostenibili e mantenibili. L'architettura deve avere determinate caratteristiche per:
\begin{itemize}
	\item Soddisfare tutti i requisiti degli stakeholder;
	\item Riuscire a gestire gli errori quando presenti;
	\item Garantire che venga eseguito il suo compito nel modo corretto;
	\item Cercare di ridurre i tempi di manutenzione;
	\item Avere componenti semplici, coesi, incapsulati e di basso accoppiamento tra di loro.
\end{itemize}

\subparagraph*{Descrizione}
La progettazione consiste nei seguenti compiti:
\begin{itemize}
	\item Controllare la complessità del prodotto suddividendo il sistema in parti di complessità trattabile;
	\item Soddisfare i requisiti garantendo qualità;
	\item Definire un’architettura logica del prodotto che dovrà avere determinate caratteristiche;
	\item Avere una progettazione dettagliata con la consapevolezza di fermarsi quando la suddivisione porterà più svantaggi che benefici;
	\item Per prendere confidenza con le tecnologie, sarà necessaria la realizzazione di un \glo{Proof of Concept} per poter affrontare la successiva fase di Codifica. Quest'ultimo sarà la nostra base dalla quale potremo attuare i successivi incrementi pianificati nel documento \PdP{}.
\end{itemize}

\subparagraph*{Architettura}
La realizzazione dell’architettura del prodotto è divisa in due parti:
\begin{itemize}
	\item \glo{Technology Baseline};
	\item \glo{Product Baseline}.
\end{itemize}

\subparagraph*{Technology Baseline}
La Technology Baseline deve dimostrare l’adeguatezza dell’architettura tramite un \glo{Proof of Concept} che rappresenta la baseline per lo sviluppo. 
La Technology Baseline deve includere:
\begin{itemize}
	\item \textbf{Le tecnologie}: Dove vengono scelte le tecnologie che \Gruppo{} utilizzerà per la realizzazione del prodotto finale. Successivamente verranno ampliate le conoscenze di esse tramite una ricerca di documentazione e materiali utili per il loro apprendimento. Verranno date delle motivazioni per la scelta delle tecnologie adottate con i rispettivi pregi e difetti;
	\item \textbf{\glo{Librerie} e \glo{Framework}}: Durante la scomposizione in sottosistemi verrà considerata la disponibilità di librerie e di framework per ottenete un maggiore facilità e affidabilità;
	\item \textbf{Creazione del \glo{Proof of Concept}}: Realizzazione di un eseguibile che permette di rappresentare e validare la Technology Baseline;
	\item \textbf{Diagrammi UML}: Viene utilizzato il linguaggio di modellazione UML per definire delle regole e dei vincoli con l'obiettivo di supportare la progettazione del sistema software. Il diagramma UML utilizzato nella Technology Baseline è il seguente:
	\begin{itemize}
	\item \textbf{Diagrammi dei casi d'uso}: Utilizzati per individuare i requisiti funzionali, descrivono le interazioni tra l'utente e il sistema.
	\end{itemize}
	\item \textbf{Test di integrazione}: Test che vengono utilizzati per verificare che le diverse componenti del
	sistema interagiscano tra di loro in modo corretto e in modo tale che siano conformi con i requisiti associati;
	\item \textbf{Tracciamento delle componenti}: Utilizzata per mettere in relazione requisiti e i componenti che li soddisfano.
\end{itemize}

\subparagraph*{Product Baseline}
La Product Baseline illustrerà la baseline architetturale del prodotto, in coerenza con la Technology Baseline e la sua definizione è fondamentale per l'attività di progettazione.
Essa deve includere un allegato che contenga:
\begin{itemize}
	\item \textbf{Diagrammi delle classi}: Utilizzati per descrivere i tipi di oggetti che vengono usati all'interno di un sistema;
	\item \textbf{Diagrammi di sequenza}: Utilizzati per la collaborazione di un insieme di oggetti che devono implementare collettivamente un comportamento;
	\item \textbf{Diagrammi di attività}: Utilizzati per rappresentare gli aspetti dinamici dei casi d'uso, mostrando l'esecuzione di un'attività;
	\item \textbf{Diagrammi di package}: Utilizzati per rappresentare un insieme di classi che condividono la stessa causa di cambiamento;
	\item \textbf{Contestualizzazione dei design pattern adottati}: Per avere un'organizzazione architetturale che riesca a gestire dei problemi ricorrenti è necessario capire quali design pattern il gruppo dovrà utilizzare e le funzionalità di ciascuno di essi;	
	\item \textbf{Requisiti del test di accettazione della produzione}: Bisogna effettuare dei test su tutto il SUT, che viene considerato black box, relativi agli use cases e ai requisiti concordati con il proponente.
\end{itemize}

Il gruppo \Gruppo{}, per i sopracitati diagrammi UML, utilizza le seguenti norme:
 
\subsubparagraph*{Diagrammi delle classi}\mbox{}\\
Per tutte le regole standard di UML riguardanti i diagrammi delle classi, consultare il seguente link:
\\
 \url{https://www.math.unipd.it/~tullio/IS-1/2019/Dispense/E01b.pdf}
\begin{itemize}
	\item Il nome della classe deve essere scritto in grassetto;
	\item Per indicare un'interfaccia si usa la keyword interface tra parentesi angolate, sopra al nome della classe;
	\item Per indicare una classe astratta si usa la keyword abstract tra parentesi graffe, sopra al nome della classe;
	\item Nella classe concreta non vengono riportate le operazioni ereditate dalle interfacce o classi astratte;
	\item Si devono applicare tutte le convenzioni del linguaggio UML. 	
\end{itemize}

\subsubparagraph*{Diagrammi di sequenza} \mbox{}\\
Per tutte le regole standard di UML riguardanti i diagrammi di sequenza, consultare il seguente link:
\\
\url{https://www.math.unipd.it/~tullio/IS-1/2019/Dispense/E02a.pdf}


\subsubparagraph*{Diagrammi di attività} \mbox{}\\
Per tutte le regole standard di UML riguardanti i diagrammi di attività, consultare il seguente link:
\\ \url{https://www.math.unipd.it/~tullio/IS-1/2019/Dispense/E02b.pdf}

\subsubparagraph*{Diagrammi di package}\mbox{}\\
Per tutte le regole standard di UML riguardanti i diagrammi dei package, consultare il seguente link: 
\\
\url{https://www.math.unipd.it/~tullio/IS-1/2019/Dispense/E01c.pdf}

\begin{itemize}
	\item Per indicare un'interfaccia all'interno di un package si usa la keyword interface tra parentesi angolate, sopra al nome della classe.
	\item Per indicare una classe astratta all'interno di un package si usa la keyword abstract tra parentesi graffe, sopra al nome della classe.
\end{itemize}
































