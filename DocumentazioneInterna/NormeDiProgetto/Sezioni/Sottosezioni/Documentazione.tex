\subsection{Documentazione}
\subsubsection{Scopo}
Lo scopo di questa sezione è di redigere e standardizzare i documenti prodotti durante tutto il ciclo di vita del software. 
Di conseguenza ci si aspetta di avere:
\begin{itemize}
\item Una struttura ben organizzata e con una facile navigabilità;
\item Una serie di norme tipografiche da rispettare.
\end{itemize}
I documenti possono essere consultati nel seguente repository di \glo{GitHub}:
\\
\url{https://github.com/qb-team/Stalker-Documentazione}.
\subsubsection{Aspettative}
Questa sezione serve a fornire degli strumenti e indicazioni comuni ai membri del gruppo per arrivare a una produzione di documentazione coesa e professionale.

\subsubsection{Descrizione}
La documentazione prodotta dal \textit{processo di documentazione} registra tutte le informazioni generate durante il ciclo di vita di un processo o un'attività.
Il \textit{processo di documentazione} deve contenere un set di attività volte a garantire la produzione, verifica e mantenimento della documentazione.

\subsubsection{Attività}

\paragraph{Implementazione del processo}
\subparagraph*{Ciclo di vita}
Ogni documento prima di essere presentato deve passare per tre stati fondamentali:
\begin{enumerate}
\item \textbf{Stesura del documento}: Creazione del documento e stesura con il linguaggio \LaTeX;
\item \textbf{Verifica del Documento}: Il documento viene assegnato ad un verificatore, il cui lavoro è controllare che il documento rispetti gli standard definiti;
\item \textbf{Approvazione del documento}: In caso la verifica risulti positiva, il documento viene consegnato al \Responsabile{} per l'approvazione al rilascio.
\end{enumerate}

\subparagraph*{Studio di Fattibilità}
Il \Responsabile{} ha il compito di convocare tutti i membri del gruppo \Gruppo{} per discutere le varie tematiche riguardanti i capitolati d'appalto disponibili.
Lo \SdF{} viene redatto dagli analisti, i quali devono analizzare il materiale disponibile e inoltre tenere in considerazione anche ciò che è stato discusso nelle riunioni sul tema.\\
Il documento è strutturato in più sezioni, ognuna riguardante un capitolato d'appalto.
Per ogni capitolato verranno trattati i seguenti punti:
\begin{itemize}
\item Titolo del capitolato:
	\begin{itemize}
	\item Nome del capitolato;
	\item Azienda proponente;
	\item Committenti.
	\end{itemize}
\item Descrizione del capitolato:
	\begin{itemize}
	\item Breve riassunto del prodotto da realizzare, secondo le specifiche richieste dal proponente.
	\end{itemize}
\item Prerequisiti e tecnologie coinvolte:
	\begin{itemize}
	\item Elenco delle tecnologie da utilizzare, con eventuali riferimenti per ulteriori approfondimenti o spiegazioni del contesto applicativo;
	\item In alcuni casi l'azienda proponente consiglia l'utilizzo di certe tecnologie.
	\end{itemize}
\item Vincoli:
	\begin{itemize}
	\item Richieste generali, tecniche e/o organizzative da parte dell'azienda proponente.
	\end{itemize}
\item Aspetti positivi:
	\begin{itemize}
	\item Vengono descritti gli aspetti ritenuti positivi dai membri del gruppo \Gruppo{} del capitolato.
	Possono essere considerati aspetti positivi, ad esempio, l'apprendimento di nuove tecnologie e/o linguaggi, disponibilità di contatto e collaborazione offerta dal proponente e documentazione disponibile online riguardante le tecnologie coinvolte.
	\end{itemize}
\item Aspetti critici:
	\begin{itemize}
	\item Vengono descritti gli aspetti ritenuti critici dai membri del gruppo \Gruppo{} del capitolato.
	Possono essere considerati aspetti critici, ad esempio, l'eccessiva mole di tecnologie da apprendere, i requisiti di vincolo da tenere in considerazione e la scarsa presenza di documentazione online riguardante le tecnologie coinvolte.
	\end{itemize}
\item Conclusioni:
	\begin{itemize}
	\item Valutazione finale motivata dai membri del gruppo \Gruppo{} nella quale vengono esposte le ragioni di interesse o disinteresse nella scelta o meno del capitolato.
	\end{itemize}
\end{itemize}
Tutte queste informazioni appena elencate vengono raccolte nel documento interno \glo{\SdF{}}, sottoposto al processo di \glo{verifica{}} da parte dei verificatori.
\subparagraph*{Piano di Progetto}
\documentclass[a4paper, oneside, dvipsnames, table]{article}
\usepackage{../../Utilita/Stiletemplate}
\usepackage{hyperref}
\usepackage{fancyhdr}
\usepackage[italian]{babel}
\usepackage{pdflscape}
\usepackage[raggedright]{titlesec}
\usepackage{blindtext}
\titleformat{\paragraph}[hang]{\normalfont\normalsize\bfseries}{\theparagraph}{1em}{}
\titlespacing*{\paragraph}{0pt}{3.25ex plus 1ex minus .2ex}{0.5em}

\newcommand{\Data}{2020-05-25}

\newcommand{\Titolo}{Verbale Riunione \Data}

\newcommand{\Redattori}{\PF{}}

\newcommand{\Verificatori}{\AT{}}

\newcommand{\Approvatore}{\CE{}}

\newcommand{\Distribuzione}{\VT{} \newline \CR{} \newline Gruppo \Gruppo{}}

\newcommand{\Uso}{Interno}

\newcommand{\DescrizioneDoc}{Questo documento si occupa di riportare quanto discusso nella riunione del \Data}

\newcommand{\pathimg}{../../../Utilita/Immagini/qbteam.png}

\newcommand{\Versionedoc}{1.0.0}
% scritto da \DF{},\AT{}

% info generali 
\newcommand{\NomeProgetto}{\textit{Stalker}}

% fornitore
\newcommand{\Gruppo}{\textit{qbteam}}
\newcommand{\Mail}{qbteamswe@gmail.com}
% \newcommand{\pathimg}{Immagini/qbteam.png}

% committenti
\newcommand{\Committente}{\VT \newline \CR}
\newcommand{\VT}{Prof. Vardanega Tullio}
\newcommand{\CR}{Prof. Cardin Riccardo}

% proponenti
\newcommand{\Proponente}{\textit{Imola Informatica}}
\newcommand{\ZD}{Zanetti Davide}
\newcommand{\CT}{Cardona Tommaso}

% qbteam
\newcommand{\AT}{Azzalin Tommaso}
\newcommand{\DF}{Drago Francesco}
\newcommand{\BR}{Baratin Riccardo}
\newcommand{\MC}{Mattei Christian}
\newcommand{\PF}{Perin Federico}
\newcommand{\CE}{Cisotto Emanuele}
\newcommand{\SE}{Salmaso Enrico}
\newcommand{\LD}{Lazzaro Davide}

% ruoli
\newcommand{\Responsabile}{Responsabile di Progetto}
\newcommand{\Amministratore}{Amministratore di Progetto}

% documenti

\newcommand{\SdF}{Studio di Fattibilità}
\newcommand{\SdFv}[1]{\textit{Studio di Fattibilità {#1}}}
\newcommand{\PdQ}{Piano di Qualifica}
\newcommand{\PdQv}[1]{\textit{Piano di Qualifica {#1}}}
\newcommand{\PdP}{Piano di Progetto}
\newcommand{\PdPv}[1]{\textit{Piano di Progetto {#1}}}
\newcommand{\NdP}{Norme di Progetto}
\newcommand{\NdPv}[1]{\textit{Norme di Progetto {#1}}}
\newcommand{\AdR}{Analisi dei Requisiti}
\newcommand{\AdRv}[1]{\textit{Analisi dei Requisiti {#1}}}
\newcommand{\Glossario}{Glossario}
\newcommand{\Glossariov}[1]{\textit{Glossario {#1}}}
\newcommand{\MM}{Manuale Manutentore}
\newcommand{\MMv}[1]{\textit{Manuale Manutentore {#1}}}
\newcommand{\MU}{Manuale Utente}
\newcommand{\MUv}[1]{\textit{Manuale Utente {#1}}}

% comandi generali
\newcommand{\glo}[1]{#1\ap{G}}

\setlength{\parindent}{-0.1em}


\begin{document}

\copertina{}
\newpage


\fancyPdP{}

\section*{Registro delle modifiche}
{
\rowcolors{2}{grigetto}{white}
\renewcommand{\arraystretch}{1.5}
\centering
\begin{longtable}{C{2cm} C{2cm}  C{3cm}  C{3cm} C{4.5cm}}
\rowcolor{rossoep}
\textcolor{white}{\textbf{Versione}} & \textcolor{white}{\textbf{Data}} & \textcolor{white}{\textbf{Nominativo}} & \textcolor{white}{\textbf{Ruolo}} & \textcolor{white}{\textbf{Descrizione}}\\	
\endhead
1.2.0 & 2020-03-07 & \CE{} & Verificatore & Verifica del documento. \\

1.1.4 & 2020-02-16 & \SE{} & Amministratore & Aggiornati 4.3.2 e 4.3.3 VERIFICATO DA \LD{}\\

1.1.3 & 2020-02-15 & \SE{} & Amministratore & Aggiornato 4.2.2 VERIFICATO DA \BR{}\\

1.1.5 & 2020-02-16 & \SE{} & Amministratore & Aggiunto 2.2.5.5 VERIFICATO DA \BR{}. \\

1.1.4 & 2020-02-15 & \SE{} & Amministratore & Aggiornato 4.2.2 VERIFICATO DA \BR{}. \\

1.1.3 & 2020-02-14 & \SE{} & Amministratore & Aggiunta 2.2.6 VERIFICATO DA \LD{}. \\

1.1.2 & 2020-02-12 & \SE{} & Amministratore & Aggiornamento 4.2.4 VERIFICATO DA \LD{}. \\ 

1.1.1 & 2020-02-12 & \BR{} & Amministratore & Aggiornamento 3.1 VERIFICATO DA \LD{}. \\ 

1.1.0 & 2020-02-12 & \LD{} & Verificatore & Verifica VERIFICATO DA \LD{}.  \\ 

1.0.3 & 2020-02-12 & \BR{} & Amministratore & Aggiunto e verificato 3.4.2 VERIFICATO DA \LD{}. \\ 

1.0.2 & 2020-02-12 & \SE{} & Amministratore & Aggiunte e verificate metriche SFIN e SFOUT VERIFICATO DA \LD{}. \\ 

1.0.1 & 2020-02-12 & \SE{} & Amministratore & Modificati e verificati i paragrafi 2.2.4.1 e 2.2.4.2 VERIFICATO DA \LD{}. \\ 

1.0.0 & 2020-01-13 & \AT{} & Amministratore & Approvazione per il rilascio.  \\

0.2.0 & 2020-01-13 & \PF{}, \CE{} & Verificatori & Verifica documento.  \\ 

0.1.9 & 2020-01-13 & \CE{} & Amministratore & Aggiunta del template dei digrammi UML dei casi d'uso. \\

0.1.8 & 2020-01-13 & \BR{} & Amministratore & Modifica dei casi d'uso d'errore, test di sistema. \\

0.1.7 & 2020-01-13 & \AT{} & Amministratore & Revisione Introduzione, Processo di fornitura, sviluppo, attività di codifica e di progettazione, processi organizzativi, documentazione. \\

0.1.6 & 2020-01-12 & \MC{} & Amministratore & Revisione e modifica strutturale dei capitoli del documento. \\

0.1.5 & 2020-01-12 & \AT{} & Amministratore & Modifica Processi Primari. \\

0.1.4 & 2020-01-11 & \MC{} & Amministratore & Stesura capitolo gestione della qualità. \\

0.1.3 & 2020-01-10 & \MC{} & Amministratore & Revisione documentazione nei Processi di supporto. \\

0.1.2 & 2020-01-06 & \AT{} & Amministratore & Modifica del processo di verifica, validazione, piano di qualifica. \\

0.1.1 & 2020-01-06 & \AT{} & Amministratore & Modifica del processo di verifica. \\

0.1.0 & 2019-12-23 & \PF{}, \CE{} & Verificatori & Verifica del documento. \\

0.0.11 & 2019-12-22 & \PF{} & Amministratore & Stesura delle sottosezioni introduzione e scopo della sezione processi di sviluppo. \\

0.0.10 & 2019-12-22 & \PF{}  & Amministratore & Stesura Sviluppo dei processi primari. \\

0.0.9 & 2019-12-21 & \PF{} & Amministratore & Stesura delle sottosezioni Gestione della qualità, Verifica e Validazione della sezione processi di supporto. \\

0.0.8 & 2019-12-20 & \MC{} & Amministratore & Modifica descrizione repository, Studio di fattibilità. \\

0.0.7 & 2019-12-19 & \SE{} & Amministratore & Revisione del documento fino ad ora redatto. \\

0.0.6 & 2019-12-19 & \CE{} & Amministratore & Modificata la sezione dei casi d’uso con le decisioni prese per la loro nomenclatura il 2019-12-18. \\

0.0.5 & 2019-12-15 & \SE{} & Amministratore & Aggiunta parte di progettazione. \\

0.0.4 & 2019-12-15 & \BR{}, \PF{}  & Amministratori & Aggiunta parte processi organizzativi, gestione delle risorse prodotte. \\

0.0.3 & 2019-12-15 & \MC{} & Amministratore & Stesura studio di fattibilità. \\

0.0.2 & 2019-12-14 & \CE{} & Amministratore & Aggiunta parte relativa all’Analisi dei requisiti. \\

0.0.1 & 2019-12-14 & \CE{} & Amministratore & Creato il documento. \\
		
\end{longtable}
}


\clearpage
\tableofcontents
\clearpage

\renewcommand{\figurename}{Diagramma}

\renewcommand{\listfigurename}{Elenco diagrammi}

\renewcommand{\listtablename}{Elenco tabelle}

\listoffigures
\clearpage
\listoftables
\clearpage

\section{Introduzione}
\subsection{Scopo del documento}
Lo scopo del documento è quello di descrivere in maniera dettagliata i requisiti e i casi d'uso che sono stati individuati durante lo studio del progetto Stalker.

\subsection{Scopo generale del prodotto}
L'obiettivo del prodotto \NomeProgetto{} di \Proponente{} è la creazione di un sistema software composto di un applicativo per cellulare e di un server, con cui interagire tramite un'interfaccia utente. La necessità nasce dal bisogno di adempiere alle normative vigenti in tema di sicurezza.
Le due componenti del sistema software, applicativo e server, devono soddisfare i seguenti obiettivi rispettivamente di:
\begin{itemize}
\item Tracciare e registrare i \glo{movimenti} di un utente in un \glo{luogo di tracciamento} di un'\glo{organizzazione}, siano essi autenticati da credenziali di un'\glo{organizzazione} oppure visitatori anonimi, il tutto nel rispetto della normativa sulla privacy;
\item Poter visionare gli accessi degli utenti autenticati e visionare il numero di visitatori anonimi all'interno di un luogo.
\end{itemize}

\subsection{Glossario}
Al fine di evitare ambiguità fra i termini, e per avere chiare fra tutti gli stakeholder le terminologie utilizzate per la realizzazione del presente documento, il gruppo \Gruppo{} ha redatto un documento denominato \Glossariov{1.0.0}.
In tale documento, sono presenti tutti i termini tecnici, ambigui, specifici del progetto e scelti dai membri del gruppo con le loro relative definizioni.
Un termine presente nel \Glossariov{1.0.0} e utilizzato in questo documento viene indicato con un apice \ap{G} alla fine della parola.

\subsection{Riferimenti}

\subsubsection{Normativi}
\begin{itemize}
\item \NdPv{1.0.0};
\item \textit{VE\_2019\_12\_13}.
\end{itemize}

\subsubsection{Informativi}
\begin{itemize}
\item \SdFv{1.0.0};
\item \textbf{Slide del capitolato C5 - Stalker}: \\ \url{https://www.math.unipd.it/~tullio/IS-1/2019/Progetto/C5.pdf}
\item \textbf{Guide to the Software Engineering Body of Knowledge};
\item \textbf{Software Engineering (10th edition) - Ian Sommerville}.
\end{itemize}
\clearpage
\section{Analisi dei Rischi}
La gestione dei rischi è un processo al quale il gruppo \Gruppo{} dà molto importanza. Questo perché incorrere in rischi potrebbe equivalere al danneggiamento del progetto, sia nella sua \glo{organizzazione} e sia nella sua qualità.
Si cerca quindi di fare una previsione dei problemi che si potrebbero verificare durante l'intero corso del progetto e, per ogni rischio identificato, si cerca una soluzione per poterlo evitare.

\subsection{Fasi della gestione dei rischi}
Il gruppo intende seguire i seguenti step nel processo di gestione dei rischi:
\begin{itemize}
	\item \textbf{Identificazione del rischio}: Questo è il primo step del processo e ci serve per identificare i rischi che potrebbero portare a dei problemi durante l'avanzamento del progetto; 
	\item \textbf{Analisi dei rischi}: Dopo aver individuato i rischi nello step precedente, per ognuno di essi viene valutata la probabilità che si verifichi e le conseguenze negative che potrebbe portare;
	\item \textbf{Pianificazione del rischio}: Nella pianificazione del rischio si sviluppano dei piani per sapere quali rimedi vanno intrapresi nel momento in cui i rischi si verificano. In tale maniera si riuscirà a risolvere i problemi prima che essi si aggravino;
	\item \textbf{Monitoraggio del rischio}: Nell'ultimo step della gestione del rischio viene verificato che le ipotesi relative ai rischi non abbiano subito delle variazioni. Quindi si cerca di valutare periodicamente la probabilità che il rischio si verifichi e i suoi possibili effetti, migliorando le strategie adottate per la loro risoluzione.
\end{itemize}

\subsection{Tipologia del rischio}
Ci sono 5 tipi di rischi che il gruppo \Gruppo{} terrà in considerazione. 
\\Ad ogni rischio verrà assegnato un codice identificativo:
\begin{itemize}
	\item Rischi Tecnologici [RT];
	\item Rischi Organizzativi [RO];
	\item Rischi Personali [RP];
	\item Rischi dei Requisiti [RR];
	\item Rischi di Stima [RS].
\end{itemize}

\subsection{Tabella dei rischi}
Nella seguente tabella vengono elencati i rischi che il gruppo \Gruppo{} potrebbe incontrare durante l'intero ciclo di vita del progetto.
Ogni riga della tabella corrisponde ad un rischio ed è composta da:
\begin{itemize}
	\item Codice [codice del tipo + numero sequenziale] e Nome del rischio;
	\item Descrizione;
	\item Rilevamento;
	\item Piano di Contingenza.
\end{itemize}

{
\rowcolors{2}{grigetto}{white}
\renewcommand{\arraystretch}{2}
\centering
\begin{longtable}{ C{2cm} C{4.5cm} C{4.5cm} C{4.5cm}}
\rowcolor{rossoep}
\textcolor{white}{\textbf{Codice Nome}} & \textcolor{white}{\textbf{Descrizione}} & \textcolor{white}{\textbf{Rilevamento}} &  \textcolor{white}{\textbf{Piano di Contingenza}}\\	

RT1 Inesperienza con le tecnologie & Il gruppo dovrà relazionarsi con tecnologie mai utilizzate precedentemente e quindi servirà del tempo per poterle utilizzare nel modo corretto & Ogni componente del gruppo sarà consapevole di saper usare o no una determinata tecnologia & Ogni componente del gruppo che ha acquisito una certa dimestichezza nell'utilizzo di una tecnologia cercherà di aiutare i componenti del gruppo che hanno più difficoltà con essa \\

RP1 Attriti Interni & Durante i verbali o incontri interni, qualche componente del gruppo potrebbe essere indisponibile & Ciascun componente del gruppo comunicherà la sua assenza nel giorno dei verbali o incontri & Si aggiornerà continuamente un calendario condiviso permettendo al responsabile di fissare gli incontri in giorni e in orari in cui tutti i componenti di qbteam (o la maggior parte di essi) siano disponibili \\ 

RP2 Comunicazione Esterna & Si potrebbero avere delle difficoltà nel comunicare con il proponente esterno & Il proponente non risponderà alle mail del responsabile di qbteam in tempi brevi & Si cercherà di far presente al proponente Davide Zanetti che la comunicazione tra fornitore e cliente è molto importante per ridurre i tempi e quindi i costi \\

RR1 Disattenzione nella definizione dei requisiti & I componenti del gruppo potrebbero interpretare male qualche requisito & I verificatori si accorgono che un requisito non è stato definito nel modo corretto & Si cercherà di condurre una precisa analisi dei requisiti chiarendo ogni dubbio di ciascuno dei componenti del gruppo \\

RS1 Stime errate delle attività & Si potrebbero fare delle stime sbagliate sui costi, tempi e risorse utilizzate delle attività & Ciascun componente comunicherà al responsabile se non avrà rispettato una delle stime di qualche attività & Si cercherà di condurre una pianificazione e un preventivo attento per essere più coerenti possibili \\

% Ecco basta che riempi i successivi Rischi, ovviamente rinominando il nome
% Te ne ho messi 5 intanto, poi se son di più o di meno non importa

RO1 Non rispetto delle milestone imposte & Potrebbe accadere che per impegni personali o mancanza delle conoscenze qualche membro del gruppo impieghi più tempo del previsto non riuscendo a portare a termine il compito all'interno della scadenza assegnatagli portando il team a sforare una milestone concordata precedentemente & Il componente che si trova in difficoltà avrà il compito di comunicarlo al gruppo, inoltre anche gli altri membri possono valutare se un componente procede a rilento e comunicare il problema al responsabile & Tutto il gruppo dovrà agire per offrire supporto in modo tale da terminare il lavoro entro la scadenza\\

RO2 Eccesso o difetto nell'assegnazione delle scadenze & Data la presenza di numerosi scenari nei quali abbiamo poca esperienza potrebbe accadere una errata assegnazione di scadenze che si rivelano sovra-stimate o sotto-stimate in relazione alla difficoltà del problema da risolvere & I componenti a cui è assegnato lo svolgimento di un compito devono riferire se la scadenza a loro imposta sia ragionevole dopo aver approfondito e compreso a fondo la difficoltà del lavoro che devono portare a termine & Se i membri del gruppo assegnati a un compito riferiscono al responsabile l'errore nella valutazione delle tempistiche si procede immediatamente ad una nuova pianificazione alla luce delle affermazioni dei membri coinvolti nel compito.\\

RO3 Assenza di comunicazione gruppo-proponente & Potrebbe accadere che presi dal lavoro si trascuri la comunicazione con Imola Informatica & Tutti i membri devono ricordarsi di mantenere un dialogo con l'azienda cercando di raccogliere dubbi e portando i nostri avanzamenti & In caso di assenza di comunicazione gruppo-azienda il responsabile deve fissare una data per un incontro, prima della prima scadenza di revisione, entro la quale il team deve impegnarsi a raccogliere domande, se presenti, e proporre tutti i progressi che sono stati fatti per ricevere feedback essenziali per la corretta riuscita del prodotto. \\

RO4 Impossibilità di stabilire un incontro tra i membri del gruppo & Potrebbe accadere che dato il numero di componenti del gruppo stabilire un incontro in cui tutti siano presenti risulti difficile e in caso si riesca non abbastanza immediato come servirebbe & Il gruppo tiene una tabella con le proprie disponibilità in settimana e si possono vedere gli orari in cui tutti possono essere presenti, oppure si può notare che in nessun giorno tutti sono liberi & Per discutere si possono usare servizi di chiamata come Hangouts, inoltre molto raramente sarà strettamente necessaria la presenza di tutti i membri del gruppo quindi diventa molto più facile organizzarsi a sotto-gruppi e stabilire un incontro.\\


\end{longtable}
}

\subsubsection{Tabella del Grado del Rischio}
Come descritto nelle fasi della gestione del rischio, è importante valutare il grado del rischio, ovvero stabilire la probabilità e la gravità che il rischio potrebbe avere durante il progetto.
//Ogni colonna riporterà il codice di ciascuno dei rischi analizzati nella tabella precedente e sarà composta da:
\begin{itemize}
	\item Codice del Rischio;
	\item Frequenza;
	\item Gravità;
\end{itemize}

{
	\rowcolors{2}{grigetto}{white}
	\renewcommand{\arraystretch}{2}
	\centering
	\begin{longtable}{ C{2cm} C{3cm} C{3cm}}
		\rowcolor{rossoep}
		\textcolor{white}{\textbf{Codice}} & \textcolor{white}{\textbf{Frequenza}} & \textcolor{white}{\textbf{Gravità}}\\	
		
		RT1 & Alta & Media\\
		
		RP1 & Media & Media\\
		
		RP2 & Media & Alta\\
		
		RR1 & Alta & Alta \\
		
		RS1 & Media & Bassa \\
		
		RO1 & Media & Alta \\
		
		RO2 & Media & Media \\
		
		RO3 & Bassa & Media \\
		
		RO4 & Alta & Bassa \\
		
	\end{longtable}
}
{
\rowcolors{2}{grigetto}{white}
\renewcommand{\arraystretch}{2}
\centering
\begin{longtable}{ C{2cm} C{4.5cm} C{4.5cm} C{4.5cm}}
\rowcolor{rossoep}
\textcolor{white}{\textbf{Codice Nome}} & \textcolor{white}{\textbf{Descrizione}} & \textcolor{white}{\textbf{Rilevamento}} &  \textcolor{white}{\textbf{Piano di Contingenza}}\\	

RT1 Inesperienza con le tecnologie & Il gruppo dovrà relazionarsi con tecnologie mai utilizzate precedentemente e quindi servirà del tempo per poterle utilizzare nel modo corretto & Ogni componente del gruppo sarà consapevole di saper usare o no una determinata tecnologia & Ogni componente del gruppo che ha acquisito una certa dimestichezza nell'utilizzo di una tecnologia cercherà di aiutare i componenti del gruppo che hanno più difficoltà con essa \\

RP1 Attriti Interni & Durante i verbali o incontri interni, qualche componente del gruppo potrebbe essere indisponibile & Ciascun componente del gruppo comunicherà la sua assenza nel giorno dei verbali o incontri & Si aggiornerà continuamente un calendario condiviso permettendo al responsabile di fissare gli incontri in giorni e in orari in cui tutti i componenti di qbteam (o la maggior parte di essi) siano disponibili \\ 

RP2 Comunicazione Esterna & Si potrebbero avere delle difficoltà nel comunicare con il proponente esterno & Il proponente non risponderà alle mail del responsabile di qbteam in tempi brevi & Si cercherà di far presente al proponente Davide Zanetti che la comunicazione tra fornitore e cliente è molto importante per ridurre i tempi e quindi i costi \\

RR1 Disattenzione nella definizione dei requisiti & I componenti del gruppo potrebbero interpretare male qualche requisito & I verificatori si accorgono che un requisito non è stato definito nel modo corretto & Si cercherà di condurre una precisa analisi dei requisiti chiarendo ogni dubbio di ciascuno dei componenti del gruppo \\

RS1 Stime errate delle attività & Si potrebbero fare delle stime sbagliate sui costi, tempi e risorse utilizzate delle attività & Ciascun componente comunicherà al responsabile se non avrà rispettato una delle stime di qualche attività & Si cercherà di condurre una pianificazione e un preventivo attento per essere più coerenti possibili \\

% Ecco basta che riempi i successivi Rischi, ovviamente rinominando il nome
% Te ne ho messi 5 intanto, poi se son di più o di meno non importa

RO1 Non rispetto delle milestone imposte & Potrebbe accadere che per impegni personali o mancanza delle conoscenze qualche membro del gruppo impieghi più tempo del previsto non riuscendo a portare a termine il compito all'interno della scadenza assegnatagli portando il team a sforare una milestone concordata precedentemente & Il componente che si trova in difficoltà avrà il compito di comunicarlo al gruppo, inoltre anche gli altri membri possono valutare se un componente procede a rilento e comunicare il problema al responsabile & Tutto il gruppo dovrà agire per offrire supporto in modo tale da terminare il lavoro entro la scadenza\\

RO2 Eccesso o difetto nell'assegnazione delle scadenze & Data la presenza di numerosi scenari nei quali abbiamo poca esperienza potrebbe accadere una errata assegnazione di scadenze che si rivelano sovra-stimate o sotto-stimate in relazione alla difficoltà del problema da risolvere & I componenti a cui è assegnato lo svolgimento di un compito devono riferire se la scadenza a loro imposta sia ragionevole dopo aver approfondito e compreso a fondo la difficoltà del lavoro che devono portare a termine & Se i membri del gruppo assegnati a un compito riferiscono al responsabile l'errore nella valutazione delle tempistiche si procede immediatamente ad una nuova pianificazione alla luce delle affermazioni dei membri coinvolti nel compito.\\

RO3 Assenza di comunicazione gruppo-proponente & Potrebbe accadere che presi dal lavoro si trascuri la comunicazione con Imola Informatica & Tutti i membri devono ricordarsi di mantenere un dialogo con l'azienda cercando di raccogliere dubbi e portando i nostri avanzamenti & In caso di assenza di comunicazione gruppo-azienda il responsabile deve fissare una data per un incontro, prima della prima scadenza di revisione, entro la quale il team deve impegnarsi a raccogliere domande, se presenti, e proporre tutti i progressi che sono stati fatti per ricevere feedback essenziali per la corretta riuscita del prodotto. \\

RO4 Impossibilità di stabilire un incontro tra i membri del gruppo & Potrebbe accadere che dato il numero di componenti del gruppo stabilire un incontro in cui tutti siano presenti risulti difficile e in caso si riesca non abbastanza immediato come servirebbe & Il gruppo tiene una tabella con le proprie disponibilità in settimana e si possono vedere gli orari in cui tutti possono essere presenti, oppure si può notare che in nessun giorno tutti sono liberi & Per discutere si possono usare servizi di chiamata come Hangouts, inoltre molto raramente sarà strettamente necessaria la presenza di tutti i membri del gruppo quindi diventa molto più facile organizzarsi a sotto-gruppi e stabilire un incontro.\\


\end{longtable}
}

\subsubsection{Tabella del Grado del Rischio}
Come descritto nelle fasi della gestione del rischio, è importante valutare il grado del rischio, ovvero stabilire la probabilità e la gravità che il rischio potrebbe avere durante il progetto.
//Ogni colonna riporterà il codice di ciascuno dei rischi analizzati nella tabella precedente e sarà composta da:
\begin{itemize}
	\item Codice del Rischio;
	\item Frequenza;
	\item Gravità;
\end{itemize}

{
	\rowcolors{2}{grigetto}{white}
	\renewcommand{\arraystretch}{2}
	\centering
	\begin{longtable}{ C{2cm} C{3cm} C{3cm}}
		\rowcolor{rossoep}
		\textcolor{white}{\textbf{Codice}} & \textcolor{white}{\textbf{Frequenza}} & \textcolor{white}{\textbf{Gravità}}\\	
		
		RT1 & Alta & Media\\
		
		RP1 & Media & Media\\
		
		RP2 & Media & Alta\\
		
		RR1 & Alta & Alta \\
		
		RS1 & Media & Bassa \\
		
		RO1 & Media & Alta \\
		
		RO2 & Media & Media \\
		
		RO3 & Bassa & Media \\
		
		RO4 & Alta & Bassa \\
		
	\end{longtable}
}
\clearpage
\section{Modello di Sviluppo}
Come modello di sviluppo il gruppo \Gruppo{} ha deciso di adottare il \textbf{modello incrementale}.
\subsection{Descrizione}
Nel modello incrementale il prodotto viene sviluppato tramite rilasci successivi. Questi rilasci hanno l'obiettivo di aggiungere funzionalità separate e accessorie a un sistema stabile in cui sono presenti requisiti di base.
Nel caso in cui un rilascio sia fallace è molto facile tornare allo stato funzionante precedente.\\
Il modello incrementale richiede, dunque, una suddivisione preliminare dei requisiti atta ad identificare quelli da sviluppare per primi e quali aggiungere al sistema stabile per incrementi. \\
Inoltre, una volta implementate le caratteristiche base del sistema lo si può sottoporre al committente e al proponente per assicurarsi di star procedendo nella giusta direzione.
In caso negativo, non è troppo tardi per cambiare la struttura del prodotto corrente. \\
Infine, non è particolarmente dispendioso riformulare degli incrementi previsti ma che devono ancora essere implementati. 

\subsection{Motivazioni}
Il gruppo ha scelto questo modello di sviluppo perché si adatta bene alle specifiche del progetto \NomeProgetto{} del proponente \Proponente{}.
Nella fattispecie, è stato facile identificare i requisiti minimi e separare molti requisiti accessori perfetti per essere implementati tramite rilasci incrementali su di un sistema stabile.\\
Inoltre, data la nostra inesperienza, il modello scelto permette a eventuali cambiamenti in corso d'opera di essere poco dispendiosi dal punto di vista sia del tempo di codifica (se circoscritti a singoli rilasci), sia del lavoro di cambiamento della documentazione. \\
In aggiunta a ciò, i rilasci successivi di funzionalità permettono di poter stabilire un confronto migliore con il proponente, riuscendo a sottoporre al suo giudizio un prodotto che sia sempre funzionante e col tempo sempre più completo e conforme alle sue aspettative. \\
Abbiamo inoltre valutato che i principali difetti del modello incrementale, quali la degradazione della struttura causata dall'aggiunta di incrementi e l'invisibilità del processo al manager, 
non influenzano il gruppo data la dimensione ridotta, relativamente ad ambienti aziendali dove i modelli di sviluppo sono sfruttati a pieno, del progetto che stiamo affrontando.


\subsection{Individuazione degli incrementi}
In seguito è riportata una tabella con indicati i requisiti che vengono sviluppati in ciascun incremento, sia dell'applicazione che del server.
I requisiti sono identificati dal loro codice identificativo e sono reperibili nel documento \AdR{}.\\
I codici dei requisiti enfatizzati in grassetto sono obbligatori, quelli non evidenziati sono requisiti desiderabili oppure opzionali.
La scelta di enfatizzare quelli grafici è puramente per una maggior comodità di consultazione.

{
\rowcolors{2}{grigetto}{white}
\renewcommand{\arraystretch}{2}
\centering
	
\begin{longtable}{C{2.5cm} C{3.2cm} C{2.8cm} C{2.8cm} C{2.8cm}}
\caption{Tabella degli incrementi}\\
\rowcolor{darkblue}
\textcolor{white}{\textbf{Incremento}} &
\textcolor{white}{\textbf{Obiettivo dell'incremento}} & 
\textcolor{white}{\textbf{Requisiti per l'app utenti}} &
\textcolor{white}{\textbf{Requisiti per il web-app admin}} &
\textcolor{white}{\textbf{Requisiti per il server}} \\
\endhead

Incremento 0 & \glo{Proof of Concept} (app utenti e web-app admin), lista delle organizzazioni e tracciamento (app utenti, web-app admin, server) (a livello dimostrativo per il \glo{Proof of Concept}) & \begin{itemize}
    % APP
    \item[ ] \textbf{R1FI1}
    \item[ ] \textbf{R1FA1.1}
    \item[ ] \textbf{R1FA1.2}
    \item[ ] \textbf{R1FA1.3}
    \item[ ] \textbf{R1FA1.4}
    \item[ ] \textbf{R1FA1.5}
    \item[ ] \textbf{R1FA2.1}
    \item[ ] \textbf{R1FA3.1}
    \item[ ] \textbf{R1FA3.8}
    \item[ ] \textbf{R1FA3.10}
    \item[ ] \textbf{R1FA6.1}
    \item[ ] \textbf{R1FA8.1}
    \item[ ] \textbf{R1FA8.4} 
\end{itemize} & \begin{itemize}
    % WEB-APP ADMIN
    \item[ ] \textbf{R1FI2}
    \item[ ] \textbf{R1FS1.1}
    \item[ ] \textbf{R1FS1.2}
    \item[ ] R2FS1.3
    \item[ ] R2FS1.4
    \item[ ] R2FS1.5
    \item[ ] R2FS1.6
    \item[ ] R2FS1.7
    \item[ ] \textbf{R1FS2.1}
    \item[ ] \textbf{R1FS3.1}
    \item[ ] \textbf{R1FS6.1}
    \item[ ] \textbf{R1FS10.1}
    \item[ ] \textbf{R1FS10.2} 
    \item[ ] \textbf{R1FS10.14}
\end{itemize} & \begin{itemize} 
    % SERVER
    \item[ ] \textbf{R1FS3.1}
    \item[ ] \textbf{R1FI5}
    \item[ ] R2FI7
    \item[ ] \textbf{R1FI8}
    \item[ ] \textbf{R1FA3.2}
    \item[ ] \textbf{R1FA6.1}

\end{itemize}\\

Incremento 1 & Funzionalità di autenticazione (app utenti e web-app admin) & \begin{itemize}
    % APP
    \item[ ] \textbf{R1FI1}
    \item[ ] \textbf{R1FA1.1}
    \item[ ] \textbf{R1FA1.2}
    \item[ ] \textbf{R1FA1.3}
    \item[ ] \textbf{R1FA1.4}
    \item[ ] \textbf{R1FA1.5}
    \item[ ] \textbf{R1FA1.6}
    \item[ ] \textbf{R1FA1.7}
    \item[ ] \textbf{R1FA2.1}
    \item[ ] \textbf{R1FA8.1}
    \item[ ] \textbf{R1FA8.2}
    \item[ ] \textbf{R1FA8.3}
    \item[ ] \textbf{R1FA8.4}
\end{itemize} & \begin{itemize} 
    % WEB-APP ADMIN
    \item[ ] \textbf{R1FI2}
    \item[ ] \textbf{R1FS1.1}
    \item[ ] \textbf{R1FS1.2}
    \item[ ] \textbf{R1FS2.1}
    \item[ ] \textbf{R1FS10.1}
    \item[ ] \textbf{R1FS10.2}
\end{itemize} & 
    % SERVER
    Nessun requisito del server previsto per questo incremento \\

Incremento 2 & Lista delle organizzazioni (app utenti e web-app admin) & \begin{itemize}
    % APP
    \item[ ] \textbf{R1FA3.1}
    \item[ ] \textbf{R1FA3.2}
    \item[ ] \textbf{R1FA3.3}
    \item[ ] \textbf{R1FA3.4}
    \item[ ] \textbf{R1FA3.5}
    \item[ ] \textbf{R1FA3.6}
    \item[ ] \textbf{R1FA3.7}
    \item[ ] \textbf{R1FA3.8}
    \item[ ] \textbf{R1FA3.9}
    \item[ ] \textbf{R1FA3.10}
    \item[ ] \textbf{R1FA3.15}
    \item[ ] \textbf{R1FA3.17}
    \item[ ] \textbf{R1FA8.5}
    \item[ ] \textbf{R1FA8.6}
\end{itemize} & \begin{itemize} 
    % WEB-APP ADMIN
    \item[ ] \textbf{R1FC3}
    \item[ ] \textbf{R1FI3}
    \item[ ] \textbf{R1FI5}
    \item[ ] \textbf{R1FI8}
    \item[ ] \textbf{R1FS3.1}
    \item[ ] \textbf{R1FS7.1}
    \item[ ] \textbf{R1FS7.2}
    \item[ ] \textbf{R1FS7.6}
\end{itemize} & \begin{itemize} 
    % SERVER
    \item[ ] \textbf{R1FC3}
    \item[ ] \textbf{R1FI3}
    \item[ ] \textbf{R1FI8} 
    \item[ ] \textbf{R1FS3.1}
    \item[ ] \textbf{R1FS7.1}
    \item[ ] \textbf{R1FS7.2}
    \item[ ] \textbf{R1FS7.6}
\end{itemize}\\

Incremento 3 & Tracciamento & \begin{itemize}
    % APP
    \item[ ] \textbf{R1FA4.1}
    \item[ ] \textbf{R1FA4.2}
    \item[ ] \textbf{R1FA4.3}
    \item[ ] \textbf{R1FA6.1}
    \item[ ] \textbf{R1FA6.2}
    \item[ ] \textbf{R1FA6.3}
    \item[ ] \textbf{R1FA6.4}
    \item[ ] \textbf{R1FA8.7}
    \item[ ] \textbf{R1FA7.1}
    \item[ ] \textbf{R1FA8.8}
    \item[ ] \textbf{R1FA7.2}
    \item[ ] \textbf{R1FA7.3}
\end{itemize}& \begin{itemize} 
    % WEB-APP ADMIN
    \item[ ] \textbf{R1FS6.1}
    \item[ ] \textbf{R1FS6.2}
    \item[ ] \textbf{R1FS6.3}
    \item[ ] \textbf{R1FS8.1}
    \item[ ] \textbf{R1FS8.2}
    \item[ ] \textbf{R1FS8.3}
    \item[ ] \textbf{R1FS8.4}

\end{itemize} & \begin{itemize} 
    % SERVER 
    \item[ ] \textbf{R1FA6.1}
    \item[ ] \textbf{R1FA6.2}
    \item[ ] \textbf{R1FA6.3}
    \item[ ] \textbf{R1FA6.4}
    \item[ ] \textbf{R1FA8.7}
    \item[ ] \textbf{R1FS6.1}
    \item[ ] \textbf{R1FS6.2}
    \item[ ] \textbf{R1FS8.1}
    \item[ ] \textbf{R1FS8.2}
    \item[ ] \textbf{R1FS8.3}
    \item[ ] \textbf{R1FS8.4}
\end{itemize}\\

Incremento 4 & Storico accessi di un utente (app utenti) e report tabellari degli accessi (web-app admin) &
    % APP
    Nessun requisito dell'applicazione previsto per questo incremento
    & \begin{itemize} 
    % WEB-APP ADMIN
    \item[ ] \textbf{R1FS4.1}
    \item[ ] \textbf{R1FS4.4}
    \item[ ] \textbf{R1FS4.5}
    \item[ ] \textbf{R1FS4.6}
    \item[ ] \textbf{R1FS4.8}
    \item[ ] \textbf{R1FS10.3}
    \item[ ] \textbf{R1FS10.4}
    \item[ ] \textbf{R1FS10.7}
    \item[ ] \textbf{R1FS10.8}
    \item[ ] \textbf{R1FS5.1}
    \item[ ] \textbf{R1FS10.9}
    \item[ ] \textbf{R1FS5.2}
    \item[ ] \textbf{R1FS5.9}
    \item[ ] \textbf{R1FS5.3}
    \item[ ] \textbf{R1FS10.10}
    \item[ ] \textbf{R1FS5.4}
    \item[ ] \textbf{R1FS5.5}
    \item[ ] \textbf{R1FS5.6}
    \item[ ] \textbf{R2FS5.7}
    \item[ ] \textbf{R1FS5.8}
\end{itemize} & \begin{itemize} 
    % SERVER
    \item[ ] \textbf{R1FS4.1}
    \item[ ] \textbf{R1FS4.4} 
    \item[ ] \textbf{R1FS4.5}
    \item[ ] \textbf{R1FS4.6}
    \item[ ] \textbf{R1FS4.8}
    \item[ ] \textbf{R1FS10.3}
    \item[ ] \textbf{R1FS10.4}
    \item[ ] \textbf{R1FS10.7}
    \item[ ] \textbf{R1FS10.8}
    \item[ ] \textbf{R1FS5.1}
    \item[ ] \textbf{R1FS10.9}
    \item[ ] \textbf{R1FS5.2}
    \item[ ] \textbf{R1FS5.9}
    \item[ ] \textbf{R1FS5.3}
    \item[ ] \textbf{R1FS10.10}
    \item[ ] \textbf{R1FS5.4}
    \item[ ] \textbf{R1FS5.5}
    \item[ ] \textbf{R1FS5.6}
    \item[ ] \textbf{R2FS5.7}
    \item[ ] \textbf{R1FS5.8}
\end{itemize} \\

Incremento 5 & Autenticazione presso l'organizzazione (app utenti), gestione amministratori  e modifica dell'organizzazione (web-app admin) & \begin{itemize}
    % APP
    \item[ ] \textbf{R1FS9.12}
    \item[ ] \textbf{R1FS9.13}
    \item[ ] \textbf{R1FS9.14}
\end{itemize} & \begin{itemize} 
    % WEB-APP ADMIN
    \item[ ] \textbf{R1FI9}
    \item[ ] \textbf{R1FI10}
    \item[ ] \textbf{R1FI11}
    \item[ ] \textbf{R1FS9.1}
    \item[ ] \textbf{R1FS9.2}
    \item[ ] \textbf{R1FS9.3}
    \item[ ] \textbf{R1FS9.4}
    \item[ ] \textbf{R1FS9.5}
    \item[ ] \textbf{R1FS9.6}
    \item[ ] \textbf{R1FS9.7}
    \item[ ] \textbf{R1FS9.8}
    \item[ ] \textbf{R1FS9.9}
    \item[ ] \textbf{R1FS9.10}
    \item[ ] \textbf{R1FS9.11}
    \item[ ] \textbf{R1FS9.12}
    \item[ ] \textbf{R1FS9.13}
    \item[ ] \textbf{R1FS9.14} 
    \item[ ] \textbf{R1FS10.11}
    \item[ ] \textbf{R1FS10.12}
    \item[ ] \textbf{R1FS10.13}
    \item[ ] \textbf{R1FS10.14}
    \item[ ] \textbf{R1FS10.15}
    \item[ ] \textbf{R1FS10.16}
    \item[ ] \textbf{R1FS10.17}
\end{itemize} & \begin{itemize}
    % SERVER
    \item[ ] \textbf{R1FI9}
    \item[ ] \textbf{R1FI10}
    \item[ ] \textbf{R1FI11}
    \item[ ] \textbf{R1FS9.2}
    \item[ ] \textbf{R1FS9.3}
    \item[ ] \textbf{R1FS9.4}
    \item[ ] \textbf{R1FS9.6}
    \item[ ] \textbf{R1FS9.7}
    \item[ ] \textbf{R1FS9.8}
    \item[ ] \textbf{R1FS9.9}
    \item[ ] \textbf{R1FS9.10}
    \item[ ] \textbf{R1FS9.12}
    \item[ ] \textbf{R1FS9.13}
    \item[ ] \textbf{R1FS10.11}
    \item[ ] \textbf{R1FS10.12}
    \item[ ] \textbf{R1FS10.14}
    \item[ ] \textbf{R1FS9.14}
    \item[ ] \textbf{R1FS10.15}
\end{itemize}\\

Incremento 6 & Funzionalità aggiuntive all'utente anonimo & \begin{itemize}
    % APP
    \item[ ] R2FA5.1
    \item[ ] R2FA5.2
    \item[ ] R2FA5.3
    \item[ ] R2FA5.4
    \item[ ] R2FA5.5
    \item[ ] R2FA5.6
    \item[ ] R2FA5.7
    \item[ ] R2FA5.8
    \item[ ] R2FA5.9
    \item[ ] R2FA5.10
    \item[ ] R2FA5.11
    \item[ ] R3FA5.12
    \item[ ] R2FA5.13
    \item[ ] R2FA5.14
    \item[ ] R3FA5.15
    \item[ ] R2FA5.16
    \item[ ] R2FA5.17
    \item[ ] R2FA8.5
    \item[ ] R2FA8.6
\end{itemize} &
    % WEB-APP ADMIN
    Nessun requisito della web-app admin previsto per questo incremento
    & \begin{itemize} 
    % SERVER
    \item[ ] R2FA5.1
    \item[ ] R2FA5.2
    \item[ ] R2FA5.3
    \item[ ] R2FA5.4
    \item[ ] R2FA5.5
    \item[ ] R2FA5.6
    \item[ ] R2FA5.7
    \item[ ] R2FA5.8
    \item[ ] R2FA5.9
    \item[ ] R2FA5.16
    \item[ ] R2FA5.17
    \item[ ] R2FA8.5
    \item[ ] R2FA8.6
\end{itemize}\\

Incremento 7 & Funzionalità per il reset della password & \begin{itemize}
    % APP
    \item[ ] R2FA1.8
    \item[ ] R2FA1.9
    \item[ ] R2FA1.10
    \item[ ] R2FA1.11
    \item[ ] R2FA1.12
\end{itemize} & \begin{itemize} 
    % WEB-APP ADMIN
    \item[ ] R2FS1.3 %poc
    \item[ ] R2FS1.4 %poc
    \item[ ] R2FS1.5 %poc
    \item[ ] R2FS1.6 %poc
    \item[ ] R2FS1.7 %poc
\end{itemize} & 
    % SERVER
    Nessun requisito del server previsto per questo incremento \\

Incremento 8 & Funzionalità aggiuntive di filtraggio/ricerca nella lista delle organizzazioni (app utenti e web-app admin) & \begin{itemize}
    % APP
    \item[ ] R2FA3.11
    \item[ ] R2FA3.12
    \item[ ] R3FA3.13
    \item[ ] R3FA3.14
    \item[ ] R2FA3.16
    \item[ ] R2FA3.18
\end{itemize} & \begin{itemize} 
    % WEB-APP ADMIN
    \item[ ] R2FI4
    \item[ ] R2FI6
    \item[ ] R2FI7
    \item[ ] R2FS7.3
    \item[ ] R2FS7.4
    \item[ ] R2FS7.5
    \item[ ] R2FS7.7 
    \item[ ] R2FS7.8 
    \item[ ] R2FS7.9
\end{itemize} & \begin{itemize} 
    % SERVER
    \item[ ] R2FI4
    \item[ ] R2FI6
    \item[ ] R2FI7
\end{itemize}\\

Incremento 9 & Funzionalità generiche aggiuntive agli amministratori e gestione errori & 
    % APP
    Nessun requisito della web-app admin previsto per questo incremento
     & \begin{itemize} 
    % WEB-APP ADMIN
    \item[ ] R2FS4.2
    \item[ ] R2FS4.3
    \item[ ] R3FS4.7
    \item[ ] R2FS10.5
    \item[ ] R2FS10.6
    \item[ ] R2FS10.18
\end{itemize} & \begin{itemize} 
    % SERVER
    \item[ ] R2FS4.2
    \item[ ] R2FS4.3
    \item[ ] R3FS4.7
    \item[ ] R2FS10.5
    \item[ ] R2FS10.6
    \item[ ] R2FS10.18
\end{itemize} \\

\end{longtable}
}
\clearpage
\section{Pianificazione}
Nella pianificazione, il \Responsabile{} suddivide il lavoro in attività e le assegna a ciascun membro del team.
Lo scopo è dimostrare come deve venire svolto il lavoro, valutare i progressi nel progetto e anticipare i problemi che potrebbero sorgere preparando delle soluzioni a tali problemi.\\
La pianificazione di progetto viene organizzata seguendo le scadenze presentate nella sezione §8.3.
Lo sviluppo del progetto viene suddiviso nelle seguenti quattro fasi: 
\begin{itemize}
	\item Analisi;
	\item Progettazione Architetturale;
	\item Progettazione di Dettaglio e Codifica;
	\item Validazione e Collaudo.
\end{itemize}
Ogni fase è suddivisa in periodi più brevi all'interno dei quali vengono elencate le diverse attività che il gruppo \Gruppo{} deve svolgere e gli incrementi previsti.


\subsection{Analisi}
Periodo: dal 2019-11-15 al 2020-01-20\\
Inizia con la formazione del gruppo e finisce con la data di consegna della Revisione dei Requisiti.\\
In questa fase viene definito il gruppo, la normazione (\glo{way of working}) e la garanzia di qualità che vuole fornire, oltre alla definizione dei requisiti del capitolato che viene scelto.
\subsubsection{Periodo 1} 
Dal 2019-11-15 al 2019-11-29\\
In questo periodo, che parte dalla formazione del gruppo e termina con la scelta del capitolato C5 \NomeProgetto{}, il gruppo ha affrontato le seguenti tematiche al fine di porre le basi per il lavoro che andava affrontato:
\begin{itemize}
	\item \textbf{Discussione capitolati}: Ogni membro del gruppo ha studiato individualmente e in seguito discusso durante gli incontri tutti i capitolati proposti, ponendo le basi per la stesura del documento \SdF{} e ha indirizzato verso la scelta del capitolato scelto;
	\item \textbf{Assegnazione e studio dei ruoli di progetto}: Ad ogni membro del gruppo è stato assegnato il ruolo principale da ricoprire nella fase di Analisi;
	\item \textbf{Definizione degli strumenti}: Vengono discusse e definite le tecnologie da usare per affrontare la fase di Analisi;
	\item \textbf{Pianificazione milestone fase di Analisi}: Vengono discusse e fissate delle \glo{milestone} intermedie da rispettare per completare la fase di Analisi entro le scadenze imposteci.
\end{itemize}
\subsubsection{Periodo 2} 
Dal 2019-11-30 al 2019-12-31\\
Questo periodo inizia con la scelta definitiva del capitolato C5 \NomeProgetto{}.\\
Dopo la scelta, sono state focalizzate le risorse del gruppo nei seguenti punti:
\begin{itemize}
	\item \textbf{Normazione}: Vengono definite le regole per la stesura dei documenti e per l'utilizzo delle tecnologie identificate in precedenza;
	\item \textbf{Approfondimento capitolati}: Vengono ulteriormente discussi tutti i capitolati in modo da terminare lo studio di fattibilità e focalizzare la nostra analisi sul capitolato scelto in modo da predisporre le basi per l'analisi dei requisiti;
	\item \textbf{Prima definizione dei casi d'uso};
	\item \textbf{Determinazione standard di qualità}: Abbiamo definito le nostre strategie per garantire la qualità di processo e la qualità di prodotto;
	\item \textbf{Verifica}: Verifica dell'andamento del gruppo in relazione alle tempistiche e allo svolgimento dei compiti assegnati.
\end{itemize}
\subsubsection{Periodo 3}
 Dal 2020-01-01 al 2020-01-14\\
 Questo periodo si estende fino alla data ultima di consegna per affrontare la Revisione dei Requisiti a cui il nostro gruppo ha deciso di partecipare.\\
 \begin{itemize}
	\item \textbf{Normazione}: Ulteriori approfondimenti alle regole per la stesura dei documenti e per l'utilizzo delle tecnologie;
	\item \textbf{Approfondimento delle tecnologie}: Vengono ampliate le conoscenze sulle tecnologie richieste dal capitolato per essere svolto;
	\item \textbf{Analisi dei requisiti}: Studio dei requisiti e raffinamento dei casi d'uso;
	\item \textbf{Pianificazione attività}: Pianificazione del lavoro da svolgere nelle fasi successive a quella di Analisi;
	\item \textbf{Verifica}: Verifica dell'andamento del team in relazione alle tempistiche e allo svolgimento dei compiti assegnati.

 \end{itemize}
\subsubsection{Periodo 4} 
Dal 2020-01-15 al 2020-01-20\\
In questo periodo, che ha inizio con la consegna dei documenti per la Revisione dei Requisiti alla presentazione pubblica della proposta, il gruppo consolida il lavoro svolto in vista delle successive fasi e della discussione per la quale serve una presentazione;
\begin{itemize}
	\item \textbf{Consolidamento}: Ogni membro del gruppo si prende del tempo per ripassare tutto il lavoro svolto e per studiare il necessario per affrontare al meglio le fasi successive;
	\item \textbf{Preparazione per la Revisione dei Requisiti}: Il gruppo produce il materiale necessario da esporre alla presentazione pubblica della nostra proposta.
\end{itemize}

\newpage
% Inizia la pagina orientata orizzontalmente
\begin{landscape}
% Ora la pagina e' in orizzontale!
\subsubsection{Diagramma di Gantt delle attività della fase di Analisi}
\pagestyle{empty}
\begin{figure}[h]
	\centering	
	\includegraphics[scale=0.455]{Sezioni/DiagrammiGantt/Analisi.png}
	\caption{Diagramma di Gantt delle attività della fase di Analisi}
\end{figure}
\end{landscape}
\clearpage

\subsection{Progettazione Architetturale}
Periodo: dal 2020-01-22 al 2020-03-15\\
Inizia al termine della fase di Analisi e finisce con la data di consegna della Revisione di Progettazione.\\
In questa fase viene definita una soluzione architetturale in modo da soddisfare i requisiti individuati nella fase di Analisi.

\subsubsection{Periodo 1} 
Dal 2020-01-22 al 2020-02-11
\begin{itemize}
	\item \textbf{Normazione}: Standardizzazione e correzione di alcune parti dei documenti che non aderiscono completamente alle \NdP{};
	\item \textbf{Analisi dei requisiti}: Correzione e modifica dei casi d'uso segnalati;
	\item \textbf{Assegnazione dei ruoli di progetto}: Assegnazione dei ruoli di ciascun membro del gruppo in base alla suddivisione oraria indicata in §5.2.1;
	\item \textbf{Pianificazione attività}: Le attività da svolgere devono essere prima pianificate e discusse dal gruppo per garantire il \glo{way of working} sancito nelle \NdP{};
	\item \textbf{Approfondimento delle tecnologie}: Ricerca di documentazione e materiali utili per l'apprendimento delle nuove tecnologie da utilizzare per la realizzazione del prodotto finale;
	\item \textbf{Verifica}: Verifica dell'andamento del team in relazione alle tempistiche e allo svolgimento dei compiti assegnati.
\end{itemize}
\subsubsection{Periodo 2} 
Dal 2020-02-12 al 2020-03-08
\begin{itemize}
<<<<<<< HEAD
	\item \textbf{Studio delle tecnologie:} IAAS \glo{Kubernetes} o \glo{PaaS}, \glo{Openshift} o \glo{Rancher}, \glo{LDAP} e \glo{GPS};
	\item \textbf{Normazione:} Decisioni ed inserimento delle nuove regole da adottare per le prossime condizioni dello sviluppo;
	\item \textbf{Miglioramento standard di qualità:} Aggiunzione, rimozione o modifica di alcuni standard per garantire qualità nei processi e prodotti software;
	\item \textbf{Technology Baseline:} Redazione della \glo{Technology Baseline}, cioè un allegato tecnico nella quale vengono stesi i design pattern che verranno utilizzati durante lo sviluppo;
	\item \textbf{Proof of Concept:} Rappresentazione della \glo{Baseline};
	\item \textbf{Codifica:} Viene codificato il \glo{Proof of Concept}, nella quale viene condiviso tramite \glo{repository} al committente e proponente in una data da definire;
	\item \textbf{Verifica:} Verifica dell'andamento del team in relazione alle tempistiche e allo svolgimento dei compiti assegnati.
=======
	\item \textbf{Studio delle tecnologie}: l'\glo{IaaS} \glo{Kubernetes} o i \glo{PaaS} \glo{Openshift} o \glo{Rancher}, \glo{LDAP} e \glo{GPS};
	\item \textbf{Normazione}: Decisioni ed inserimento delle nuove regole da adottare per le prossime condizioni dello sviluppo;
	\item \textbf{Miglioramento standard di qualità}: Aggiunta, rimozione o modifica di alcune metriche per garantire le qualità di processo e di prodotto affermate nel \PdQ{};
	\item \textbf{Technology Baseline}: Redazione della \glo{Technology Baseline}, cioè un allegato tecnico nel quale vengono indicate le tecnologie e i design pattern che vengono utilizzati durante lo sviluppo del prodotto;
	\item \textbf{Proof of Concept}: Creazione di un eseguibile che permetta di dimostrare la validità del prodotto che si vuole fornire, concretizzando la \glo{Technology Baseline};
	\item \textbf{Codifica}: Viene codificato il \glo{Proof of Concept} e successivamente condiviso tramite i \glo{repository} del gruppo al committente e al proponente in una data da definire;
	\item \textbf{Verifica}: Verifica dell'andamento del team in relazione alle tempistiche e allo svolgimento dei compiti assegnati.
>>>>>>> b3dd61d2b76f06616fbece5634acabe2aacf85cc
\end{itemize}
\subsubsection{Periodo 3} 
Dal 2020-03-09 al 2020-03-15
\begin{itemize}
	\item \textbf{Consolidamento}: Ogni membro si prende del tempo per ripassare tutto il lavoro svolto e per studiare il necessario per affrontare al meglio le fasi successive;
	\item \textbf{Preparazione per la Revisione di Progettazione}: Il gruppo produce il materiale necessario da esporre alla presentazione pubblica della nostra proposta.
\end{itemize}

\newpage
% Inizia la pagina orientata orizzontalmente
\begin{landscape}
% Ora la pagina e' in orizzontale!
\subsubsection{Diagramma di Gantt delle attività della fase di Progettazione Architetturale}
\pagestyle{empty}
\begin{figure}[h]
	\centering
	\includegraphics[scale=1.48]{Sezioni/DiagrammiGantt/ProgettazioneArchitetturale.png}
	\caption{Diagramma di Gantt delle attività della fase di Progettazione Architetturale}	
\end{figure}
\end{landscape}

\subsection{Progettazione di Dettaglio e Codifica}
Dal 2020-03-16 al 2020-04-19\\
Inizia al termine della Progettazione Architetturale e finisce con la data di consegna della Revisione di Qualifica.\\
In questa fase si definisce nel dettaglio e si implementa l'architettura logica costruita nella fase di Progettazione Architetturale.\\


\subsubsection{Periodo 1} 
Dal 2020-03-16 al 2020-03-27\\
\begin{itemize}
	\item \textbf{Approfondimento delle tecnologie:} Ricerca documentazione e materiali utili per l'apprendimento delle nuove tecnologie da utilizzare per la realizzazione del prodotto finale;
	\item \textbf{Normazione:} Standardizzazione e correzione di alcune parti dei documenti che non aderiscono completamente alle norme;
	\item \textbf{Assegnazione dei ruoli di progetto:} Ad ogni membro del gruppo viene assegnato il ruolo principale da ricoprire nella fase di progettazione di dettaglio e codifica;
	\item \textbf{Pianificazione delle attività:} Le attività da svolgere devono essere prima pianificate e discusse dal gruppo per garantire successivamente un buon \glo{way of working};
	\item \textbf{Progettazione:} Ricerca di una soluzione soddisfacente per tutti gli \glo{stakeholder}, descrive l'architettura del prodotto prima di pensare al codice ed attua un approccio sintetico;
	\item \textbf{Codifica:} Implementazione dei requisiti di base identificati per ottenere un sistema stabile;
	\item \textbf{Manuali:} Stesura manuale utente e manuale manutentore in relazione alle funzionalità di base del sistema.
\end{itemize}
\subsubsection{Periodo 2} 
Dal 2020-03-28 al 2020-04-08\\
\begin{itemize}
	\item \textbf{Implementazione della Product Baseline:} Seguendo le specifiche della \glo{Technology Baseline};
	\item \textbf{Codifica incrementale:} Aggiunta di requisiti al sistema tramite incrementi;
	\item \textbf{Incremento e verifica:} Incrementi e verifiche con eventuali aggiunte al lavoro svolto in precedenza;
	\item \textbf{Manuali:} Aggiunta nel manuale utente e nel manuale manutentore delle funzionalità inserite incrementalmente nel sistema;
\end{itemize}
\subsubsection{Periodo 3}
Dal 2020-04-09 al 2020-04-12\\
\begin{itemize}
	\item \textbf{Primo rilascio del prodotto} Pubblicazione del prodotto in un apposito \glo{repository} condiviso dai membri del gruppo;
	\item \textbf{Verifica:} Verifica dell'andamento del team in relazione alle tempistiche e allo svolgimento dei compiti assegnati;
\end{itemize}
\subsubsection{Periodo 4} 
Dal 2020-04-13 al 2020-04-19\\
\begin{itemize}
	\item \textbf{Consolidamento:} Ogni membro si prende del tempo per ripassare tutto il lavoro svolto e per studiare il necessario per affrontare al meglio le fasi successive;
	\item \textbf{Preparazione per la Revisione di Qualifica:} Il gruppo produce il materiale necessario da esporre alla presentazione pubblica della nostra proposta.
\end{itemize}

\newpage
% Inizia la pagina orientata orizzontalmente
\begin{landscape}
	% Ora la pagina e' in orizzontale!
	\subsubsection{Diagramma di Gantt delle attività}
	\pagestyle{empty}
	\begin{figure}[h]
			
		\begin{center}	
				\includegraphics[scale=0.5]{Sezioni/DiagrammiGantt/ProgettazioneDiDettaglio.png}	
		\end{center}
	\caption{Diagramma di Gantt delle attività di Progettazione di Dettaglio e Codifica}	
	\end{figure}
\end{landscape}

\subsection{Validazione e Collaudo}
Inizia al termine della progettazione di dettaglio e codifica e finisce con la data di consegna della revisione di accettazione.
\\In questo fase vengono definite le attività che servono per verificare che il prodotto corrisponde a quello desiderato dal cliente.
\subsubsection{Periodo 1} 
Dal 2020-04-21 al 2020-04-28
\begin{itemize}
	\item \textbf{Normazione:} Standardizzazione e correzione di alcune parti dei documenti che non aderiscono completamente alle norme;
	\item \textbf{Assegnazione dei ruoli di progetto:} Ad ogni membro del gruppo viene assegnato il ruolo principale da ricoprire nella fase di progettazione di validazione e collaudo;
	\item \textbf{Soddisfazione dei requisiti:} Controllo che i requisiti siano soddisfatti;
	\item \textbf{Pianificazione attività:} Le attività da svolgere devono essere prima pianificate e discusse dal gruppo per garantire successivamente un buon \glo{way of working};
	\item \textbf{Verifica} Verifica dell'andamento del team in relazione alle tempistiche e allo svolgimento dei compiti assegnati.
\end{itemize}
\subsubsection{Periodo 2} 
Dal 2020-04-29 al 2020-05-10
\begin{itemize}
	\item \textbf{Codifica:} Esecuzione dell'ultimo versionamento del prodotto;
	\item \textbf{Verifica:} Accertamento che le esecuzioni delle attività siano esenti da errori;
	\item \textbf{Validazione:} Verifica se il prodotto realizzato sia conforme alle attese, e validazione finale in caso di esito positivo;
	\item \textbf{Scrittura dei manuali:} Esecuzione del secondo versionamento del manuale utente e del manuale manutentore;
	\item \textbf{Collaudo:} Vengono eseguiti gli ultimi test sul prodotto per verificare se le funzionalità rispettano i risultati attesi.
\end{itemize}
\subsubsection{Periodo 3} 
Dal 2020-05-11 al 2020-05-17
\begin{itemize}
	\item \textbf{Preparazione per la Revisione di Accettazione:} Il gruppo produce il materiale necessario da esporre alla presentazione pubblica della nostra proposta.
\end{itemize}


\newpage
% Inizia la pagina orientata orizzontalmente
\begin{landscape}
	% Ora la pagina e' in orizzontale!
	\subsubsection{Diagramma di Gantt delle attività}
	\pagestyle{empty}
	\begin{figure}[h]
		
		\begin{center}	
			\includegraphics[scale=1.6]{Sezioni/DiagrammiGantt/Validazione.png}
		\end{center}
	\caption{Diagramma di Gantt delle attività di Validazione e Collaudo}	
	\end{figure}
\end{landscape}
\clearpage
\section{Preventivo}
Sigle identificative per i ruoli indicati nelle tabelle e nei grafici
\begin{itemize}
    \item RE: Responsabile;
    \item AM: Amministratore;
    \item AN: Analista;
    \item PT: Progettista;
    \item PR: Programmatore;
    \item VE: Verificatore.
\end{itemize}
{
	
	\rowcolors{2}{grigetto}{white}
	\renewcommand{\arraystretch}{2}
	\begin{table}[h!]
		\centering
		\begin{longtable}{ C{2cm} C{3cm}}
			\rowcolor{rossoep}
			\textcolor{white}{\textbf{Ruolo}} & \textcolor{white}{\textbf{Costo per ora espresso in euro}}\\	
			
			Responsabile & 30\\
			Amministratore & 20\\
			Analista & 25\\
			Progettista & 22\\
			Programmatore & 15\\
			Verificatore & 15\\
			
		\end{longtable}
		\caption{  \ref{table:5} rappresenta il costo orario associato a ciascun ruolo}
		\label{table:5}
	\end{table}
}

\subsection{Fase di Analisi}

\subsubsection{Divisione Oraria}
La seguente tabella §5.1.1 rappresenta la distribuzione oraria dei ruoli per ogni componente del gruppo.
{
	\rowcolors{2}{grigetto}{white}
	\renewcommand{\arraystretch}{2}
	\centering
	\begin{longtable}{ C{5cm} C{1cm} C{1cm} C{1cm} C{1cm} C{1cm} C{1cm} C{3cm}}
		\rowcolor{rossoep}
		\textcolor{white}{\textbf{Nome membro del gruppo}} & \textcolor{white}{\textbf{RE}} & \textcolor{white}{\textbf{AM}} & \textcolor{white}{\textbf{AN}} & \textcolor{white}{\textbf{PT}} & \textcolor{white}{\textbf{PR}} & \textcolor{white}{\textbf{VE}} & \textcolor{white}{\textbf{Ore complessive}}\\	
        
        
        Christian Mattei     &  & 7 & 12 &  & & 11 & 30 \\
		Davide Lazzaro       &  & 5 & 16 &  &  & 9 & 30 \\
        Emanuele Cisotto     &  &  & 21 &  &  & 9 & 30 \\
        Enrico Salmaso       & 15 & 2 & 8  &  &  & 5 & 30 \\
        Federico Perin       &  &  & 21 &  &  & 9 & 30 \\
        Francesco Drago      &  & 7 & 16 &  &  & 7 & 30 \\
        Riccardo Baratin     &  & 5 & 11 &  &  & 14 & 30 \\
        Tommaso Azzalin      & 4 & 12 & 9  &  &  & 5 & 30 \\
        \textbf{Ore totali ruolo} & 19 & 38 & 114 &  &  & 69 & 240 \\
		
	\end{longtable}
}

La quantità di ore che ciascun componente del gruppo ha svolto per ogni ruolo viene rappresentata nel seguente istogramma:

\begin{figure}[h]
	\centering
	\includegraphics[scale=2]{sezioni/Istogrammi/IstogrammaAnalisi.png}
	\caption{Istogramma della disposizione ore per ruolo di ciascun componente del fase di analisi}
\end{figure}

\subsubsection{Costo Risultate}
La seguente tabella §5.1.2 rappresenta, per ruolo, le ore totali investite e il corrispondente costo in euro.
{
	\rowcolors{2}{grigetto}{white}
	\renewcommand{\arraystretch}{2}
	\centering
	\begin{longtable}{ C{3cm} C{2cm} C{4cm}}
		\rowcolor{rossoep}
		\textcolor{white}{\textbf{Ruolo}} & \textcolor{white}{\textbf{Totale ore}} & \textcolor{white}{\textbf{Costo Ruolo in euro}}\\	
        
        Responsabile & 19 & 570\\
        Amministratore & 38 & 760\\
        Analista & 114 & 2850 \\
        Progettista & 0 & 0 \\
        Programmatore & 0 & 0 \\
        Verificatore & 69 & 1035 \\
        \textbf{Totale} & 240 & 5215 \\
		
	\end{longtable}
}

La quantità di ore totali di ciascun ruolo viene rappresentata nel seguente aerogramma:

\begin{figure}[h]
	\centering
	\includegraphics[scale=2]{sezioni/Aerogrammi/AerogrammaAnalisi.png}
	\caption{Suddivisione ore per ruolo della fase di analisi}
\end{figure}

\subsection{Progettazione Architetturale}

\subsubsection{Divisione Oraria}
La seguente tabella §5.2.1 rappresenta la distribuzione oraria dei ruoli per ogni componente del gruppo.
{
	\rowcolors{2}{grigetto}{white}
	\renewcommand{\arraystretch}{2}
	\centering
	\begin{longtable}{ C{5cm} C{1cm} C{1cm} C{1cm} C{1cm} C{1cm} C{1cm} C{3cm}}
		\rowcolor{rossoep}
		\textcolor{white}{\textbf{Nome membro del gruppo}} & \textcolor{white}{\textbf{RE}} & \textcolor{white}{\textbf{AM}} & \textcolor{white}{\textbf{AN}} & \textcolor{white}{\textbf{PT}} & \textcolor{white}{\textbf{PR}} & \textcolor{white}{\textbf{VE}} & \textcolor{white}{\textbf{Ore complessive}}\\	
        
        Christian Mattei & 5 & & & 14 & 10 & 4 & 33\\
        Davide Lazzaro & 8 & & & 18 & & 7 & 33 \\
        Emanuele Cisotto & & 8 & & 10 & 5 & 6 & 29 \\
        Enrico Salmaso & & 10 & 6 & 10 & & 6 & 32 \\
        Federico Perin & & 6 & & 5 & 7 & 13 &  31\\
        Francesco Drago & & & 13 & 5 & 7 & 6 & 31 \\
        Riccardo Baratin & 6 & & & & 17 & 8 & 31\\
        Tommaso Azzalin & & & 8 & 7 & & 16 & 31\\
        \textbf{Ore totali ruolo} & 19 & 24 & 27 & 69 & 46 & 66 & 251\\
		
	\end{longtable}
}

La suddivisione di quante ore ha svolto ciascun componente del gruppo per ogni ruolo viene rappresentata nel seguente istogramma:

\begin{figure}[h]
	\centering
	\includegraphics[scale=3]{sezioni/Istogrammi/IstogrammaProgettArchitetturale.png}
	\caption{Istogramma della disposizione ore per ruolo di ciascun componente della fase di Progettazione Architetturale}
\end{figure}

\subsubsection{Costo Risultate}
La seguente tabella §5.2.2 rappresenta, per ruolo, le ore totali investite e il corrispondente costo in euro.
{
	\rowcolors{2}{grigetto}{white}
	\renewcommand{\arraystretch}{2}
	\centering
	\begin{longtable}{ C{3cm} C{2cm} C{4cm}}
		\rowcolor{rossoep}
		\textcolor{white}{\textbf{Ruolo}} & \textcolor{white}{\textbf{Totale ore}} & \textcolor{white}{\textbf{Costo Ruolo in euro}}\\	
        
        Responsabile & 19 & 570 \\
        Amministratore & 24 & 480 \\
        Analista & 27 & 675 \\
        Progettista & 69 & 1518 \\
        Programmatore & 46 & 690 \\
        Verificatore & 66 & 990 \\
        \textbf{Totale} & 251 & 4923 \\
		
	\end{longtable}
}

La suddivisione della quantità di ore totali di ciascun ruolo viene rappresentata nel seguente aerogramma:

\begin{figure}[h]
	\centering
	\includegraphics[scale=2]{sezioni/Aerogrammi/AerogrammaProgettArchitetturale.png}
	\caption{Suddivisione ore per ruolo della fase di Progettazione Architetturale}
\end{figure}

\subsection{Progettazione di dettaglio e codifica}

\subsubsection{Divisione Oraria}
La seguente tabella §5.3.1 rappresenta la distribuzione oraria dei ruoli per ogni componente del gruppo.
{
	\rowcolors{2}{grigetto}{white}
	\renewcommand{\arraystretch}{2}
	\centering
	\begin{longtable}{ C{5cm} C{1cm} C{1cm} C{1cm} C{1cm} C{1cm} C{1cm} C{3cm}}
		\rowcolor{rossoep}
		\textcolor{white}{\textbf{Nome membro del gruppo}} & \textcolor{white}{\textbf{RE}} & \textcolor{white}{\textbf{AM}} & \textcolor{white}{\textbf{AN}} & \textcolor{white}{\textbf{PT}} & \textcolor{white}{\textbf{PR}} & \textcolor{white}{\textbf{VE}} & \textcolor{white}{\textbf{Ore complessive}}\\	
        
        Christian Mattei & & 10 & & 12 & 15 & 10 & 47\\
        Davide Lazzaro & 6 & & & 10 & 15 & 16 & 47\\
        Emanuele Cisotto & & 5 & & 13 & 16 & 13 & 47 \\
        Enrico Salmaso & & 4 & & 11 & 18 & 14 & 47\\
        Federico Perin & & 8 & & 8 & 19 & 12 & 47\\
        Francesco Drago & 10 & & & 7 & 17 & 13 & 47\\
        Riccardo Baratin & & 3 & & 12 & 19 & 13 & 47\\
        Tommaso Azzalin & 10 & & & 10 & 17 & 10 & 47 \\
        \textbf{Ore totali ruolo} & 26 & 30 & & 83 & 136 & 101 & 376\\
		
	\end{longtable}
}

La suddivisione di quante ore ha svolto ciascun componente del gruppo per ogni ruolo viene rappresentata nel seguente istogramma:

\begin{figure}[h]
	\centering
	\includegraphics[scale=3]{sezioni/Istogrammi/IstogrammaDiDettaglio.png}
	\caption{Istogramma della disposizione ore per ruolo di ciascun componente della fase di progettazione di dettaglio e codifica}
\end{figure}

\subsubsection{Costo Risultate}
La seguente tabella §5.3.2 rappresenta, per ruolo, le ore totali investite e il corrispondente costo in euro.
{
	\rowcolors{2}{grigetto}{white}
	\renewcommand{\arraystretch}{2}
	\centering
	\begin{longtable}{ C{3cm} C{2cm} C{4cm}}
		\rowcolor{rossoep}
		\textcolor{white}{\textbf{Ruolo}} & \textcolor{white}{\textbf{Totale ore}} & \textcolor{white}{\textbf{Costo Ruolo in euro}}\\	
        
        Responsabile & 26 & 780 \\
        Amministratore & 30 & 600 \\
        Analista & 0 & 0 \\
        Progettista & 83 & 1826 \\
        Programmatore & 136 & 2040 \\
        Verificatore & 101 & 1515\\
        \textbf{Totale} & 376 & 6761 \\
		
	\end{longtable}
}

La suddivisione della quantità di ore totali di ciascun ruolo viene rappresentata nel seguente aerogramma:

\begin{figure}[h]
	\centering
	\includegraphics[scale=2]{sezioni/Aerogrammi/AerogrammaDiDettaglio.png}
	\caption{Suddivisione ore per ruolo della fase di Progettazione di dettaglio e codifica}
\end{figure}

\subsection{Validazione e Collaudo}

\subsubsection{Divisione Oraria}
La seguente tabella §5.4.1 rappresenta la distribuzione oraria dei ruoli per ogni componente del gruppo.
{
	\rowcolors{2}{grigetto}{white}
	\renewcommand{\arraystretch}{2}
	\centering
	\begin{longtable}{ C{5cm} C{1cm} C{1cm} C{1cm} C{1cm} C{1cm} C{1cm} C{3cm}}
		\rowcolor{rossoep}
		\textcolor{white}{\textbf{Nome membro del gruppo}} & \textcolor{white}{\textbf{RE}} & \textcolor{white}{\textbf{AM}} & \textcolor{white}{\textbf{AN}} & \textcolor{white}{\textbf{PT}} & \textcolor{white}{\textbf{PR}} & \textcolor{white}{\textbf{VE}} & \textcolor{white}{\textbf{Ore complessive}}\\	
        
        Christian Mattei & & & 7 & & 4 & 8 & 19\\
        Davide Lazzaro & & 4 & & 5 & & 10 & 19\\
        Emanuele Cisotto & 5 & & & & 6 & 12 & 23\\ 
        Enrico Salmaso & & & & 6 & 5 & 9 & 20 \\
        Federico Perin & 6 & & & & 7 & 8 & 21\\
        Francesco Drago & & 6 & & & 9 & 6 & 21\\
        Riccardo Baratin & & 5 & & & 6 & 10 & 21\\
        Tommaso Azzalin & & & 4 & 10 & & 7 & 21\\
        \textbf{Ore totali ruolo} & 11 & 15 & 11 & 21 & 37 & 70 & 165\\
		
	\end{longtable}
}


La suddivisione di quante ore ha svolto ciascun componente del gruppo per ogni ruolo viene rappresentata nel seguente istogramma:

\begin{figure}[h]
	\centering
	\includegraphics[scale=2.5]{sezioni/Istogrammi/IstogrammaValidazione.png}
	\caption{Istogramma della disposizione ore per ruolo di ciascun componente della fase di Validazione e Collaudo}
\end{figure}

\subsubsection{Costo Risultate}
La seguente tabella §5.4.2 rappresenta, per ruolo, le ore totali investite e il corrispondente costo in euro.
{
	\rowcolors{2}{grigetto}{white}
	\renewcommand{\arraystretch}{2}
	\centering
	\begin{longtable}{ C{3cm} C{2cm} C{4cm}}
		\rowcolor{rossoep}
		\textcolor{white}{\textbf{Ruolo}} & \textcolor{white}{\textbf{Totale ore}} & \textcolor{white}{\textbf{Costo Ruolo in euro}}\\	
        
        Responsabile & 11 & 330\\
        Amministratore & 15 & 300 \\
        Analista & 11 & 275\\
        Progettista & 21 & 462\\
        Programmatore & 37 & 555\\
        Verificatore & 70 & 1050\\
        \textbf{Totale} & 169 & 2972\
		
	\end{longtable}
}

La suddivisione della quantità di ore totali di ciascun ruolo viene rappresentata nel seguente aerogramma:

\begin{figure}[h]
	\centering
	\includegraphics[scale=2.5]{sezioni/Aerogrammi/AerogrammaValidazione.png}
	\caption{Suddivisione ore per ruolo della fase di Validazione e Collaudo}
\end{figure}


\clearpage
\subsection{Preventivo finale} 
Nel preventivo riportiamo la spesa totale che il committente dovrà affrontare, derivata dal totale delle ore rendicontate e preventivate nelle fasi di Progettazione Architetturale, Progettazione di Dettaglio e Codifica, Validazione e Collaudo.\\

\subsubsection{Divisione oraria complessiva} 
Nella seguente tabella viene mostrata la distribuzione oraria dei ruoli per ogni componente del gruppo.\\
{
	\rowcolors{2}{grigetto}{white}
	\renewcommand{\arraystretch}{2}
	\centering
	\begin{longtable}{ C{5cm} C{1cm} C{1cm} C{1cm} C{1cm} C{1cm} C{1cm} C{3cm}}
		\rowcolor{rossoep}
		\textcolor{white}{\textbf{Nome membro del gruppo}} & \textcolor{white}{\textbf{RE}} & \textcolor{white}{\textbf{AM}} & \textcolor{white}{\textbf{AN}} & \textcolor{white}{\textbf{PT}} & \textcolor{white}{\textbf{PR}} & \textcolor{white}{\textbf{VE}} & \textcolor{white}{\textbf{Ore complessive}}\\	
        
        Christian Mattei & 5 & 10 & 7 & 26 & 29 & 22 & 99 \\
        Davide Lazzaro & 14 & 4 & & 33 & 15 & 33 & 99\\
        Emanuele Cisotto & 5 & 13 & & 23 & 27 & 31 & 99 \\
        Enrico Salmaso & & 14 & 6 & 27 & 23 & 29 & 99\\
        Federico Perin & 6 & 14 & & 13 & 33 & 33 & 99\\
        Francesco Drago & 10 & 6 & 13 & 12 & 33 & 25 & 99 \\
        Riccardo Baratin & 6 & 8 & & 12 & 42 & 31 & 99 \\
        Tommaso Azzalin & 10 & & 12 & 27 & 17 & 33 & 99 \\
        \textbf{Ore totali ruolo} & 56 & 69 & 38 & 173 & 219 & 236 &  792 \\

	\end{longtable}
}
\subsubsection{Costo complessivo per ruolo}
Nella seguente tabella viene illustrato il monte ore risultante per ogni ruolo con il costo ad esso associato.\\
{
	\rowcolors{2}{grigetto}{white}
	\renewcommand{\arraystretch}{2}
	\centering
	\begin{longtable}{ C{3cm} C{2cm} C{4cm}}
		\rowcolor{rossoep}
		\textcolor{white}{\textbf{Ruolo}} & \textcolor{white}{\textbf{Totale ore}} & \textcolor{white}{\textbf{Costo Ruolo in euro}}\\	
        
        Responsabile & 56 &  1680\\
        Amministratore & 69 & 1380 \\
        Analista & 38 & 950 \\
        Progettista & 173 & 3806 \\
        Programmatore & 219 & 3285 \\
        Verificatore & 237 & 3555 \\
		
	\end{longtable}
}

\subsubsection{Costo complessivo}
Nella seguente tabella vengono riportati i costi complessivi delle varie fasi e infine l'importo proposto da qbTeam per la realizzazione del progetto Stalker.\\
{
	\rowcolors{2}{grigetto}{white}
	\renewcommand{\arraystretch}{2}
	\centering
	\begin{longtable}{ C{5cm} C{5cm}}
		\rowcolor{rossoep}
		\textcolor{white}{\textbf{Fase}} & \textcolor{white}{\textbf{Costo Fase}}\\	
		
		Progettazione Architetturale & 4923 \\
		Progettazione di Dettaglio e Codifica & 6761 \\
		Validazione e Collaudo & 2972 \\
		\textbf{Totale} & 14656\\
		
	\end{longtable}
}




\clearpage
\section{Consuntivo}

Nel Consuntivo approfondiamo il bilancio effettivo della fase di Analisi, ovvero la differenza tra le ore preventivate e quelle effettive.

\subsection{Analisi dei Requisiti}

Il bilancio della fase di Analisi è negativo, ovvero il gruppo ha impiegato più ore di quelle anticipate, quindi il costo finale della fase di Analisi supera l'aspettaviva calcolata in precedenza.\\
La seguente tabella illustra la differenza oraria ed economica rilevata a posteriori.

\rowcolors{2}{grigetto}{white}
\renewcommand{\arraystretch}{2}
\centering
\begin{longtable}{ C{5cm} C{1cm} C{1cm} C{1cm} C{1cm} C{1cm} C{1cm} C{3cm}}
	\rowcolor{rossoep}
	\textcolor{white}{\textbf{Ruolo}} & \textcolor{white}{\textbf{Ore preventivate}} & \textcolor{white}{\textbf{Ore supplementari}} & \textcolor{white}{\textbf{Costo preventivato}} & \textcolor{white}{\textbf{Costo effettivo}} & \textcolor{white}{\textbf{Differenza di costo}}\\	
	
	Responsabile & & & & & \\
	Amministratore & & & & & \\
	Analista & & & & & \\
	Progettista & & & & & \\
	Programmatore & & & & & \\
	Verificatore & & & & & \\
	\textbf{Totale} & & & & & & \\
	
\end{longtable}
\clearpage
\section{Organigramma}
\subsection{Redazione}
{
	\rowcolors{2}{grigetto}{white}
	\renewcommand{\arraystretch}{2}
	\begin{longtable}{ C{5cm} C{4cm} C{5cm} }
		\rowcolor{darkblue}
		\textcolor{white}{\textbf{Nome}} & \textcolor{white}{\textbf{Data di redazione}} & \textcolor{white}{\textbf{Firma}}\\	\endhead
        
        \LD{} & 2020-01-07 &
        \includegraphics[scale=0.60]{sezioni/Firme/Davide.png}  \\
        \SE{} & 2020-01-07 &
        \includegraphics[scale=0.70]{sezioni/Firme/Enrico.png}  \\
        		
	\end{longtable}
}

\subsection{Approvazione}
{
	\rowcolors{2}{grigetto}{white}
	\renewcommand{\arraystretch}{2}
	\centering
	\begin{longtable}{ C{5cm} C{4cm} C{5cm} }
		\rowcolor{darkblue}
		\textcolor{white}{\textbf{Nome}} & \textcolor{white}{\textbf{Data di Approvazione}} & \textcolor{white}{\textbf{Firma}}\\	\endhead
		
		
		\SE{} & 2020-01-13 &  
		\includegraphics[scale=0.70]{sezioni/Firme/Enrico.png}\\
		\VT{} &  & \\
		\CR{} & &  \\
		
	\end{longtable}
}

\subsection{Accettazione dei componenti}
{
	\rowcolors{2}{grigetto}{white}
	\renewcommand{\arraystretch}{2}
	\begin{longtable}{ C{5cm} C{4cm} C{5cm} }
		\rowcolor{darkblue}
		\textcolor{white}{\textbf{Nome}} & \textcolor{white}{\textbf{Data di Accettazione}} & \textcolor{white}{\textbf{Firma}}\\	\endhead
		
		
		\MC{} & 2020-01-10 & \includegraphics[scale=0.70]{sezioni/Firme/Christian.png}\\
		\LD{} & 2020-01-10 & \includegraphics[scale=0.60]{sezioni/Firme/Davide.png}\\
		\CE{} & 2020-01-10 & \includegraphics[scale=0.70]{sezioni/Firme/Emanuele.png} \\
		\SE{} & 2020-01-10 & \includegraphics[scale=0.70]{sezioni/Firme/Enrico.png}\\
		\PF{} & 2020-01-10 & \includegraphics[scale=0.50]{sezioni/Firme/Federico.png}\\
		\DF{} & 2020-01-10 & \includegraphics[scale=0.70]{sezioni/Firme/Francesco.png} \\
		\BR{} & 2020-01-10& \includegraphics[scale=0.70]{sezioni/Firme/Riccardo.png} \\
		\AT{} & 2020-01-10 & \includegraphics[scale=0.70]{sezioni/Firme/Tommaso.png} \\
		
		
	\end{longtable}
}

\clearpage
\subsection{Componenti}
{
	\rowcolors{2}{grigetto}{white}
	\renewcommand{\arraystretch}{2}
	\begin{longtable}{ C{4cm} C{2cm} C{8cm} }
		\rowcolor{darkblue}
		\textcolor{white}{\textbf{Nome}} & \textcolor{white}{\textbf{Matricola}} & \textcolor{white}{\textbf{Indirizzo di posta elettronica}}\\\endhead	
		
		\MC{} & 1121305 & christian.mattei@studenti.unipd.it \\
		\LD{} & 1162190 & davidemaria.lazzaro@studenti.unipd.it\\
		\CE{} & 1161514 & emanuele.cisotto@studenti.unipd.it\\
		\SE{} & 1166175 & enrico.salmaso.2@studenti.unipd.it \\
		\PF{} & 1170747 & federico.perin.1@studenti.unipd.it \\
		\DF{} & 1146928 & francesco.drago.1@studenti.unipd.it \\
		\BR{} & 1148142 & riccardo.baratin@studenti.unipd.it \\
		\AT{} & 1169740 & tommaso.azzalin@studenti.unipd.it \\
		
		
	\end{longtable}
}
\clearpage


\end{document}
\subparagraph*{Piano di Qualifica}
Nel documento \PdQ{} verrà descritta la strategia utilizzata dai verificatori per effettuare nel miglior modo possibile la verifica e la validazione di tutti i documenti prodotti da \Gruppo{}.

Lo scopo nel dirigere il \PdQ{} è quello di:
\begin{itemize}
	\item Illustrare come si intende gestire la qualità di processo e di prodotto;
	\item Elencare le varie metriche definite per aderire alle definizioni degli standard;
	\item Elencare i test per verificare la corretta soddisfazione dei requisiti del prodotto software.
\end{itemize}

La qualità di processo e la qualità di prodotto sono due aspetti chiaramente coordinati, ma vengono gestiti separatamente. \\ \\
Le sezioni principali del documento sono le seguenti:
\begin{itemize}
    \item \textbf{Qualità di processo:} Sezione dove vengono elencate le metriche inerenti ai \glo{processi};
    \item \textbf{Qualità di prodotto:} Sezione dove vengono elencate le metriche inerenti al prodotto;
    \item \textbf{Strategia di testing:} Sezione dove viene elencato il piano di testing delle componenti e del sistema software nel suo complesso;
\end{itemize}


\subparagraph*{Analisi dei Requisiti}

L'analisi dei requisiti è un'attività che avviene prima di quella di sviluppo.\\
Il documento \AdR{}, redatto dagli analisti, ha come scopo i seguenti punti:
\begin{itemize}
\item Definire lo scopo del prodotto da realizzare;
\item Fissare le funzionalità del progetto concordate col proponente;
\item Fornire ai progettisti riferimenti precisi ed affidabili per la progettazione dell'architettura software;
\item Definire una base per integrare i raffinamenti che permettono un miglioramento continuo del prodotto e del \glo{processo} di sviluppo;
\item Fornire ai verificatori dei riferimenti per l’attività di controllo;
\item Fornire una stima del quantitativo di lavoro da svolgere per tracciare una stima dei costi. 
\end{itemize}

L'obiettivo è quello di creare un documento formale contenente tutti i \glo{requisiti} richiesti e concordati col proponente.\\
Deve essere possibile fare riferimento a quanto redatto nel documento \AdR{} qualora sorgessero incomprensioni e dubbi al momento del collaudo del prodotto.

\subparagraph*{Verbali}

I verbali vengono redatti per tenere traccia delle decisioni prese negli incontri sia interni che esterni, in modo che ogni membro in caso non possa presenziare ad un incontro possa ritrovare le decisioni e aggiornarsi sui temi discussi.

\subparagraph*{Glossario}

Il Glossario raccoglie tutte le parole che riteniamo debbano essere disambiguate o contestualizzate nell'ambito del progetto per offrire una lettura che sia il meno possibile confusionaria.

\subparagraph*{Manuale Utente}
Il Manuale Utente è un documento il cui scopo è fornire delle spiegazioni semplici sull'utilizzo del sistema \NomeProgetto{} realizzato dal gruppo \Gruppo{}.\\
Il gruppo, dopo essersi confrontato e avendo analizzato come negli ultimi tempi vengano prodotti i manuali utenti degli altri software, ha deciso di non realizzare un documento nello stesso formato degli altri documenti di progetto: non viene quindi realizzato un file PDF in \LaTeX{} contenenti le istruzioni per utilizzare il sistema.
Si è deciso invece di realizzare un sito web, disponibile e visitabile da tutti i dispositivi con una connessione a internet (quindi non solo per PC) puntando molto sui contenuti multimediali invece di quelli testuali.
Il sito web è formato dalle seguenti pagine principali:
\begin{itemize}
    \item Introduzione: Pagina in cui viene introdotto al lettore lo scopo e le funzionalità offerte da \NomeProgetto{};
    \item Funzionalità (Sezione di App utenti e Web-app amministratori): Pagine in cui vengono mostrate le funzionalità dell'app (sezione §2.1 del sito) e della web-app (sezione §3.1 del sito), diviso in piccole sezioni monotematiche che presentano un breve testo contestualizzante, delle immagini e un breve video in cui viene illustrato l'utilizzo della funzionalità;
    \item Installazione (Sezione di App utenti e Web-app amministratori): Attualmente non disponibile, sono delle pagine in cui viene indicato, sempre con contenuti misti multimediali e testuali, come installare nel proprio smartphone e/o tablet l'applicazione per utenti e come utilizzare e/o installare la web-app per amministratori.
\end{itemize}

Il sito web in questione è accessibile presso questo link: \href{https://stalker-manuale-utente.readthedocs.io/}{https://stalker-manuale-utente.readthedocs.io/}.

\subparagraph*{Manuale Manutentore}
Il Manuale Manutentore è un documento il cui scopo è fornire delle spiegazioni dettagliate su come è possibile sviluppare nuove funzionalità per tutte e tre le componenti del sistema \NomeProgetto{} realizzato dal gruppo \Gruppo{}.\\
Come per il Manuale Utente, si è deciso di realizzare un sito web al posto di un documento PDF. A maggior ragione, è diventata ormai prassi per il software e per le librerie, ancor di più quelli open source (tra cui il nostro), realizzare siti web comodamente navigabili in cui consultare la documentazione del codice sorgente.
Oltre al codice sorgente, nel Manuale Manutentore realizzato dal gruppo vi sono anche le spiegazioni sulle scelte architetturali e i vari diagrammi UML di classi, package e sequenza.
Il sito web è formato dalle seguenti sezioni principali:
\begin{itemize}
    \item Introduzione: Pagina in cui viene introdotto al lettore lo scopo e le funzionalità offerte da \NomeProgetto{};
    \item App utenti: Sezione in cui viene illustrato come l'app utenti è composta;
    \item Web-app amministratori: Sezione in cui viene illustrato come la web-app amministratori è composta;
    \item Backend: Sezione in cui viene illustrato come il backend è composto;
    \item REST API: Sezione in cui viene illustrato come le REST API possono essere utilizzate (sezione scritta in lingua inglese per raggiungere un pubblico di sviluppatori maggiore).
\end{itemize}

Ciascuna delle sezioni App utenti, Web-app amministratori e Backend, è composta
\begin{itemize}
    \item Introduzione: Pagina in cui vengono introdotte le funzionalità offerte da ciascuna delle componenti del sistema \NomeProgetto{}, specificatamente di quelle della sezione a cui fanno riferimento;
    \item Requisiti e installazione: Pagina in cui viene indicato, passo dopo passo, a uno sviluppatore esterno (o interno) al gruppo ciò che è necessario e cosa fare per poter sviluppare o effettuare attività di manutenzione nelle componenti software del sistema;
    \item Estendibilità: Pagina in cui viene indicato quali sono, oltre all'attività di manutenzione del codice già esistente, le possibilità di estensione e modifica (più o meno retro-compatibili) del sistema;
    \item Architettura: Pagina in cui vengono illustrate le scelte architetturali della app, web-app e backend;
    \item Diagrammi dei package, classi, sequenza: Pagine in cui vengono illustrate le scelte di progettazione di dettaglio delle componenti dell'architettura di ciascuna delle parti che compongono il sistema Stalker;
\end{itemize}

La sezione REST API invece, è formata dalle seguenti pagine e sezioni:
\begin{itemize}
    \item Introduction: Pagina in cui viene introdotto gli scopi delle REST API di \NomeProgetto{};
    \item Overview: Pagina in cui viene mostrata la lista di tutte le API accessibili da un client HTTP per utilizzare le funzionalità di \NomeProgetto{};
    \item Model: Sezione composta dalle pagine che mostrano in dettaglio come sono composte le entità del sistema \NomeProgetto{};
    \item Api: Sezione composta dalle pagine che mostrano in dettaglio gli end-point accessibili per implementare all'interno di un applicativo software le funzionalità offerte dal sistema \NomeProgetto{}.
\end{itemize}

Il sito web in questione è accessibile presso questo link: \href{https://stalker-manuale-manutentore.readthedocs.io/}{https://stalker-manuale-manutentore.readthedocs.io/}.

\paragraph*{Progettazione e Sviluppo}

\paragraph*{Template}
È presente un template per i documenti, realizzato anch'esso in \LaTeX{}, per standardizzare e velocizzare la stesura della documentazione.
Ogni documento che viene redatto deve includere al suo interno il file di stile Stiletemplate.sty presente nella cartella Utilita/.
Il template fornisce i seguenti comandi:
\begin{itemize}
\item \textbf{\textbackslash copertina\{\}}: Da inserire all'inizio del documento, ne inserisce la prima pagina (o copertina);
\item \textbf{\textbackslash fancydoc\{\}}: Da inserire dopo l'indice, inserisce l'intestazione e il piè di pagina del documento. Va utilizzato in ogni documento, ad eccezione dei verbali;
\item \textbf{\textbackslash fancyverbale\{\}}: Da inserire dopo l'indice, inserisce l'intestazione e il piè di pagina del verbale. Da utilizzare solamente nei verbali.
\end{itemize}

\paragraph*{Struttura dei documenti}
Ogni documento è caratterizzato da una struttura che dovrà seguire obbligatoriamente.
Di seguito viene elencato, per ogni sezione del documento, le sue caratteristiche e la sua posizione.

\subparagraph*{Prima pagina - Copertina} 
\begin{itemize}
\item \textbf{Logo del gruppo}: Il logo viene posizionato al centro della pagina, in alto;
\item \textbf{Titolo del documento}: Indica il nome del documento ed è posizionato sotto al logo del gruppo;
\item \textbf{Informazioni gruppo}: Nome del gruppo (\Gruppo{}) e del capitolato, seguiti subito dopo dall'e-mail del gruppo; 
\item \textbf{Informazioni documento}: Tabella posizionata al centro della pagina contenente le seguenti informazioni sul documento:
\begin{itemize}
\item Versione;
\item Approvatori;
\item Redattori;
\item Verificatori;
\item Uso;
\item Distribuzione.
\end{itemize}
\item \textbf{Descrizione documento}: Breve descrizione relativa al documento posizionata in fondo alla pagina, centrata.
\end{itemize}
\subparagraph*{Registro delle modifiche}
Tabella contenente diverse informazioni sul ciclo di vita del documento, composta come segue:
\begin{itemize}
\item Versione;
\item Data;
\item Nominativo;
\item Ruolo;
\item Descrizione.
\end{itemize}
Questa tabella non deve rispettare la norma definita in "Elementi grafici", poiché non deve presentare una didascalia.

\subparagraph*{Indice}
Contiene i titoli di tutte le sezioni e sottosezioni del documento, rendendo più facile la navigazione.
Se sono presenti tabelle o immagini all'interno del documento, esse possono essere riepilogate in un indice separato, se ritenuto necessario.

\subparagraph*{Riferimenti}
Tutti i riferimenti bibliografici, normativi ed informativi, vengono inseriti in ogni documento in cui siano necessari come ultima sezione del documento, sotto forma di elenco puntato.

\subparagraph*{Intestazione - Piè di pagina}
Il contenuto del documento è posto tra intestazione e il piè di pagina:

\begin{itemize}
\item In alto a sinistra è presente il logo del gruppo \Gruppo{};
\item In alto una linea orizzontale separa l’intestazione dal contenuto;
\item In basso a sinistra è presente il titolo del documento;
\item In basso a destra è presente il numero della pagina;
\item In basso una linea orizzontale separa il piè di pagina dal contenuto.
\end{itemize}

\subparagraph*{Elementi grafici}
\begin{itemize}
\item \textbf{Tabelle}: Centrate, con la didascalia posizionata al di sopra di esse;
\item \textbf{Diagrammi}: Centrati, con la didascalia posizionata al di sotto di essi;
\item \textbf{Immagini}: Centrate, con la didascalia posizionata al di sotto di esse.
\end{itemize}

\subparagraph*{Stile del testo}
\begin{itemize}
    \item \textbf{Grassetto}:
    \begin{itemize}
        \item Utilizzato per evidenziare le voci di un elenco puntato, che vengono descritte in loco;
        \item Parole ritenute particolarmente importanti in un testo;
        \item Termini del documento \Glossario{};
        \item Codici identificativi, quando introdotti per la prima volta.
    \end{itemize}
    \item \textit{Corsivo}:
    \begin{itemize}
        \item Nome del capitolato;
        \item Nome del proponente;
        \item Nome del gruppo.
    \end{itemize}
    \item \textcolor{blue}{Link ipertestuali}: Tutti i link ipertestuali devono essere di colore blu.
\end{itemize}

\subparagraph*{Glossario}
Ogni termine di un documento che contiene una definizione nel \Glossario{} viene identificato nel seguente modo:
\begin{center}
    \glo{termine}
\end{center}

\subparagraph*{Data}
Si è deciso di seguire uno dei formati più diffusi per la rappresentazione della data:
\begin{center}
\textbf{YYYY-MM-DD}
\end{center}
in cui \textbf{YYYY} rappresenta l'anno, \textbf{MM} il mese e \textbf{DD} il giorno.
\subparagraph*{Elenchi puntati/numerati}
Ogni voce di un elenco ordinato o numerato comincia con la lettera maiuscola e termina con punto e virgola (\textbf{";"}) , tranne per l'ultimo elemento dell'elenco che termina con un punto fermo (\textbf{"."}).
Nel caso di elenchi che definiscono un termine:
\begin{itemize}
    \item Il termine deve essere in grassetto;
    \item Il termine deve essere seguito da due punti (\textbf{":"});
    \item La definizione del termine inizia con la lettera maiuscola.
\end{itemize}

\subparagraph*{Nomenclatura dei documenti}
I nomi di file (escludendo l'estensione) e cartelle sono scritti usando la convenzione "\glo{Camel Case}".
I file della documentazione prodotti avranno i seguenti nomi:
\begin{itemize}
    \item \textbf{AnalisiDeiRequisiti.pdf}: Contenente il documento \AdR{};
    \item \textbf{PianoDiProgetto.pdf}: Contenente il documento \PdP{};
    \item \textbf{PianoDiQualifica.pdf}: Contenente il documento \PdQ{};
    \item \textbf{NormeDiProgetto.pdf}: Contenente il documento \NdP{};
    \item \textbf{StudioDiFattibilita.pdf}: Contenente il documento \SdF{}.
\end{itemize}
I file dei verbali, esterni e interni, avranno i seguenti nomi:
\begin{itemize}
    \item \textbf{VI\_[YYYY]\_[MM]\_[DD].pdf}: Contenente il verbale interno del [YYYY]-[MM]-[DD];
    \item \textbf{VE\_[YYYY]\_[MM]\_[DD].pdf}: Contenente il verbale esterno del [YYYY]-[MM]-[DD].
\end{itemize}
con:
\begin{itemize}
    \item \textbf{[YYYY]} l'anno in cui si è tenuto l'\glo{incontro formale};
    \item \textbf{[MM]} il mese in cui si è tenuto l'\glo{incontro formale};
    \item \textbf{[DD]} il giorno in cui si è tenuto l'\glo{incontro formale}.
\end{itemize}

\subparagraph*{Tabelle nei documenti}
Le tabelle utilizzate nei documenti si differenziano sostanzialmente dalla loro struttura e dal colore.
Le tabelle con intestazione di colore:
\begin{itemize}
    \item \textcolor{rossoep}{\textbf{Rosso}}: Sono solamente le tabelle del "Registro delle modifiche";
    \item \textcolor{darkblue}{\textbf{Blu}}: Sono tutte le altre tabelle.
\end{itemize}
Le intestazioni delle tabelle devono ripetersi in ogni pagina, qualora si dovessero spezzare per mancanza di spazio.

\subsubsection{Metriche}

\paragraph{Indice di Gulpease}
\begin{itemize}
	\item \textbf{Codice:} MPC6
	\item \textbf{Descrizione:} È l'indice di leggibilità di un determinato testo. Calcola la lunghezza delle parole e delle frasi rispetto al numero totale delle lettere. Il valore è un intero da 0 a 100; se esso è inferiore a 80 sarà difficile da leggere per chi ha la licenza elementare, mentre se è inferiore a 40 sarà difficili da leggere per chi ha un diploma superiore;
	\item \textbf{Processo di riferimento:} Documentazione;
	\item \textbf{Sigla:} $IG$
	\item \textbf{Formula:} $$IG = 89 + {\frac{300 \; \cdot \; |frasi| \; - \; 10 \; \cdot \; |lettere|}{|parole|}}$$
	\item \textbf{Strumenti utilizzati:} \url{https://farfalla-project.org/readability_static/}
\end{itemize}

\subsubsection{Strumenti}
\paragraph*{\LaTeX}
Per quanto riguarda la stesura dei documenti (eccetto i manuali) il gruppo ha scelto di utilizzare il linguaggio \LaTeX{}, il quale consente una migliore qualità tipografica rispetto ai normali software di videoscrittura, ma soprattutto facilita il versionamento e la suddivisione in parti di un documento.

\paragraph*{Linguaggio UML}
Per la realizzazione dei diagrammi dei casi d'uso, delle classi, dei package, delle attività e delle sequenze viene utilizzato il linguaggio di modellazione \textbf{UML}, in particolare la versione v2.0 dello standard.

\paragraph*{Markdown e Mkdocs}
Per quanto riguarda la stesura dei manuali il gruppo ha scelto di utilizzare il linguaggio Markdown attraverso lo strumento Mkdocs, il quale consente di realizzare un sito web di documentazione in modo veramente facile e veloce, con risultati molto soddisfacenti. Come per \LaTeX{}, facilita il versionamento e la suddivisione in parti di un documento.

\paragraph*{Read the Docs}
Piattaforma online in cui il gruppo rende visitabile i due manuali, utente e sviluppatore, e con cui rende pubblico anche l'Allegato Tecnico per la revisione agile della Product Baseline.

\paragraph*{Editor \LaTeX}
Per quanto riguarda gli editor per l'utilizzo e la scrittura di \LaTeX{} vengono scelti come ufficiali:
\begin{itemize}
	\item \href{https://www.xm1math.net/texmaker/}{Texmaker}, che include editor e supporto alla correzione ortografica;
	\item \href{https://code.visualstudio.com/}{Visual Studio Code}, insieme ai plugin \href{https://github.com/James-Yu/LaTeX-Workshop}{LaTeX Workshop} (per il supporto a \LaTeX) e \href{https://github.com/bartosz-antosik/vscode-spellright}{Spell Right} (per il supporto alla correzione ortografica).
\end{itemize}
I file di \LaTeX{} (file \textbf{.tex}), per generare file PDF distribuibili, devono essere compilati.
Per \LaTeX{} sono disponibili parecchi compilatori, ma avendo definito come ufficiali i soli editor TexStudio e Visual Studio Code,
i compilatori e strumenti necessari per la compilazione sono:
\begin{itemize}
    \item \textbf{MiKTeX}: Disponibile a \href{https://miktex.org/}{questo indirizzo}, se si sceglie l'utilizzo dell'editor Texmaker;
    \item \textbf{TeX Live}: Disponibile a \href{https://www.tug.org/texlive/}{questo indirizzo}, se si sceglie l'utilizzo dell'editor Visual Studio Code (nota: è una distribuzione completa di \LaTeX, quindi offre anche strumenti che possono non essere utili ai fini del progetto);
    \item \textbf{Perl}: Disponibile a \href{http://strawberryperl.com/}{questo indirizzo} per sistemi Windows, mentre, seguendo la guida di \href{https://learn.perl.org/installing/unix_linux.html}{quest'altro} per sistemi Unix/Linux, se si sceglie l'utilizzo dell'editor Visual Studio Code.
\end{itemize}
Per compilare un file \textbf{.tex} e generare un file PDF corretto deve essere invocato tre volte il comando \textbf{pdflatex $nomeFile$}, dove $nomeFile$ è il percorso ad un file con estensione \textbf{.tex} contenente al suo interno le istruzioni:
\begin{itemize}
    \item \textbackslash begin\{document\};
    \item \textbackslash end\{document\}.
\end{itemize}
Questi file hanno sempre il nome del documento che producono, con le convenzioni di nomenclatura indicate nelle \NdP{}.

\paragraph*{UML}
Per realizzare tutti i diagrammi in UML vengono utilizzati i software \href{https://draw.io}{Draw.io} e \href{http://staruml.io/}{StarUML}, entrambi disponibile come applicazione desktop, mentre Draw.io è anche disponibile come applicazione web, permettendo la collaborazione in tempo reale.
Draw.io permette sia di salvare i file di lavoro per poterli modificare, sia di esportarli.
Il salvataggio dei file di lavoro, con Draw.io, può avvenire in file in formato:
\begin{itemize}
    \item \textbf{XML}: Il file ha estensione \textbf{.xml};
    \item \textbf{Draw.io}: Il file ha estensione \textbf{.drawio}.
\end{itemize}
Il salvataggio delle immagini da includere nei documenti deve essere in formato \textbf{PNG}, con estensione \textbf{.png}, ed esportata (con Draw.io) con i seguenti parametri:
\begin{itemize}
    \item \textbf{Spessore bordo}: 20;
    \item \textbf{DPI}: 400dpi;
    \item \textbf{Sfondo}: Non trasparente (spunta rimossa).
\end{itemize}
Le seguenti impostazioni di esportazioni si trovano in File $\rightarrow$ Esporta come $\rightarrow$ Avanzate\dots; il nome del file e lo zoom, così come profondità e altezza, sono indifferenti.
