Il documento \PdP, che sarà stilato dal Responsabile con l'aiuto dell'Amministratore, ha l'obiettivo di descrivere come sarà svolto il lavoro per la realizzazione del prodotto ed elencare tutte le attività che \Gruppo{} svolgerà durante il ciclo di vita del progetto.


Le sezioni principali del documento sono le seguenti:
\begin{itemize}
	\item \textbf{Analisi dei Rischi:} verrà redatto un elenco di tutti i possibili rischi che potrebbero verificarsi durante lo sviluppo del progetto con la rispettiva probabilità, gravità e rimedio;
	\item \textbf{Modello di sviluppo:} verrà data una descrizione del modello di sviluppo incrementale con delle motivazioni che ne giustificano la scelta;
	\item \textbf{Pianificazione:} verranno elencate tutte le attività e la durata approssimativa di ciascuna di essa, che il gruppo svolgerà per ogni fase;
	\item \textbf{Preventivo:} verrà proposto il costo, in base alle ore totali svolte da ciascun componente per ciascun ruolo, che \Gruppo{} presenterà al committente per la realizzazione del progetto \NomeProgetto{}.
	\item \textbf{Consuntivo:} al termine di ogni fase si redige un consuntivo di periodo per tenere traccia della differenza tra le ore preventivate e quelle effettive.
\end{itemize}