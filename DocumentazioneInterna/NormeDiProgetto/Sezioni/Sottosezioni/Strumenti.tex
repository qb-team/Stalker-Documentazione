\subsubsection{Strumenti}
Di seguito sono elencati gli strumenti utilizzati dal gruppo per sviluppare il progetto \glo{\NomeProgetto{}}:

\paragraph{Android Studio}
IDE ufficiale di Android che ha l'obiettivo di accelerare lo sviluppo e la creazione di applicazioni mobile per i dispositivi Android.
Offre, al gruppo, degli strumenti su misura per il debugging e per il testing dell'applicazione. È disponibile per il download e per la documentazione sull'utilizzo presso il \href{https://developer.android.com/studio}{sito ufficiale}.

\paragraph{Visual Studio Code}
Visual Studio Code è un editor di codice sorgente sviluppato da Microsoft per Windows, Linux e macOS. Include il supporto per debugging, un controllo per Git integrato, Syntax highlighting, IntelliSense, Snippet e refactoring del codice.
È integrabile con molti plugin per estenderne le funzionalità. È disponibile per il download e per la documentazione sull'utilizzo presso il \href{https://code.visualstudio.com/}{sito ufficiale}.

\paragraph{Angular}
Angular è un framework open source per lo sviluppo di applicazioni web, evoluzione di AngularJS. Mentre AngularJS utilizzava JavaScript, il linguaggio usato per Angular è TypeScript, un linguaggio derivato da JavaScript.
È disponibile la documentazione sull'utilizzo e le regole per l'installazione tramite \glo{npm} presso il \href{https://angular.io/}{sito ufficiale}.

\paragraph{Swagger}
Framework software open source supportato da numerosi strumenti che aiutano gli sviluppatori a progettare costruire e documentare servizi web \glo{RESTful}.
I suoi strumenti, da usare online e offline, e la sua documentazione sono disponibili presso il \href{https://swagger.io/}{sito ufficiale}

\paragraph{OpenAPI}
La specifica OpenAPI (conosciuta originariamente come la specifica Swagger) è una specifica per file di interfaccia leggibili dalle macchine per descrivere, produrre, consumare e visualizzare servizi web RESTful.
La sua documentazione è disponibile presso il \href{https://www.openapis.org/}{sito ufficiale}. 

\paragraph{Firebase}
Firebase è una piattaforma per lo sviluppo di applicazioni web e mobile. Il gruppo utilizza il servizio Authentication per l'autenticazione degli utenti sia dell'applicazione che dell'interfaccia web.
I servizi e la documentazione sono disponibili presso il \href{https://firebase.google.com/?hl=it}{sito ufficiale}.

\paragraph{Docker}
Docker è un progetto open-source che automatizza il deployment di applicazioni all'interno di contenitori software, fornendo un'astrazione aggiuntiva grazie alla virtualizzazione a livello di sistema operativo di Linux.
È disponibile per il download e per la documentazione sull'utilizzo presso il \href{https://www.docker.com/}{sito ufficiale}.

\paragraph{MySQL}
MySQL (Community Edition) è un sistema open source di gestione di database relazionali.
È disponibile per il download e per la documentazione sull'utilizzo presso il \href{https://www.mysql.com/it/}{sito ufficiale}.

\paragraph{Redis}
Redis è un software per la memorizzazione di coppie chiave-valore open source residente in memoria, caratterizzato da una velocità nella gestione delle operazioni di lettura/scrittura di molto superiore a un database relazionale.
È disponibile per il download e per la documentazione sull'utilizzo presso il \href{https://redis.io/}{sito ufficiale}.

\paragraph{Google Maps Platform}
Google Maps Platform è un insieme di API e SDK che permette agli sviluppatori di integrare nella loro applicazione i servizi, oppure di recuperare i dati, da Google Maps.
È disponibile la documentazione presso il \href{https://cloud.google.com/maps-platform?hl=it}{sito ufficiale}.

\paragraph{Google Play Services}
Google Play Services è una libreria che contiene le interfacce per i singoli servizi di Google e consente di ottenere l'autorizzazione da parte degli utenti per sfruttare questi servizi con le loro credenziali.
È disponibile la documentazione presso il \href{https://developers.google.com/android/guides/overview}{sito ufficiale}.

\paragraph{Spring}
Spring è un framework open source per lo sviluppo di applicazioni su piattaforma Java.
A questo framework sono associati tanti altri progetti, che hanno nomi composti come Spring Boot, Spring Data, Spring Batch. Il gruppo ha scelto di utilizzare particolarmente Spring Boot, Spring Data e Spring Security.
È disponibile la documentazione presso il \href{https://spring.io/}{sito ufficiale}.

\paragraph{Volley}
Volley è una libreria che cura tutti gli aspetti di accesso alla rete, ad esempio la gestione autonoma delle richieste e delle connessione multiple, gestione delle priorità e caching delle risposte sia in memoria che su disco.
È disponibile la guida per il download e per la documentazione sull'utilizzo presso il \href{https://developer.android.com/training/volley}{sito ufficiale}.

\paragraph{Maven}
È uno strumento di gestione e comprensione di progetti software. Esso si basa sul concetto di un modello di progetto ad oggetti ed è in grado di gestire la costruzione, 
il reporting e la documentazione di un progetto da un'informazione centrale.
È disponibile la guida per il download e per la documentazione sull'utilizzo presso il \href{https://maven.apache.org/}{sito ufficiale}.

\paragraph{Gradle}
È un sistema open source per l'automazione dello sviluppo che introduce un domain-specific language (DSL) basato su Groovy.
Esso è stato progettato per sviluppi multi-progetto che possono crescere fino a divenire abbastanza grandi e supporta sviluppi incrementali determinando in modo intelligente.
È disponibile la guida per il download e per la documentazione sull'utilizzo presso il \href{https://gradle.org/}{sito ufficiale}.

\paragraph{npm (Node Package Manager)}
È un gestore di pacchetti per il linguaggio di programmazione JavaScript. È il gestore di pacchetti predefinito per l'ambiente di runtime JavaScript Node.js. 
Consiste in un client da linea di comando, chiamato anch'esso npm e un database online di pacchetti pubblici e privati, chiamato npm registry.
È disponibile la documentazione sull'utilizzo presso il \href{https://www.npmjs.com/}{sito ufficiale}.

\paragraph{StarUML}
StarUML è uno strumento di modellazione software open source che supporta il framework UML (Unified Modeling Language) per la modellazione di sistemi e software. Una parte del gruppo
ha deciso di utilizzare questo software per la realizzazione dei diagrammi delle: classi, package e attività.
È disponibile la guida per il download e per la documentazione sull'utilizzo presso il \href{http://staruml.io/}{sito ufficiale}.

\paragraph{Draw.io}
Draw.io è uno strumento di modellazione software open source che supporta il framework UML (Unified Modeling Language) per la modellazione di sistemi e software. Una parte del gruppo
ha deciso di utilizzare questo software per la realizzazione dei diagrammi delle: classi, package e attività.
È disponibile l'utilizzo di questo tool direttamente sul \href{https://app.diagrams.net/}{sito ufficiale}.