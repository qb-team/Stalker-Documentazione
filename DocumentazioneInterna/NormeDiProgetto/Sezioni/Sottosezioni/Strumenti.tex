\subsubsection{Strumenti}
Di seguito sono elencati gli strumenti utilizzati dal gruppo per sviluppare il progetto \glo{\textit{Stalker}}:

\paragraph{AndroidStudio}\mbox{}\\ \\
IDE ufficiale di Android che ha l'obbiettivo di accelerare lo sviluppo e la creazione di applicazioni mobile per i dispositivi Android.
Offre, al gruppo,  strumenti su misura per il debugging e per il testing dell'applicazione.
\href{https://developer.android.com/studio}{https://developer.android.com/studio}

\paragraph{VisualStudioCode}\mbox{}\\ \\
Visual Studio Code è un editor di codice sorgente sviluppato da Microsoft per Windows, Linux e macOS. Include il supporto per debugging, un controllo per Git integrato, Syntax highlighting, IntelliSense, Snippet e refactoring del codice.
\href{https://code.visualstudio.com/}{https://code.visualstudio.com/}

\paragraph{Angular}\mbox{}\\ \\
Angular 2+ è un framework open source per lo sviluppo di applicazioni web con licenza MIT, evoluzione di AngularJS, il linguaggio usato è TypeScript.
\href{https://angular.io/}{https://angular.io/}

\paragraph{Swagger}\mbox{}\\ \\
Framework software opensource supportato da numerosi strumenti che aiutano gli sviluppatori a progettare costruire e documentare servizi web \glo{RESTful}
\href{https://swagger.io/}{https://swagger.io/}

\paragraph{OpenApi}\mbox{}\\ \\
La Specifica OpenAPI (conosciuta originariamente come la Specifica Swagger) è una specifica per file di interfaccia leggibili dalle macchine per descrivere, produrre, consumare e visualizzare servizi web RESTful.
\href{https://www.openapis.org/}{https://www.openapis.org/}

\paragraph{Firebase}\mbox{}\\ \\
Firebase è una piattaforma per lo sviluppo di applicazioni web e mobile. Il gruppo utilizza il servizio Firebase Auth per l'autenticazione degli utenti sia dell'applicazione che dell'interfaccia web.
\href{https://firebase.google.com/?hl=it}{https://firebase.google.com/?hl=it}

\paragraph{Docker}\mbox{}\\ \\
Docker è un progetto open-source che automatizza il deployment di applicazioni all'interno di contenitori software, fornendo un'astrazione aggiuntiva grazie alla virtualizzazione a livello di sistema operativo di Linux.
\href{https://www.docker.com/}{https://www.docker.com/}

\paragraph{MySQL}\mbox{}\\ \\
MySQL è un sistema opensource di gestione di database relazionali.
\href{https://www.mysql.com/it/}{https://www.mysql.com/it/}

\paragraph{Redis}\mbox{}\\ \\
Redis è un key-value store open source residente in memoria con persistenza facoltativa, caratterizzato da una velocità nella gestione delle operazioni di lettura/scrittura di molto superiore a un database relazionale.
\href{https://redis.io/}{https://redis.io/}

\paragraph{Google Maps Platform}\mbox{}\\ \\
Google Maps Platform è un insieme di API e SDK che permette agli sviluppatori di integrare nella loro applicazione i servizi, oppure di recuperare i dati, da Google Maps.
\href{https://cloud.google.com/maps-platform?hl=it}{https://cloud.google.com/maps-platform?hl=it}

\paragraph{Google Play Services}\mbox{}\\ \\
Google Play Services è una libreria che contiene le interfacce per i singoli servizi di Google e consente di ottenere l'autorizzazione da parte degli utenti per sfruttare questi servizi con le loro credenziali.
\href{https://developers.google.com/android/guides/overview}{https://developers.google.com/android/guides/overview}

\paragraph{Spring}\mbox{}\\ \\
Spring è un framework open source per lo sviluppo di applicazioni su piattaforma Java.
A questo framework sono associati tanti altri progetti, che hanno nomi composti come Spring Boot, Spring Data, Spring Batch; il gruppo ha scelto di utilizzare Spring Boot.
\href{https://spring.io/projects/spring-boot}{https://spring.io/projects/spring-boot}


\paragraph{Volley}\mbox{}\\ \\
Volley è una libreria che cura tutti gli aspetti di accesso alla rete, ad esempio la gestione autonoma delle richieste e delle connessione multiple, gestione delle priorità e caching delle risposte sia in memoria che su disco.
\href{https://developer.android.com/training/volley}{https://developer.android.com/training/volley}