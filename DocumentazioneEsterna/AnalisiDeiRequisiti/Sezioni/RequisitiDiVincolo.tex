\renewcommand{\o}{Obbligatorio}
\renewcommand{\d}{Desiderabile}
\subsection{Requisiti di vincolo}
{
\rowcolors{2}{grigetto}{white}
\renewcommand{\arraystretch}{2}
\centering
\begin{longtable}{ c C{6.5cm} c c}
\caption{Tabella dei Requisiti di vincolo}\\
\rowcolor{darkblue}
\textcolor{white}{\textbf{Identificativo}} & \textcolor{white}{\textbf{Descrizione}} & \textcolor{white}{\textbf{Classificazione}} & \textcolor{white}{\textbf{Fonti}}\\	
\endfirsthead
\rowcolor{darkblue}
\textcolor{white}{\textbf{Identificativo}} & \textcolor{white}{\textbf{Descrizione}} & \textcolor{white}{\textbf{Classificazione}} & \textcolor{white}{\textbf{Fonti}}\\
\endhead

R1VC1.1 & Deve essere sviluppato un server back-end. & \o & Capitolato \\
R1VC1.2 & Il server deve essere correlato di una \glo{UI} per la gestione delle funzioni implementate. & \o & Capitolato \\
R1VC1.3 & Deve essere sviluppata una applicazione mobile per sistemi operativi Android o iOS. & \o & Capitolato \\
R1VC2.1 & Le comunicazioni di tracciamento tra applicazione cellulare e server devono avvenire solo al momento d’ingresso ed uscita dai luoghi designati. & \o & Capitolato \\
R2VC3.1 & Possibile utilizzo di \glo{Java} (versione 8 o superiore) per sviluppo server back-end. & \d & Capitolato \\
R2VC3.2 & Possibile utilizzo di \glo{Python} per sviluppo server back-end. & \d & Capitolato \\
R2VC3.3 & Possibile utilizzo di \glo{Node.js} per sviluppo server back-end. & \d & Capitolato \\
R2VC4.1 & Possibile utilizzo di \glo{protocolli asincroni} per le comunicazioni app mobile-server. & Desiderato & Capitolato \\
R2VC5.1 & Possibile utilizzo del pattern di \glo{Publisher/Subscriber}. & \d & Capitolato \\
R2VC6.1 & Possibile utilizzo dell’IAAS \glo{Kubernetes} per la gestione della \glo{scalabilità orizzontale}. & \d & Capitolato \\
R2VC6.2 & Possibile utilizzo di \glo{PAAS} per la gestione della \glo{scalabilità orizzontale}. & \d & Capitolato \\
R2VC6.3 & Possibile utilizzo di \glo{Openshift} per la gestione della \glo{scalabilità orizzontale}. & \d & Capitolato \\
R2VC6.4 & Possibile utilizzo di \glo{Rancher} per la gestione della \glo{scalabilità orizzontale}. & \d & Capitolato \\
R1VC7.1 & Il server deve esporre oltre a eventuali protocolli richiesti per l’interazione con il servizio, delle API \glo{REST} necessarie per permettere l’utilizzo applicativo, come alternativa alle \glo{API} \glo{REST} è possibile utilizzare \glo{gRPC}. & \o & Capitolato \\
R2VC8.1 & Utilizzo di tecnologie network-\glo{GPS} per il tracciamento della posizione. & \d & Capitolato \\
R1VC8.2 & Si deve fornire un resoconto sulle scelte fatte e sui test effettuati relativi al tracciamento della posizione. & \o & Capitolato \\
R1VC9.1 & Si deve correlare a tutte le componenti applicative dei test di unità e d’integrazione. & \o & Capitolato \\
R1VC9.2 & Si deve testare tramite test end to end il sistema nella sua interezza. & \o & Capitolato \\
R1VC9.3 & Devono essere effettuati test di carico per testare il corretto funzionamento della scalabilità. & \o & Capitolato \\
R1VC9.4 & Si deve avere una copertura dei test maggiore o uguale al 80\%. & \o & Capitolato \\
R1VC9.5 & Tutti i test effettuati devono essere correlati da un report. & \o & Capitolato \\
R1VC10.1 & Deve essere scritta la documentazione relativa alle scelte implementative e progettuali con annessa motivazione. & \o & Capitolato \\
R1VC10.2 & Deve essere scritta la documentazione relativa ai problemi aperti e eventuali soluzioni proposte da esplorare. & \o & Capitolato \\
R2VC11.1 & Cifratura di tutte le comunicazioni fra App e Server. & \d & Capitolato  \\
R1VC12.1 & Si deve utilizzare il protocollo LDAP per autenticare i singoli dipendenti e gli amministratori di una organizzazione per poter effettuare il \glo{tracciamento autenticato} della posizione. & \o & Capitolato \\	
R1VV1.1 & Deve essere garantita la privacy e quindi verificare che tutte le informazioni raccolte rispettino le \glo{GDPR}. & \o & VE\_2019\_12\_16 \\
R1VV1.2 & Le credenziali per accedere alla applicazione e le credenziali per accedere e autenticarsi in una organizzazione che richiede \glo{tracciamento autenticato} devono essere diverse. & \o & VE\_2019\_12\_16 \\
R1VV1.3 & I server LDAP delle organizzazioni che richiedono autenticazione devono essere esterni al sistema.  & \o & VE\_2019\_12\_16 \\
R1VV1.4 & Per gli utenti che non devono essere autenticati, si deve assegnare a loro un codice che ne certifichi la validità dell’ingresso o dell'uscita da un luogo in modo tale da non conoscere chi lo ha generato. & \o & VE\_2019\_12\_16 \\
\end{longtable}
}