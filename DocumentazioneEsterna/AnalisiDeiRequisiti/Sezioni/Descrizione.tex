\subsection{Contesto d'uso del prodotto}
Il prodotto è orientato ai seguenti utenti: proprietari di \glo{organizzazioni} per la sezione della web-app, mentre l'app sarà orientata a visitatori e clienti di \glo{organizzazioni} pubbliche e dipendenti di \glo{organizzazioni} private.
Il prodotto servirà per tracciare gli utenti dell'applicazione al fine di rispettare le norme vigenti sulla sicurezza nei luoghi pubblici oppure per agevolare la gestione e la tracciabilità dei dipendenti dell'azienda.
Alcune funzionalità del prodotto, come la creazione di un'\glo{organizzazione} e la sua eliminazione non verranno eseguite dagli utenti a cui è rivolto lo stesso. Sarà compito degli amministratori del sistema \NomeProgetto{} occuparsene, che non rientrano tra gli attori del prodotto.

\subsection{Funzioni del prodotto}
Il prodotto garantirà le seguenti funzionalità:

\begin{itemize}
\item
  \textbf{Amministratori}: gli amministratori dovranno essere in grado,
  attraverso la web-app, di gestire la propria organizzazione,
  visualizzare gli accessi dei dipendenti e nominare altri
  amministratori per assisterli nella gestione e monitoraggio.
  L'amministratore avrà accesso alle seguenti funzionalità:

  \begin{itemize}
  \item
    \textbf{Modifica dei parametri dell'organizzazione}:
    L'amministratore può ridefinire il nome, la descrizione, l'immagine
    e l'indirizzo dell'organizzazione selezionata;
  \item
    \textbf{Modifica delle superfici geografiche di tracciamento
    dell'organizzazione}: Può modificare il perimetro di tracciamento
    dell'organizzazione e quello di specifici luoghi, inserendo un
    numero a piacere di coordinate per delimitarne la superficie di
    tracciamento (manualmente o tramite Google Maps API);
  \item
    \textbf{Gestione degli amministratori}: È possibile nominare e/o
    eliminare amministratori e modificarne i privilegi;
  \item
    \textbf{Monitoraggio degli utenti tracciati}: L'amministratore può
    sapere, in tempo reale, quanti utenti si trovano all'interno dei
    vari luoghi dell'organizzazione o dell'organizzazione in generale.
    Qualora l'organizzazione monitorata fosse gestita con tracciamento
    riconosciuto, l'amministratore è anche in grado di sapere l'identità
    dei vari utenti tracciati;
  \item
    \textbf{Visualizzazione degli accessi effettuati}: L'amministratore
    ha la possibilità di visualizzare lo storico degli accessi degli
    utenti che hanno effettuato l'accesso nell'organizzazione, qualora
    quest'ultima fosse monitorata con tracciamento riconosciuto. Per
    ogni accesso di uno specifico utente viene mostrato: il timestamp di
    ingresso, quello di uscita e il tempo di permanenza presso
    l'organizzazione.
  \item
    \textbf{Estrapolazione di report tabellari riguardanti gli accessi
    dei dipendenti e gli accesi ai vari luoghi dell'organizzazione}:

    L'amministratore può ricavare tabelle dei seguenti tipi:

    \begin{itemize}
    \item
      Ore di entrata e di uscita da un luogo di uno specifico utente;
    \item
      Totale di ore spese in ogni luogo per uno specifico utente;
    \item
      Il numero di dipendenti e il totale delle ore da loro trascorse in
      ogni luogo dell'organizzazione.
    \end{itemize}
  \end{itemize}
\item
  \textbf{Utenti:} gli utenti avranno la possibilità, con
  l'applicazione, di registrarsi e autenticarsi nell'app, di venire
  tracciati nelle organizzazioni e autenticarsi presso quelle che lo
  richiedono. Agli utenti saranno fornite le seguenti funzionalità:

  \begin{itemize}
  \item
    \textbf{Funzionalità di registrazione e autenticazione}: L'utente
    può registrarsi con delle nuove credenziali o, alternativamente,
    effettuare l'accesso con un account già registrato nel sistema.
    Qualora l'utente avesse smarrito la password, avrebbe comunque la
    possibilità di effettuarne il reset;
  \item
    \textbf{Possibilità di scaricare e aggiornare la lista delle
    organizzazioni}: L'utente ha la possibilità di scaricare la lista
    delle organizzazioni, sia quelle con tracciamento autenticato che
    quelle senza. Può inoltre effettuare l'aggiornamento della lista in
    maniera manuale, tramite un pulsante o temporizzazione;
  \item
    \textbf{Venire tracciati nelle organizzazioni desiderate}: L'utente
    verrà tracciato qualora effettuasse un movimento all'interno
    dell'organizzazione;
  \item
    \textbf{Gestione delle organizzazioni preferite}: L'utente può
    selezionare un'organizzazione e aggiungerla ai preferiti.
  \item
    \textbf{Visualizzare gli accessi effettuati presso le varie
    organizzazioni e i relativi luoghi}: L'utente ha a disposizione un
    registro degli accessi in cui sarà visualizzato l'orario di entrata
    e uscita da una determinata organizzazione, o luogo, e il tempo
    trascorso al suo interno;
  \item
    \textbf{Passare in tracciabilità anonima}: Un utente riconosciuto
    potrà decidere di passare all'anonimato, cioè di diventare un utente
    anonimo, selezionando l'apposita funzionalità.
  \end{itemize}
\end{itemize}
\subsection{Vincoli generali}
Per gli amministratori è sufficiente un browser (su di un computer con connessione ad internet); per gli utenti dell'applicazione un dispositivo con SO Android, una connessione a internet e/o un modulo GPS.