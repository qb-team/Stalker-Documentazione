\section{Introduzione}
\subsection{Scopo del documento}
Lo scopo del documento è quello di descrivere in maniera dettagliata i requisiti e i casi d'uso che sono stati individuati durante lo studio del progetto Stalker.

\subsection{Scopo generale del prodotto}
L'obbiettivo del prodotto \NomeProgetto{} di \Proponente{} è la creazione di un sistema software composto di un applicativo per cellulare e di un server, con cui interagire tramite un'interfaccia utente. La necessità nasce dal bisogno di adempiere alle normative vigenti in tema di sicurezza.
Le due componenti del sistema software, applicativo e server, devono soddisfare i seguenti obiettivi rispettivamente di:
Tracciare e registrare i \glo{movimenti} di un utente in un \glo{luogo di tracciamento} di un'\glo{organizzazione}, siano essi autenticati da credenziali di un'\glo{organizzazione} oppure visitatori anonimi, il tutto nel rispetto della normativa sulla privacy;
Poter visionare gli accessi degli utenti autenticati e visionare il numero di visitatori anonimi all'interno di un luogo.

\subsection{Glossario}
Al fine di evitare ambiguità fra i termini, e per avere chiare fra tutti gli stakeholder le terminologie utilizzate per la realizzazione del presente documento, il gruppo \Gruppo{} ha redatto un documento denominato “\Glossariov{1.0.0}”.
In tale documento, sono presenti tutti i termini tecnici, ambigui, specifici del progetto e scelti dai membri del gruppo con le loro relative definizioni.
Un termine presente nel \Glossariov{1.0.0} e utilizzato in questo documento viene indicato con un'apice \ap{G} alla fine della parola.

\subsection{Riferimenti}

\subsubsection{Riferimenti normativi}
\begin{itemize}
\item \NdPv{1.0.0};
\item \textit{VE\_2019\_12\_13}.
\end{itemize}

\subsubsection{Riferimenti informativi}
\begin{itemize}
\item \SdFv{1.0.0};
\item \textbf{Slide del capitolato C5 - Stalker}: https://www.math.unipd.it/~tullio/IS-1/2019/Progetto/C5.pdf
\item \textbf{Guide to the Software Engineering Body of Knowledge};
\item \textbf{Software Engineering (10th edition) - Ian Sommerville}.
\end{itemize}
