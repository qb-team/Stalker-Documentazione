\subsection{Requisiti qualitativi}
{
\rowcolors{2}{grigetto}{white}
\renewcommand{\arraystretch}{1.5}
\centering
\begin{longtable}{ c C{6.5cm} c c}
\caption{Tabella dei Requisiti qualitativi}\\
\rowcolor{darkblue}
\rowcolor{darkblue}
\textcolor{white}{\textbf{Identificativo}} & \textcolor{white}{\textbf{Descrizione}} & \textcolor{white}{\textbf{Classificazione}} & \textcolor{white}{\textbf{Fonti}}\\	
\endfirsthead
\rowcolor{darkblue}
\textcolor{white}{\textbf{Identificativo}} & \textcolor{white}{\textbf{Descrizione}} & \textcolor{white}{\textbf{Classificazione}} & \textcolor{white}{\textbf{Fonti}}\\
\endhead

R1QI1 & Le norme e le metriche definite nei documenti \NdP{} e \PdQ{} sono state rispettate. & Obbligatorio & Interno\\

R1QI2 & Deve essere prodotto un manuale utente. & Obbligatorio & Interno\\

R1QI3 & Deve essere prodotto un manuale amministratore. & Obbligatorio & Interno\\

R1QI4 & Il codice sorgente deve essere caricato in una repository su GitHub. & Obbligatorio & Interno\\

R1QI5 & La documentazione riguardante il software deve essere disponibile in lingua italiana. & Obbligatorio & Interno\\

R1QV1 & Si deve scegliere una licenza tra \glo{GNU}, \glo{GPL}, \glo{LGPL} e \glo{MIT}. & Obbligatorio & VE\_2019\_12\_16 \\

R1QC1 & Si deve correlare a tutte le componenti applicative dei test di unità e d’integrazione. & \o & Capitolato \\

R1QC2 & Si deve testare tramite test end to end il sistema nella sua interezza. & \o & Capitolato \\

R1QC3 & Devono essere effettuati test di carico per testare il corretto funzionamento della scalabilità. & \o & Capitolato \\

R1QC4 & Si deve avere una copertura dei test maggiore o uguale al 80\%. & \o & Capitolato \\

R1QC5 & Tutti i test effettuati devono essere correlati da un report. & \o & Capitolato \\

\end{longtable}
}