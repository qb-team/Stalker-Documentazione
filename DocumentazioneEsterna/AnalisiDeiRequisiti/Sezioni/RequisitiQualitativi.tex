\subsection{Requisiti qualitativi}
{
\rowcolors{2}{grigetto}{white}
\renewcommand{\arraystretch}{1.5}
\centering
\begin{longtable}{ c C{8cm} c c}
\caption{Tabella dei Requisiti qualitativi}\\
\rowcolor{darkblue}
\rowcolor{darkblue}
\textcolor{white}{\textbf{Identificativo}} & \textcolor{white}{\textbf{Descrizione}} & \textcolor{white}{\textbf{Classificazione}} & \textcolor{white}{\textbf{Fonti}}\\	
\endfirsthead
\rowcolor{darkblue}
\textcolor{white}{\textbf{Identificativo}} & \textcolor{white}{\textbf{Descrizione}} & \textcolor{white}{\textbf{Classificazione}} & \textcolor{white}{\textbf{Fonti}}\\
\endhead

R1QI1 & Le norme e le metriche definite nei documenti \textit{NormeDiProgetto} e \textit{PianoDiQualifica} sono state rispettate. & Obbligatorio & Interno\\

R1QI2 & Deve essere prodotto un manuale utente. & Obbligatorio & Interno\\

R1QI3 & Deve essere prodotto un manuale amministratore. & Obbligatorio & Interno\\

R1QI4 & Il codice sorgente deve essere caricato in una repository su GitHub. & Obbligatorio & Interno\\

R1QI5 & La documentazione riguardante il software deve essere disponibile in lingua italiana. & Obbligatorio & Interno\\

R1QV1 & Si deve scegliere una licenza tra GNU\ap{G}, GPL\ap{G}, LGPL\ap{G} e MIT\ap{G}. & Obbligatorio & VE\_2019\_12\_16 \\

\end{longtable}
}