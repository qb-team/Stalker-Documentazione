\section{Analisi dei Rischi}
La gestione dei rischi è un processo al quale il gruppo qbteam dà molto importanza. Questo perchè incorrere contro ad un rischio potrebbe equivalere al daneggiamento del progetto, sia nella sua organizzazione e sia nella sua qualità.
Quindi si cerca di fare una previsione dei problemi che si potranno verificare durante l'intero corso del progetto e ad ogni problema riscontrato si cerca una soluzione per poterlo evitare.

\subsection{Fasi della gestione dei rischi}
Il gruppo seguirà il processo di gestione dei rischi che è formato dalle seguenti fasi:
\item Identificazione del rischio 
\\ Questa è la prima fase del processo e serve per identificare i rischi che potrebbero portare a dei problemi durante l'avanzamento del progetto; 
\item Analisi dei rischi 
\\ Dopo aver individuato i rischi nella fase precedente, per ognuno di essi si valuterà la probabilità che esso si verifichi e le conseguenze negative che potrebbe portare;
\item Pianificazione del rischio 
\\ In Pianificazione del rischio si sviluppano dei piani per sapere quali rimedi bisognerà intraprendere nel momento in cui i rischi si verificano. In tale modo si riuscirà a risolvere i problemi prima che essi diventano gravi.
\item Monitoraggio del rischio 
\\ Nell'ultima fase della gestione del rischio si verifica che le ipotesi relative ai rischi non abbiano subito delle variazioni. Quindi si cerca di valutare periodicamente la probabilità che il rischio si verifichi e i suoi possibili effetti cercando di adottare strategie migliori alla loro risoluzione.

\subsection{Tipologia di rischio}
Ci sono 6 tipi di rischi che il gruppo qtbeam terrà in considerazione:
\begin{itemize}
	\item Rischi Tecnologici;
	\item Richi Organizzativi;
	\item Rischi Personali;
	\item Rischi dei Requisiti;
	\item Rischi Strumentali;
	\item Rischi di stima;
\end{itemize}

\subsection{Tabella dei rischi}
Nella seguente tabella veranno elencati i rischi che il gruppo qbteam potrebbe incontrare durante l'intero ciclo di vita del progetto.
Ogni colonna di un rischio sarà composto da:
\begin{itemize}
	\item nome;
	\item tipo;
	\item probabilità e gravità;
	\item descrizione;
	\item rilevamento;
	\item risoluzione;
\end{itemize}
