\section{Modello di Sviluppo}
Come modello di sviluppo il gruppo \Gruppo{} ha deciso di adottare il \textbf{modello incrementale}.
\subsection{Descrizione}
Nel modello incrementale il prodotto viene sviluppato tramite rilasci successivi. Questi rilasci hanno l'obiettivo di aggiungere funzionalità separate e accessorie a un sistema stabile in cui sono presenti requisiti di base.
Nel caso in cui un rilascio sia fallace è molto facile tornare allo stato funzionante precedente.\\
Il modello incrementale richiede, dunque, una suddivisione preliminare dei requisiti atta ad identificare quelli da sviluppare per primi e quali aggiungere al sistema stabile per incrementi. \\
Inoltre, una volta implementate le caratteristiche base del sistema lo si può sottoporre al committente e al proponente per assicurarsi di star procedendo nella giusta direzione.
In caso negativo, non è troppo tardi per cambiare la struttura del prodotto corrente. \\
Infine, non è particolarmente dispendioso riformulare degli incrementi previsti ma che devono ancora essere implementati. 

\subsection{Motivazioni}
Il gruppo ha scelto questo modello di sviluppo perché si adatta bene alle specifiche del progetto \NomeProgetto{} del proponente \Proponente{}.
Nella fattispecie, è stato facile identificare i requisiti minimi e separare molti requisiti accessori perfetti per essere implementati tramite rilasci incrementali su di un sistema stabile.\\
Inoltre, data la nostra inesperienza, il modello scelto permette a eventuali cambiamenti in corso d'opera di essere poco dispendiosi dal punto di vista sia del tempo di codifica (se circoscritti a singoli rilasci), sia del lavoro di cambiamento della documentazione. \\
In aggiunta a ciò, i rilasci successivi di funzionalità permettono di poter stabilire un confronto migliore con il proponente, riuscendo a sottoporre al suo giudizio un prodotto che sia sempre funzionante e col tempo sempre più completo e conforme alle sue aspettative. \\
Abbiamo inoltre valutato che i principali difetti del modello incrementale, quali la degradazione della struttura causata dall'aggiunta di incrementi e l'invisibilità del processo al manager, 
non influenzano il gruppo data la dimensione ridotta, relativamente ad ambienti aziendali dove i modelli di sviluppo sono sfruttati a pieno, del progetto che stiamo affrontando.