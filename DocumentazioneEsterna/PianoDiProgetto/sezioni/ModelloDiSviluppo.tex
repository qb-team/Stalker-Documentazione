\section{Modello di Sviluppo}
Come modello di sviluppo il gruppo \Gruppo{} ha deciso di adottare il \textbf{modello incrementale}.
\subsection{Descrizione}
Nel modello incrementale il prodotto viene sviluppato tramite rilasci successivi. Questi rilasci hanno l'obiettivo di aggiungere funzionalità separate e accessorie a un sistema stabile in cui sono presenti requisiti di base.
Nel caso in cui un rilascio sia fallace è molto facile tornare allo stato funzionante precedente.\\
Il modello incrementale richiede, dunque, una suddivisione preliminare dei requisiti atta ad identificare quelli da sviluppare per primi e quali aggiungere al sistema stabile per incrementi. \\
Inoltre, una volta implementate le caratteristiche base del sistema lo si può sottoporre al committente e al proponente per assicurarsi di star procedendo nella giusta direzione.
In caso negativo, non è troppo tardi per cambiare la struttura del prodotto corrente. \\
Infine, non è particolarmente dispendioso riformulare degli incrementi previsti ma che devono ancora essere implementati. 

\subsection{Motivazioni}
Il gruppo ha scelto questo modello di sviluppo perché si adatta bene alle specifiche del progetto \NomeProgetto{} del proponente \Proponente{}.
Nella fattispecie, è stato facile identificare i requisiti minimi e separare molti requisiti accessori perfetti per essere implementati tramite rilasci incrementali su di un sistema stabile.\\
Inoltre, data la nostra inesperienza, il modello scelto permette a eventuali cambiamenti in corso d'opera di essere poco dispendiosi dal punto di vista sia del tempo di codifica (se circoscritti a singoli rilasci), sia del lavoro di cambiamento della documentazione. \\
In aggiunta a ciò, i rilasci successivi di funzionalità permettono di poter stabilire un confronto migliore con il proponente, riuscendo a sottoporre al suo giudizio un prodotto che sia sempre funzionante e col tempo sempre più completo e conforme alle sue aspettative. \\
Abbiamo inoltre valutato che i principali difetti del modello incrementale, quali la degradazione della struttura causata dall'aggiunta di incrementi e l'invisibilità del processo al manager, 
non influenzano il gruppo data la dimensione ridotta, relativamente ad ambienti aziendali dove i modelli di sviluppo sono sfruttati a pieno, del progetto che stiamo affrontando.


\subsection{Individuazione degli incrementi}
In seguito è riportata una tabelle con indicati i requisiti che vengono sviluppati in ciascun incremento, sia dell'applicazione che del server.
I requisiti sono identificati dal loro codice identificativo e sono reperibili nel documento \AdR{}.\\
I codici dei requisiti enfatizzati in grassetto sono obbligatori, quelli non evidenziati sono requisiti desiderabili oppure opzionali.
La scelta di enfatizzare quelli grafici è puramente per una maggior comodità di consultazione.

{
\rowcolors{2}{grigetto}{white}
\renewcommand{\arraystretch}{2}
\centering
	
\begin{longtable}{C{4cm} C{6cm} C{4cm}}
\caption{Tabella degli incrementi}\\
\rowcolor{darkblue}

\textcolor{white}{\textbf{Incremento}} &
\textcolor{white}{\textbf{Obiettivo dell'incremento}} & 
\textcolor{white}{\textbf{Requisiti}}\\
\endhead

Incremento 1 & Funzionalità di autenticazione (applicazione e server) & \begin{itemize}
    % APP
    \item[ ] \textbf{R1FI1}
    \item[ ] \textbf{R1FA1.1}
    \item[ ] \textbf{R1FA1.2}
    \item[ ] \textbf{R1FA1.3}
    \item[ ] \textbf{R1FA1.4}
    \item[ ] \textbf{R1FA1.5}
    \item[ ] \textbf{R1FA1.6}
    \item[ ] \textbf{R1FA1.7}
    \item[ ] R2FA1.8
    \item[ ] R2FA1.9
    \item[ ] R2FA1.10
    \item[ ] R2FA1.11
    \item[ ] R2FA1.12 
    \item[ ] \textbf{R1FA2.1}
    \item[ ] \textbf{R1FA8.1}
    \item[ ] \textbf{R1FA8.2}
    \item[ ] \textbf{R1FA8.3}
    \item[ ] \textbf{R1FA8.4}
    % SERVER
    \item[ ] \textbf{R1FI2}
    \item[ ] \textbf{R1FS1.1}
    \item[ ] \textbf{R1FS1.2}
    \item[ ] R2FS1.3
    \item[ ] R2FS1.4
    \item[ ] R2FS1.5
    \item[ ] R2FS1.6
    \item[ ] R2FS1.7
    \item[ ] \textbf{R1FS2.1}
    \item[ ] \textbf{R1FS10.1}
    \item[ ] \textbf{R1FS10.2}
\end{itemize}\\

Incremento 2 & Lista delle \glo{organizzazioni} (applicazione e server) & \begin{itemize}
    % APP
    \item[ ] \textbf{R1FA3.1}
    \item[ ] \textbf{R1FA3.2}
    \item[ ] \textbf{R1FA3.3}
    \item[ ] \textbf{R1FA3.4}
    \item[ ] \textbf{R1FA3.5}
    \item[ ] \textbf{R1FA3.6}
    \item[ ] \textbf{R1FA3.7}
    \item[ ] \textbf{R1FA3.8}
    \item[ ] \textbf{R1FA3.9}
    \item[ ] \textbf{R1FA3.10}
    \item[ ] R2FA3.11
    \item[ ] R2FA3.12
    \item[ ] R3FA3.13
    \item[ ] R3FA3.14
    \item[ ] \textbf{R1FA3.15}
    \item[ ] R2FA3.16
    \item[ ] \textbf{R1FA3.17}
    \item[ ] R2FA3.18 
    \item[ ] \textbf{R1FA8.5}
    \item[ ] \textbf{R1FA8.6}
    % SERVER
    \item[ ] \textbf{R1FC3}
    \item[ ] \textbf{R1FI3}
    \item[ ] \textbf{R1FI5}
    \item[ ] \textbf{R1FI8}
    \item[ ] \textbf{R1FS3.1}
    \item[ ] \textbf{R1FS7.1}
    \item[ ] \textbf{R1FS7.2}
    \item[ ] R2FS7.3
    \item[ ] R2FS7.4
    \item[ ] R2FS7.5 
    \item[ ] \textbf{R1FS7.6}
    \item[ ] R2FI4
    \item[ ] R2FI6
    \item[ ] R2FI7
\end{itemize}\\

Incremento 3 & Tracciamento & \begin{itemize}
    % APP
    \item[ ] \textbf{R1FA4.1}
    \item[ ] \textbf{R1FA4.2}
    \item[ ] \textbf{R1FA4.3}
    \item[ ] R2FA6.1
    \item[ ] R2FA6.2
    \item[ ] R2FA6.3
    \item[ ] R2FA6.4
    \item[ ] R2FA6.5
    \item[ ] R2FA6.6
    \item[ ] R2FA6.7
    \item[ ] R2FA6.8 
    \item[ ] R2FA6.9
    \item[ ] R2FA8.7
    % SERVER 
    \item[ ] \textbf{R1FS5.1}
    \item[ ] \textbf{R1FS5.2}
    \item[ ] \textbf{R1FS5.3}
    \item[ ] \textbf{R1FS5.4}
    \item[ ] \textbf{R1FS5.5}
    \item[ ] \textbf{R1FS5.6}
    \item[ ] \textbf{R1FS5.7}
    \item[ ] \textbf{R1FS5.8}
    \item[ ] \textbf{R1FS5.9}
    \item[ ] \textbf{R1FS6.1}
    \item[ ] \textbf{R1FS6.2}
    \item[ ] \textbf{R1FS6.3}
    \item[ ] \textbf{R1FS10.9}
    \item[ ] \textbf{R1FS10.10}
\end{itemize}\\

Incremento 4 & Storico accessi di un utente (applicazione) e report tabellari degli accessi (server) & \begin{itemize}
    % APP
    \item[ ] R2FA5.1
    \item[ ] R2FA5.2
    \item[ ] R2FA5.3
    \item[ ] R2FA5.4
    \item[ ] R2FA5.5
    \item[ ] R2FA5.6
    \item[ ] R2FA5.7
    \item[ ] R2FA5.8
    \item[ ] R2FA5.9
    \item[ ] R2FA5.10
    \item[ ] R2FA5.11
    \item[ ] R3FA5.12
    \item[ ] R2FA5.13
    \item[ ] R2FA5.14
    \item[ ] R2FA5.15
    \item[ ] R2FA5.16
    \item[ ] R2FA5.17
    \item[ ] R2FA5.18
    \item[ ] R2FA8.5
    \item[ ] R2FA8.6
    % SERVER
    \item[ ] \textbf{R1FS8.1}
    \item[ ] \textbf{R1FS8.2}
    \item[ ] \textbf{R1FS8.3}
    \item[ ] \textbf{R1FS8.4}         
\end{itemize}\\

Incremento 5 & Autenticazione presso l'organizzazione (applicazione) e modifica dell'organizzazione (server) & \begin{itemize}
    % APP
    \item[ ] \textbf{R1FA7.1}
    \item[ ] \textbf{R1FA7.2}
    \item[ ] \textbf{R1FA7.3}
    \item[ ] \textbf{R1FA8.8}
    % SERVER
    \item[ ] \textbf{R1FS4.1}
    \item[ ] R2FS4.2
    \item[ ] R2FS4.3
    \item[ ] \textbf{R1FS4.4}
    \item[ ] \textbf{R1FS4.5}
    \item[ ] \textbf{R1FS4.6}
    \item[ ] R3FS4.7
    \item[ ] \textbf{R1FS4.8}
    \item[ ] \textbf{R1FS10.3}
    \item[ ] \textbf{R1FS10.4}
    \item[ ] R2FS10.5
    \item[ ] R2FS10.6
    \item[ ] \textbf{R1FS10.7}
    \item[ ] \textbf{R1FS10.8}
\end{itemize}\\

Incremento 6 & Gestione amministratori (server) & \begin{itemize}
    % SERVER
    \item[ ] \textbf{R1FI9}
    \item[ ] \textbf{R1FI10}
    \item[ ] \textbf{R1FI11}
    \item[ ] \textbf{R1FS9.1}
    \item[ ] \textbf{R1FS9.2}
    \item[ ] \textbf{R1FS9.3}
    \item[ ] \textbf{R1FS9.4}
    \item[ ] \textbf{R1FS9.5}
    \item[ ] \textbf{R1FS9.6}
    \item[ ] \textbf{R1FS9.7}
    \item[ ] \textbf{R1FS9.8}
    \item[ ] \textbf{R1FS9.9}
    \item[ ] \textbf{R1FS9.10}
    \item[ ] \textbf{R1FS9.11}
    \item[ ] \textbf{R1FS10.11}
    \item[ ] \textbf{R1FS10.12}
    \item[ ] \textbf{R1FS10.13}
    \item[ ] \textbf{R1FS10.14}
    \item[ ] \textbf{R1FS10.15}
    \item[ ] \textbf{R1FS10.16}
    \item[ ] \textbf{R1FS10.17}
    \item[ ] R2FS10.18
\end{itemize}\\

\end{longtable}
}