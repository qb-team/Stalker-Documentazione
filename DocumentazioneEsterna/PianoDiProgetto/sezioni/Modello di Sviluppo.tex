\section{Modello di Sviluppo}
Come modello di sviluppo è stato deciso di adottare il \textbf{modello incrementale}.
\subsection{Descrizione}
Nel modello incrementale si sviluppa il prodotto tramite rilasci successivi, questi rilasci hanno l'obbiettivo di aggiungere funzionalità separate e accessorie 
a un sistema stabile in cui sono presenti requisiti di base; nel caso in cui un rilascio sia fallace è molto facile tornare allo stato funzionante precedente.\\
Il modello incrementale richiede, dunque, una suddivisione preliminare dei requisiti atta ad identificare quelli da sviluppare per primi e quali aggiungere al sistema 
stabile per incrementi. \\
Inoltre una volta implementate le caratteristiche base del sistema lo si può sottoporre al committente per assicurarsi di star procedendo nella giusta direzione e, in caso negativo,
non è troppo tardi per cambiare la struttura del prodotto corrente. Inoltre, non è particolarmente dispendioso riformulare degli incrementi che devono ancora essere implementati. 

\subsection{Motivazioni}
Il nostro gruppo ha scelto questo modello di sviluppo perché si adatta bene alle specifiche del capitolato \textit{Stalker},
nella fattispecie  è stato facile identificare i requisiti minimi e separare molti requisiti accessori perfetti per essere implementati
tramite rilasci incrementali sul sistema stabile.\\
Inoltre, data la nostra inesperienza, il modello scelto permette a eventuali cambiamenti in corso d'opera di essere poco dispendiosi dal punto di vista
sia del tempo di codifica, se sono circoscritti a singoli rilasci, sia del lavoro di cambiamento della documentazione. \\
In aggiunta a ciò i rilasci successivi di funzionalità ci permettono di poter stabilire un confronto migliore con \textit{Imola Informatica}
riuscendo a sottoporre al loro giudizio un prodotto che sia sempre funzionante e nel tempo sempre più completo e conforme alle loro aspettative. \\
Abbiamo inoltre valutato che i principali difetti del modello incrementale, quali la degradazione della struttura causata dall'aggiunta di 
incrementi e l'invisibilità del processo al manager, non ci influenzano data la dimensione ridotta, in relazione a 
ambienti aziendali dove i modelli di sviluppo sono sfruttati a pieno, del progetto che stiamo affrontando.
 
