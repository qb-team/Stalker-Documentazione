\section{Pianificazione}
Nella pianificazione, il responsabile suddividerà il lavoro in attività e le assegnerà a ciascun membro del team qbteam.
Lo scopo è quello di mostrare come verrà svolto il lavoro, valutare i progressi nel progetto e anticipare i problemi che potrebbero sorgere preparando così delle soluzioni a tali problemi. 
La pianificazione di progetto è stata organizzata seguendo le scadenze presenti nella sezione Scadenze.
Lo sviluppo del progetto è stato suddiviso nei seguenti 5 macro periodi: 
\begin{itemize}
	\item Analisi;
	\item Progettazione architetturale;
	\item Progettazione di dettaglio e codifica;
	\item Validazione e collaudo;
\end{itemize}
Ogni macro periodo sarò suddiviso a sua volta in altri periodi più brevi in cui verranno elencate le diverse attività che il gruppo qbteam svolgerà.


\subsection{Analisi}

\subsubsection{Periodo 1} 
Dal 15/11/2019 al 29/11/2019\\
In questo periodo, che parte dalla formazione del gruppo e termina con la scelta del capitolato C5, abbiamo affrontato le seguenti tematiche al fine di porre le basi per il lavoro che dovevamo affrontare\\
\begin{itemize}
	\item \textbf{Discussione capitolati:} per prima cosa abbiamo studiato individualmente e in seguito discusso durante gli incontri tutti i capitolati proposti, questo ha posto le basi per la stesura del documento Studio di Fattibilità e ci ha indirizzati verso la scelta del capitolato che avremmo affrontato;
	\item \textbf{Spartizione e studio dei ruoli:} a ogni membro del gruppo è stato assegnato il ruolo che svolgerà nella fase di Analisi;
	\item \textbf{Definizione degli strumenti:} Abbiamo discusso e definito le tecnologie che avremmo usato per affrontare la fase di Analisi;
	\item \textbf{Pianificazione milesto fase di Analisi:} Abbiamo discusso e fissato delle milestone da rispettare per completare la fase di Analisi entro le scadenze imposteci;
\end{itemize}
\subsubsection{Periodo 2} 
Dal 29/11/2019 al 14/01/2020\\
Questo periodo inizia con la scelta definitiva del capitolato C5 e termina con la scadenza della consegna per affrontare la revisione dei requisiti fissata in data 21/01/2020.\\
Dopo la scelta abbiamo focalizzato le risorse del gruppo sui seguenti punti:
\begin{itemize}
	\item \textbf{Normazione: }Abbiamo definito le regole per la stesura dei documenti e per l'utilizzo delle tecnologie identificate in precedenza;
	\item \textbf{Approfondimento capitolati: }Abbiamo ulteriormente discusso tutti i capitolati in modo da terminare lo studio di fattibilità e focalizzato la nostra analisi su quello scelto in modo da predisporre le basi per l'analisi dei requisiti;
	\item \textbf{Prima definizione dei casi d'uso};
	\item \textbf{Determinazione della qualità: }Abbiamo definito le nostre strategie per garantire la qualità di processo e di prodotto;
	\item \textbf{Approfondimento delle tecnologie: }Abbiamo ampliato le nostre conoscenze sulle tecnologie richieste dal capitolato per essere svolto;
	\item \textbf{Analisi dei requisiti: }Studio dei requisiti e raffinazione dei casi d'uso;
	\item \textbf{Pianificazione attività: }Pianificazionde del lavoro da svolgere nelle fasi successive a quella di analisi;
\end{itemize}
\subsubsection{Periodo 3} 
Dal 14/01/2020 al 21/01/2020\\
In questo periodo che parte dalla consegna dei documenti per la revisione dei requisiti alla presentazione pubblica della proposta il gruppo consolida il lavoro svolto in vista delle successive fasi e della discussione per la quale serve una presentazione;
\begin{itemize}
	\item \textbf{Consolidamento:} Ogni membro si prende del tempo per ripassare tutto il lavoro svolto e per studiare il necessario per affrontare al meglio le fasi successive;
	\item \textbf{Preparazione alla discussione:} Il gruppo produce il materiale necessario da esporre alla presentazione pubblica della nostra proposta;
\end{itemize}
\subsubsection{Diagramma di Gantt delle attività}


\subsection{Progettazione architetturale}
Periodo: dal 2019-01-22 al 2019-03-16.
\\Inizia al termine dell'Analisi dei Requisiti e finisce con la data di consegna della Revisione di Progettazione.
\\In questo macro periodo viene definita una soluzione architetturale in modo da soddisfare i requisiti individuati nel periodo di Analisi dei Requisiti.

\subsubsection{Periodo 1} 
Dal 2019-01-22 al 2019-02-17
\begin{itemize}
	\item \textbf{normazione} aggiornamento della normazione;
	\item \textbf{analisi dei requisiti} aggiornamento dell'analisi dei requisiti;
	\item \textbf{pianificazione} aggiornamento della pianificazione;
	\item \textbf{};
\end{itemize}
\subsubsection{Periodo 2} 
Dal 2019-02-18 al 2019-03-08
\begin{itemize}
	\item \textbf{};
	\item \textbf{};
	\item \textbf{};
	\item \textbf{};
\end{itemize}
\subsubsection{Periodo 3} 
Dal 2019-03-09 al 2019-03-16
\begin{itemize}
	\item \textbf{};
	\item \textbf{};
	\item \textbf{};
	\item \textbf{};
\end{itemize}
\subsubsection{Diagramma di Gantt delle attività}


\subsection{Progettazione di dettaglio e codifica}

\subsubsection{Periodo 1} 
Dal al
\begin{itemize}
	\item \textbf{};
	\item \textbf{};
	\item \textbf{};
	\item \textbf{};
\end{itemize}
\subsubsection{Periodo 2} 
Dal al
\begin{itemize}
	\item \textbf{};
	\item \textbf{};
	\item \textbf{};
	\item \textbf{};
\end{itemize}
\subsubsection{Periodo 3} 
Dal 2019-04-13 al 2019-04-20
\begin{itemize}
	\item \textbf{};
	\item \textbf{};
	\item \textbf{};
	\item \textbf{};
\end{itemize}
\subsubsection{Diagramma di Gantt delle attività}


\subsection{Validazione e collaudo}

\subsubsection{Periodo 1} 
Dal al
\begin{itemize}
	\item \textbf{};
	\item \textbf{};
	\item \textbf{};
	\item \textbf{};
\end{itemize}
\subsubsection{Periodo 2} 
Dal al
\begin{itemize}
	\item \textbf{};
	\item \textbf{};
	\item \textbf{};
	\item \textbf{};
\end{itemize}
\subsubsection{Periodo 3} 
Dal 2019-05-11 al 2019-05-18
\begin{itemize}
	\item \textbf{};
	\item \textbf{};
	\item \textbf{};
	\item \textbf{};
\end{itemize}
\subsubsection{Diagramma di Gantt delle attività}