\section{Introduzione}
\subsection{Scopo del documento}
In questo documento viene illustrato come, il gruppo \textit{qbTeam}, affonta tematiche cruciali nello sviluppo del progetto \textit{Stalker} quali:
\begin{itemize}
    \item Analisi dei rischi;
    \item Decisione e giustificazione del modello di sviluppo;
    \item Temporizzazione delle varie fasi in relazione alle milestone imposte con conseguente spartimento del lavoro tra ogni membro del gruppo;
    \item Stime dei costi economici e orari necessari al compimento del progetto \textit{Stalker}.
\end{itemize}

\subsection{Struttura del documento}
In relazione alle tematiche precedentemente elencate il documento viene così sviluppato:
\begin{itemize}
    \item Introduzione, con riferimenti e scadenze di progetto;
    \item Analisi dei rischi;
    \item Modello di sviluppo;
    \item Pianificazione;
    \item Preventivo;
    \item Consuntivo.
\end{itemize}

\subsection{Scopo del Prodotto}
L'obbiettivo nel progetto \textit{Stalker} è la creazione di un applicativo per cellulare e di un server con interfaccia web che adempiano alle funzioni, rispettivamente, di 
tracciare e registrare la posizione in tempo reale dei possessori dell'applicativo, siano essi autenticati da credenziali aziendali oppure anonimi visitatori.
Questi dati raccolti devono poter essere visionati dagli amministratori delle stesse aziende e al contempo deve essere garantita la privacy degli utenti al di fuori del perimetro dell'organizzazione dalla quale vogliono farsi tracciare.

\subsection{Glossario}
Per completezza ogni termine che riteniamo non essere di immediata comprensione verrà contrassegnato con una G a pedice per indicare che nel documento \textit{Glossario v1.0.0} è presente una spiegazione inerente al contesto nel quale viene utilizzato,
in modo da evitare incomprensioni o erronee interpretazioni e rendere meno frustrante la lettura al committente.

\subsection{Riferimenti}
\subsubsection{Normativi}
\begin{itemize}
    \item \textbf{Norme di Progetto} v1.0.0;
    \item \textbf{Organigramma}: \url(https://www.math.unipd.it/~tullio/IS-1/2019/Progetto/RO.html).
\end{itemize}   
\subsubsection{Informativi}
\begin{itemize}
    \item \textbf{Capitolato d'appalto C5 - Stalker}: \url(math.unipd.it/~tullio/IS-1/2019/Progetto/C5.pdf);
    \item \textbf{Libro di Testo}: Software Engineering (10th edition)- Ian Sommerville- Pearson Education | Addison-Wesley.
\end{itemize}

\subsection{Scadenze}
Il gruppo \textit{qbTeam} ha concordato di rispettare le seguenti milestone per il progetto \textit{Stalker}
\begin{itemize}
    \item \textbf{Revisione dei Requisiti : } 21/01/2020;
    \item \textbf{Revisione di Progettazione : } 16/03/2020;
    \item \textbf{Revisione di Qualifica : } 20/04/2020;
    \item \textbf{Revisione di Accettazione : } 18/05/2020.
\end{itemize}