{
\rowcolors{2}{grigetto}{white}
\renewcommand{\arraystretch}{2}
\centering
\begin{longtable}{ C{2cm} C{4.5cm} C{4.5cm} C{4.5cm}}
\rowcolor{rossoep}
	\textcolor{white}{\textbf{Codice Nome}} & 
	\textcolor{white}{\textbf{Descrizione}} & 
	\textcolor{white}{\textbf{Rilevamento}} &  
	\textcolor{white}{\textbf{Piano di Contingenza}}\\	
\endhead


RT1 Inesperienza con le tecnologie & Il gruppo dovrà relazionarsi con tecnologie mai utilizzate precedentemente e quindi servirà del tempo per poterle utilizzare nel modo corretto & Ogni componente del gruppo sarà consapevole di saper usare o no una determinata tecnologia & Ogni componente del gruppo, che ha acquisito una certa dimestichezza nell'utilizzo di una tecnologia, cercherà di aiutare i componenti del gruppo che hanno più difficoltà con essa. \\

RP1 Conflitti decisionali & Alcuni componenti del gruppo potrebbero essere in disaccordo su alcune decisioni prese creando una situazione di malessere & Il componente del gruppo comunicherà la sua controversia riguardo una decisione  & Si cercherà di ascoltare e commentare tutte le opinioni di ciascun componente del team cercando di scegliere la soluzione più adatta per il bene del progetto. \\ 

RP2 Comunicazione Esterna & Si potrebbero avere delle difficoltà nel comunicare con il proponente esterno & Il proponente non risponderà alle e-mail del responsabile di \Gruppo{} in tempi brevi & Si cercherà di far presente al proponente \ZD{} che la comunicazione tra fornitore e cliente è molto importante per ridurre i tempi e quindi i costi. \\

RR1 Disattenzione nella definizione dei requisiti & I componenti del gruppo potrebbero interpretare male qualche requisito & I verificatori si accorgono che un requisito non è stato definito nel modo corretto & Si cercherà di condurre una precisa analisi dei requisiti chiarendo ogni dubbio di ciascuno dei componenti del gruppo. \\

RS1 Stime errate delle attività & Si potrebbero fare delle stime sbagliate sui costi, tempi e risorse utilizzate delle attività & Ciascun componente comunicherà al responsabile se non avrà rispettato una delle stime di qualche attività & Si cercherà di condurre una pianificazione e un preventivo attento per essere più coerenti possibili. \\


RO1 Non rispetto delle milestone imposte & Potrebbe accadere che a causa di impegni personali o mancanza di alcune conoscenze, qualche membro del gruppo impieghi più tempo del previsto per portare a termine il compito assegnatogli. Come conseguenza immediata, il completamento del compito sforerà oltre la scadenza temporale preposta, comportando il mancatto rispetto temporale delle milestones precedentemente concordate & Il componente che si trova in difficoltà avrà il compito di comunicarlo al gruppo. Tuttavia, anche gli altri membri possono valutare se un componente dovesse procede a rilento e, in tal caso, comunicare il problema al responsabile & Tutto il gruppo dovrà agire per offrire supporto in modo tale da terminare il lavoro entro la scadenza. \\

RO2 Eccesso o difetto nell'assegnazione delle scadenze & Data la presenza di numerosi scenari nei quali abbiamo poca esperienza, potrebbe venire realizzata un'assegnazione di scadenze sovra-stimate o sotto-stimate in relazione alla difficoltà del problema da risolvere & I componenti a cui è assegnato lo svolgimento di un compito devono riferire se la scadenza a loro imposta sia ragionevole dopo aver approfondito e compreso a fondo la difficoltà del lavoro che devono portare a termine & Se i membri del gruppo assegnati a un compito riferiscono al responsabile l'errore nella valutazione delle tempistiche si procede immediatamente ad una nuova pianificazione alla luce delle affermazioni dei membri coinvolti nel compito.\\

RO3 Assenza di comunicazione gruppo-proponente & Potrebbe accadere che si trascuri la comunicazione con \Proponente{} & Tutti i membri devono ricordarsi di mantenere un dialogo con l'azienda, cercando di raccogliere dubbi e avanzamenti per poi illustrarli alla stessa & In caso di assenza di comunicazione gruppo-azienda il responsabile deve fissare una data per un incontro, prima della prima scadenza di revisione, entro la quale il team deve impegnarsi a raccogliere domande, se presenti, e proporre tutti i progressi che sono stati fatti per ricevere feedback essenziali per il corretto avanzamento del prodotto. \\

RO4 Impossibilità di stabilire un incontro tra i membri del gruppo & Data l'elevata numerosità del gruppo, potrebbe risultare difficile fissare una data per un incontro che risulti valida per tutti i membri e sufficientemente vicina temporalmente & Il gruppo condivide una tabella con le disponibilità di ciascun membro nei vari giorni della settimana. La tabella permette di identificare con facilità i giorni in cui i membri del gruppo sono più disponibili per trovarsi e quelli in cui lo sono meno. & Per confrontarsi e organizzarsi si possono usare servizi di chiamata telematici come Discord. Inoltre, molto raramente sarà strettamente necessaria la presenza di tutti i membri del gruppo, pertanto risulterà più facile organizzarsi in sotto-gruppi e stabilire una data valida per il ritrovo.\\

\end{longtable}
}

\clearpage

\subsubsection{Tabella del Grado del Rischio}
Come descritto nelle fasi della gestione del rischio, è importante valutare il grado del rischio, ovvero stabilire la probabilità e la gravità che il rischio potrebbe avere durante il progetto.\\
Ogni colonna riporterà il codice di ciascuno dei rischi analizzati nella tabella precedente e sarà composta da:
\begin{itemize}
	\item Codice del Rischio;
	\item Frequenza;
	\item Gravità.
\end{itemize}

{
\rowcolors{2}{grigetto}{white}
\renewcommand{\arraystretch}{2}
\centering
	
\begin{longtable}{C{2cm} C{3cm} C{3cm}}
\rowcolor{rossoep}
	\textcolor{white}{\textbf{Codice}} & 
	\textcolor{white}{\textbf{Frequenza}} & 
	\textcolor{white}{\textbf{Gravità}}\\	
\endhead
		
		RT1 & Alta & Media\\
		
		RP1 & Media & Media\\
		
		RP2 & Media & Alta\\
		
		RR1 & Alta & Alta\\
		
		RS1 & Media & Bassa\\
		
		RO1 & Media & Alta\\
		
		RO2 & Media & Media\\
		
		RO3 & Bassa & Media\\
		
		RO4 & Alta & Bassa\\
		
	
	\end{longtable}
	
}
