{
\rowcolors{2}{grigetto}{white}
\renewcommand{\arraystretch}{2}
\centering
\begin{longtable}{ C{2cm} C{4.5cm} C{4.5cm} C{4.5cm}}
\rowcolor{rossoep}
\textcolor{white}{\textbf{Codice Nome}} & \textcolor{white}{\textbf{Descrizione}} & \textcolor{white}{\textbf{Rilevamento}} &  \textcolor{white}{\textbf{Piano di Contingenza}}\\	

RT1 Inesperienza con le tecnologie & Il gruppo dovrà relazionarsi con tecnologie mai utilizzate precedentemente e quindi servirà del tempo per poterle utilizzare nel modo corretto & Ogni componente del gruppo sarà consapevole di saper usare o no una determinata tecnologia & Ogni componente del gruppo che ha acquisito una certa dimestichezza nell'utilizzo di una tecnologia cercherà di aiutare i componenti del gruppo che hanno più difficoltà con essa \\

RP1 Comunicazione Interna & Durante i verbali o incontri interni, qualche componente del gruppo potrebbe essere indisponibile & Ciascun componente del gruppo comunicherà la sua assenza nel giorno dei verbali o incontri & Si aggiornerà continuamente un calendario condiviso permettendo al responsabile di fissare gli incontri in giorni e in orari in cui tutti i componenti di qbteam (o la maggior parte di essi) siano disponibili \\ 

RP2 Comunicazione Esterna & Si potrebbero avere delle difficoltà nel comunicare con il proponente esterno & Il proponente non risponderà alle mail del responsabile di qbteam in tempi brevi & Si cercherà di far presente al proponente Davide Zanetti che la comunicazione tra fornitore e cliente è molto importante per ridurre i tempi e quindi i costi \\

RR1 Disattenzione nella definizione dei requisiti & I componenti del gruppo potrebbero interpretare male qualche requisito & I verificatori si accorgono che un requisito non è stato definito nel modo corretto & Si cercherà di condurre una precisa analisi dei requisiti chiarendo ogni dubbio di ciascuno dei componenti del gruppo \\

RS1 Stime errate delle attività & Si potrebbero fare delle stime sbagliate sui costi, tempi e risorse utilizzate delle attività & Ciascun componente comunicherà al responsabile se non avrà rispettato una delle stime di qualche attività & Si cercherà di condurre una pianificazione e un preventivo attento per essere più coerenti possibili \\

% Ecco basta che riempi i successivi Rischi, ovviamente rinominando il nome
% Te ne ho messi 5 intanto, poi se son di più o di meno non importa

RO1 Non rispetto delle milestone imposte & Potrebbe accadere che per impegni personali o mancanza delle conoscenze qualche membro del gruppo impieghi più tempo del previsto non riuscendo a portare a termine il compito all'interno della scadenza assegnatagli portando il team a sforare una milestone concordata precedentemente & Il componente che si trova in difficoltà avrà il compito di comunicarlo al gruppo, inoltre anche gli altri membri possono valutare se un componente procede a rilento e comunicare il problema al responsabile & Tutto il gruppo dovrà agire per offrire supporto in modo tale da terminare il lavoro entro la scadenza\\

RO2 Eccesso o difetto nell'assegnazione delle scadenze & Data la presenza di numerosi scenari nei quali abbiamo poca esperienza potrebbe accadere una errata assegnazione di scadenze che si rivelano sovra-stimate o sotto-stimate in relazione alla difficoltà del problema da risolvere & I componenti a cui è assegnato lo svolgimento di un compito devono riferire se la scadenza a loro imposta sia ragionevole dopo aver approfondito e compreso a fondo la difficoltà del lavoro che devono portare a termine & Se i membri del gruppo assegnati a un compito riferiscono al responsabile l'errore nella valutazione delle tempistiche si procede immediatamente ad una nuova pianificazione alla luce delle affermazioni dei membri coinvolti nel compito.\\

RO3 Assenza di comunicazione gruppo-proponente & Potrebbe accadere che presi dal lavoro si trascuri la comunicazione con Imola Informatica & Tutti i membri devono ricordarsi di mantenere un dialogo con l'azienda cercando di raccogliere dubbi e portando i nostri avanzamenti & In caso di assenza di comunicazione gruppo-azienda il responsabile deve fissare una data per un incontro, prima della prima scadenza di revisione, entro la quale il team deve impegnarsi a raccogliere domande, se presenti, e proporre tutti i progressi che sono stati fatti per ricevere feedback essenziali per la corretta riuscita del prodotto. \\

RO4 Impossibilità di stabilire un incontro tra i membri del gruppo & Potrebbe accadere che dato il numero di componenti del gruppo stabilire un incontro in cui tutti siano presenti risulti difficile e in caso si riesca non abbastanza immediato come servirebbe & Il gruppo tiene una tabella con le proprie disponibilità in settimana e si possono vedere gli orari in cui tutti possono essere presenti, oppure si può notare che in nessun giorno tutti sono liberi & Per discutere si possono usare servizi di chiamata come Hangouts, inoltre molto raramente sarà strettamente necessaria la presenza di tutti i membri del gruppo quindi diventa molto più facile organizzarsi a sotto-gruppi e stabilire un incontro.\\


\end{longtable}
}

\subsubsection{Tabella del Grado del Rischio}
Come descritto nelle fasi della gestione del rischio, è importante valutare il grado del rischio, ovvero stabilire la probabilità e la gravità che il rischio potrebbe avere durante il progetto.
//Ogni colonna riporterà il codice di ciascuno dei rischi analizzati nella tabella precedente e sarà composta da:
\begin{itemize}
	\item Codice del Rischio;
	\item Frequenza;
	\item Gravità;
\end{itemize}

{
	\rowcolors{2}{grigetto}{white}
	\renewcommand{\arraystretch}{2}
	\centering
	\begin{longtable}{ C{2cm} C{3cm} C{3cm}}
		\rowcolor{rossoep}
		\textcolor{white}{\textbf{Codice}} & \textcolor{white}{\textbf{Frequenza}} & \textcolor{white}{\textbf{Gravità}}\\	
		
		RT1 & Alta & Alta\\
		
		RP1 & Media & Media\\
		
		RP2 & Media & Alta\\
		
		RT1 & Alta & Alta \\
		
		RT1 & Media & Media \\
		
		% Stessa cosa di prima
		
		RO1 & Media & Alta \\
		
		RO2 & Media & Media \\
		
		RO3 & Bassa & Media \\
		
		RO4 & Alta & Bassa \\
		
	\end{longtable}
}