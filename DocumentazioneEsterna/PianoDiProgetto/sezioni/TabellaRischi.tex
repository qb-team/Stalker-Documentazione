{
\rowcolors{2}{grigetto}{white}
\renewcommand{\arraystretch}{2}
\centering
\begin{longtable}{ C{2cm} C{4.5cm} C{4.5cm} C{4.5cm}}
\rowcolor{rossoep}
\textcolor{white}{\textbf{Codice Nome}} & \textcolor{white}{\textbf{Descrizione}} & \textcolor{white}{\textbf{Rilevamento}} &  \textcolor{white}{\textbf{Piano di Contingenza}}\\	

RT1 Inesperienza con le tecnologie & Il gruppo dovrà relazionarsi con tecnologie mai utilizzate precedentemente e quindi servirà del tempo per poterle utilizzare nel modo corretto & Ogni componente del gruppo sarà consapevole di saper usare o no una determinata tecnologia & Ogni componente del gruppo che ha acquisito una certa dimestichezza nell'utilizzo di una tecnologia, cercherà di aiutare i componenti del gruppo che hanno più difficoltà con essa \\

RP1 Comunicazione Interna & Durante i verbali o incontri interni, qualche componente del gruppo potrebbe essere indisponibile & Ciascun componente del gruppo comunicherà la sua assenza nel giorno dei verbali o incontri & Si aggiornerà continuamente un calendario condiviso permettendo al responsabile di fissare gli incontri in giorni e in orari in cui tutti i componenti di qbteam (o la maggior parte di essi) siano disponibili \\ 

RP2 Comunicazione Esterna & Si potrebbero avere delle difficoltà nel comunicare con il proponente esterno & Il proponente non risponderà alle mail del responsabile di qbteam in tempi brevi & Si cercherà di far presente al proponente Davide Zanetti che la comunicazione tra fornitore e cliente è molto importante per ridurre i tempi e quindi i costi \\

RR1 Disattenzione nella definizione dei requisiti & I componenti del gruppo potrebbero interpretare male qualche requisito & I verificatori si accorgono che un requisito non è stato definito nel modo corretto & Si cercherà di condurre una precisa analisi dei requisiti chiarendo ogni dubbio di ciascuno dei componenti del gruppo \\

RS1 Stime errate delle attività & Si potrebbero fare delle stime sbagliate sui costi, tempi e risorse utilizzate delle attività & Ciascun componente comunicherà al responsabile se non avrà rispettato una delle stime di qualche attività & Si cercherà di condurre una pianificazione e un preventivo attento per essere più coerenti possibili \\

% Ecco basta che riempi i successivi Rischi, ovviamente rinominando il nome
% Te ne ho messi 5 intanto, poi se son di più o di meno non importa

RO1 & & & \\

RO1 & & & \\

RO1 & & & \\

RO1 & & & \\
		
RO1 & & & \\

\end{longtable}
}

\subsubsection{Tabella del Grado del Rischio}
Come descritto nelle fasi della gestione del rischio, è importante valutare il grado del rischio, ovvero stabilire la probabilità e la gravità che il rischio potrebbe avere durante il progetto.
//Ogni colonna riporterà il codice di ciascuno dei rischi analizzati nella tabella precedente e sarà composta da:
\begin{itemize}
	\item Codice del Rischio;
	\item Frequenza;
	\item Gravità;
\end{itemize}

{
	\rowcolors{2}{grigetto}{white}
	\renewcommand{\arraystretch}{2}
	\centering
	\begin{longtable}{ C{2cm} C{3cm} C{3cm}}
		\rowcolor{rossoep}
		\textcolor{white}{\textbf{Codice}} & \textcolor{white}{\textbf{Frequenza}} & \textcolor{white}{\textbf{Gravità}}\\	
		
		RT1 & Alta & Alta\\
		
		RP1 & Media & Media\\
		
		RP2 & Media & Alta\\
		
		RT1 & Alta & Alta \\
		
		RT1 & Media & Media \\
		
		% Stessa cosa di prima
		
		RO1 & & \\
		
		RO1 & & \\
		
		RO1 & & \\
		
		RO1 & & \\
		
		RO1 & & \\
		
	\end{longtable}
}