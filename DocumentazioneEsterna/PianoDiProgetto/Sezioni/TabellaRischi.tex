{
\rowcolors{2}{grigetto}{white}
\renewcommand{\arraystretch}{2}
\centering
\begin{longtable}{ C{2.1cm} C{4cm} C{4cm} C{4.9cm}}
\caption{Tabella dei rischi}\\
\rowcolor{darkblue}
\textcolor{white}{\textbf{Codice Nome}} & 
\textcolor{white}{\textbf{Descrizione}} & 
\textcolor{white}{\textbf{Rilevamento}} &  
\textcolor{white}{\textbf{Piano di Contingenza}}\\	
\endhead

RT1 \newline Inesperienza con le tecnologie &
Il gruppo dovrà relazionarsi con tecnologie mai utilizzate precedentemente e quindi servirà del tempo per poterle utilizzare nel modo corretto. &
Ogni componente del gruppo sarà consapevole di saper usare o no una determinata tecnologia. &
Ogni componente del gruppo, che ha acquisito una certa dimestichezza nell'utilizzo di una tecnologia, cercherà di aiutare i componenti del gruppo che hanno più difficoltà con essa. \\

RP1 \newline Conflitti decisionali &
Alcuni componenti del gruppo potrebbero essere in disaccordo su alcune decisioni prese creando una situazione di malessere. &
Il componente del gruppo comunicherà la sua controversia riguardo una decisione. &
Si cercherà di ascoltare e commentare tutte le opinioni di ciascun componente del gruppo cercando di scegliere la soluzione più adatta per il bene del progetto. \\ 

RP2 \newline Comunicazione Esterna &
Si potrebbero avere delle difficoltà nel comunicare con il proponente esterno. &
Il proponente non risponderà alle e-mail del responsabile di \Gruppo{} in tempi brevi. &
Si cercherà di far presente al proponente \Proponente{} che la comunicazione tra fornitore e cliente è molto importante per ridurre i tempi e quindi i costi. \\

RR1 \newline Disattenzione nella definizione dei requisiti &
I componenti del gruppo potrebbero interpretare male qualche requisito. &
I verificatori si accorgono che un requisito non è stato definito nel modo corretto. &
Si cercherà di condurre una precisa analisi dei requisiti chiarendo ogni dubbio di ciascuno dei componenti del gruppo. \\

RS1 \newline Stime errate delle attività &
Si potrebbero fare delle stime sbagliate sui costi, tempi e risorse utilizzate delle attività. &
Ciascun componente comunicherà al responsabile se non avrà rispettato una delle stime di qualche attività. &
Si cercherà di condurre una pianificazione e un preventivo attento per essere più coerenti possibili. \\

RO1 \newline Non rispetto delle \glo{milestone} imposte &
Potrebbe accadere che a causa di impegni personali o mancanza di alcune conoscenze, qualche membro del gruppo impieghi più tempo del previsto per portare a termine il compito assegnatogli. Come conseguenza immediata, il completamento del compito sforerà oltre la scadenza temporale preposta, comportando il mancato rispetto temporale delle \glo{milestone} precedentemente concordate. &
Il componente che si trova in difficoltà avrà il compito di comunicarlo al gruppo. Tuttavia, anche gli altri membri possono valutare se un componente dovesse procede a rilento e, in tal caso, comunicare il problema al responsabile. &
Tutto il gruppo dovrà agire per offrire supporto in modo tale da terminare il lavoro entro la scadenza. \\

RO2 \newline Eccesso o difetto nell'assegnazione delle scadenze &
Data la presenza di numerosi scenari nei quali i membri del gruppo hanno poca esperienza, potrebbe venire realizzata un'assegnazione di scadenze sovra-stimate o sotto-stimate in relazione alla difficoltà del problema da risolvere. &
I componenti a cui è assegnato lo svolgimento di un compito devono riferire se la scadenza a loro imposta sia ragionevole dopo aver approfondito e compreso a fondo la difficoltà del lavoro che devono portare a termine. &
Se i membri del gruppo assegnati a un compito riferiscono al responsabile l'errore nella valutazione delle tempistiche si procede immediatamente ad una nuova pianificazione alla luce delle affermazioni dei membri coinvolti nel compito. \\

RO3 \newline Assenza di comunicazione gruppo-proponente &
Potrebbe accadere che si trascuri la comunicazione con il proponente \Proponente{}. &
Tutti i membri devono ricordarsi di mantenere un dialogo con l'azienda, cercando di raccogliere dubbi e avanzamenti per poi illustrarli alla stessa. &
In caso di assenza di comunicazione gruppo-azienda il responsabile deve fissare una data per un incontro, prima della prima scadenza di revisione, entro la quale il team deve impegnarsi a raccogliere domande, se presenti, e proporre tutti i progressi che sono stati fatti per ricevere feedback essenziali per il corretto avanzamento del prodotto. \\

RO4 \newline Impossibilità di stabilire un incontro tra i membri del gruppo &
Data l'elevata numerosità del gruppo, potrebbe risultare difficile fissare una data per un incontro che risulti valida per tutti i membri e sufficientemente vicina temporalmente. &
Il gruppo condivide una tabella con le disponibilità di ciascun membro nei vari giorni della settimana. La tabella permette di identificare con facilità i giorni in cui i membri del gruppo sono più disponibili per trovarsi e quelli in cui lo sono meno. &
Per confrontarsi e organizzarsi si possono usare servizi di chiamata telematici come \glo{Discord}. Inoltre, molto raramente sarà strettamente necessaria la presenza di tutti i membri del gruppo, pertanto risulterà più facile organizzarsi in sotto-gruppi e stabilire una data valida per il ritrovo. \\

\end{longtable}
}

\subsubsection{Tabella del Grado del Rischio}
Come descritto nelle fasi della gestione del rischio, è importante valutare il grado del rischio, ovvero stabilire la probabilità e la gravità che il rischio potrebbe avere durante il progetto.\\
Ogni colonna riporterà il codice di ciascuno dei rischi analizzati nella tabella precedente e sarà composta da:
\begin{itemize}
	\item Codice del Rischio;
	\item Frequenza;
	\item Gravità.
\end{itemize}

{
\rowcolors{2}{grigetto}{white}
\renewcommand{\arraystretch}{2}
\centering
	
\begin{longtable}{C{2cm} C{3cm} C{3cm}}
\caption{Tabella del Grado del Rischio}\\
\rowcolor{darkblue}

\textcolor{white}{\textbf{Codice}} & 
\textcolor{white}{\textbf{Frequenza}} & 
\textcolor{white}{\textbf{Gravità}}\\	
\endhead

RT1 &  Alta & Media \\
RP1 & Media & Media \\
RP2 & Media &  Alta \\
RR1 &  Alta &  Alta \\
RS1 & Media & Bassa \\
RO1 & Media &  Alta \\
RO2 & Media & Media \\
RO3 & Bassa & Media \\
RO4 &  Alta & Bassa \\

\end{longtable}
}

\clearpage

\subsection{Occorrenza delle situazioni di rischio}
In questa sezione viene descritto quanto i rischi indicati in §2.3 si sono effettivamente verificati durante lo svolgimento del progetto.
\subsubsection{Fase di Analisi}
{
\rowcolors{2}{grigetto}{white}
\renewcommand{\arraystretch}{2}
\centering
\begin{longtable}{C{2cm} C{3cm} C{10cm}}
\caption{Tabella occorrenza e mitigazione nella Fase di Analisi}\\
\rowcolor{darkblue}

\textcolor{white}{\textbf{Codice}} & 
\textcolor{white}{\textbf{Occorrenza}} & 
\textcolor{white}{\textbf{Descrizione e risoluzione}}\\	
\endhead

RT1 &
Bassa &
Tutti i membri del gruppo hanno dovuto apprendere il linguaggio \LaTeX{} inoltre Christian e Riccardo l'utilizzo di \glo{Git}, in quanto mai usato prima. Entrambi non hanno portato gravi ripercussioni sull'avanzamento del progetto in quanto rapidamente apprendibili (per i nostri scopi). \\

RP1 &
Bassa &
Le decisioni prese sono state generalmente accolte positivamente da tutti i membri del gruppo. Nella definizione degli attori e dei requisiti ci sono state maggiori difficoltà ad accordarsi, poi risolte durante l'incontro con il proponente. In generale, le maggiori difficoltà sono state comunque dovute all'incomprensione, poi risolte con la redazione del \Glossario. \\

RP2 &
Bassa &
Non ci sono stati particolari problemi di comunicazione con il proponente. È stato richiesto un incontro per poter discutere di dubbi che si sono presentati durante l'analisi ed è stato proficuo. \\

RR1 &
Media &
Come indicato in RP1, c'è stata disattenzione nella definizione di alcuni requisiti sfociata in difficoltà di comprensione fra i membri del gruppo. Dopo l'incontro con il proponente, queste situazioni di incomprensione sono state risolte. \\

RS1 &
Alta &
Considerando il poco tempo a disposizione e gli impegni di ogni membro del gruppo sono state richieste più ore di quante preventivate per lo svolgimento delle attività. Complice di questo errore è anche non aver preventivato in maniera ottimale la suddivisione dei tempi. Sono stati quindi suddivisi nuovamente i compiti e cercato di collaborare e comunicare il più possibile in remoto, al fine di raggiungere gli obiettivi prestabiliti. \\

RO1 &
Media &
Pur di rispettare le \glo{milestone} ogni membro del gruppo ha cercato di supportarsi a vicenda nel tentativo di rispettare le scadenze, purtroppo non sempre con successo. \\

RO2 &
Media &
Per mancata esperienza l'assegnazione di alcune scadenze sono risultate sbagliate o pianificate male. Alla luce degli errori commessi, deve essere effettuata una miglior pianificazione delle attività. \\

RO3 &
Alta &
È stato organizzato un solo incontro con il proponente che, seppur proficuo, non è sufficiente per accertarsi di essere in sintonia con esso. Nei prossimi periodi deve esserci una maggior comunicazione, anche attraverso l'utilizzo di \glo{Slack} per una maggiore interazione. \\

RO4 &
Bassa &
Essendo ancora periodo di lezioni, non ci sono stati problemi nell'incontrarci, anche perché gli incontri sono sempre stati fissati con almeno una settimana di anticipo. \\

\end{longtable}	
}

\subsubsection{Fase di Progettazione Architetturale}
{
\rowcolors{2}{grigetto}{white}
\renewcommand{\arraystretch}{2}
\centering
\begin{longtable}{C{2cm} C{3cm} C{10cm}}
\caption{Tabella occorrenza e mitigazione nella Fase di Progettazione Architetturale}\\
\rowcolor{darkblue}

\textcolor{white}{\textbf{Codice}} & 
\textcolor{white}{\textbf{Occorrenza}} & 
\textcolor{white}{\textbf{Descrizione e risoluzione}}\\	
\endhead

RT1 &
Alta &
Tutti i membri del gruppo hanno dovuto apprendere dei linguaggi e strumenti in base al compito che gli è stato assegnato. I linguaggi/strumenti presi in causa sono i seguenti: \glo{CSS}, \glo{HTML5}, \glo{JSON}, \glo{TypeScript}, \glo{Node.js}, \glo{MySQL}, \glo{Redis}, \glo{YAML}, \glo{Swagger}, \glo{OpenAPI}, \glo{Spring}, \glo{Maven}, \glo{Java} e Android Studio. Il tempo per apprendere queste tecnologie in maniera sufficiente è stato abbastanza dispendioso. In alcuni casi lo studio ha occupato più tempo del previsto ma in conclusione siamo riuscito a realizzare il \glo{PoC} funzionante. \\

RP1 &
Bassa &
Le decisioni prese sono state generalmente accolte positivamente da tutti i membri del gruppo. C'è stata qualche difficoltà sulla scelta per la realizzazione dei collegamenti tra le varie parti del \glo{PoC} (app, web-app e server) e farli funzionare correttamente; anche per l'utilizzo di certe tecnologie, varie compatibilità e versioni. In generale dopo una discussione tra i vari membri del gruppo si riusciva a venirsi incontro con le decisioni. \\

RP2 &
Bassa &
Non ci sono stati particolari problemi di comunicazione con il proponente. È stato richiesto un incontro in videoconferenza per discutere di alcuni dubbi che si sono presentati durante la scelta di certe tecnologie, linguaggi, versioni e librerie che ha aiutato alla realizzazione del \glo{PoC}. \\

RR1 &
Bassa &
Ci sono stati dubbi su alcuni requisiti e casi d'uso presentati nell'\AdRv{2.0.0}. È stata svolta una videochiamata sulla piattaforma \glo{Hangouts} con \CR{} per risolvere dubbi ed incomprensioni. \\

RS1 &
Bassa &
Per quanto riguarda lo studio delle tecnologie, la realizzazione del \glo{PoC} e la correzione dei documenti quasi tutti i tempi richiesti sono stati rispettati (in particolar modo rispetto alla precedente fase in cui l'occorrenza era stata alta). \\

RO1 &
Bassa &
Pur di rispettare le \glo{milestone} ogni membro del gruppo ha cercato di supportarsi a vicenda nel tentativo di rispettare le scadenze. In particolar modo, nonostante la suddivisione del lavoro in tre parti (e quindi gruppi di membri) ci si è comunque aiutati pur di completare il lavoro. \\

RO2 &
Bassa &
Avendo acquisito maggior esperienza nell'assegnazione di alcune scadenze sono state quasi tutte rispettate e non è servita uno stravolgimento della pianificazione. Alla luce degli errori commessi nel RR, la pianificazione ha ottenuto maggiore importanza. \\

RO3 &
Media &
È stato organizzato un solo incontro con il proponente che è stato molto proficuo. Alcuni dubbi si sono risolti attraverso l'utilizzo di \glo{Slack} per una maggiore interazione. \\

RO4 &
Media &
Essendo la fase cominciata nel periodo esami, ogni membro del gruppo aveva altri impegni importanti da rispettare.
In aggiunta ad essi, alla fine della sessione si è verificata un'altra situazione che ha complicato l'avanzamento dei lavori, ovvero la diffusione del Covid-19 che ha portato disagi come la chiusura delle aule universitarie, limitazioni di trasporti e di comunicazioni.
Il lavoro è stato necessariamente compiuto sfruttando al massimo le potenzialità che gli strumenti \glo{Discord}, \glo{Slack} e \glo{Hangouts} offrono.
L'organizzazione avveniva su \glo{Slack} mentre su \glo{Discord} avvenivano le chiamate vocali e le condivisioni degli schermi dei computer in live streaming.
I colloqui con committente e proponente sono dovuti avvenire attraverso \glo{Hangouts}, con la importante limitazione del numero di utenti in una chiamata (fisso a 6). \\

\end{longtable}	
}