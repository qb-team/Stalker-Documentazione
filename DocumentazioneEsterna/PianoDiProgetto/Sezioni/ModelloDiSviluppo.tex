\section{Modello di sviluppo}
Come modello di sviluppo il gruppo \Gruppo{} ha deciso di adottare il \textbf{modello incrementale}.
\subsection{Descrizione}
Nel modello incrementale il prodotto viene sviluppato tramite rilasci successivi. Questi rilasci hanno l'obiettivo di aggiungere funzionalità separate e accessorie a un sistema stabile in cui sono presenti requisiti di base.
Nel caso in cui un rilascio sia fallace è molto facile tornare allo stato funzionante precedente.\\
Il modello incrementale richiede, dunque, una suddivisione preliminare dei requisiti atta ad identificare quelli da sviluppare per primi e quali aggiungere al sistema stabile per incrementi. \\
Inoltre, una volta implementate le caratteristiche base del sistema lo si può sottoporre al committente e al proponente per assicurarsi di star procedendo nella giusta direzione.
In caso negativo, non è troppo tardi per cambiare la struttura del prodotto corrente. \\
Infine, non è particolarmente dispendioso riformulare degli incrementi previsti ma che devono ancora essere implementati. 

\subsection{Motivazioni}
Il gruppo ha scelto questo modello di sviluppo perché si adatta bene alle specifiche del progetto \NomeProgetto{} del proponente \Proponente{}.
Nella fattispecie, è stato facile identificare i requisiti minimi e separare molti requisiti accessori perfetti per essere implementati tramite rilasci incrementali su di un sistema stabile.\\
Inoltre, data la nostra inesperienza, il modello scelto permette a eventuali cambiamenti in corso d'opera di essere poco dispendiosi dal punto di vista sia del tempo di codifica (se circoscritti a singoli rilasci), sia del lavoro di cambiamento della documentazione. \\
In aggiunta a ciò, i rilasci successivi di funzionalità permettono di poter stabilire un confronto migliore con il proponente, riuscendo a sottoporre al suo giudizio un prodotto che sia sempre funzionante e col tempo sempre più completo e conforme alle sue aspettative. \\
Abbiamo inoltre valutato che i principali difetti del modello incrementale, quali la degradazione della struttura causata dall'aggiunta di incrementi e l'invisibilità del processo al manager, 
non influenzano il gruppo data la dimensione ridotta, relativamente ad ambienti aziendali dove i modelli di sviluppo sono sfruttati appieno, del progetto che stiamo affrontando.


\subsection{Individuazione degli incrementi}
In seguito è riportata una tabella con indicati i requisiti che vengono sviluppati in ciascun incremento, sia dell'applicazione che del server.
I requisiti sono identificati dal loro codice identificativo e sono reperibili nel documento \AdR{}.\\
\subsubsection{Incrementi con requisiti obbligatori}
I requisiti degli incrementi dall'Incremento 1 all'Incremento 5 sono requisiti solamente obbligatori.
\subsubsection{Incrementi con requisiti desiderabili e opzionali}
I requisiti contenuti dallo Incremento 6 all'Incremento 9 sono desiderabili oppure opzionali. 

{
\rowcolors{2}{grigetto}{white}
\renewcommand{\arraystretch}{2}
\centering
	
\begin{longtable}{C{2.5cm} C{3.2cm} C{2.8cm} C{2.8cm} C{2.8cm}}
\caption{Tabella degli incrementi}\\
\rowcolor{darkblue}
\textcolor{white}{\textbf{Incremento}} &
\textcolor{white}{\textbf{Obiettivo dell'incremento}} & 
\textcolor{white}{\textbf{Requisiti per l'app utenti}} &
\textcolor{white}{\textbf{Requisiti per il web-app admin}} &
\textcolor{white}{\textbf{Requisiti per il server}} \\
\endhead

Incremento 0 & \glo{Proof of Concept} (app utenti e web-app admin), lista delle organizzazioni e tracciamento (app utenti, web-app admin, server) (a livello dimostrativo per il \glo{Proof of Concept}) & \begin{itemize}
    % APP
    \item[ ] R1FI1
    \item[ ] R1FA1.1
    \item[ ] R1FA1.2
    \item[ ] R1FA1.3
    \item[ ] R1FA1.4
    \item[ ] R1FA1.5
    \item[ ] R1FA2.1
    \item[ ] R1FA3.1
    \item[ ] R1FA3.8
    \item[ ] R1FA3.10
    \item[ ] R1FA6.1
    \item[ ] R1FA8.1
    \item[ ] R1FA8.4
\end{itemize} & \begin{itemize}
    % WEB-APP ADMIN
    \item[ ] R1FI2
    \item[ ] R1FS1.1
    \item[ ] R1FS1.2
    \item[ ] R2FS1.3
    \item[ ] R2FS1.4
    \item[ ] R2FS1.5
    \item[ ] R2FS1.6
    \item[ ] R2FS1.7
    \item[ ] R1FS2.1
    \item[ ] R1FS3.1
    \item[ ] R1FS6.1
    \item[ ] R1FS10.1
    \item[ ] R1FS10.2
    \item[ ] R1FS10.14
\end{itemize} & \begin{itemize} 
    % SERVER
    \item[ ] R1FS3.1
    \item[ ] R1FI5
    \item[ ] R2FI7
    \item[ ] R1FI8
    \item[ ] R1FA3.2
    \item[ ] R1FA6.1

\end{itemize}\\

Incremento 1 & Funzionalità di autenticazione (app utenti e web-app admin) & \begin{itemize}
    % APP
    \item[ ] R1FI1
    \item[ ] R1FA1.1
    \item[ ] R1FA1.2
    \item[ ] R1FA1.3
    \item[ ] R1FA1.4
    \item[ ] R1FA1.5
    \item[ ] R1FA1.6
    \item[ ] R1FA1.7
    \item[ ] R1FA2.1
    \item[ ] R1FA8.1
    \item[ ] R1FA8.2
    \item[ ] R1FA8.3
    \item[ ] R1FA8.4
\end{itemize} & \begin{itemize}
    % WEB-APP ADMIN
    \item[ ] R1FI2
    \item[ ] R1FS1.1
    \item[ ] R1FS1.2
    \item[ ] R1FS2.1
    \item[ ] R1FS10.1
    \item[ ] R1FS10.2
\end{itemize} & \begin{itemize}
    % SERVER
    \item[ ] R1FA1.1
    \item[ ] R1FA1.2
    \item[ ] R1FA1.3
    \item[ ] R1FA1.4
    \item[ ] R1FS1.1
    \item[ ] R1FS1.2
\end{itemize}\\

Incremento 2 & Lista delle organizzazioni (app utenti e web-app admin), gestione preferiti (app utenti) & \begin{itemize}
    % APP
    \item[ ] R1FA3.1
    \item[ ] R1FA3.2
    \item[ ] R1FA3.3
    \item[ ] R1FA3.4
    \item[ ] R1FA3.5
    \item[ ] R1FA3.6
    \item[ ] R1FA3.7
    \item[ ] R1FA3.8
    \item[ ] R1FA3.9
    \item[ ] R1FA3.10
    \item[ ] R1FA3.15
    \item[ ] R1FA3.17
    \item[ ] R1FA8.5
    \item[ ] R1FA8.6
\end{itemize} & \begin{itemize} 
    % WEB-APP ADMIN
    \item[ ] R1FC3
    \item[ ] R1FI3
    \item[ ] R1FI5
    \item[ ] R1FI8
    \item[ ] R1FS3.1
    \item[ ] R1FS7.1
    \item[ ] R1FS7.2
    \item[ ] R1FS7.6
\end{itemize} & \begin{itemize} 
    % SERVER
    \item[ ] R1FC3
    \item[ ] R1FI3
    \item[ ] R1FI8
    \item[ ] R1FS3.1
    \item[ ] R1FS7.1
    \item[ ] R1FS7.2
    \item[ ] R1FS7.6
\end{itemize}\\

Incremento 3 & Tracciamento & \begin{itemize}
    % APP
    \item[ ] R1FA4.1
    \item[ ] R1FA4.2
    \item[ ] R1FA4.3
    \item[ ] R1FA6.1
    \item[ ] R1FA6.2
    \item[ ] R1FA6.3
    \item[ ] R1FA6.4
    \item[ ] R1FA8.7
    \item[ ] R1FA7.1
    \item[ ] R1FA8.8
    \item[ ] R1FA7.2
    \item[ ] R1FA7.3
\end{itemize}& \begin{itemize} 
    % WEB-APP ADMIN
    \item[ ] R1FS6.1
    \item[ ] R1FS6.2
    \item[ ] R1FS6.3
    \item[ ] R1FS8.1
    \item[ ] R1FS8.2
    \item[ ] R1FS8.3
    \item[ ] R1FS8.4

\end{itemize} & \begin{itemize} 
    % SERVER 
    \item[ ] R1FA6.1
    \item[ ] R1FA6.2
    \item[ ] R1FA6.3
    \item[ ] R1FA6.4
    \item[ ] R1FA8.7
    \item[ ] R1FS6.1
    \item[ ] R1FS6.2
    \item[ ] R1FS8.1
    \item[ ] R1FS8.2
    \item[ ] R1FS8.3
    \item[ ] R1FS8.4
\end{itemize}\\

Incremento 4 & Tracciamento movimenti (app utenti) e gestione organizzazione e luoghi (web-app amministratori) & \begin{itemize}
    % APP
    \item[ ] R1FA4.1
    \item[ ] R1FA4.2
    \item[ ] R1FA4.3
    \item[ ] R1FA6.1
    \item[ ] R1FA6.2
    \item[ ] R1FA6.3
    \item[ ] R1FA6.4
    \item[ ] R1FA8.7
\end{itemize} & \begin{itemize} 
    % WEB-APP ADMIN
    \item[ ] R1FS4.1
    \item[ ] R1FS4.4
    \item[ ] R1FS4.5
    \item[ ] R1FS4.6
    \item[ ] R1FS4.8
    \item[ ] R1FS10.3
    \item[ ] R1FS10.4
    \item[ ] R1FS10.7
    \item[ ] R1FS10.8
    \item[ ] R1FS5.1
    \item[ ] R1FS10.9
    \item[ ] R1FS5.2
    \item[ ] R1FS5.9
    \item[ ] R1FS5.3
    \item[ ] R1FS10.10
    \item[ ] R1FS5.4
    \item[ ] R1FS5.5
    \item[ ] R1FS5.6
    \item[ ] R2FS5.7
    \item[ ] R1FS5.8
\end{itemize} & \begin{itemize} 
    % SERVER
    \item[ ] R1FS4.1
    \item[ ] R1FS4.4
    \item[ ] R1FS4.5
    \item[ ] R1FS4.6
    \item[ ] R1FS4.8
    \item[ ] R1FS10.3
    \item[ ] R1FS10.4
    \item[ ] R1FS10.7
    \item[ ] R1FS10.8
    \item[ ] R1FS5.1
    \item[ ] R1FS10.9
    \item[ ] R1FS5.2
    \item[ ] R1FS5.9
    \item[ ] R1FS5.3
    \item[ ] R1FS10.10
    \item[ ] R1FS5.4
    \item[ ] R1FS5.5
    \item[ ] R1FS5.6
    \item[ ] R2FS5.7
    \item[ ] R1FS5.8
\end{itemize} \\

Incremento 5 & Autenticazione presso l'organizzazione (app utenti), gestione amministratori  e modifica dell'organizzazione (web-app admin) & \begin{itemize}
    % APP
    \item[ ] R1FS9.12
    \item[ ] R1FS9.13
    \item[ ] R1FS9.14
\end{itemize} & \begin{itemize} 
    % WEB-APP ADMIN
    \item[ ] R1FI9
    \item[ ] R1FI10
    \item[ ] R1FI11
    \item[ ] R1FS9.1
    \item[ ] R1FS9.2
    \item[ ] R1FS9.3
    \item[ ] R1FS9.4
    \item[ ] R1FS9.5
    \item[ ] R1FS9.6
    \item[ ] R1FS9.7
    \item[ ] R1FS9.8
    \item[ ] R1FS9.9
    \item[ ] R1FS9.10
    \item[ ] R1FS9.11
    \item[ ] R1FS9.12
    \item[ ] R1FS9.13
    \item[ ] R1FS9.14
    \item[ ] R1FS10.11
    \item[ ] R1FS10.12
    \item[ ] R1FS10.13
    \item[ ] R1FS10.14
    \item[ ] R1FS10.15
    \item[ ] R1FS10.16
    \item[ ] R1FS10.17
\end{itemize} & \begin{itemize}
    % SERVER
    \item[ ] R1FI9
    \item[ ] R1FI10
    \item[ ] R1FI11
    \item[ ] R1FS9.2
    \item[ ] R1FS9.3
    \item[ ] R1FS9.4
    \item[ ] R1FS9.6
    \item[ ] R1FS9.7
    \item[ ] R1FS9.8
    \item[ ] R1FS9.9
    \item[ ] R1FS9.10
    \item[ ] R1FS9.12
    \item[ ] R1FS9.13
    \item[ ] R1FS10.11
    \item[ ] R1FS10.12
    \item[ ] R1FS10.14
    \item[ ] R1FS9.14
    \item[ ] R1FS10.15
\end{itemize}\\

Incremento 6 & Funzionalità aggiuntive all'utente anonimo & \begin{itemize}
    % APP
    \item[ ] R2FA5.1
    \item[ ] R2FA5.2
    \item[ ] R2FA5.3
    \item[ ] R2FA5.4
    \item[ ] R2FA5.5
    \item[ ] R2FA5.6
    \item[ ] R2FA5.7
    \item[ ] R2FA5.8
    \item[ ] R2FA5.9
    \item[ ] R2FA5.10
    \item[ ] R2FA5.11
    \item[ ] R3FA5.12
    \item[ ] R2FA5.13
    \item[ ] R2FA5.14
    \item[ ] R3FA5.15
    \item[ ] R2FA5.16
    \item[ ] R2FA5.17
    \item[ ] R2FA8.5
    \item[ ] R2FA8.6
\end{itemize} &
    % WEB-APP ADMIN
    Nessun requisito della web-app admin previsto per questo incremento
    & \begin{itemize} 
    % SERVER
    \item[ ] R2FA5.1
    \item[ ] R2FA5.2
    \item[ ] R2FA5.3
    \item[ ] R2FA5.4
    \item[ ] R2FA5.5
    \item[ ] R2FA5.6
    \item[ ] R2FA5.7
    \item[ ] R2FA5.8
    \item[ ] R2FA5.9
    \item[ ] R2FA5.16
    \item[ ] R2FA5.17
    \item[ ] R2FA8.5
    \item[ ] R2FA8.6
\end{itemize}\\

Incremento 7 & Funzionalità per il reset della password & \begin{itemize}
    % APP
    \item[ ] R2FA1.8
    \item[ ] R2FA1.9
    \item[ ] R2FA1.11
    \item[ ] R2FA1.12
\end{itemize} & \begin{itemize} 
    % WEB-APP ADMIN
    \item[ ] R2FS1.3
    \item[ ] R2FS1.4
    \item[ ] R2FS1.6
    \item[ ] R2FS1.7
\end{itemize} & 
    % SERVER
    Nessun requisito del server previsto per questo incremento \\

Incremento 8 & Funzionalità aggiuntive di filtraggio/ricerca nella lista delle organizzazioni (app utenti e web-app admin) & \begin{itemize}
    % APP
    \item[ ] R2FA3.11
    \item[ ] R2FA3.12
    \item[ ] R3FA3.13
    \item[ ] R3FA3.14
    \item[ ] R2FA3.16
    \item[ ] R2FA3.18
\end{itemize} & \begin{itemize} 
    % WEB-APP ADMIN
    \item[ ] R2FI4
    \item[ ] R2FI6
    \item[ ] R2FI7
    \item[ ] R2FS7.3
    \item[ ] R2FS7.4
    \item[ ] R2FS7.5
    \item[ ] R2FS7.7 
    \item[ ] R2FS7.8 
    \item[ ] R2FS7.9
\end{itemize} & \begin{itemize} 
    % SERVER
    \item[ ] R2FI4
    \item[ ] R2FI6
    \item[ ] R2FI7
\end{itemize}\\

Incremento 9 & Funzionalità generiche aggiuntive agli amministratori e gestione errori & 
    % APP
    Nessun requisito dell'app previsto per questo incremento
     & \begin{itemize} 
    % WEB-APP ADMIN
    \item[ ] R2FS4.2
    \item[ ] R2FS4.3
    \item[ ] R3FS4.7
    \item[ ] R2FS10.5
    \item[ ] R2FS10.6
    \item[ ] R2FS10.18
\end{itemize} & \begin{itemize} 
    % SERVER
    \item[ ] R2FS4.2
    \item[ ] R2FS4.3
    \item[ ] R3FS4.7
    \item[ ] R2FS10.5
    \item[ ] R2FS10.6
    \item[ ] R2FS10.18
\end{itemize} \\

\end{longtable}
}