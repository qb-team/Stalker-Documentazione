\subsection{Progettazione Architetturale}

\subsubsection{Divisione oraria}
La seguente tabella rappresenta la distribuzione oraria dei ruoli per ogni componente del gruppo:
{
\rowcolors{2}{grigetto}{white}
\renewcommand{\arraystretch}{2}
\begin{longtable}[h!] { C{4cm} C{1cm} C{1cm} C{1cm} C{1cm} C{1cm} C{1cm} C{3cm}}
\caption{Tabella della divisione oraria della Progettazione Architetturale}\\
\rowcolor{darkblue}

\textcolor{white}{\textbf{Membro del gruppo}} & 
\textcolor{white}{\textbf{RE}} & 
\textcolor{white}{\textbf{AM}} & 
\textcolor{white}{\textbf{AN}} & 
\textcolor{white}{\textbf{PT}} & 
\textcolor{white}{\textbf{PR}} & 
\textcolor{white}{\textbf{VE}} & 
\textcolor{white}{\textbf{Ore complessive}}\\	
\endhead
        
\MC{}                     &  5 &  - &  - & 14 & 10 &  4 &  33 \\
\LD{}                     &  8 &  - &  - & 18 &  - &  7 &  33 \\
\CE{}                     &  - &  8 &  - & 10 &  5 &  6 &  29 \\
\SE{}                     &  - & 10 &  6 & 10 &  - &  6 &  32 \\
\PF{}                     &  - &  6 &  - &  5 &  7 & 13 &  31 \\
\DF{}                     &  - &  - & 13 &  5 &  7 &  6 &  31 \\
\BR{}                     &  6 &  - &  - &  - & 17 &  8 &  31 \\
\AT{}                     &  - &  - &  8 &  7 &  - & 16 &  31 \\
\textbf{Ore totali ruolo} & 19 & 24 & 27 & 69 & 46 & 66 & 251 \\
		
\end{longtable}
}

La suddivisione delle ore svolte da ciascun componente del gruppo per ogni ruolo viene rappresentata nel seguente istogramma:
\begin{center}
	\pgfplotsset{width=17cm, height=8.5cm}
	\begin{tikzpicture}
		\begin{axis}[
			ybar stacked,
			bar width=20pt,
			legend style={
				at={(0.5,-0.15)},
				anchor=north,
				legend columns=-1
			},
			symbolic x coords={Christian, Davide, Emanuele, Enrico, Federico, Francesco, Riccardo, Tommaso},
			xtick=data
		]
			\legend{Responsabile, Amministratore, Analista, Progettista, Programmatore, Verificatore}
			% Responsabile
			\addplot coordinates {\ColonnaIstogramma{5}{8}{0}{0}{0}{0}{6}{0}};
			% Amministratore
			\addplot coordinates {\ColonnaIstogramma{0}{0}{8}{10}{6}{0}{0}{0}};
			% Analista
			\addplot coordinates {\ColonnaIstogramma{0}{0}{0}{6}{0}{13}{0}{8}};
			% Progettista
			\addplot coordinates {\ColonnaIstogramma{14}{18}{10}{10}{5}{5}{0}{7}};
			% Programmatore
			\addplot coordinates {\ColonnaIstogramma{10}{0}{5}{0}{7}{7}{17}{0}};
			% Verificatore
			\addplot coordinates {\ColonnaIstogramma{4}{7}{6}{6}{13}{6}{8}{16}};
		\end{axis}
	\end{tikzpicture}
\end{center}

\clearpage

\subsubsection{Costo risultante}
La seguente tabella rappresenta per ogni ruolo le ore totali investite e il corrispondente costo in euro:
{
\rowcolors{2}{grigetto}{white}
\renewcommand{\arraystretch}{2}
\begin{longtable}{ C{3cm} C{2cm} C{4cm}}
\caption{Tabella del costo risultante della Progettazione Architetturale}\\
\rowcolor{darkblue}

\textcolor{white}{\textbf{Ruolo}} & 
\textcolor{white}{\textbf{Totale ore}} & 
\textcolor{white}{\textbf{Costo ruolo (in \euro{})}}\\	
\endhead
        
Responsabile    &  19 &  570 \\
Amministratore  &  24 &  480 \\
Analista        &  27 &  675 \\
Progettista     &  69 & 1518 \\
Programmatore   &  46 &  690 \\
Verificatore    &  66 &  990 \\
\textbf{Totale} & 251 & 4923 \\	
        	
\end{longtable}
}

La quantità di ore totali per ciascun ruolo viene rappresentata nel seguente areogramma:
\begin{center}
	\begin{tikzpicture}
		\pie[rotate = 180] {
			8/Responsabile,
			10/Amministratore,
			11/Analista,
			27/Progettista,
			18/Programmatore,
			26/Verificatore
		}
	\end{tikzpicture}
\end{center}