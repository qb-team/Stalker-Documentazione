\subsection{Progettazione Architetturale}
Periodo: dal 2020-01-22 al 2020-03-15\\
Inizia al termine della \glo{fase} di Analisi e finisce con la data di consegna per la Revisione di Progettazione.\\
In questa \glo{fase} viene definita una soluzione architetturale in modo da soddisfare i requisiti individuati nella \glo{fase} di Analisi.

\subsubsection{Periodo 1} 
Dal 2020-01-22 al 2020-02-20
\begin{itemize}
	\item \textbf{Normazione}: Standardizzazione e correzione di alcune parti della documentazione che non aderiscono completamente alle \NdP{};
	\item \textbf{\AdR{}}: Avvio della correzione e modifica dei casi d'uso segnalati, concentrandosi sui casi d'uso e requisiti segnalati (e non) utili al \glo{PoC};
	\item \textbf{Assegnazione dei ruoli di progetto}: Assegnazione dei ruoli di ciascun membro del gruppo in base alla suddivisione oraria indicata in §5.2.1;
	\item \textbf{Pianificazione delle attività}: Le attività da svolgere devono essere prima pianificate e discusse dal gruppo per garantire il \glo{way of working} sancito nelle \NdP{};
	\item \textbf{Approfondimento e studio delle tecnologie}: Ricerca di documentazione e materiali utili per l'apprendimento delle nuove tecnologie da utilizzare per la realizzazione del prodotto finale.
	In particolare:
	\begin{itemize}
		\item per l'app per gli utenti: Java per Android, l'\glo{IDE} Android Studio, le \glo{API} di Google per Google Maps, Firebase, Google Play, il formato \glo{JSON}, la libreria client HTTP Volley;
		\item per la web app per gli amministratori: il tool di build automation \glo{Node.js}, i framework \glo{Angular} e \glo{Bootstrap}, \glo{TypeScript};
		\item per il server: il tool di build automation Maven e i suoi \glo{plugin}, gli strumenti di \glo{Swagger} e lo standard \glo{OpenAPI}, il framework Spring per Java, il database \glo{Redis}.
	\end{itemize}
	\item \textbf{Verifica}: \glo{Verifica} dell'andamento del team in relazione alle tempistiche e allo svolgimento dei compiti assegnati.
\end{itemize}

\subsubsection{Periodo 2} 
Dal 2020-02-21 al 2020-03-02
\begin{itemize}
	\item \textbf{Approfondimento delle tecnologie}: \glo{LDAP}, \glo{GPS} per l'app per gli utenti, \glo{Angular} e \glo{Bootstrap} per la web-app per gli amministratori, Spring, \glo{Swagger} e \glo{Redis} per il server;
	\item \textbf{Normazione}: Decisioni ed inserimento delle nuove regole da adottare per le attività di progettazione e sviluppo;
	\item \textbf{Incrementi}: Per facilitare l'organizzazione del lavoro di progettazione e di implementazione, vengono indicati qui di seguito gli incrementi che vengono portati avanti (come indicato in §3.3):
		\begin{itemize}
			\item \textbf{Incremento 1}: Vengono progettate e successivamente implementate le funzionalità di \glo{autenticazione} per l'utente e per l'amministratore;
			\item \textbf{Incremento 2}: Vengono progettate e successivamente implementate la gestione delle liste delle \glo{organizzazioni} dell'applicazione e del server;
			\item \textbf{Incremento 3}: Vengono progettate e successivamente implementate la gestione delle \glo{modalità} di tracciamento;
			\item \textbf{Incremento 4}: Viene progettato e successivamente implementato lo storico degli accessi di un utente nell'applicazione e il report tabellare degli accessi nel server;
			\item \textbf{Incremento 5}: Viene progettata e successivamente implementata l'autenticazione presso l'organizzazione nell'applicazione e la modifica dell'organizzazione nel server;
			\item \textbf{Incremento 6}: Viene progettata e successivamente implementata la gestione degli amministratori nel server.
		\end{itemize}
		L'obiettivo è progettare almeno i requisiti obbligatori degli incrementi e avere un \glo{Proof of Concept} che li sappia dimostrare correttamente, in modo da avere una solida \glo{baseline} per le successive fasi;
	\item \textbf{Progettazione}: Ricerca di una soluzione soddisfacente per tutti gli \glo{stakeholder}, che descriva l'architettura del prodotto prima di pensare al codice, seguendo gli incrementi definiti.
	La definizione dell'architettura deve essere proposta al proponente per ricevere feedback sulla sua bontà;
	\item \textbf{Technology Baseline}: Base per la realizzazione del \glo{Proof of Concept} nell'immediato e della \glo{Product Baseline} successivamente, che dimostra le tecnologie (librerie, framework, linguaggi) selezionati per lo sviluppo del prodotto;
	\item \textbf{Proof of Concept}: Creazione di uno più eseguibili che permettano di dimostrare la validità del prodotto che si vuole fornire, concretizzando la \glo{Technology Baseline}.
	Il \glo{Proof of Concept} realizzato è composto di:
	\begin{itemize}
		\item un'app per utenti sviluppata per dispositivi Android, che fornisca le funzionalità di autenticazione e logout, scaricamento della lista delle organizzazioni, verifica di presenza o meno all'interno dell'\glo{organizzazione} e visionare lo storico degli accessi;
		\item una web app per amministratori che fornisca le funzionalità di autenticazione e logout, scaricamento della lista delle organizzazioni, visualizzazione delle informazioni di un'\glo{organizzazione}, controllo del numero di utenti presenti all'interno dell'\glo{organizzazione};
		\item un server che permetta all'app e alla web-app di ottenere i dati per fornire a utenti e amministratori rispettivamente le funzionalità richieste.
	\end{itemize}
	\item \textbf{Codifica}: Viene codificato il \glo{Proof of Concept} e successivamente condiviso tramite i \glo{repository} del gruppo al committente e al proponente per il colloquio TB in data 3 marzo;
	\item \textbf{Verifica}: \glo{Verifica} dell'andamento del gruppo in relazione alle tempistiche e allo svolgimento dei compiti assegnati.
\end{itemize}

\subsubsection{Periodo 3} 
Dal 2020-03-03 al 2020-03-08
\begin{itemize}
	\item \textbf{Miglioramento standard di qualità}: Aggiunta, rimozione o modifica di alcune metriche per garantire le qualità di \glo{processo} e di prodotto affermate nel \PdQ{};
	\item \textbf{\AdR{}}: Continuazione della correzione del documento alla luce della correzione e del colloquio con il committente in data 2020-02-18;
	\item \textbf{Pianificazione attività}: Aggiustamento delle previsioni precedentemente realizzate sulla base dell'esperienza maturata, in particolar modo dopo la realizzazione del \glo{PoC};
\end{itemize}

\subsubsection{Periodo 4} 
Dal 2020-03-09 al 2020-03-15
\begin{itemize}
	\item \textbf{Consolidamento}: Ogni membro si prende del tempo per ripassare tutto il lavoro svolto e per studiare il necessario per affrontare al meglio le \glo{fasi} successive;
	\item \textbf{Avvio allo studio delle tecnologie mancanti}: Prima dell'avvio della successiva \glo{fase}, cominciare lo studio delle tecnologie non impiegate per il \glo{Proof of Concept} ma necessarie per il prodotto;
	\item \textbf{Preparazione per la Revisione di Progettazione}: Il gruppo produce il materiale necessario da esporre alla presentazione pubblica della propria proposta.
\end{itemize}

%PAGINA ORIZZONTALE
\newpage
\paperwidth=\pdfpageheight
\paperheight=\pdfpagewidth
\pdfpageheight=\paperheight
\pdfpagewidth=\paperwidth

\begingroup 
\subsubsection{Diagramma di Gantt delle attività della fase di Progettazione Architetturale}
\pagestyle{empty}
\begin{figure}[h]
	\centering
	\includegraphics[scale=0.26]{Sezioni/DiagrammiGantt/ProgettazioneArchitetturale.png}
	\caption{Diagramma di Gantt delle attività della fase di Progettazione Architetturale}	
\end{figure}

\textwidth=\hsize
\textheight=\vsize

\endgroup
\newpage
\paperwidth=\pdfpageheight
\paperheight=\pdfpagewidth
\pdfpageheight=\paperheight
\pdfpagewidth=\paperwidth
\headwidth=\textwidth

\clearpage