\subsection{Progettazione di Dettaglio e Codifica}
Dal 2020-03-16 al 2020-04-19\\
Inizia al termine della fase di Progettazione Architetturale e finisce con la data di consegna della Revisione di Qualifica.\\
In questa fase si definisce nel dettaglio e si implementa l'architettura logica costruita nella fase di Progettazione Architetturale.

\subsubsection{Periodo 1} 
Dal 2020-03-16 al 2020-03-27\\
\begin{itemize}
	\item \textbf{Approfondimento delle tecnologie}: Ricerca documentazione e materiali utili per l'apprendimento delle nuove tecnologie da utilizzare per la realizzazione del prodotto finale;
	\item \textbf{Normazione}: Standardizzazione e correzione di alcune parti della documentazione e che non aderiscono completamente alle \NdP{};
	\item \textbf{Correzioni}: Correzioni di difetti notati dal committente (ove presenti) nella \glo{Technology Baseline};
	\item \textbf{Assegnazione dei ruoli di progetto}: Assegnazione dei ruoli di ciascun membro del gruppo in base alla suddivisione oraria indicata in §5.3.1;
	\item \textbf{Pianificazione delle attività}: Le attività da svolgere devono essere prima pianificate e discusse dal gruppo per garantire il \glo{way of working} sancito nelle \NdP{};
	\item \textbf{Progettazione incrementale}: Gli incrementi definiti nella sezione §3.3 vengono rivisti ed analizzati per poterli poi svolgere nel periodo successivo.
\end{itemize}
\subsubsection{Periodo 2} 
Dal 2020-03-28 al 2020-04-08\\
\begin{itemize}
	\item \textbf{Progettazione di dettaglio}: A partire dalla progettazione architetturale, viene terminata la progettazione delle parti non ancora sviluppate del sistema, seguendo l'approccio indicato in §3.3;
	\item \textbf{Implementazione della Product Baseline}: Seguendo le specifiche della \glo{Technology Baseline}, viene realizzata una prima versione stabile del prodotto, \glo{baseline} per il lavoro futuro;
	\item \textbf{Codifica incrementale}: Per facilitare l'organizzazione del lavoro di progettazione e di implementazione, vengono indicati qui di seguito gli incrementi che vengono portati avanti (come indicato in §3.3):
	\begin{itemize}
		\item \textbf{Incremento 1}: Vengono progettate e successivamente implementate le funzionalità di \glo{autenticazione} per l'utente e per l'amministratore;
		\item \textbf{Incremento 2}: Vengono progettate e successivamente implementate la gestione delle liste delle \glo{organizzazioni} dell'applicazione e del server;
		\item \textbf{Incremento 3}: Vengono progettate e successivamente implementate la gestione delle \glo{modalità} di tracciamento;
		\item \textbf{Incremento 4}: Viene progettato e successivamente implementato lo storico degli accessi di un utente nell'applicazione e il report tabellare degli accessi nel server;
		\item \textbf{Incremento 5}: Viene progettata e successivamente implementata l'autenticazione presso l'organizzazione nell'applicazione e la modifica dell'organizzazione nel server;
		\item \textbf{Incremento 6}: Viene progettata e successivamente implementata la gestione degli amministratori nel server.
	\end{itemize}
	L'obiettivo è progettare almeno i requisiti obbligatori degli incrementi e un prodotto di \glo{qualità} che li sappia dimostrare correttamente;
	\item \textbf{Verifica}: \glo{Verifiche} (tramite i test) per assicurarsi della bontà dei requisiti implementati;
	\item \textbf{Manuali}: Stesura del Manuale Utente e del Manuale Manutentore in relazione alle funzionalità di base del sistema.
\end{itemize}
\subsubsection{Periodo 3}
Dal 2020-04-09 al 2020-04-12\\
\begin{itemize}
	\item \textbf{Primo rilascio del prodotto}: Pubblicazione del prodotto eseguibile all'interno dei \glo{repository} del gruppo;
	\item \textbf{Verifica}: \glo{Verifica} dell'andamento del team in relazione alle tempistiche e allo svolgimento dei compiti assegnati.
\end{itemize}
\subsubsection{Periodo 4} 
Dal 2020-04-13 al 2020-04-19\\
\begin{itemize}
	\item \textbf{Consolidamento}: Ogni membro si prende del tempo per ripassare tutto il lavoro svolto e per studiare il necessario per affrontare al meglio le fasi successive;
	\item \textbf{Preparazione per la Revisione di Qualifica}: Il gruppo produce il materiale necessario da esporre alla presentazione pubblica della propria proposta.
\end{itemize}

%PAGINA ORIZZONTALE
\newpage
\paperwidth=\pdfpageheight
\paperheight=\pdfpagewidth
\pdfpageheight=\paperheight
\pdfpagewidth=\paperwidth
\headwidth=\textheight

\begingroup 
\vsize=\textwidth
\hsize=\textheight

\subsubsection{Diagramma di Gantt delle attività di Progettazione di Dettaglio e Codifica}
\pagestyle{empty}
\begin{figure}[h]
	\centering
	\includegraphics[scale=0.38]{Sezioni/DiagrammiGantt/ProgettazioneDiDettaglio.png}
	\caption{Diagramma di Gantt delle attività di Progettazione di Dettaglio e Codifica}
\end{figure}

\textwidth=\hsize
\textheight=\vsize

\endgroup
\newpage
\paperwidth=\pdfpageheight
\paperheight=\pdfpagewidth
\pdfpageheight=\paperheight
\pdfpagewidth=\paperwidth
\headwidth=\textwidth