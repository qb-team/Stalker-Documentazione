\subsubsection{Fase di Progettazione di Dettaglio e Codifica}
{
\rowcolors{2}{grigetto}{white}
\renewcommand{\arraystretch}{2}
\centering
\begin{longtable}{C{2cm} C{3cm} C{10cm}}
\caption{Tabella occorrenza e mitigazione nella Fase di Progettazione di Dettaglio e Codifica}\\
\rowcolor{darkblue}

\textcolor{white}{\textbf{Codice}} & 
\textcolor{white}{\textbf{Occorrenza}} & 
\textcolor{white}{\textbf{Descrizione e risoluzione}}\\	
\endhead

RT1 &
Alta &
Nonostante gran parte del lavoro di apprendimento delle tecnologie fosse già stato fatto per la realizzazione del \glo{PoC}, al fine di ottenere un prodotto software più mantenibile è stato necessario occupare altro tempo per riuscire ad ottenere un livello di dettaglio maggiore. Questo non ha purtroppo dato vantaggi dal punto di vista del rispetto dei tempi, certamente però ha migliorato la qualità del prodotto e la comprensione delle tecnologie. Nella prossima e ultima fase è necessario che le tecnologie da usare siano consolidate per evitare che il problema si ripresenti, così facendo si riuscirebbe a evitare questo rischio. \\

RP1 &
Media &
Il metodo fin qui adottato di discussione fra le componenti del gruppo in disaccordo, discutendo sui pro e contro di ogni proposta, si è rivelato efficace. Per valutare pro e contro si sono sempre confrontate le proposte con le best practice di dominio (riguardanti le architetture e le tecnologie coinvolte). \\

RP2 &
Bassa &
In caso il proponente non risponda alle e-mail, è possibile contattarlo tramite un canale \glo{Telegram} che ci è stato messo a disposizione per comunicare in maniera più diretta. Essendo questo canale utilizzabile da tutti i gruppi che stanno svolgendo il progetto \NomeProgetto{}, è bene non abusarne. \\

RR1 &
Bassa &
I requisiti raccolti sono risultati sufficienti a portare avanti lo sviluppo del progetto. I problemi evidenziati dal committente al documento \AdR{} sono stati risolti senza particolari problemi, anche grazie alle ultime lezioni di esercitazione del corso di Ingegneria del Software. \\

RS1 &
Alta &
Conseguenza diretta di RT1 e RO1. \\

RO1 &
Alta &
Alla luce di quanto indicato in RT1, e in vista dell'imminente esame del corso di Ingegneria del Software è stato deciso, per evitare di risultare inconcludenti, di posticipare le scadenze. I periodi della fase corrente sono stati raggruppati a quelli della fase successiva. La scelta è stata corretta, soprattutto a posteriori. Alla fine del periodo 1 dell'ultima fase del progetto è bene esplorare nuovamente questa ipotesi, se dovesse essere necessario (l'auspicio è che non serva). \\

RO2 &
Media &
Con le nuove scadenze fissate, come indicato, in RO1 il rispetto delle \glo{milestone} è stato fatto con più agevolezza. \\

RO3 &
Bassa &
A questo stadio di sviluppo la comunicazione con il proponente è stata ridotta al solo necessario per alcuni dettagli implementativi, per le parti al gruppo poco chiare. Queste modalità fino ad ora adottate si sono rivelate valide. \\

RO4 &
Bassa &
Non ci sono stati problemi di comunicazione in quanto è avvenuta solamente tramite i canali di comunicazione così come indicate nelle \NdP{}: \glo{Discord} e \glo{Slack}. \\

\end{longtable}	
}
\newpage