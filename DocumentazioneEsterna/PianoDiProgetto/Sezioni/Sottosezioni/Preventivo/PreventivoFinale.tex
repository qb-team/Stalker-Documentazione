\subsection{Preventivo finale} 
Nel preventivo riportiamo la spesa totale che il committente dovrà affrontare, derivata dal totale delle ore rendicontate e preventivate nelle fasi di Progettazione Architetturale, Progettazione di Dettaglio e Codifica, Validazione e Collaudo.

\subsubsection{Divisione oraria complessiva} 
La seguente tabella rappresenta la distribuzione oraria dei ruoli per ogni componente del gruppo:
{
\rowcolors{2}{grigetto}{white}
\renewcommand{\arraystretch}{2}
\begin{longtable}[h!] { C{4cm} C{1cm} C{1cm} C{1cm} C{1cm} C{1cm} C{1cm} C{3cm}}
\caption{Tabella della divisione oraria complessiva}\\
\rowcolor{darkblue}

\textcolor{white}{\textbf{Membro del gruppo}} & 
\textcolor{white}{\textbf{RE}} & 
\textcolor{white}{\textbf{AM}} & 
\textcolor{white}{\textbf{AN}} & 
\textcolor{white}{\textbf{PT}} & 
\textcolor{white}{\textbf{PR}} & 
\textcolor{white}{\textbf{VE}} & 
\textcolor{white}{\textbf{Ore complessive}}\\	
\endhead
        
\MC{}                     &  5 & 10 &  7 &  27 &  30 &  23 & 102 \\
\LD{}                     & 14 &  4 &  - &  34 &  16 &  34 & 102 \\
\CE{}                     &  5 & 13 &  - &  24 &  28 &  32 & 102 \\
\SE{}                     &  - & 14 &  6 &  28 &  24 &  30 & 102 \\
\PF{}                     &  6 & 14 &  - &  14 &  34 &  34 & 102 \\
\DF{}                     & 10 &  6 & 13 &  13 &  34 &  26 & 102 \\
\BR{}                     &  6 &  8 &  - &  13 &  43 &  32 & 102 \\
\AT{}                     & 10 &  - & 12 &  28 &  18 &  34 & 102 \\
\textbf{Ore totali ruolo} & 56 & 69 & 38 & 181 & 227 & 245 & 816 \\

\end{longtable}
}

\subsubsection{Costo complessivo per ruolo}
Nella seguente tabella viene illustrato il monte ore risultante per ogni ruolo con il costo ad esso associato:
{
\rowcolors{2}{grigetto}{white}
\renewcommand{\arraystretch}{2}
\begin{longtable}{ C{3cm} C{2cm} C{4cm}}
\caption{Tabella del costo complessivo per ruolo}\\
\rowcolor{darkblue}

\textcolor{white}{\textbf{Ruolo}} & 
\textcolor{white}{\textbf{Totale ore}} & 
\textcolor{white}{\textbf{Costo ruolo (in \euro{})}}\\	
\endhead
        
Responsabile   &  56 & 1680 \\
Amministratore &  69 & 1380 \\
Analista       &  38 &  950 \\
Progettista    & 181 & 3982 \\
Programmatore  & 227 & 3405 \\
Verificatore   & 245 & 3675 \\
        	
\end{longtable}
}

% La quantità di ore totali per ciascun ruolo (rendicontate e non) viene rappresentata nel seguente areogramma:
% \begin{center}
% 	\begin{tikzpicture}
% 		\pie[rotate = 180, color={blue, yellow, red, green, grigetto, orange}] {
% 			7/Responsabile, % 75/1056 circa 7%
% 			10/Amministratore, % 107/1056 circa 10%
% 			14/Analista, % 152/1056 circa 14%
% 			17/Progettista, % 181/1056 circa 17%
% 			22/Programmatore, % 227/1056 circa 22%
% 			30/Verificatore % 314/1056 circa 30%
% 		}
% 	\end{tikzpicture}
% \end{center}

Nel seguente areogramma viene rappresentata la distribuzione dei costi in percentuale sulla spesa totale da affrontare:
\begin{center}
	\begin{tikzpicture}
		\pie[rotate = 180, color={blue, yellow, red, green, grigetto, orange}] {
			7/Responsabile, % 56/816 circa 7%
			8/Amministratore, % 69/816 circa 8%
			5/Analista, % 38/816 circa 5%
			22/Progettista, % 181/816 circa 22%
			28/Programmatore, % 227/816 circa 28%
			30/Verificatore % 245/816 circa 30%
		}
	\end{tikzpicture}
\end{center}

\subsubsection{Costo complessivo}
Nella seguente tabella vengono riportati i costi complessivi delle varie fasi e infine l'importo proposto da \Gruppo{} per la realizzazione del progetto \NomeProgetto{}:\\
{
\rowcolors{2}{grigetto}{white}
\renewcommand{\arraystretch}{2}
\begin{longtable}{ C{5cm} C{5cm}}
\caption{Tabella del costo complessivo}\\
\rowcolor{darkblue}

\textcolor{white}{\textbf{Fase}} &
\textcolor{white}{\textbf{Costo fase (in \euro{})}}\\	
\endhead
		
Progettazione Architetturale          &  4923 \\
Progettazione di Dettaglio e Codifica &  7177 \\
Validazione e Collaudo                &  2972 \\
\textbf{Totale}                       & 15072 \\

\end{longtable}
}