\subsection{Progettazione di Dettaglio e Codifica}
\subsubsection{Bilancio}

{
\rowcolors{2}{grigetto}{white}
\renewcommand{\arraystretch}{2}
\begin{longtable}[h]{ C{2.5cm} C{2.5cm} C{2.5cm} C{2.5cm} C{1.5cm} C{2.5cm}}
\caption{Tabella del costo complessivo per ruolo}\\
\rowcolor{darkblue}

\textcolor{white}{\textbf{Ruolo}} & 
\textcolor{white}{\textbf{Ore preventivate}} & 
\textcolor{white}{\textbf{Variazione oraria}} & 
\textcolor{white}{\textbf{Costo preventivato (in \euro{})}} & 
\textcolor{white}{\textbf{Costo effettivo (in \euro{})}} & 
\textcolor{white}{\textbf{Variazione di costo (in \euro{})}}\\	
	
Responsabile    &  26 &  -3 &  780 &  690 &  -90 \\
Amministratore  &  30 &  -1 &  600 &  580 &  -20 \\
Analista        &   - &  +6 &    - &  150 & +150 \\
Progettista     &  91 &  +5 & 2002 & 2112 & +110 \\
Programmatore   & 144 &  +3 & 2160 & 2205 &  +45 \\
Verificatore    & 109 &   0 & 1635 & 1635 &    0 \\
\textbf{Totale} & 400 & +10 & 7177 & 7372 & +195 \\	

\end{longtable}
}

\subsubsection{Bilancio degli incrementi}
La seguente tabella rappresenta la distribuzione delle ore effettivamente investite durante il periodo in cui vengono svolti gli incrementi e il corrispondente costo in euro.
Quanto segnato fra parentesi e dopo il segno di addizione, in ogni cella, corrisponde alla variazione rispetto alla pianificazione.

{
\rowcolors{2}{grigetto}{white}
\renewcommand{\arraystretch}{1.65}
\centering
\begin{longtable}{ C{2.1cm} C{2.7cm} C{3cm} C{3cm} C{3.3cm} }
\caption{Tabella del costo risultante di ogni incremento}\\
\rowcolor{darkblue}
\textcolor{white}{\textbf{Incremento}} & 
\textcolor{white}{\textbf{Ore progettista}} &
\textcolor{white}{\textbf{Ore programmatore}}&
\textcolor{white}{\textbf{Ore verificatore}}&
\textcolor{white}{\textbf{Costo totale incremento (in \euro{})}}\\
\endhead

1 & 16      & 25      &  9 &  862       \\
2 & 16      & 25      &  9 &  862       \\
3 & 19      & 30      & 12 & 1048       \\
4 & 24 (+2) & 39 (+3) & 15 & 1338 (+89) \\
5 & 16 (+3) & 25      &  9 &  862 (+66) \\

\end{longtable}
}

\subsubsection{Conclusioni}
Come riportato dalla tabella, il bilancio risulta essere abbastanza diverso da quello preventivato per tutti i ruoli eccetto il verificatore, mentre ci sono stati alcuni cambiamenti per gli altri. \\
In particolar modo:
\begin{itemize}
    \item \textbf{Analista}: Alcuni membri del gruppo hanno dovuto sistemare alcune parti del documento di \AdR{} segnalate alla precedente revisione e notate da essi nel documento;
	\item \textbf{Amministratore e Responsabile}: Come per la precedente fase, nonostante il lieve discostamento, possiamo notare che sono stati preventivati correttamente i numeri di ore richieste per il ruolo;
    \item \textbf{Progettista}: Visto che nella precedente fase sono state impegnate molte meno ore di progettista, in questa fase c'è stato l'impegno a soddisfare a pieno il monte ore preventivato e sono state quindi impiegate 5 ore extra per evitare di dover, nella prossima fase (e quindi nei prossimi incrementi), ritoccare aspetti importanti dell'architettura;
    \item \textbf{Programmatore}: La parte di programmazione ha richiesto anche in questo caso alcune ore in più, in particolare per sistemare alcuni difetti dovuti all'utilizzo improprio di alcune tecnologie;
	\item \textbf{Verificatore}: Il lavoro necessario per il ruolo di verificatore è stato preventivato correttamente. 
\end{itemize}
\subsubsection{Ragionamento sugli scostamenti}
Una cosa che può essere notata fin da subito è che il ruolo di analista poteva essere messo a preventivo fin da subito, considerando la possibilità di dover aggiustare alcune parti del documento di \AdR{}. L'altro scostamento più rilevante, quello del ruolo di progettista, più che preventivabile, è dovuto principalmente allo scarto importante di ore del ruolo avute nella precedente fase.\\
Le ore di programmatore in più, dovute alle tecnologie impiegate, sarebbero state difficilmente preventivabili, ma una scelta di queste più accurata, magari puntando su altre tecnologie più diffuse, semplici da usare o meglio documentate sarebbe stato meglio. Per fare un esempio, uno strumento come Redis potrebbe essere stato sostituito da una tecnologia simile ma più semplice da usare (o meglio, configurare).

\subsubsection{Preventivo a finire}
Il preventivo a finire in fase di Progettazione Architetturale era inferiore di \euro{} 196. In questa fase, l'attuazione degli incrementi e ruoli come l'analista hanno portato ad un aumento del costo della fase di \euro{} 195. Per questa ragione, il consuntivo attuale è aumentato, tornando molto vicino alla soglia preventivata (\euro{} 15072).\\
Il prezzo (attuale) totale per la realizzazione del prodotto è \euro{} 15071.