\subsection{Progettazione di Dettaglio e Codifica}
\subsubsection{Bilancio}

{
\rowcolors{2}{grigetto}{white}
\renewcommand{\arraystretch}{2}
\begin{longtable}[h]{ C{2.5cm} C{2.5cm} C{2.5cm} C{2.5cm} C{1.5cm} C{2.5cm}}
\caption{Tabella del costo complessivo per ruolo}\\
\rowcolor{darkblue}

\textcolor{white}{\textbf{Ruolo}} & 
\textcolor{white}{\textbf{Ore preventivate}} & 
\textcolor{white}{\textbf{Variazione oraria}} & 
\textcolor{white}{\textbf{Costo preventivato (in \euro{})}} & 
\textcolor{white}{\textbf{Costo effettivo (in \euro{})}} & 
\textcolor{white}{\textbf{Variazione di costo (in \euro{})}}\\	
	
Responsabile    &  26 &  0 & 780 &   &   \\
Amministratore  &  30 &  0 & 600 &   &    \\
Analista        &  -  &  - &  -  & - &  - \\
Progettista     &  91 &  0 & 2002 &   &  \\
Programmatore   & 144 &  0 & 2160 &  &  \\
Verificatore    & 106 &  0 & 1635 &  &   \\
\textbf{Totale} & 400 &  0 & 7177 &  &  \\	

\end{longtable}
}

\subsubsection{Conclusioni}
Come riportato dalla tabella, il bilancio risulta essere particolarmente diverso da quello preventivato per i due ruoli di progettista e programmatore, mentre ci sono stati alcuni cambiamenti per gli altri.\\
In particolar modo:
\begin{itemize}
	\item \textbf{Amministratore}: Questo ruolo non si è discostato, dovute al fatto che le attività di suo interesse erano già state portate avanti correttamente a sufficienza nella precedente \glo{fase} e non sono state necessarie ore extra;
    \item \textbf{Progettista}: A differenza della pianificazione svolta, la parte progettuale si è attuata in tempi maggiori. Questo è dovuto dal fatto che la progettazione e la 
    realizzazione dei diagrammi delle classi, package e di sequenza hanno richiesto più tempo del dovuto aumentandone i controlli per assicurarne la loro corretta validità.
	Per questo motivo ci si è concentrati maggiormente sui dettagli di grana più fine;
    \item \textbf{Programmatore}: La parte di programmazione si è attutata nei tempi previsti. Ciò è dovuto da un aumento dell'esperienza nella scrittura del codice da parte dei 
    membri del gruppo \Gruppo{} e anche dalle numerose ore di studio delle tecnologie all'inizio di questa fase, sul come integrare in maniera corretta le unità software tenendo sempre conto di tutti i diagrammi progettati precedentemente e mantenendo
    un aggiornamento continuo su come realizzare le comunicazioni tra le varie componenti del prodotto software: app, \glo{backend} e web-app;
	\item \textbf{Verificatore}: Il lavoro necessario per il ruolo di verificatore è stato preventivato correttamente. 
\end{itemize}
\subsubsection{Ragionamento sugli scostamenti}
Dalle conclusioni si può notare ...


\subsubsection{Preventivo a finire}
Il preventivo a finire è attualmente ...