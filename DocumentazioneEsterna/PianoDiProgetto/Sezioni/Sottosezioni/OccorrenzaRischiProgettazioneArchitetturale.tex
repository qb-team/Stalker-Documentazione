\subsubsection{Fase di Progettazione Architetturale}
{
\rowcolors{2}{grigetto}{white}
\renewcommand{\arraystretch}{2}
\centering
\begin{longtable}{C{2cm} C{3cm} C{10cm}}
\caption{Tabella occorrenza e mitigazione}\\
\rowcolor{darkblue}

\textcolor{white}{\textbf{Codice}} & 
\textcolor{white}{\textbf{Occorrenza}} & 
\textcolor{white}{\textbf{Descrizione e risoluzione}}\\	
\endhead

RT1 &
Alta &
Tutti i membri del gruppo hanno dovuto apprendere dei linguaggi e strumenti in base al compito che gli è stato assegnato. I linguaggi/strumenti presi in causa sono i seguenti: \glo{CSS}, \glo{HTML5}, \glo{JSON}, \glo{TypeScript}, \glo{Node.js}, \glo{MySQL}, \glo{Redis}, \glo{YAML}, \glo{Swagger}, \glo{OpenAPI}, \glo{Spring}, \glo{Maven}, \glo{Java} e Android Studio. Il tempo per apprendere queste tecnologie in maniera sufficiente è stato abbastanza dispendioso. In alcuni casi lo studio ha occupato più tempo del previsto ma in conclusione siamo riuscito a realizzare il \glo{PoC} funzionante. \\

RP1 &
Bassa &
Le decisioni prese sono state generalmente accolte positivamente da tutti i membri del gruppo. C'è stata qualche difficoltà sulla scelta per la realizzazione dei collegamenti tra le varie parti del \glo{PoC} (app, web-app e server) e farli funzionare correttamente; anche per l'utilizzo di certe tecnologie, varie compatibilità e versioni. In generale dopo una discussione tra i vari membri del gruppo si riusciva a venirsi incontro con le decisioni. \\

RP2 &
Bassa &
Non ci sono stati particolari problemi di comunicazione con il proponente. È stato richiesto un incontro in videoconferenza per discutere di alcuni dubbi che si sono presentati durante la scelta di certe tecnologie, linguaggi, versioni e librerie che ha aiutato alla realizzazione del \glo{PoC}. \\

RR1 &
Bassa &
Ci sono stati dubbi su alcuni requisiti e casi d'uso presentati nell'\AdRv{1.0.0}. È stata svolta una videochiamata sulla piattaforma \glo{Hangouts} con \CR{} per risolvere dubbi ed incomprensioni. \\

RS1 &
Bassa &
Per quanto riguarda lo studio delle tecnologie, la realizzazione del \glo{PoC} e la correzione dei documenti quasi tutti i tempi richiesti sono stati rispettati (in particolar modo rispetto alla precedente fase in cui l'occorrenza era stata alta). \\

RO1 &
Bassa &
Pur di rispettare le \glo{milestone} ogni membro del gruppo ha cercato di supportarsi a vicenda nel tentativo di rispettare le scadenze. In particolar modo, nonostante la suddivisione del lavoro in tre parti (e quindi gruppi di membri) ci si è comunque aiutati pur di completare il lavoro. \\

RO2 &
Bassa &
Avendo acquisito maggior esperienza nell'assegnazione di alcune scadenze sono state quasi tutte rispettate e non è servita uno stravolgimento della pianificazione. Alla luce degli errori commessi nel RR, la pianificazione ha ottenuto maggiore importanza. \\

RO3 &
Media &
È stato organizzato un solo incontro con il proponente che è stato molto proficuo. Alcuni dubbi si sono risolti attraverso l'utilizzo di \glo{Slack} per una maggiore interazione. \\

RO4 &
Media &
Essendo la fase cominciata nel periodo esami, ogni membro del gruppo aveva altri impegni importanti da rispettare.
In aggiunta ad essi, alla fine della sessione si è verificata un'altra situazione che ha complicato l'avanzamento dei lavori, ovvero la diffusione del Covid-19 che ha portato disagi come la chiusura delle aule universitarie, limitazioni di trasporti e di comunicazioni.
Il lavoro è stato necessariamente compiuto sfruttando al massimo le potenzialità che gli strumenti \glo{Discord}, \glo{Slack} e \glo{Hangouts} offrono.
L'organizzazione avveniva su \glo{Slack} mentre su \glo{Discord} avvenivano le chiamate vocali e le condivisioni degli schermi dei computer in live streaming.
I colloqui con committente e proponente sono dovuti avvenire attraverso \glo{Hangouts}, con la importante limitazione del numero di utenti in una chiamata (fisso a 6). \\

\end{longtable}	
}