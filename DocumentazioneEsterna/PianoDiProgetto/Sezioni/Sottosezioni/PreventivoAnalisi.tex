\subsection{Analisi}

\subsubsection{Divisione oraria}
La seguente tabella rappresenta la distribuzione oraria dei ruoli per ogni componente del gruppo:
{
\rowcolors{2}{grigetto}{white}
\renewcommand{\arraystretch}{2}
\begin{longtable}[h!] { C{4cm} C{1cm} C{1cm} C{1cm} C{1cm} C{1cm} C{1cm} C{3cm}}
\caption{Tabella della divisione oraria di Analisi}	\\
\rowcolor{darkblue}

\textcolor{white}{\textbf{Membro del gruppo}} & 
\textcolor{white}{\textbf{RE}} & 
\textcolor{white}{\textbf{AM}} & 
\textcolor{white}{\textbf{AN}} & 
\textcolor{white}{\textbf{PT}} & 
\textcolor{white}{\textbf{PR}} & 
\textcolor{white}{\textbf{VE}} & 
\textcolor{white}{\textbf{Ore complessive}}\\	
\endhead

\MC{}                     &  - &  7 &  12 & - & - & 11 &  30 \\
\LD{}                     &  - &  5 &  16 & - & - &  9 &  30 \\
\CE{}                     &  - &  - &  21 & - & - &  9 &  30 \\
\SE{}                     & 15 &  2 &   8 & - & - &  5 &  30 \\
\PF{}                     &  - &  - &  21 & - & - &  9 &  30 \\
\DF{}                     &  - &  7 &  16 & - & - &  7 &  30 \\
\BR{}                     &  - &  5 &  11 & - & - & 14 &  30 \\
\AT{}                     &  4 & 12 &   9 & - & - &  5 &  30 \\
\textbf{Ore totali ruolo} & 19 & 38 & 114 & - & - & 69 & 240 \\

\end{longtable}
}

La suddivisione delle ore svolte da ciascun componente del gruppo per ogni ruolo viene rappresentata nel seguente istogramma:
\begin{center}
	\pgfplotsset{width=17cm, height=8.5cm}
	\begin{tikzpicture}
		\begin{axis}[
			ybar stacked,
			bar width=20pt,
			legend style={
				at={(0.5,-0.15)},
				anchor=north,
				legend columns=-1
			},
			symbolic x coords={Christian, Davide, Emanuele, Enrico, Federico, Francesco, Riccardo, Tommaso},
			xtick=data
		]
			\legend{Responsabile, Amministratore, Analista, Progettista, Programmatore, Verificatore}
			% Responsabile
			\addplot coordinates {\ColonnaIstogramma{0}{0}{0}{15}{0}{0}{0}{4}};
			% Amministratore
			\addplot coordinates {\ColonnaIstogramma{7}{5}{0}{2}{0}{7}{5}{12}};
			% Analista
			\addplot coordinates {\ColonnaIstogramma{12}{16}{21}{8}{21}{16}{11}{9}};
			% Progettista
			\addplot coordinates {\ColonnaIstogramma{0}{0}{0}{0}{0}{0}{0}{0}};
			% Programmatore
			\addplot coordinates {\ColonnaIstogramma{0}{0}{0}{0}{0}{0}{0}{0}};
			% Verificatore
			\addplot coordinates {\ColonnaIstogramma{11}{9}{9}{5}{9}{7}{14}{5}};
		\end{axis}
	\end{tikzpicture}
\end{center}
\clearpage

\subsubsection{Costo risultante}
La seguente tabella rappresenta per ogni ruolo le ore totali investite e il corrispondente costo in euro:
{
\rowcolors{2}{grigetto}{white}
\renewcommand{\arraystretch}{2}
\begin{longtable}{ C{3cm} C{2cm} C{4cm}}
\caption{Tabella del costo risultante di Analisi}\\
\rowcolor{darkblue}

\textcolor{white}{\textbf{Ruolo}} & 
\textcolor{white}{\textbf{Totale ore}} & 
\textcolor{white}{\textbf{Costo ruolo (in \euro{})}}\\	
\endhead

Responsabile    &  19 &  570 \\
Amministratore  &  38 &  760 \\
Analista        & 114 & 2850 \\
Progettista     &   - &    - \\
Programmatore   &   - &    - \\
Verificatore    &  69 & 1035 \\
\textbf{Totale} & 240 & 5215 \\
		
\end{longtable}
}

La quantità di ore totali per ciascun ruolo viene rappresentata nel seguente areogramma:
\begin{center}
	\begin{tikzpicture}
		\pie[rotate = 180] {
			8/Responsabile,
			16/Amministratore,
			47/Analista,
			29/Verificatore
		}
	\end{tikzpicture}
\end{center}