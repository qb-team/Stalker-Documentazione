\section{Pianificazione}
Nella pianificazione, il \Responsabile{} suddivide il lavoro in attività e le assegna a ciascun membro del gruppo.
Lo scopo è dimostrare come deve essere svolto il lavoro, valutare i progressi nel progetto e anticipare i problemi che potrebbero sorgere preparando delle soluzioni per essi.\\
La pianificazione di progetto viene organizzata seguendo le scadenze presentate nella sezione §1.6.
Lo sviluppo del progetto viene suddiviso nelle seguenti quattro \glo{fasi}: 
\begin{itemize}
	\item Analisi;
	\item Progettazione Architetturale;
	\item Progettazione di Dettaglio e Codifica;
	\item Validazione e Collaudo.
\end{itemize}
Alla fine di ciascuna di queste fasi corrisponde una \glo{milestone} e il gruppo può disporre di una o più \glo{baseline}, base su cui continuare il lavoro nelle successive fasi.\\
Essendo queste fasi di durata di uno o due mesi circa, il gruppo ha ritenuto necessario fare un'ulteriore suddivisione. Ad ogni \glo{fase} corrispondono quindi periodi più brevi, all'interno dei quali vengono elencate le diverse attività che il gruppo \Gruppo{} deve svolgere e gli incrementi previsti.
Alla fine di ciascun periodo corrisponde una \glo{milestone} interna. A differenza delle \glo{milestone} relative alle \glo{fasi}, le cui date vengono stabilite dal calendario del committente, le \glo{milestone} interne sono scelte dal \Responsabile{} del gruppo \Gruppo{}.

\subsection{Analisi}
Periodo: dal 2019-11-15 al 2020-01-20\\
Inizia con la formazione del gruppo e finisce con la data di consegna per la Revisione dei Requisiti.\\
In questa fase viene definito il gruppo, la normazione (\glo{way of working}) e la garanzia di qualità che si vuole fornire, oltre alla definizione dei requisiti del capitolato che viene scelto.

\subsubsection{Periodo 1}
Dal 2019-11-15 al 2019-11-29\\
In questo periodo, che parte dalla formazione del gruppo e termina con la scelta del capitolato C5 \NomeProgetto{}, il gruppo intende affrontare le seguenti tematiche al fine di porre le basi per il lavoro che va affrontato:
\begin{itemize}
	\item \textbf{Discussione capitolati}: Ogni membro del gruppo studia individualmente e in seguito discute con gli altri membri durante gli incontri tutti i capitolati proposti, ponendo le basi per la stesura del documento \SdF{} e indirizzando la scelta del capitolato;
	\item \textbf{Assegnazione e studio dei ruoli di progetto}: Ad ogni membro del gruppo viene assegnato il ruolo principale da ricoprire nella fase di Analisi;
	\item \textbf{Definizione degli strumenti}: Vengono discusse e definite le tecnologie da usare per affrontare la fase di Analisi;
	\item \textbf{Pianificazione milestone fase di Analisi}: Vengono discusse e fissate delle \glo{milestone} intermedie da rispettare per completare la fase di Analisi entro le scadenze imposteci.
\end{itemize}

\subsubsection{Periodo 2} 
Dal 2019-11-30 al 2019-12-31\\
Questo periodo inizia con la scelta definitiva del capitolato C5 \NomeProgetto{}.\\
Dopo la scelta, vanno focalizzate le risorse del gruppo nei seguenti punti:
\begin{itemize}
	\item \textbf{Normazione}: Vengono definite le regole per la stesura dei documenti e per l'utilizzo delle tecnologie identificate in precedenza;
	\item \textbf{Approfondimento capitolati}: Vengono ulteriormente discussi tutti i capitolati in modo da terminare lo \SdF{} e focalizzare l'attenzione sull'analisi del capitolato scelto in modo da predisporre le basi per l'\AdR{};
	\item \textbf{Prima definizione dei casi d'uso}: Attività dove vengono identificati ed analizzati i requisiti del capitolato, cercando di comprendere come il sistema debba essere realizzato;
	\item \textbf{Determinazione standard di qualità}: Vengono definite le strategie per garantire la qualità di \glo{processo} e la qualità di prodotto;
	\item \textbf{Verifica}: \glo{Verifica} dell'andamento del gruppo in relazione alle tempistiche e allo svolgimento dei compiti assegnati.
\end{itemize}

\subsubsection{Periodo 3}
Dal 2020-01-01 al 2020-01-14\\
Questo periodo si estende fino alla data ultima di consegna per affrontare la Revisione dei Requisiti a cui il gruppo ha deciso di partecipare.
\begin{itemize}
	\item \textbf{Normazione}: Ulteriori approfondimenti alle regole per la stesura dei documenti e per l'utilizzo delle tecnologie;
	\item \textbf{Approfondimento delle tecnologie}: Vengono ampliate le conoscenze sulle tecnologie richieste dal capitolato per essere svolto;
	\item \textbf{\AdR{}}: Studio dei requisiti e raffinamento dei casi d'uso;
	\item \textbf{Pianificazione delle attività}: Pianificazione del lavoro da svolgere nelle fasi successive a quella di Analisi;
	\item \textbf{Verifica}: \glo{Verifica} dell'andamento del gruppo in relazione alle tempistiche e allo svolgimento dei compiti assegnati.
\end{itemize}

\subsubsection{Periodo 4} 
Dal 2020-01-15 al 2020-01-20\\
In questo periodo, che ha inizio con la consegna della documentazione prodotta per la Revisione dei Requisiti alla presentazione pubblica della proposta, il gruppo consolida il lavoro svolto in vista delle successive fasi e della discussione per la quale serve una presentazione:
\begin{itemize}
	\item \textbf{Consolidamento}: Ogni membro del gruppo si prende del tempo per ripassare tutto il lavoro svolto e per studiare il necessario per affrontare al meglio le fasi successive;
	\item \textbf{Preparazione per la Revisione dei Requisiti}: Il gruppo produce il materiale necessario da esporre alla presentazione pubblica della propria proposta.
\end{itemize}

\newpage
% Inizia la pagina orientata orizzontalmente
\begin{landscape}
% Ora la pagina e' in orizzontale!
\subsubsection{Diagramma di Gantt delle attività della fase di Analisi}
\pagestyle{empty}
\begin{figure}[h]
	\centering	
	\includegraphics[scale=0.32]{Sezioni/DiagrammiGantt/Analisi.png}
	\caption{Diagramma di Gantt delle attività della fase di Analisi}
\end{figure}
\end{landscape}
\clearpage

\subsection{Progettazione Architetturale}
Periodo: dal 2020-01-22 al 2020-03-15\\
Inizia al termine della fase di Analisi e finisce con la data di consegna per la Revisione di Progettazione.\\
In questa fase viene definita una soluzione architetturale in modo da soddisfare i requisiti individuati nella fase di Analisi.

\subsubsection{Periodo 1} 
Dal 2020-01-22 al 2020-02-11
\begin{itemize}
	\item \textbf{Normazione}: Standardizzazione e correzione di alcune parti della documentazione che non aderiscono completamente alle \NdP{};
	\item \textbf{\AdR{}}: Correzione e modifica dei casi d'uso segnalati;
	\item \textbf{Assegnazione dei ruoli di progetto}: Assegnazione dei ruoli di ciascun membro del gruppo in base alla suddivisione oraria indicata in §5.2.1;
	\item \textbf{Pianificazione delle attività}: Le attività da svolgere devono essere prima pianificate e discusse dal gruppo per garantire il \glo{way of working} sancito nelle \NdP{};
	\item \textbf{Approfondimento delle tecnologie}: Ricerca di documentazione e materiali utili per l'apprendimento delle nuove tecnologie da utilizzare per la realizzazione del prodotto finale;
	\item \textbf{Verifica}: \glo{Verifica} dell'andamento del team in relazione alle tempistiche e allo svolgimento dei compiti assegnati.
\end{itemize}

\subsubsection{Periodo 2} 
Dal 2020-02-12 al 2020-03-08
\begin{itemize}
	\item \textbf{Studio delle tecnologie}: l'\glo{IaaS} \glo{Kubernetes} o i \glo{PaaS} \glo{Openshift} o \glo{Rancher}, \glo{LDAP} e \glo{GPS};
	\item \textbf{Normazione}: Decisioni ed inserimento delle nuove regole da adottare per le attività di progettazione e sviluppo;
	\item \textbf{Miglioramento standard di qualità}: Aggiunta, rimozione o modifica di alcune metriche per garantire le qualità di \glo{processo} e di prodotto affermate nel \PdQ{};
	\item \textbf{Incrementi}: Per facilitare l'organizzazione del lavoro di progettazione e di implementazione, vengono indicati qui di seguito gli incrementi che vengono portati avanti (come indicato in §3.3):
		\begin{itemize}
			\item \textbf{Incremento 1}: Vengono progettate e successivamente implementate le funzionalità di \glo{autenticazione} per l'utente e per l'amministratore;
			\item \textbf{Incremento 2}: Vengono progettate e successivamente implementate la gestione delle liste delle \glo{organizzazioni} dell'applicazione e del server;
			\item \textbf{Incremento 3}: Vengono progettate e successivamente implementate la gestione delle \glo{modalità} di tracciamento;
			\item \textbf{Incremento 4}: Viene progettato e successivamente implementato lo storico degli accessi di un utente nell'applicazione e il report tabellare degli accessi nel server;
			\item \textbf{Incremento 5}: Viene progettata e successivamente implementata l'autenticazione presso l'organizzazione nell'applicazione e la modifica dell'organizzazione nel server;
			\item \textbf{Incremento 6}: Viene progettata e successivamente implementata la gestione degli amministratori nel server.
		\end{itemize}
		L'obiettivo è progettare almeno i requisiti obbligatori degli incrementi e avere un \glo{Proof of Concept} che li sappia dimostrare correttamente, in modo da avere una solida \glo{baseline} per le successive fasi;
	\item \textbf{Progettazione}: Ricerca di una soluzione soddisfacente per tutti gli \glo{stakeholder}, che descriva l'architettura del prodotto prima di pensare al codice, seguendo gli incrementi definiti;
	\item \textbf{Technology Baseline}: Redazione della \glo{Technology Baseline}, cioè un allegato tecnico nel quale vengono indicate le tecnologie e i design pattern che vengono utilizzati durante lo sviluppo del prodotto;
	\item \textbf{Proof of Concept}: Creazione di un eseguibile che permetta di dimostrare la validità del prodotto che si vuole fornire, concretizzando la \glo{Technology Baseline};
	\item \textbf{Codifica}: Viene codificato il \glo{Proof of Concept} e successivamente condiviso tramite i \glo{repository} del gruppo al committente e al proponente in una data da definire;
	\item \textbf{Verifica}: \glo{Verifica} dell'andamento del gruppo in relazione alle tempistiche e allo svolgimento dei compiti assegnati.
\end{itemize}

\subsubsection{Periodo 3} 
Dal 2020-03-09 al 2020-03-15
\begin{itemize}
	\item \textbf{Consolidamento}: Ogni membro si prende del tempo per ripassare tutto il lavoro svolto e per studiare il necessario per affrontare al meglio le fasi successive;
	\item \textbf{Preparazione per la Revisione di Progettazione}: Il gruppo produce il materiale necessario da esporre alla presentazione pubblica della propria proposta.
\end{itemize}


\newpage
% Inizia la pagina orientata orizzontalmente
\begin{landscape}
% Ora la pagina e' in orizzontale!
\subsubsection{Diagramma di Gantt delle attività della fase di Progettazione Architetturale}
\pagestyle{empty}
\begin{figure}[h]
	\centering
	\includegraphics[scale=0.34]{Sezioni/DiagrammiGantt/ProgettazioneArchitetturale.png}
	\caption{Diagramma di Gantt delle attività della fase di Progettazione Architetturale}	
\end{figure}
\end{landscape}

\subsection{Progettazione di Dettaglio e Codifica}
Dal 2020-03-16 al 2020-04-19\\
Inizia al termine della fase di Progettazione Architetturale e finisce con la data di consegna della Revisione di Qualifica.\\
In questa fase si definisce nel dettaglio e si implementa l'architettura logica costruita nella fase di Progettazione Architetturale.

\subsubsection{Periodo 1} 
Dal 2020-03-16 al 2020-03-27\\
\begin{itemize}
	\item \textbf{Approfondimento delle tecnologie}: Ricerca documentazione e materiali utili per l'apprendimento delle nuove tecnologie da utilizzare per la realizzazione del prodotto finale;
	\item \textbf{Normazione}: Standardizzazione e correzione di alcune parti della documentazione e che non aderiscono completamente alle \NdP{};
	\item \textbf{Correzioni}: Correzioni di difetti notati dal committente (ove presenti) nella \glo{Technology Baseline};
	\item \textbf{Assegnazione dei ruoli di progetto}: Assegnazione dei ruoli di ciascun membro del gruppo in base alla suddivisione oraria indicata in §5.3.1;
	\item \textbf{Pianificazione delle attività}: Le attività da svolgere devono essere prima pianificate e discusse dal gruppo per garantire il \glo{way of working} sancito nelle \NdP{};
	\item \textbf{Codifica}: Implementazione dei requisiti di base identificati per ottenere un sistema stabile;
	\item \textbf{Manuali}: Stesura del Manuale Utente e del Manuale Manutentore in relazione alle funzionalità di base del sistema.
\end{itemize}
\subsubsection{Periodo 2} 
Dal 2020-03-28 al 2020-04-08\\
\begin{itemize}
	\item \textbf{Progettazione di dettaglio}: A partire dalla progettazione architetturale, viene terminata la progettazione delle parti non ancora sviluppate del sistema, seguendo l'approccio indicato in §3.3;
	\item \textbf{Implementazione della Product Baseline}: Seguendo le specifiche della \glo{Technology Baseline}, viene realizzata una prima versione stabile del prodotto, \glo{baseline} per il lavoro futuro;
	\item \textbf{Codifica incrementale}: Implementazione di requisiti nel sistema seguendo gli incrementi definiti in §3.3;
	\item \textbf{Verifica}: \glo{Verifiche} (tramite i test) per assicurarsi della bontà dei requisiti implementati;
	\item \textbf{Manuali}: Aggiunta nel Manuale Utente e del Manuale Manutentore delle funzionalità inserite incrementalmente nel sistema.
\end{itemize}
\subsubsection{Periodo 3}
Dal 2020-04-09 al 2020-04-12\\
\begin{itemize}
	\item \textbf{Primo rilascio del prodotto}: Pubblicazione del prodotto eseguibile all'interno dei \glo{repository} del gruppo;
	\item \textbf{Verifica}: \glo{Verifica} dell'andamento del team in relazione alle tempistiche e allo svolgimento dei compiti assegnati.
\end{itemize}
\subsubsection{Periodo 4} 
Dal 2020-04-13 al 2020-04-19\\
\begin{itemize}
	\item \textbf{Consolidamento}: Ogni membro si prende del tempo per ripassare tutto il lavoro svolto e per studiare il necessario per affrontare al meglio le fasi successive;
	\item \textbf{Preparazione per la Revisione di Qualifica}: Il gruppo produce il materiale necessario da esporre alla presentazione pubblica della propria proposta.
\end{itemize}

\newpage
% Inizia la pagina orientata orizzontalmente
\begin{landscape}
% Ora la pagina e' in orizzontale!
\subsubsection{Diagramma di Gantt delle attività di Progettazione di Dettaglio e Codifica}
\pagestyle{empty}
\begin{figure}[h]
	\centering
	\includegraphics[scale=0.35]{Sezioni/DiagrammiGantt/ProgettazioneDiDettaglio.png}
	\caption{Diagramma di Gantt delle attività di Progettazione di Dettaglio e Codifica}
\end{figure}
\end{landscape}

\subsection{Validazione e Collaudo}
Dal 2020-04-21 al 2020-05-17\\
Inizia al termine della fase di Progettazione di Dettaglio e Codifica e finisce con la data di consegna per la Revisione di Accettazione.\\
In questo fase vengono definite le attività che servono a verificare che il prodotto corrisponda a quello desiderato dal committente e dal proponente.

\subsubsection{Periodo 1} 
Dal 2020-04-21 al 2020-04-28
\begin{itemize}
	\item \textbf{Normazione}: Standardizzazione e correzione di alcune parti della documentazione che non aderiscono completamente alle \NdP{};
	\item \textbf{Correzioni}: Correzioni di difetti notati dal committente (ove presenti) nella \glo{Product Baseline};
	\item \textbf{Assegnazione dei ruoli di progetto}: Assegnazione dei ruoli di ciascun membro del gruppo in base alla suddivisione oraria indicata in §5.4.1;
	\item \textbf{Soddisfazione dei requisiti}: Controllo che i requisiti siano soddisfatti;
	\item \textbf{Pianificazione attività}: Le attività da svolgere devono essere prima pianificate e discusse dal gruppo per garantire il \glo{way of working} sancito nelle \NdP{};
	\item \textbf{Verifica}: \glo{Verifica} dell'andamento del gruppo in relazione alle tempistiche e allo svolgimento dei compiti assegnati.
\end{itemize}

\subsubsection{Periodo 2} 
Dal 2020-04-29 al 2020-05-10
\begin{itemize}
	\item \textbf{Codifica}: Esecuzione dell'ultimo versionamento del prodotto;
	\item \textbf{Verifica}: Accertamento che le esecuzioni delle attività siano esenti da errori (condizione necessaria ma non sufficiente: superamento dei test di unità, di integrazione, di sistema);
	\item \textbf{Validazione}: Verifica se il prodotto realizzato sia conforme alle attese, e validazione finale in caso di esito positivo;
	\item \textbf{Scrittura dei manuali}: Esecuzione del secondo versionamento del Manuale Utente e del Manuale Manutentore;
	\item \textbf{Collaudo}: Vengono eseguiti gli ultimi test sul prodotto per verificare se le funzionalità rispettano i risultati attesi.
\end{itemize}

\subsubsection{Periodo 3} 
Dal 2020-05-11 al 2020-05-17
\begin{itemize}
	\item \textbf{Preparazione per la Revisione di Accettazione}: Il gruppo produce il materiale necessario da esporre alla presentazione pubblica della propria proposta.
\end{itemize}


\newpage
% Inizia la pagina orientata orizzontalmente
\begin{landscape}
% Ora la pagina e' in orizzontale!
\subsubsection{Diagramma di Gantt delle attività di Validazione e Collaudo}
\pagestyle{empty}
\begin{figure}[h]
	\centering
	\includegraphics[scale=0.34]{Sezioni/DiagrammiGantt/Validazione.png}
	\caption{Diagramma di Gantt delle attività di Validazione e Collaudo}
\end{figure}
\end{landscape}