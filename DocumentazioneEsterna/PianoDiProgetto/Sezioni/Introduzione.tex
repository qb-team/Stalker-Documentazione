\section{Introduzione}
\subsection{Scopo del documento}
In questo documento viene illustrato come  il gruppo \Gruppo{} affronta tematiche cruciali nello sviluppo del progetto \NomeProgetto{}, quali:
\begin{itemize}
    \item Analisi dei rischi;
    \item Decisione e giustificazione del modello di sviluppo;
    \item Temporizzazione delle varie fasi in relazione alle \glo{milestone} imposte con conseguente suddivisione del lavoro tra ogni membro del gruppo;
    \item Stime dei costi economici e orari necessari al compimento del progetto \NomeProgetto{}.
\end{itemize}

\subsection{Struttura del documento}
In relazione alle tematiche precedentemente elencate il documento viene così sviluppato:
\begin{itemize}
    \item Introduzione, con riferimenti e scadenze di progetto;
    \item Analisi dei rischi;
    \item Modello di sviluppo;
    \item Pianificazione;
    \item Preventivo;
    \item Consuntivo;
    \item Organigramma.
\end{itemize}

\subsection{Scopo generale del prodotto}
L'obiettivo del prodotto \NomeProgetto{} di \Proponente{} è la creazione di un sistema software composto di un applicativo per cellulare e di un server, con cui interagire tramite un'interfaccia utente. La necessità nasce dal bisogno di adempiere alle normative vigenti in tema di sicurezza.
Le due componenti del sistema software, applicativo e server, devono soddisfare i seguenti obiettivi rispettivamente di:
\begin{itemize}
\item Tracciare e registrare i \glo{movimenti} di un utente in un \glo{luogo di tracciamento} di un'\glo{organizzazione}, siano essi autenticati da credenziali di un'\glo{organizzazione} oppure visitatori anonimi, il tutto nel rispetto della normativa sulla privacy;
\item Poter visionare gli accessi degli utenti autenticati e visionare il numero di visitatori anonimi all'interno di un luogo.
\end{itemize}

\subsection{Glossario}
Al fine di evitare ambiguità fra i termini, e per avere chiare fra tutti gli stakeholder le terminologie utilizzate per la realizzazione del presente documento, il gruppo \Gruppo{} ha redatto un documento denominato \Glossariov{1.0.0}.
In tale documento, sono presenti tutti i termini tecnici, ambigui, specifici del progetto e scelti dai membri del gruppo con le loro relative definizioni.
Un termine presente nel \Glossariov{1.0.0} e utilizzato in questo documento viene indicato con un apice \ap{G} alla fine della parola.

\subsection{Riferimenti}
\subsubsection{Normativi}
\begin{itemize}
    \item \NdPv{}{1.0.0};
    \item \textbf{Organigramma}: \url{https://www.math.unipd.it/~tullio/IS-1/2019/Progetto/RO.html}.
\end{itemize}

\subsubsection{Informativi}
\begin{itemize}
    \item \textbf{Capitolato d'appalto C5 - Stalker}: \url{https://www.math.unipd.it/~tullio/IS-1/2019/Progetto/C5.pdf};
    \item \textbf{Libro di Testo}: Software Engineering (10th edition) - Ian Sommerville - Pearson Education | Addison-Wesley.\\
    Parti:
    \begin{itemize}
        \item Sezione 2.1.2 "Incremental Development"
        \item Sezione 22.1 "Risk Management"
    \end{itemize}
\end{itemize}

\subsection{Scadenze}
Il gruppo \Gruppo{} ha concordato di rispettare le seguenti \glo{milestone} per il progetto \NomeProgetto{}:
\begin{itemize}
    \item \textbf{Revisione dei Requisiti}: 2020-01-21;
    \item \textbf{Revisione di Progettazione}: 2020-03-16;
    \item \textbf{Revisione di Qualifica}: 2020-04-20;
    \item \textbf{Revisione di Accettazione}: 2020-05-18.
\end{itemize}