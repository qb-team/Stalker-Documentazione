\section{Analisi dei Rischi}
La gestione dei rischi è un processo al quale il gruppo \Gruppo{} dà molto importanza. Questo perché incorrere in rischi potrebbe equivalere al danneggiamento del progetto, sia nella sua \glo{organizzazione} e sia nella sua qualità.
Si cerca quindi di fare una previsione dei problemi che si potrebbero verificare durante l'intero corso del progetto e, per ogni rischio identificato, si cerca una soluzione per poterlo evitare.

\subsection{Fasi della gestione dei rischi}
Il gruppo intende seguire i seguenti step nel processo di gestione dei rischi:
\begin{itemize}
	\item \textbf{Identificazione del rischio}: Questo è il primo step del processo e ci serve per identificare i rischi che potrebbero portare a dei problemi durante l'avanzamento del progetto; 
	\item \textbf{Analisi dei rischi}: Dopo aver individuato i rischi nello step precedente, per ognuno di essi viene valutata la probabilità che si verifichi e le conseguenze negative che potrebbe portare;
	\item \textbf{Pianificazione del rischio}: Nella pianificazione del rischio si sviluppano dei piani per sapere quali rimedi vanno intrapresi nel momento in cui i rischi si verificano. In tale maniera si riuscirà a risolvere i problemi prima che essi si aggravino;
	\item \textbf{Monitoraggio del rischio}: Nell'ultimo step della gestione del rischio viene verificato che le ipotesi relative ai rischi non abbiano subito delle variazioni. Quindi si cerca di valutare periodicamente la probabilità che il rischio si verifichi e i suoi possibili effetti, migliorando le strategie adottate per la loro risoluzione.
\end{itemize}

\subsection{Tipologia del rischio}
Ci sono 5 tipi di rischi che il gruppo \Gruppo{} terrà in considerazione. 
\\Ad ogni rischio verrà assegnato un codice identificativo:
\begin{itemize}
	\item Rischi Tecnologici [RT];
	\item Rischi Organizzativi [RO];
	\item Rischi Personali [RP];
	\item Rischi dei Requisiti [RR];
	\item Rischi di Stima [RS].
\end{itemize}

\subsection{Tabella dei rischi}
Nella seguente tabella vengono elencati i rischi che il gruppo \Gruppo{} potrebbe incontrare durante l'intero ciclo di vita del progetto.
Ogni riga della tabella corrisponde ad un rischio ed è composta da:
\begin{itemize}
	\item Codice [codice del tipo + numero sequenziale] e Nome del rischio;
	\item Descrizione;
	\item Rilevamento;
	\item Piano di Contingenza.
\end{itemize}

{
\rowcolors{2}{grigetto}{white}
\renewcommand{\arraystretch}{1.5}
\centering
\begin{longtable}{ c c  C{4cm}  C{3cm} C{4cm}}
\rowcolor{darkblue}
\textcolor{white}{\textbf{Codice}} & \textcolor{white}{\textbf{Nome}} & \textcolor{white}{\textbf{Descrizione}} & \textcolor{white}{\textbf{Rilevamento}} & \textcolor{white}{\textbf{Probabilità e Gravità}} & \textcolor{white}{\textbf{Risoluzione}}\\	

RT1 & Inesperienza alle tecnologie & Il gruppo dovrà affrontare tecnologie mai utilizzate precedentemente e quindi servirà del tempo per poter imparare ad utilizzarle nel modo corretto & Ogni componente del gruppo sarà consapevole di saper usare una determinata tecnologia & Probabilità Alta Gravità Media & Ogni componente del gruppo che ha acquisito una certa abilità nell'utilizzo di una tecnologia, cercherà di aiutare i componenti del gruppo che hanno più difficoltà \\
		

\end{longtable}
}