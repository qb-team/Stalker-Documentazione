\section{Introduzione}
\subsection{Scopo del documento}
Lo scopo del documento è quello di descrivere in maniera dettagliata i requisiti e i casi d'uso che sono stati individuati durante lo studio del progetto Stalker.
\subsection{Scopo del prodotto}
L'obbiettivo del prodotto è la realizzazione di un'applicazione per smartphone, nello specifico Android, che permetta di segnalare ad un server dedicato sia l'ingresso che l'uscita dell'utilizzatore dalle aree d'interesse (Organizzazioni o luoghi all'interno di esse).
Il server, il quale sarà utilizzato dagli amministratori, deve essere in grado di gestire più organizzazioni e monitorare gli utenti all'interno di esse.
\subsection{Glossario}
Nel documento sono presenti termini che possono assumere significati diversi a seconda del contesto. Per questo motivo è stato creato un Glossario contenente tali termini, i quali saranno descritti in maniera approfondita. I termini presenti nel \textit{Glossario v.1} saranno identificabili con una G posta all'apice di essi.

\subsection{Riferimenti}


\subsubsection{Riferimenti normativi}
\begin{itemize}
\item \textbf{Norme di progetto}: \textit{Norme di progetto v.1.0.0};
\item \textbf{Verbale esterno}: \textit{Verbale esterno 13-12-2019}.
\end{itemize}

\subsubsection{Riferimenti informativi}
\begin{itemize}
\item \textbf{Studio di fattibilità}: \textit{Studio di fattibilità v.1.0.0}.
\item \textbf{Slide del capitolato C5 - Stalker}: https://www.math.unipd.it/~tullio/IS-1/2019/Progetto/C5.pdf
\end{itemize}
