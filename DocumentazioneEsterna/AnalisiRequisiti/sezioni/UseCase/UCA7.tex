\newpage
%scritto da Federico Perin
\subsubsection{UCA 7 - Raccolta di messaggi di errore lato applicazione}
\subsubsection{UCA 7.1.1 - Visualizzazione messaggio di errore in caso di e-mail già presente durante la registrazione}%fish level
\begin{itemize}
\item \textbf{Attori primari:} Utente non autenticato;
%\item \textbf{Attori secondari:}%opzionale
\item \textbf{Precondizione:} L'utente non è registrato e ha inserito nel campo indirizzo e-mail un valore già presente nel sistema;
\item \textbf{Postcondizione:} L'indirizzo e-mail inserito dall'utente è già presente nel sistema, pertanto riceve un messaggio di errore esplicativo che lo esorta a scegliere una nuova password.
\end{itemize}

\subsubsection{UCA 7.1.2 - Visualizzazione messaggio di errore in caso di password troppo debole}%fish level
\begin{itemize}
\item \textbf{Attori primari:} Utente non autenticato;
%\item \textbf{Attori secondari:}%opzionale
\item \textbf{Precondizione:} L'utente non è registrato e ha inserito una password poco sicura;
\item \textbf{Postcondizione:} La password inserita dall'utente risulta essere poco sicura per il sistema, pertanto riceve un messaggio di errore che gli intima di sceglierne una più sicura.
\end{itemize}


\subsubsection{UCA 7.1.3 - Visualizzazione messaggio di errore in caso di password e conferma password diverse}%fish level
\begin{itemize}
\item \textbf{Attori primari:} Utente non autenticato;
%\item \textbf{Attori secondari:}%opzionale
\item \textbf{Precondizione:} L'utente non è registrato e ha inserito nei campi relativi alla password e conferma password valori differenti;
\item \textbf{Postcondizione:} La password inserita dall'utente è diversa da quella inserita nel campo conferma password, pertanto riceve un messaggio di errore esplicativo.
\end{itemize}

\subsubsection{UCA 7.3.1 - Visualizzazione di un messaggio di errore che informa del tentativo fallito di scaricare la lista}%fish level
\begin{itemize}
\item \textbf{Attori primari:} Utente anonimo;
\item \textbf{Precondizione:} L'utente esegue l'azione di scaricare la lista delle organizzazioni\ap{G} ma fallisce;
\item \textbf{Postcondizione:} Viene visualizzato un messaggio d'errore che informa che il tentativo di scaricare la lista è fallito.

\end{itemize}

\subsubsection{UCA 7.3.2 - Visualizzazione di un messaggio di errore che informa che non è salvata nessuna lista delle organizzazioni}%fish level
\begin{itemize}
	\item \textbf{Attori primari:} Utente anonimo;
	\item \textbf{Precondizione:} La lista delle organizzazioni\ap{G} non è presente nel dispositivo dell'utente;
	\item \textbf{Postcondizione:} Viene visualizzato un messaggio di errore che informa che nel dispositivo non è presente la lista delle organizzazioni\ap{G}.
\end{itemize}

\subsubsection{UCA 7.5.1 - Avviso di assenza di organizzazioni preferite}
\begin{itemize}
    \item \textbf{Attori primari:} Utente anonimo, Utente riconosciuto;
    \item \textbf{Precondizione:} L'utente non ha aggiunto alcuna organizzazione\ap{G} come preferita;
    \item \textbf{Postcondizione:} L'utente visualizza nella schermata un messaggio che lo avvisa dell'assenza di organizzazioni\ap{G}.
\end{itemize}

\subsubsection{UCA 7.5.2 - Avviso di assenza di accessi presso i luoghi di un'organizzazione preferita}
\begin{itemize}
    \item \textbf{Attori primari:} Utente anonimo, Utente riconosciuto;
    \item \textbf{Precondizione:} L'utente ha selezionato dalla lista delle organizzazioni preferite un'organizzazione di cui visualizzare i propri accessi, ma non ha mai effettuato accesso ai luoghi dell'organizzazione;
    \item \textbf{Postcondizione:} L'utente visualizza un messaggio che lo avvisa della mancanza di accessi nei luoghi dell'organizzazione.
\end{itemize}

\subsubsection{UCA 7.6.1 - Notifica di errore se procedimento di memorizzazione dell'accesso al luogo è fallito}
\begin{itemize}
	\item \textbf{Attori primari:} Utente riconosciuto, Utente anonimo;
	\item \textbf{Precondizione:} Avviene un errore durante la registrazione dell'accesso dell'utente;
	\item \textbf{Postcondizione:} Viene notificato l'errore all'utente.
\end{itemize}