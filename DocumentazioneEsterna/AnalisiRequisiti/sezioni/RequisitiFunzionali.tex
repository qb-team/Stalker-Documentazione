\renewcommand{\o}{Obbligatorio}
\renewcommand{\d}{Desiderabile}
\newcommand{\op}{Opzionale}
\subsection{Requisiti funzionali}
{
\rowcolors{2}{grigetto}{white}
\renewcommand{\arraystretch}{1.5}
\centering
\begin{longtable}{ c C{4cm} c c}
\rowcolor{rossoep}
\textcolor{white}{\textbf{Identificativo}} & \textcolor{white}{\textbf{Descrizione}} & \textcolor{white}{\textbf{Classificazione}} & \textcolor{white}{\textbf{Fonti}}\\	

R1FI1 & Un utente non autenticato non può effettuare alcuna azione a meno di autenticazione e registrazione & Obbligatorio & Interno\\

R1FA1.1 & L'autenticazione da parte di un utente necessita di e-mail & Obbligatorio & UCA 1.1.1 Interno\\

R1FA1.2 & L'autenticazione da parte di un utente necessita di password & Obbligatorio & UCA 1.1.2 Interno\\

R1FA7.1 & L'autenticazione viene negata qualora l'utente tenti di autenticarsi con delle credenziali errate. Viene inoltre visualizzato un messaggio d'errore & Obbligatorio & UCA 7.1.4 Interno\\

R1FA1.3 & La registrazione da parte di un utente necessita di e-mail & Obbligatorio & UCA 1.2.1 Interno\\

R1FA7.2 & Il processo di registrazione dell'utente viene negato qualora l'e-mail inserita fosse già registrata nel sistema. Viene visualizzato inoltre un messaggio d'errore & Obbligatorio & UCA 7.1.1 Interno\\

R1FA1.4 & La registrazione da parte di un utente necessita di password & Obbligatorio & UCA 1.2.2 Interno\\

R1FA1.5 & La registrazione da parte di un utente necessita di conferma della password & Obbligatorio & UCA 1.2.3 Interno\\

R2FA1.6 & L'utente deve essere in grado di effettuare il reset della password qualora se la fosse dimenticata & Desiderabile & UCA 1.3 Interno\\

R1FA7.3 & Il processo di autenticazione viene negato qualora la password inserita non sia abbastanza sicura. Viene visualizzato inoltre un messaggio d'errore & \o & UCA 1.3 Interno\\

R1FA7.4 & Il processo di registrazione viene negato se password e conferma password inserita non combaciano. Viene visualizzato inoltre un messaggio d'errore & \o & UCA 1.3 Interno\\



AAAAAAAAA & AAAAAAAAAAA & \o \d \op & AAAAAAAAAAAAAAAA\\
R1FI2 & Un amministratore non autenticato non può effettuare alcuna azione a meno di autenticazione & \o & Interno \\
R1FS1.1 & L’autenticazione da parte di un amministratore necessita di e-mail & \o & UCS 1.1.1 Interno\\
R1FS10.1 & L’autenticazione viene negata qualora l'amministratore tenti di autenticarsi con delle credenziali errate. Viene inoltre visualizzato un messaggio d’errore & \o & UCS 10.1.1 Interno \\
R1FS1.2 & L’autenticazione da parte di un amministratore necessita di password & \o & UCS 1.1.2 Interno\\
R2FS1.3 & L'amministratore deve essere in grado di effettuare il reset della password qualora se la fosse dimenticata & \d & UCS 1.3 Interno\\
R1FS2.1 & L'amministratore deve essere in grado di effettuare il logout & \o & UCS 2 Interno\\
R1FC3 & L'amministratore deve poter visualizzare le organizzazioni disponibili & \o & Capitolato\\
R1FS3.1 & L'amministratore deve poter selezionare un'organizzazione tra quelle da lui visualizzate & \o & UCS 3 Interno\\
R1FS4.1 & L'amministratore deve poter modificare il nome dell'organizzazione & \o & UCS 4.1.1 Interno\\
R2FS4.2 & L'amministratore deve poter modificare l'immagine dell'organizzazione & \d & UCS 4.1.2 Interno\\
R2FS4.3 & L'amministratore deve poter modificare la descrizione dell'organizzazione & \d & UCS 4.1.3 Interno\\
R1FS4.4 & L'amministratore deve poter modificare l'indirizzo dell'organizzazione & \o & UCS 4.1.4 Interno\\
R1FS10.2 & Se il nome dell'organizzazione inserito dall'amministratore non rispetta i vincoli imposti viene mostrato un messaggio d'errore & \o & UCS 10.4.2\\
R1FS10.3 & Se il nome dell'organizzazione inserito dall'amministratore dovesse essere già presente nel sistema (associato ad un'altra organizzazione) viene mostrato un messaggio d'errore & \o & UCS 10.4.3\\
R2FS10.4 & Se l'immagine dell'organizzazione selezionata dall'amministratore non rispetta i vincoli imposti viene mostrato un messaggio d'errore & \d & UCS 10.4.4\\
R2FS10.5 & Se la descrizione dell'organizzazione inserita dall'amministratore non rispetta i vincoli imposti viene mostrato un messaggio d'errore & \d & UCS 10.4.5\\
R1FS10.6 & Se l'indirizzo dell'organizzazione inserito dall'amministratore non rispetta i vincoli imposti viene mostrato un messaggio d'errore & \o & UCS 10.4.6\\
R1FS4.5 & L'amministratore deve avere la possibilità di inviare la richiesta di eliminazione per un'organizzazione & \o & UCS 4.2 Capitolato\\
R3FS4.6 & L'amministratore deve poter inserire una motivazione per la richiesta di eliminazione dell'organizzazione & \op & UCS 4.2.1 Interno \\
R1FS4.7 & L'amministratore deve poter annullare le modifiche che sta apportando & \o & UCS 4.3 Interno\\
 &  &  & \\


AAAAAAAAA & AAAAAAAAAAA & \o \d \op & AAAAAAAAAAAAAAAA\\

R1FA2 & L'utente deve essere in grado di effettuare il logout & \o & UCA 2 Interno\\
R1FA3.1 & L'utente può gestire la propria lista delle organizzazioni\ap{G}& \o & UCA 3 Interno\\
R1FA3.2 & L'utente deve poter essere in grado di scaricare la lista di tutte le organizzazioni\ap{G} & \o & UCA 3.1 Capitolato \\
R1FA7.3 & Qualora fallisca lo scaricamento della lista delle organizzazioni\ap{G} deve venire visualizzato un messaggio d'errore che lo informa di tale evento & \o & UCA 7.3.1 Interno \\
R1FA3.3 & L’utente deve poter essere in grado di gestire la propria lista delle organizzazioni preferite\ap{G} & \o & UCA 3.2 Interno \\
R1FA3.4 & L’utente può inserire una organizzazione\ap{G} presente nella lista delle organizzazioni\ap{G}, nella propria lista delle organizzazioni preferite\ap{G} & \o & UCA 3.2.1 Interno \\
R1FA3.5 & Qualora l’utente inserisca un'organizzazione\ap{G} nella propria lista delle organizzazioni preferite\ap{G} che richiede autenticazione con credenziali LDAP, deve autenticarsi con credenziali LDAP\ap{G} & \o & UCA 3.2.2 Capitolato\\
R1FA3.6 & L’utente può rimuovere una organizzazione\ap{G} presente nella propria lista delle organizzazioni preferite\ap{G} & \o & UCA 3.2.3 Interno \\
R1FA7.4 & Qualora non sia memorizzata nessuna lista delle organizzazioni\ap{G} nel dispositivo, viene informato l’utente di questo fatto & \o & UCA 7.3.2 Interno \\
R1FA3.7 & L’utente ha la possibilità di aggiornare la lista delle organizzazioni\ap{G} & \o & UCA 3.3 Interno \\
R1FA3.8 & L’utente può aggiornare la lista delle organizzazioni\ap{G} tramite refresh manuale\ap{G} & \o & UCA 3.3.1 Interno \\
R1FA3.9 & L’utente può aggiornare la lista delle organizzazioni\ap{G} tramite temporizzazione\ap{G} & \o & UCA 3.3.2 Interno \\
R1FA3.10 & L’utente può visualizzare la lista delle organizzazioni\ap{G} & \o & UCA 3.4 Interno \\
R2FA3.11 & L’utente ha la possibilità di visualizzare la lista delle organizzazioni\ap{G} ordinate alfabeticamente, dalla A alla Z & \d & UCA 3.4.1 Interno \\
R2FA3.12 & L’utente ha la possibilità di visualizzare la lista delle organizzazioni\ap{G} ordinate secondo politica FIFO\ap{G} & \d & UCA 3.4.2 Interno \\
R3FA3.13 & L’utente ha la possibilità di visualizzare la lista delle organizzazioni\ap{G} che permettono il tracciamento anonimo\ap{G} & \op & UCA 3.4.3 Interno \\
R3FA3.14 & L’utente ha la possibilità di visualizzare la lista delle organizzazioni\ap{G} che permettono il tracciamento autenticato\ap{G} & \op & UCA 3.4.4 Interno \\
R1FA3.15 & L’utente può effettuare ricerche personalizzate per cercare le organizzazioni\ap{G} presenti nella lista delle organizzazioni\ap{G} & \o & UCA 3.5 Interno\\
R2FA3.16 & L’utente può ricercare organizzazioni\ap{G} presenti nella lista delle organizzazioni\ap{G} appartenenti alla nazione indicata dall’utente & \d & UCA 3.5.1 Interno \\
R1FA3.17 & L’utente può ricercare organizzazioni\ap{G} presenti nella lista delle organizzazioni\ap{G} che hanno nel nome una sottostringa scelta dall'utente & \o & UCA 3.5.2 Interno \\
R2FA3.18 & L’utente può ricercare organizzazioni\ap{G} presenti nella lista delle organizzazioni\ap{G} appartenenti alla città indicata dall’utente & \d & UCA 3.5.3 Interno \\
R1FA4.1 & L’utente deve poter inserire la modalità di tracciamento\ap{G} che preferisce & \o & UCA 4 Capitolato \\
R1FA4.2 & L’utente può selezionare la modalità di tracciamento anonimo\ap{G} & \o & UCA 4.1 Capitolato \\
R1FA4.3 & L’utente può selezionare la modalità di tracciamento autenticato\ap{G} & \o & UCA 4.2 Capitolato \\
R2FA5.1 & L’utente ha la possibilità di visualizzare il proprio storico degli accessi & \d & UCA 5 Capitolato \\
R2FA5.2 & L’utente ha la possibilità di visualizzare il proprio storico degli accessi presso una organizzazione \ap{G} & \d & UCA 5.1 Capitolato \\
R2FA5.3 & L'utente nella visualizzazione del proprio storico degli accessi nell'organizzazione visualizza la data per ogni accesso di quando è stato fatto & \d &  UCA 5.1 \\
R2FA5.4 & L'utente nella visualizzazione del proprio storico degli accessi nell'organizzazione visualizza il luogo per ogni accesso di quando è stato fatto & \d &  UCA 5.1 \\
R2FA5.5 & L'utente nella visualizzazione del proprio storico degli accessi nell'organizzazione visualizza il tempo trascorso per ogni accesso di quando è stato fatto & \d &  UCA 5.1 \\
R2FA5.6 & L’utente ha la possibilità di visualizzare il proprio storico degli accessi presso un luogo dell’organizzazione\ap{G} & \d & UCA 5.2 Capitolato\\
R2FA5.7 & L'utente nella visualizzazione del proprio storico degli accessi nel luogo del organizzazione visualizza la data per ogni accesso di quando è stato fatto & \d &  UCA 5.1 \\
R2FA5.8 & L'utente nella visualizzazione del proprio storico degli accessi nel luogo del organizzazione visualizza il luogo per ogni accesso di quando è stato fatto & \d &  UCA 5.1 \\
R2FA5.9 & L'utente nella visualizzazione del proprio storico degli accessi nel luogo del organizzazione visualizza il tempo trascorso per ogni accesso di quando è stato fatto & \d &  UCA 5.1 \\
R2FA5.10 & L’utente può visualizzare la propria lista degli accessi in una organizzazione\ap{G} ordinata per data decrescente & \d & UCA 5.1.1 \\
R2FA5.11 & L’utente può visualizzare la propria lista degli accessi in una organizzazione\ap{G} ordinata per data crescente & \d & UCA 5.1.2 \\
R3FA5.12 & L’utente può effettuare una ricerca degli accessi presso un'organizzazione\ap{G} in un giorno specifico& \op & UCA 5.1.3 \\
R2FA5.13 & L’utente può visualizzare la propria lista degli accessi presso un luogo dell’organizzazione \ap{G} ordinata per data decrescente & \d & UCA 5.2.1 \\
R2FA5.14 & L’utente può visualizzare la propria lista degli accessi presso un luogo dell’organizzazione \ap{G} ordinata per data crescente & \d & UCA 5.2.2 \\
R3FA5.15 & L’utente può effettuare una ricerca degli accessi presso un luogo dell’organizzazione \ap{G} in un giorno specifico & \op & UCA 5.2.3 \\
R2FA5.16 & L’utente se si trova all’interno dell’organizzazione\ap{G} ha la possibilità di visualizzare il tempo passato all’interno dall'ultimo ingresso effettuato & \d & UCA 5.1 Capitolato \\
R2FA5.17 & L’utente se si trova all’interno dell’luogo dell’organizzazione\ap{G} ha la possibilità di visualizzare il tempo passato all’interno dall'ultimo ingresso effettuato & \d & UCA 5.2 Capitolato \\
R2FA7.5 & Qualora non ci sono accessi effettuati presso l'organizzazione selezionata, l'utente deve essere informato di ciò & \d & UCA 7.5.1 Interno \\
R2FA7.6 & Qualora non ci sono accessi effettuati presso il luogo selezionato, l'utente deve essere informato di ciò & \d & UCA 7.5.2 Interno \\
R2FA6.1 & L’utente che effettua un movimento nell’organizzazione, deve essere registrato il tracciamento della sua azione & \d & UCA 6 Capitolato \\
R2FA6.2 & Nella registrazione del tracciamento di un movimento dell’utente, deve essere memorizzata la data di quando è stato fatto & \d & UCA 6.1.1 \\
R2FA6.3 & Nella registrazione del tracciamento di un movimento dell’utente, deve essere memorizzata l’ora di quando è stato fatto & \d & UCA 6.1.1 \\
R2FA6.4 & Nella registrazione del tracciamento di un movimento dell’utente, deve essere memorizzata da chi è stata fatta & \d & UCA 6 Interno \\
R2FA6.5 & L’utente che effettua un ingresso nell’organizzazione, deve essere registrato il tracciamento della sua azione secondo la modalità di tracciamento autenticata & \d & UCA 6.1 Capitolato \\
R2FA6.6 & L’utente che effettua l’uscita dall’organizzazione, deve essere registrato il tracciamento della sua azione secondo la modalità di tracciamento autenticata & \d & UCA 6.2 Capitolato \\
R2FA6.7 & L’utente che effettua un ingresso nell’organizzazione, deve essere registrato il tracciamento della sua azione secondo la modalità di tracciamento anonima & \d & UCA 6.3 Capitolato \\
R2FA6.8 & L’utente che effettua l’uscita dall’organizzazione, deve essere registrato il tracciamento della sua azione secondo la modalità di tracciamento anonima & \d & UCA 6.4 Capitolato \\
R2FA 7.7 & Qualora non vengano memorizzate le informazioni necessarie per la registrazione del movimento effettuato dall’utente, deve essere notificato tale evento all’utente & \d & UCA 7.6.1 \\
R2FA6.9 & Deve essere notificato all’utente che è avvenuta la corretta registrazione del suo movimento & \d & UCA 6.1.3 \\

\end{longtable}
}