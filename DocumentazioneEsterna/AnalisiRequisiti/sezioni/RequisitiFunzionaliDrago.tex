%R1FS5 & L'amministratore deve essere in grado di modificare le funzionalità di tracciamento dell'organizzazione & \o & UCS 5 Interno\\
R1FS5.1 & L'amministratore può selezionare un nuovo perimetro di tracciamento dell'organizzazione & \o & UCS 5.1 Capitolato\\
R1FA10.7 & La modifica del perimetro dell'organizzazione viene negata qualora l'amministratore selezioni un area che non rispetta i vincoli imposti. Viene visualizzato un messaggio di errore & \o & UCS 10.5.1 Capitolato \\
R1FS5.2 & L'amministratore deve essere in grado di creare dei nuovi luoghi di tracciamento nell'organizzazione & \o & UCS 5.2 Capitolato\\
R1FS5.3 & L'amministratore deve essere in grado di modificare i luoghi di tracciamento dell'organizzazione  & \o & UCS 5.3 Capitolato\\
R1FS10.8 & La creazione di nuovi luoghi e la modifica dell'area di tracciamento di essi vengono negati qualora l'amministratore selezioni un area che fuoriesce dal perimetro. Viene visualizzato un messaggio di errore & \o & UCS 10.5.2 Capitolato \\
R1FS5.4 & L'amministratore deve essere in grado di eliminare i luoghi di tracciamento dell'organizzazione  & \o & UCS 5.4 Capitolato\\
R1FS5.5 & L'amministratore deve essere in grado di selezionare un'area geografica per il tracciamento del luogo scelto  & \o & UCS 5.5 Capitolato\\
R1FS5.6 &  L'amministratore può scegliere l'area di tracciamento tramite l'inserimento delle coordinate geografiche & \o & UCS 5.5.1 Capitolato\\
R2FS5.7 & L'amministratore può scegliere l'area di tracciamento tramite Google Maps API & \d & UCS 5.5.2 Interno\\
R1FS5.8 & L'amministratore deve poter annullare la selezione dell'area geografica per il tracciamento & \o & UCS 5.5.3 Interno\\
R1FS6 & L'amministratore può monitorare l'organizzazione visualizzando il numero di utenti anonimi presenti nell'organizzazione & \o & UCS 6 Capitolato\\
R1FS6.1 & L'amministratore può monitorare l'organizzazione visualizzando il numero di utenti anonimi presenti in un specifico luogo dell'organizzazione  & \o & UCS 6.1 Capitolato\\
R1FS7.1 & L'amministratore può monitorato gli accessi effettuati dagli utenti riconosciuti visualizzandone il nome e il cognome & \o & UCS 7 Capitolato\\
R1FS7.2 & L'amministratore può monitorare gli accessi effettuati da un specifico utente riconosciuto visualizzandone il nome e il cognome & \o & UCS 7.1 Capitolato\\
R2FS7.3 & L'amministratore può monitorare gli accessi filtrandoli in base a una data & \d & UCS 7.2 Interno\\
R2FS7.4 & Bisogna inserire una data per filtrare gli accessi in base a essa & \d & UCS 7.2.1 Interno\\
R1FS8.1 & L'amministratore può ricevere un report tabellare degli accessi ai luoghi dell'organizzazione & \d & UCS 8 Capitolato\\
R1FS8.2 &  Tabella delle entrate e uscite degli utenti nei luoghi dell'organizzazione generabile dall'amministratore di un organizzazione a tracciamento autenticato & \d & UCS 8.1.1 Capitolato\\
R1FS8.3 & Tabella delle ore spese dagli utenti nei luoghi dell'organizzazione generabile dall'amministratore di un organizzazione a tracciamento autenticato & \d & UCS 8.1.2 Capitolato\\
R1FS8.4 & Tabella contenente il numero degli utenti e il totale delle ore passate da essi nei luoghi dell'organizzazione generabile dall'amministratore di un organizzazione a tracciamento autenticato o anonimo & \d & UCS 8.1.3 Capitolato\\
%aggiungere tabelle più particolareggiate

%R3FS8.1 & Le tabelle generabili dall'amministratore possono  & \op & UCS 8.1 Interno\\




