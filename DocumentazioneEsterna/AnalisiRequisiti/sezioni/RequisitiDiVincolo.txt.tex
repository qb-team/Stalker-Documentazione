\renewcommand{\o}{Obbligatorio}
\renewcommand{\d}{Desiderabile}

\subsection{Requisiti di vincolo}
{
\rowcolors{2}{grigetto}{white}
\renewcommand{\arraystretch}{1.5}
\centering
\begin{longtable}{ c C{4cm} c c}
\rowcolor{rossoep}
\textcolor{white}{\textbf{Identificativo}} & \textcolor{white}{\textbf{Descrizione}} & \textcolor{white}{\textbf{Classificazione}} & \textcolor{white}{\textbf{Fonti}}\\	

R1VC1.1 & Deve essere sviluppato un server back-end & \o & Capitolato \\
R1VC1.2 & Il server deve essere correlato di una UI\ap{G} per la gestione delle funzioni implementate& \o & Capitolato \\
R1VC1.3 & Deve essere sviluppata una applicazione mobile per sistemi operativi Android o IOS & \o & Capitolato \\
R1VC2.1 & Le comunicazioni di tracciamento tra applicazione cellulare e Server devono avvenire solo al momento d’ingresso ed uscita dai luoghi designati & \o & Capitolato \\
R2VC3.1 & Utilizzo di Java\ap{G} (versioni 8 o superiori) per sviluppo Server back-end & \d & Capitolato \\
R2VC3.2 & Utilizzo di Python\ap{G} per sviluppo server back-end & \d & Capitolato \\
R2VC3.3 & Utilizzo di Nodejs\ap{G}per sviluppo server back-end & \d & Capitolato \\
R2VC4.1 & Utilizzo di protocolli asincroni\ap{G} per le comunicazioni app mobile-server & Desiderato & Capitolato \\
R2VC5.1 & Utilizzo del pattern di Publish/Subscriber\ap{G} & \d & Capitolato \\
R2VC6.1 & Utilizzo dell’IAAS Kubernetes\ap{G} per la gestione della scalabilità orizzontale\ap{G}& \d & Capitolato \\
R2VC6.2 & Utilizzo di PAAS\ap{G} per la gestione della scalabilità orizzontale\ap{G}& \d & Capitolato \\
R2VC6.3 & Utilizzo di Openshift\ap{G} per la gestione della scalabilità orizzontale\ap{G}& \d & Capitolato \\
R2VC6.4 & Utilizzo di Rancher\ap{G} per la gestione della scalabilità orizzontale\ap{G}& \d & Capitolato \\
R1VC7.1 & Il server deve esporre oltre ad eventuali altri protocolli richiesti per l’interazione con il servizio, delle API Rest\ap{G} necessarie per permettere l’utilizzo applicativo, come alternativa alle API Rest\ap{G} è possibile utilizzare gRPC\ap{G} & \o & Capitolato \\
R1VC8.1 & Deve essere garantita una precisione sufficiente affinché si possibile certificare che la persona tracciata sia all’interno degli edifici & \o & Capitolato \\
R2VC9.1 & Utilizzo di tecnologie network-GPS\ap{G} per il tracciamento della posizione & \d & Capitolato \\
R1VC9.2 & Si deve fornire un resoconto sulle scelte fatte e sui test effettuati relativi al tracciamento della posizione & \o & Capitolato \\
R1VC10.1 & Si deve correlare a tutte le componenti applicative dei test di unità e d’integrazione & \o & Capitolato \\
R1VC10.2 & Si deve testare tramite test end to end il sistema nella sua interezza & \o & Capitolato \\
R1VC10.3 & Devono essere effettuati test di carico per testare il corretto funzionamento della scalabilità & \o & Capitolato \\
R1VC10.4 & Si deve avere una copertura dei test maggiore o uguale all’80 percento & \o & Capitolato \\
R1VC10.5 & Tutti i test effettuati devono essere correlati da un report & \o & Capitolato \\
R1VC11.1 & Deve essere scritta documentazione relativa alle scelte implementative e progettuali con annessa motivazione & \o & Capitolato \\
R1VC11.2 & Deve essere scritta documentazione relativa ai problemi aperti e eventuali soluzioni proposte da esplorare & \o & Capitolato \\
R2VC12.1 & Cifratura di tutte le comunicazioni fra App e Server & \d & Capitolato  \\
R1VC13.1 & Si deve utilizzare il protocollo LDAP per autenticare i singoli dipendenti e gli amministratori di una organizzazione per poter effettuare il tracciamento della posizione autenticato & \o & Capitolato \\

\end{longtable}
}