R1FI2 & Un amministratore non autenticato non può effettuare alcuna azione a meno di autenticazione & \o & Interno \\
R1FS1.1 & L’autenticazione da parte di un amministratore necessita di e-mail & \o & UCS 1.1.1 Interno\\
R1FS10.1 & L’autenticazione viene negata qualora l'amministratore tenti di autenticarsi con delle credenziali errate & \o & UCS 10.1.1 Interno \\
R1FS10.2 & Qualora l'amministratore tenti di autenticarsi con le credenziali errate viene visualizzato un messaggio d’errore & \o & UCS 10.1.1 Interno \\
R1FS1.2 & L’autenticazione da parte di un amministratore necessita di password & \o & UCS 1.1.2 Interno\\
R2FS1.3 & L'amministratore deve essere in grado di effettuare il reset della password qualora se la fosse dimenticata & \d & UCS 1.3 Interno\\
R1FS2.1 & L'amministratore deve essere in grado di effettuare il logout & \o & UCS 2 Interno\\
R1FC3 & L'amministratore deve poter visualizzare le organizzazioni disponibili & \o & Capitolato\\
R1FI3 & Deve venire mostrato il nome dell'organizzazione durante la sua visualizzazione da parte di un amministratore & \o & Interno\\
R2FI4 & Deve venire mostrata l'immagine dell'organizzazione durante la sua visualizzazione da parte di un amministratore & \d & Interno\\
R1FS3.1 & L'amministratore deve poter selezionare un'organizzazione tra quelle da lui visualizzate & \o & UCS 3 Interno\\
R1FI5 & Deve venire mostrato il nome dell'organizzazione selezionata durante la sua visualizzazione da parte di un amministratore & \o & Interno\\
R2FI6 & Deve venire mostrata l'immagine dell'organizzazione selezionata durante la sua visualizzazione da parte di un amministratore & \d & Interno\\
R2FI7 & Deve venire mostrata la descrizione dell'organizzazione selezionata durante la sua visualizzazione da parte di un amministratore & \d & Interno\\
R1FI8 & Deve venire mostrato l'indirizzo dell'organizzazione selezionata durante la sua visualizzazione da parte di un amministratore & \o & Interno\\
R1FS4.1 & L'amministratore deve poter inserire un nuovo nome dell'organizzazione & \o & UCS 4.1.1 Interno\\
R1FS4.2 & L'amministratore deve poter modificare il nome dell'organizzazione & \o & UCS 4.1.1 Interno\\
R2FS4.3 & L'amministratore deve poter inserire una nuova immagine dell'organizzazione & \d & UCS 4.1.2 Interno\\
R2FS4.4 & L'amministratore deve poter modificare l'immagine dell'organizzazione & \d & UCS 4.1.2 Interno\\
R2FS4.5 & L'amministratore deve poter inserire una nuova descrizione dell'organizzazione & \d & UCS 4.1.3 Interno\\
R2FS4.6 & L'amministratore deve poter modificare la descrizione dell'organizzazione & \d & UCS 4.1.3 Interno\\
R1FS4.7 & L'amministratore deve poter inserire un nuovo indirizzo dell'organizzazione & \o & UCS 4.1.4 Interno\\
R1FS4.8 & L'amministratore deve poter modificare l'indirizzo dell'organizzazione & \o & UCS 4.1.4 Interno\\
R1FS10.3 & Se il nome dell'organizzazione inserito dall'amministratore non rispetta i vincoli imposti viene mostrato un messaggio d'errore & \o & UCS 10.4.2\\
R1FS10.4 & Se il nome dell'organizzazione inserito dall'amministratore dovesse essere già presente nel sistema e associato ad un'altra organizzazione viene mostrato un messaggio d'errore & \o & UCS 10.4.3\\
R2FS10.5 & Se l'immagine dell'organizzazione selezionata dall'amministratore non rispetta i vincoli imposti viene mostrato un messaggio d'errore & \d & UCS 10.4.4\\
R2FS10.6 & Se la descrizione dell'organizzazione inserita dall'amministratore non rispetta i vincoli imposti viene mostrato un messaggio d'errore & \d & UCS 10.4.5\\
R1FS10.7 & Se l'indirizzo dell'organizzazione inserito dall'amministratore non rispetta i vincoli imposti viene mostrato un messaggio d'errore & \o & UCS 10.4.6\\
R1FS4.9 & L'amministratore deve avere la possibilità di inviare la richiesta di eliminazione per un'organizzazione & \o & UCS 4.2 Capitolato\\
R3FS4.10 & L'amministratore deve poter inserire una motivazione per la richiesta di eliminazione dell'organizzazione & \op & UCS 4.2.1 Interno \\
R1FS4.11 & L'amministratore deve poter annullare le modifiche che sta apportando & \o & UCS 4.3 Interno\\

R1FS9.1 & L'amministratore proprietario ha la possibilità di entrare nella sezione di gestione degli amministratori (per la nomina, eliminazione e modifica dei privilegi ad altri amministratori) & \o & UCS 9 Capitolato \\
R1FI8 & L'amministratore proprietario ha la possibilità di visualizzare gli amministratori da esso nominati una volta entrato nella gestione degli amministratori & \o & Interno \\
R1FI9 & La visualizzazione di un amministratore deve mostrare la sua e-mail & \o & Interno \\
R1FI10 & La visualizzazione di un amministratore deve mostrare i suoi privilegi & \o & Interno \\
R1FS9.2 & L'amministratore proprietario ha la possibilità di nominare un nuovo amministratore & \o & UCS 9.1 Capitolato\\
R1FS9.3 & L'amministratore proprietario deve poter inserire un'e-mail per il nuovo amministratore da nominare & \o & UCS 9.1.1 Interno\\
R1FS10.8 & Viene mostrato un messaggio d'errore qualora l'e-mail del nuovo amministratore da nominare risulti già registrata nel sistema & \o & UCS 10.9.1 Interno\\
R1FS10.9 & Viene mostrato un messaggio d'errore qualora la password del nuovo amministratore da nominare risulti troppo debole & \o & UCS 10.9.2 Interno\\
R1FS10.10 & Viene mostrato un messaggio d'errore qualora la password non combaci con la conferma password del nuovo amministratore da nominare & \o & UCS 10.9.3 Interno\\
R1FS9.4 & L'amministratore proprietario deve poter inserire una nuova password per il nuovo amministratore da nominare & \o & UCS 9.1.2 Interno\\
R1FS9.5 & L'amministratore proprietario deve poter inserire la conferma della nuova password per il nuovo amministratore da nominare & \o & UCS 9.1.3 Interno\\
R1FS9.6 & L'amministratore proprietario deve poter selezionare i privilegi per il nuovo amministratore da nominare & \o & UCS 9.1.4 Interno\\
R1FS9.7 & L'amministratore proprietario ha la possibilità di eliminare un amministratore & \o & UCS 9.2 Capitolato\\
R1FS9.8 & L'amministratore proprietario ha la possibilità di inserire la e-mail dell'account amministratore da eliminare & \o & UCS 9.2.1 Interno\\
R1FS10.11 & Viene mostrato un messaggio d'errore qualora l'e-mail dell'amministratore da eliminare inserita non risulti registrata nel sistema & \o & UCS 10.9.4 Interno\\
R1FS9.9 & L'amministratore proprietario ha la possibilità di modificare i privilegi di un amministratore & \o & UCS 9.3 Interno\\
R1FS9.10 & L'amministratore proprietario ha la possibilità di inserire la e-mail dell'account amministratore a cui desidera modificare i privilegi & \o & UCS 9.3.1 Interno\\
R1FS10.12 & Viene mostrato un messaggio d'errore qualora l'e-mail dell'amministratore a cui si vuole modificare i privilegi non risulti registrata nel sistema & \o & UCS 10.9.4 Interno\\
R1FS9.11 & L'amministratore proprietario ha la possibilità annullare le modifiche che sta apportando agli amministratori & \o & UCS 9.4 Interno\\