% scritto da\AT{}
\section{Introduzione}
\subsection{Scopo del documento}
Il presente documento ha lo scopo di descrivere le strategie che il gruppo \Gruppo{} intende applicare per garantire la qualità di processo e di prodotto per l’intera durata del progetto.
Al fine di rispettare questi obiettivi vengono descritte le modalità in cui vengono effettuate la verifica e la validazione del prodotto.
In questo modo è consentita la rilevazione e correzione di problemi o incongruenze in breve tempo, senza correre il rischio di sprechi di risorse.

\subsection{Scopo generale del prodotto}
L'obiettivo del prodotto \NomeProgetto{} di \Proponente{} è la creazione di un sistema software composto di un applicativo per cellulare e di un server, con cui interagire tramite un'interfaccia utente. La necessità nasce dal bisogno di adempiere alle normative vigenti in tema di sicurezza.
Le due componenti del sistema software, applicativo e server, devono soddisfare i seguenti obiettivi rispettivamente di:
\begin{itemize}
\item Tracciare e registrare i \glo{movimenti} di un utente in un \glo{luogo di tracciamento} di un'\glo{organizzazione}, siano essi autenticati da credenziali di un'\glo{organizzazione} oppure visitatori anonimi, il tutto nel rispetto della normativa sulla privacy;
\item Poter visionare gli accessi degli utenti autenticati e visionare il numero di visitatori anonimi all'interno di un luogo.
\end{itemize}

\subsection{Glossario}
Al fine di evitare ambiguità fra i termini, e per avere chiare fra tutti gli stakeholder le terminologie utilizzate per la realizzazione del presente documento, il gruppo \Gruppo{} ha redatto un documento denominato \Glossariov{2.0.0}.
In tale documento, sono presenti tutti i termini tecnici, ambigui, specifici del progetto e scelti dai membri del gruppo con le loro relative definizioni.
Un termine presente nel \Glossariov{2.0.0} e utilizzato in questo documento viene indicato con un apice \ap{G} alla fine della parola.

\subsection{Standard di progetto}
Il gruppo \Gruppo{} ha deciso di gestire i propri processi del ciclo di vita del software istanziando alcune parti dello standard \textbf{ISO/IEC 12207} come definito nel documento \NdP{}, dove vengono descritte le parti selezionate in quanto ritenute importanti, da parte del gruppo, per il progetto; 
come modello di qualità del prodotto software si è istanziato secondo le necessità parte dello standard \textbf{ISO/IEC 9126}, anch'esso definito nelle \NdP{}.

\subsection{Riferimenti}

\subsubsection{Normativi}
\begin{itemize}
    \item \textbf{Capitolato d'appalto C5 - Stalker}\\     
    \url{https://www.math.unipd.it/~tullio/IS-1/2019/Progetto/C5.pdf};
    \item \NdPv{2.0.0}.
\end{itemize}

\subsubsection{Informativi}
\begin{itemize}
    \item \textbf{Indice di Gulpease}\\
    \url{https://it.wikipedia.org/wiki/Indice_Gulpease};
    \item \textbf{Metriche di progetto}\\
    \url{https://it.wikipedia.org/wiki/Metriche_di_progetto};
    \item \textbf{Varie metriche}\\
    \url{http://torlone.dia.uniroma3.it/sistelab/annipassati/sbavaglia.pdf};
    \item \textbf{Ciclo di Deming - Plan Do Check Act}\\
    \url{https://it.wikipedia.org/wiki/Ciclo_di_Deming};
    \item \textbf{ISO/IEC 12207-1995:}\\     
    \url{https://www.math.unipd.it/~tullio/IS-1/2009/Approfondimenti/ISO_12207-1995.pdf};\\
    \url{http://www.colonese.it/SviluppoSw_Standard_ISO12207.html};
    \item \textbf{ISO/IEC 9126:}\\
    \url{http://www.colonese.it/00-Manuali_Pubblicatii/07-ISO-IEC9126_v2.pdf};\\
    \url{https://en.wikipedia.org/wiki/ISO/IEC_9126};
\end{itemize}