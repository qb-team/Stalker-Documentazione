% scritto da Tommaso Azzalin
\section{Introduzione}
\subsection{Scopo del documento}
Il presente documento ha lo scopo di descrivere le strategie che il gruppo \Gruppo{} intende applicare per garantire la qualità di processo e di prodotto per l’intera durata del progetto.
Al fine di rispettare questi obiettivi vengono descritte le modalità in cui vengono effettuate la verifica e la validazione del prodotto.
In questo modo è consentita la rilevazione e correzione di problemi o incongruenze in breve tempo, senza correre il rischio di sprechi di risorse.

\subsection{Scopo del Prodotto}
L'obbiettivo nel progetto \textit{Stalker} è la creazione di un applicativo per cellulare e di un server con interfaccia web che adempiano alle funzioni, rispettivamente, di 
tracciare e registrare la posizione in tempo reale dei possessori dell'applicativo, siano essi autenticati da credenziali aziendali oppure anonimi visitatori.
Questi dati raccolti devono poter essere visionati dagli amministratori delle stesse aziende e al contempo deve essere garantita la privacy degli utenti al di fuori del perimetro dell'organizzazione\ap{G} dalla quale vogliono farsi tracciare.

\subsection{Glossario}
Al fine di evitare ambiguità fra i termini, e per avere chiare fra tutti gli stakeholder le terminologie utilizzate per la realizzazione del presente documento, il gruppo \Gruppo{} ha redatto un documento denominato “Glossario”.
In tale documento, sono presenti tutti i termini tecnici, ambigui, specifici del progetto e scelti dai membri del gruppo con le loro relative definizioni.
Un termine presente nel “Glossario” e utilizzato in questo documento viene indicato con un'apice \ap{G} alla fine della parola.

\subsection{Standard di progetto}
Il gruppo qbteam ha deciso di gestire i propri processi del ciclo di vita del software così come definito dallo standard \textbf{ISO/IEC 12207}; 
mentre ha deciso di applicare come modello di qualità del prodotto software lo standard \textbf{ISO/IEC 9126}.