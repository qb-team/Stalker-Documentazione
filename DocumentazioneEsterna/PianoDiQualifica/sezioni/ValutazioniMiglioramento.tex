\clearpage
\section{Valutazioni per il miglioramento}
In questa sezione si vogliono elencare i problemi che sono stati riscontrati dal gruppo nel corso del progetto e di inserire i miglioramenti che sono stati fatti. Vengono perciò indicati i problemi sorti e la loro relativa soluzione inoltre viene indicato cosa e come è stato migliorato.
I problemi e i miglioramenti tracciati riguarderanno i seguenti ambiti:
\begin{itemize}
\item Organizzazione;
\item Ruoli;
\item Strumenti.
\end{itemize}
Per ogni problema e per ogni miglioramento verrà indicato il periodo di registrazione.
\subsection{Processo di miglioramento continuo}
Applicare il ciclo di processo di miglioramento continuo permette di ottenere dei sensibili miglioramenti sia in termini di efficienza e sia in termini di efficacia. Questo perché si assume di avere un approccio proattivo, cioè si cerca di attuare dei miglioramenti che cerchino di migliorare il \textsl{way of working}, facendolo prima che si manifestino delle problematiche. Purtroppo le attività di miglioramento che sono state applicate dal gruppo per buona parte della vita del progetto, sono state di tipo reattivo, cioè si applicavano delle correzione solo quando si riscontravano delle problematiche o degli errori, e quindi risultavano certamente meno costose, ma molto meno efficienti e efficaci. Questo approccio reattivo adottato del gruppo, era figlio dell'inesperienza perché non si sapeva come applicare il processo di miglioramento continuo. Al fine di poter applicare il ciclo di PDCA, devono essere stabili dei obiettivi da raggiungere nel effettuare un azione di miglioramento, questo perché permette di misurare l'attività di miglioramento fatta, e valutare se tale attività ha portato dei miglioramenti o no.
\subsection{Valutazioni sull'organizzazione}
{
	\rowcolors{2}{grigetto}{white}
	\renewcommand{\arraystretch}{1.5}
	\centering
	\begin{longtable}{ C{3cm} C{4cm} C{3cm} C{4cm}}
		\caption{Elenco dei cambiamenti effettuati}\\
		\rowcolor{darkblue}
		\textcolor{white}{\textbf{Periodo}} & \textcolor{white}{\textbf{Problema/Da migliorare}} & \textcolor{white}{\textbf{Obiettivo di miglioramento}} & \textcolor{white}{\textbf{Cambiamento nel way of working}}\\
		\endfirsthead
		\rowcolor{darkblue}
		\textcolor{white}{\textbf{Periodo}} & \textcolor{white}{\textbf{Problema/Da migliorare}} &
		\textcolor{white}{\textbf{Obiettivo di miglioramento}} & \textcolor{white}{\textbf{Cambiamento nel way of working}}\\
		\endhead
		Periodo 1 della Progettazione Architetturale & A causa della sessione di esami vi è stato un lasso di tempo durante il quale non è stato possibile portare avanti il progetto. Come diretta conseguenza, si è dovuto svolgere i compiti in un ridotto lasso temporale. & - & Essere coscienti sin dall'inizio dei periodi nei quali il gruppo risulta mediamente più occupato in modo da distribuire il carico di lavoro di conseguenza. \\
		
		Periodo 2 della Progettazione Architetturale & Da fine febbraio a causa del COVID-19 non è più possibile effettuare incontri fisici ma solo da remoto. & - & Bisogna adottarsi di norme che regolino la comunicazione da remoto e i strumenti per la comunicazione da remoto adottati. \\
		
		Periodo 1 della Validazione e Collaudo & Implementazione delle corrette funzionalità che il prodotto deve avere. & Si vuole evitare di implementare funzionalità sbagliate e quindi evitare di sprecare tempo. & Si è aumentato il numero dei colloqui e quindi l'interazione con il proponente, al fine di capire se ciò che si è sviluppo era adatto alle richieste di quest'ultimo. \\
		
		Periodo 2 della Validazione e Collaudo & Attività di verifica del corretto funzionamento del prodotto. & Si vuole implementare un attività di verifica che non faccia sprecare tempo in doppie verifiche della stessa funzionalità implementata in parti del prodotto diverse. & Durante la verifica del corretto funzionamento ogni funzionalità dovrà essere analizzata da almeno un componente che si è occupato della web-app, un componente che si è occupato del backend e infine un componente che si è occupato della app Android, al fine di effettuare un'unica verifica sulla funzionalità sull'intero prodotto evitando di ripetere singole verifiche su singoli componenti del prodotto già fatte in precedenza.
		
	\end{longtable}
}

\subsection{Valutazione dei ruoli}

{
	\rowcolors{2}{grigetto}{white}
	\renewcommand{\arraystretch}{1.5}
	\centering
	\begin{longtable}{ C{3cm} C{4cm} C{3cm} C{4cm}}
		\caption{Elenco dei cambiamenti effettuati}\\
		\rowcolor{darkblue}
		\textcolor{white}{\textbf{Periodo}} & \textcolor{white}{\textbf{Problema/Da migliorare}} & \textcolor{white}{\textbf{Obiettivo di miglioramento}} & \textcolor{white}{\textbf{Cambiamento nel way of working}}\\
		\endfirsthead
		\rowcolor{darkblue}
		\textcolor{white}{\textbf{Periodo}} & \textcolor{white}{\textbf{Problema/Da migliorare}} &
		\textcolor{white}{\textbf{Obiettivo di miglioramento}} & \textcolor{white}{\textbf{Cambiamento nel way of working}}\\
		\endhead
		
		Periodo 2 della Progettazione Architetturale & Per il \glo{Proof of Concept} sono state previste un quantità di ore di progettista molto elevata rispetto a quella necessaria, infatti la maggior parte delle ore sono state impiegate per la codifica. & - & Le ore di progettista non svolte in questa fase saranno distribuite nella prossima, riducendo quelle di codifica.\\
		
		Periodo 1 della Progettazione di Dettaglio e Codifica & Durante la correzione del \PdP{} il responsabile ha avuto delle problematiche nella risoluzione dei problemi legati al versionamento errato. & - & È stato richiesto un incontro remoto con il \VT{} per risolvere i dubbi sorti.\\
		
		Periodo 2 della Progettazione di Dettaglio e Codifica & Durante la progettazione si sono avuti dei problemi nell'utilizzo di design pattern. & - & È stato richiesto un incontro remoto con il \CR{} per risolvere i dubbi sorti da parte dei progettisti.\\
		
		Periodo 1 della Progettazione di Dettaglio e Codifica & Durante la progettazione delle componenti ci si è accorti che alcune scelte implementative fatte per il \glo{Proof of Concept} non erano le soluzioni migliori. & - & Si sono riprogettare le varie componenti da migliorare mentre, ciò che era stato implementato è stato riadattato secondo le specifiche di progettazione. \\
		
		Periodo 1 della Progettazione di Dettaglio e Codifica & L'amministratore ha segnalato la difficoltà a calcolare alcune metriche perché non si sono trovati strumenti adatti per calcolarle automaticamente. & - & Sono stata tolte tali metriche e sostituite con altre dove è possibile calcolarle automaticamente attraverso strumenti automatici. \\
		
		Periodo 1 della Validazione e Collaudo & Pianificazione delle ore per ruolo. & Togliere le ore pianificati nei ruoli dove si prevede un sotto utilizzo e aggiungerle in ruoli dove c'è molta richiesta di ore. & Si prevede che le ore pianificate per il ruolo di analista, amministratore e responsabile di progetto saranno sotto utilizzate per ciò saranno tolte delle ore e date a i ruoli di progettista, programmatore e verificatore che certamente saranno usate.
		
		
	\end{longtable}
}

\subsection{Valutazione sugli strumenti}

{
	\rowcolors{2}{grigetto}{white}
	\renewcommand{\arraystretch}{1.5}
	\centering
	\begin{longtable}{ C{3cm} C{4cm} C{3cm} C{4cm}}
		\caption{Elenco dei cambiamenti effettuati}\\
		\rowcolor{darkblue}
		\textcolor{white}{\textbf{Periodo}} & \textcolor{white}{\textbf{Problema/Da migliorare}} & \textcolor{white}{\textbf{Obiettivo di miglioramento}} & \textcolor{white}{\textbf{Cambiamento nel way of working}}\\
		\endfirsthead
		\rowcolor{darkblue}
		\textcolor{white}{\textbf{Periodo}} & \textcolor{white}{\textbf{Problema/Da migliorare}} &
		\textcolor{white}{\textbf{Obiettivo di miglioramento}} & \textcolor{white}{\textbf{Cambiamento nel way of working}}\\
		\endhead
	
	Periodo 2 della progettazione architetturale & Per il calcolo delle metriche è stato utilizzato \glo{SonarQube}, il quale è risultato essere di difficile comprensione a causa della scarsa documentazione. Come conseguenza, il calcolo di alcune metriche è stato arduo. & - &  Studi più approfonditi dello strumento (\glo{SonarQube}).\\
	
	Periodo 2 della progettazione architetturale & Durante l'implementazione dell'applicazione mobile i programmatori hanno avuto problemi con la compatibilità delle librerie con \glo{Grandle}. & - & I programmatori si sono temporaneamente fermati per risolvere insieme il problema, si è dovuto perciò leggere la documentazione dedicata disponibile su Internet e grazie a ciò si è trovato una soluzione.\\
	
	Periodo 1 della progettazione di dettaglio e codifica & Per lo sviluppo di \glo{unit test} per la applicazione web sono sorti delle difficoltà nel utilizzo di \glo{Jasmine} come mezzo per eseguire i test. & - & Come soluzione i programmatori si sono temporaneamente fermati per risolvere insieme il problema, si è dovuto perciò leggere la documentazione dedicata disponibile su Internet e grazie a ciò si è trovato una soluzione.\\
	
	Periodo 1 della progettazione di dettaglio e codifica& Lo strumento per la creazione di diagrammi \glo{Draw.io} risulta essere troppo limitante in termini di di tempo, infatti viene speso troppo tempo per poter produrre diagrammi di classe e di sequenza. & - & Si è deciso di abbandonare lo strumento per la creazione di diagrammi \glo{Draw.io} a favore dello strumento \glo{StarUML} ottenendo un risparmio di tempo significativo e avvolte in alcune casi un risultato visivo migliore.\\
	
	Periodo 2 della progettazione di dettaglio e codifica & Scrittura dei nuovi documenti \MM{} e \MU{}. & Sostituire \LaTeX{} con l 'utilizzo di uno strumento più efficiente per la scrittura perché risulta essere troppo dispendioso in termini di tempo speso, e poco attrattivo. & Come soluzione migliorativa è stato scelto di adottare lo strumento \glo{MkDocs} per la scrittura dei documenti \MM{} e \MU{}.\\
	
	
\end{longtable}
}






