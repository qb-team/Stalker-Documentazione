{
\rowcolors{2}{grigetto}{white}
\renewcommand{\arraystretch}{1.5}
\centering
\begin{longtable}{ c C{13cm} C{1cm}}
\caption{Elenco dei test di integrazione}\\
\rowcolor{darkblue}
\textcolor{white}{\textbf{Codice}} & \textcolor{white}{\textbf{Descrizione}} & \textcolor{white}{\textbf{Stato}}\\
\endfirsthead
\rowcolor{darkblue}
\textcolor{white}{\textbf{Codice}} & \textcolor{white}{\textbf{Descrizione}} & \textcolor{white}{\textbf{Stato}}\\
\endhead

TIA1 & Si verifichi che l’integrazione tra l’applicazione Android e servizi di \glo{Firebase} permetta l’autenticazione ad un utente. & S \\
TIA2 & Si verifichi che l’integrazione tra l’applicazione Android e servizi di \glo{Firebase} permetta la registrazione di un utente. & S \\
TIA3 & Si verifichi che l’integrazione tra l’applicazione Android e servizi di \glo{Firebase} permetta di effettuare il logout dell'utente. & S \\
TIA4 & Si verifichi che l’integrazione tra l’applicazione Android e il server LDAP permetta l’autenticazione. & S \\
TIA5 & Si verifichi che l’integrazione tra l’applicazione Android e i servizi di localizzazione permettano il corretto tracciamento. & S \\
TIA6 & Si verifichi che l’integrazione tra i servizi REST-API e \glo{backend} permetta la comunicazione fra loro. & S \\
TIA7 & Si verifichi che l’integrazione tra l’applicazione Android e REST-API permetta di scaricare la lista delle organizzazioni dal \glo{backend}. & S \\
TIA8 & Si verifichi che l’integrazione tra l’applicazione Android e REST-API permetta di scaricare la lista delle organizzazioni preferite dal \glo{backend}. & S \\
TIA9 & Si verifichi che l’integrazione tra l’applicazione Android e REST-API permetta di aggiungere una organizzazione tra le organizzazioni preferite nel \glo{backend}. & S \\
TIA10 & Si verifichi che l’integrazione tra l’applicazione Android e REST-API permetta di rimuovere una organizzazione tra le organizzazioni preferite nel \glo{backend}. & S \\
TIA11 & Si verifichi che l’integrazione tra l’applicazione Android e REST-API permetta di aggiungere una organizzazione con credenziali LDAP tra le organizzazioni preferite nel \glo{backend}. & S \\
TIA12 & Si verifichi che l’integrazione tra l’applicazione Android e REST-API permetta di rimuovere una organizzazione con credenziali LDAP tra le organizzazioni preferite nel \glo{backend}. & S \\
TIA13 & Si verifichi che l’integrazione tra l’applicazione Android e REST-API permetta di avvisare il \glo{backend} qualora l'utente sia entrato in un'organizzazione preferita ed in risposta ricevere l'exit token. & S \\
TIA14 & Si verifichi che l’integrazione tra l’applicazione Android e REST-API permetta di avvisare il \glo{backend} qualora l'utente sia uscito da un'organizzazione preferita e di riconsegnare l'exit token. & S \\
TIA15 & Si verifichi che l’integrazione tra l’applicazione Android e REST-API permetta di avvisare il \glo{backend} qualora l'utente sia entrato in un luogo di un'organizzazione preferita ed in risposta ricevere l'exit token. & S \\
TIA16 & Si verifichi che l’integrazione tra l’applicazione Android e REST-API permetta di avvisare il \glo{backend} qualora l'utente sia uscito da un luogo di un'organizzazione preferita e di riconsegnare l'exit token. & S \\
TIA17 & Si verifichi che l’integrazione tra l’applicazione Android e REST-API permetta di avvisare il \glo{backend} qualora l'utente sia entrato in un'organizzazione preferita inviando il codice di autenticazione LDAP ed in risposta ricevere l'exit token. & S \\
TIA18 & Si verifichi che l’integrazione tra l’applicazione Android e REST-API permetta di avvisare il \glo{backend} qualora l'utente sia uscito da un'organizzazione preferita inviando il codice di autenticazione LDAP e di riconsegnare l'exit token. & S \\
TIA19 & Si verifichi che l’integrazione tra l’applicazione Android e REST-API permetta di avvisare il \glo{backend} qualora l'utente sia entrato in un luogo di un'organizzazione preferita inviando il codice di autenticazione LDAP ed in risposta ricevere l'exit token. & S \\
TIA20 & Si verifichi che l’integrazione tra l’applicazione Android e REST-API permetta di avvisare il \glo{backend} qualora l'utente sia uscito da un luogo di un'organizzazione preferita inviando il codice di autenticazione LDAP e di riconsegnare l'exit token. & S \\
TIB1 & Verificare che una richiesta (qualsiasi) al backend senza token di accesso ottenuto da Firebase restituisca codice di stato HTTP 401. & S \\
TIB2 & Verificare che una richiesta (qualsiasi) al backend con token di accesso non valido o scaduto ottenuto da Firebase restituisca codice di stato HTTP 401. & S \\
TIB3 & Verificare che una richiesta (qualsiasi) al backend con token di accesso valido di un utente dell'app (non ha permessi di amministrazione $\leftrightarrow$ è un utente dell'app) ad API per gli amministratori restituisca codice di stato HTTP 403. & S \\
TIB4 & Verificare che una richiesta (qualsiasi) al backend con token di accesso valido di un amministratore della web-app (ha permessi di amministrazione $\rightarrow$ è un amministratore della web-app) ad API per gli utenti restituisca codice di stato HTTP 403. & S \\
TIB5 & Verificare che una richiesta (qualsiasi) che richieda come parametro l'identificativo di un'organizzazione che però non esiste restituisca codice di stato HTTP 404. & S \\
TIB6 & Verificare che una richiesta (qualsiasi) che richieda come parametro l'identificativo di un luogo che però non esiste restituisca codice di stato HTTP 404. & S \\
TIB7 & Verificare che una richiesta di associazione di un'amministratore (già esistente) presso un'organizzazione il cui identificativo o identificativo dell'amministratore non esistano restituisca codice di stato HTTP 404. & S \\
TIB8 & Verificare che una richiesta di associazione di un'amministratore (inesistente) presso un'organizzazione il cui identificativo non esiste restituisca codice di stato HTTP 404. & S \\
TIB9 & Verificare che una richiesta di dissociazione di un'amministratore che non ha un permesso presso l'organizzazione restituisca codice di stato HTTP 404. & S \\
TIB10 & Verificare che una richiesta di modifica permessi di un'amministratore (già esistente) presso un'organizzazione il cui identificativo o identificativo dell'amministratore non esistano restituisca codice di stato HTTP 404. & S \\
TIB11 & Verificare che una richiesta di aggiunta ai preferiti di un'organizzazione il cui identificativo non esiste restituisca codice di stato HTTP 404. & S \\
TIB12 & Verificare che una richiesta di rimozione dai preferiti di un preferito dalla lista dei preferiti inesistente restituisca codice di stato HTTP 404. & S \\
TIB13 & Verificare che una richiesta di tracciamento di un movimento presso un'organizzazione il cui identificativo non esiste restituisca codice di stato HTTP 404. & S \\
TIB14 & Verificare che una richiesta di tracciamento di un movimento presso un luogo il cui identificativo non esiste restituisca codice di stato HTTP 404. & S \\
TIB15 & Verificare che una richiesta di aggiornamento di un'organizzazione il cui identificativo non esiste restituisca codice di stato HTTP 404. & S \\
TIB16 & Verificare che una richiesta di creazione di un luogo il cui identificativo dell'organizzazione non esiste restituisca codice di stato HTTP 404. & S \\
TIB17 & Verificare che una richiesta di aggiornamento di un luogo il cui identificativo non esiste restituisca codice di stato HTTP 404. & S \\
TIB18 & Verificare che una richiesta di creazione o di aggiornamento dei permessi di un amministratore non sia concessa se i permessi da inserire sono maggiori di quelli dell'amministratore che li sta impostando restituendo codice di stato HTTP 400. & S \\
TIB19 & Verificare che una richiesta di creazione di un profilo per un amministratore della web-app con una mail già associata ad un altro profilo non sia concessa restituendo codice di stato HTTP 400. & S \\
TIB20 & Verificare che una richiesta di aggiornamento di un permesso per un amministratore comprendente altro oltre al livello di permesso restituisca codice di stato HTTP 400. & S \\
TIB21 & Verificare che una richiesta di creazione di un preferito per un utente dell'app che è già esistente restituisca codice di stato HTTP 400. & S \\
TIB22 & Verificare che una richiesta della lista di preferiti con un identificativo dell'utente diverso da quello del proprietario del token restituisca codice di stato HTTP 400. & S \\
TIB23 & Verificare che una richiesta di tracciamento di un movimento di uscita da un'organizzazione o luogo senza exitToken restituisca codice di stato HTTP 400. & S \\
TIB24 & Verificare che una richiesta di aggiornamento di un'organizzazione, dell'area di tracciamento di un'organizzazione o di un luogo restituisca codice di stato HTTP 400. & S \\
TIB25 & Verificare che una richiesta di accessi anonimi presso un'organizzazione dati degli exitToken vengano restituiti con successo assieme ad un codice di stato HTTP 200. & S \\
TIB26 & Verificare che una richiesta di accessi anonimi presso un luogo dati degli exitToken vengano restituiti con successo assieme ad un codice di stato HTTP 200. & S \\
TIB27 & Verificare che una richiesta di accessi autenticati presso un'organizzazione dati degli identificativi LDAP vengano restituiti con successo assieme ad un codice di stato HTTP 200. & S \\
TIB28 & Verificare che una richiesta di accessi autenticati presso un'organizzazione dati degli identificativi LDAP vengano restituiti con successo assieme ad un codice di stato HTTP 200. & S \\
TIB29 & Verificare che una richiesta della lista di amministratori presso un'organizzazione la restituisca assieme ad un codice di stato HTTP 200. & S \\
TIB30 & Verificare che una richiesta della lista di permessi presso un'organizzazione la restituisca assieme ad un codice di stato HTTP 200. & S \\
TIB31 & Verificare che una richiesta della lista dei preferiti di un utente dato l'identificativo dell'utente venga restituita con successo assieme ad un codice di stato HTTP 200. & S \\
TIB32 & Verificare che una richiesta di un'organizzazione dato il suo identificativo venga restituita con successo assieme ad un codice di stato HTTP 200. & S \\
TIB33 & Verificare che una richiesta della lista delle organizzazioni (non vuota) venga restituita con successo assieme ad un codice di stato HTTP 200. & S \\
TIB34 & Verificare che una richiesta di aggiornamento dell'organizzazione aggiorni l'organizzazione e la restituisca aggiornata con successo assieme ad un codice di stato HTTP 200. & S \\
TIB35 & Verificare che una richiesta di aggiornamento dell'area di tracciamento dell'organizzazione la aggiorni e la restituisca aggiornata con successo assieme ad un codice di stato HTTP 200. & S \\
TIB36 & Verificare che una richiesta della lista dei luoghi (non vuota) venga restituita con successo assieme ad un codice di stato HTTP 200. & S \\
TIB37 & Verificare che una richiesta di aggiornamento del luogo aggiorni l'organizzazione e lo restituisca aggiornato con successo assieme ad un codice di stato HTTP 200. & S \\
TIB38 & Verificare che una richiesta del contatore di presenze presso un'organizzazione venga restituito con successo assieme ad un codice di stato HTTP 200. & S \\
TIB39 & Verificare che una richiesta del contatore di presenze presso un luogo venga restituito con successo assieme ad un codice di stato HTTP 200. & S \\
TIB40 & Verificare che una richiesta del report di ore spese dagli utenti presso un luogo venga restituito assieme ad un codice di stato HTTP 200. & S \\
TIB41 & Verificare che una richiesta di aggiunta di un permesso di un amministratore presso un'organizzazione vada a buon fine restituendo il record di Permission assieme ad un codice di stato HTTP 201. & S \\
TIB42 & Verificare che una richiesta di creazione e aggiunta di un permesso di un amministratore presso un'organizzazione vada a buon fine restituendo il record di Permission assieme ad un codice di stato HTTP 201. & S \\
TIB43 & Verificare che una richiesta di aggiornamento di un permesso di un amministratore presso un'organizzazione vada a buon fine restituendo il record di Permission assieme ad un codice di stato HTTP 201. & S \\
TIB44 & Verificare che una richiesta di aggiunta di un preferito alla lista di organizzazioni aggiunga il preferito e lo ritorni assieme ad un codice di stato HTTP 201. & S \\
TIB45 & Verificare che una richiesta di tracciamento di un movimento di ingresso in un'organizzazione vada a buon fine restituendo l'istanza del movimento con l'exitToken assieme ad un codice di stato HTTP 201. & S \\
TIB46 & Verificare che una richiesta di tracciamento di un movimento di ingresso in un luogo vada a buon fine restituendo l'istanza del movimento con l'exitToken assieme ad un codice di stato HTTP 201. & S \\
TIB47 & Verificare che una richiesta di inserimento di un luogo di un'organizzazione memorizzi il luogo e lo ritorni assieme ad un codice di stato HTTP 201. & S \\
TIB48 & Verificare che una richiesta di tracciamento di un movimento di uscita in un'organizzazione vada a buon fine restituendo un codice di stato HTTP 202. & S \\
TIB49 & Verificare che una richiesta di tracciamento di un movimento di uscita in un luogo vada a buon fine restituendo un codice di stato HTTP 202. & S \\
TIB50 & Verificare che una richiesta di accessi anonimi presso un'organizzazione dati degli exitToken, vuota, venga restituito un codice di stato HTTP 204. & S \\
TIB51 & Verificare che una richiesta di accessi anonimi presso un luogo dati degli exitToken, vuota, venga restituito un codice di stato HTTP 204. & S \\
TIB52 & Verificare che una richiesta di accessi autenticati presso un'organizzazione dati degli identificativi LDAP, vuota, venga restituito un codice di stato HTTP 204. & S \\
TIB53 & Verificare che una richiesta di accessi autenticati presso un'organizzazione dati degli identificativi LDAP, vuota, venga restituito un codice di stato HTTP 204. & S \\
TIB54 & Verificare che una richiesta della lista di permessi presso un'organizzazione, vuota, venga restituito un codice di stato HTTP 204. & S \\
TIB55 & Verificare che una richiesta della lista dei preferiti di un utente, vuota, dato l'identificativo dell'utente venga restituita con successo assieme ad un codice di stato HTTP 204. & S \\
TIB56 & Verificare che una richiesta della lista delle organizzazioni, vuota, venga restituita con successo assieme ad un codice di stato HTTP 204. & S \\
TIB57 & Verificare che una richiesta della lista dei luoghi, vuota, venga restituita con successo assieme ad un codice di stato HTTP 204. & S \\
TIB58 & Verificare che una richiesta di aggiornamento del luogo aggiorni l'organizzazione e lo restituisca aggiornato con successo assieme ad un codice di stato HTTP 200. & S \\
TIB59 & Verificare che una richiesta del report di ore spese dagli utenti presso un luogo, vuoto, venga restituito assieme ad un codice di stato HTTP 204. & S \\
TIB60 & Verificare che una richiesta di eliminazione di un'organizzazione restituisca codice di stato HTTP 204. & S \\
TIB61 & Verificare che una richiesta di dissociazione di un amministratore di un'organizzazione restituisca codice di stato HTTP 204. \\
TIB62 & Verificare che una richiesta di rimozione di un preferito lo rimuova e restituisca un codice di stato HTTP 204. & S \\
TIB63 & Verificare che una richiesta di rimozione di un luogo lo rimuova e restituisca un codice di stato HTTP 204. & S \\
TIW1 & Si verifichi l’integrazione tra l’applicazione web e servizi di \glo{Firebase} permetta l’autenticazione a un amministratore. & S \\
TIW2 & Si verifichi l’integrazione tra l’applicazione web e servizi di \glo{Firebase} permetta il reset della password a un amministratore. & S \\
TIW3 & Si verifichi l’integrazione tra applicazione web sviluppata in Angular 2+ e \glo{Firebase}.& S \\
TIW4 & Si verifichi l’integrazione tra l’applicazione web e il server LDAP permetta l’autenticazione.& S \\
TIW5 & Si verifichi l’integrazione tra REST-API e \glo{backend} permetta la comunicazione fra loro.& S \\
TIW6 & Si verifichi l’integrazione tra l’applicazione web e REST-API permetta di recuperare la lista dei permessi che ha l’amministratore autenticato.& S \\
TIW7 & Si verifichi l’integrazione tra l’applicazione web e REST-API permetta di recuperare la lista delle organizzazioni che ha l’amministratore autenticato.& S \\
TIW8 & Si verifichi l’integrazione tra l’applicazione web e REST-API permetta di recuperare la lista dei luoghi di un’organizzazione che ha l’amministratore autenticato.& S \\
TIW9 & Si verifichi l’integrazione tra l’applicazione web e REST-API permetta di monitorare gli accessi all’interno di un’organizzazione.& S \\
TIW10 & Si verifichi l’integrazione tra l’applicazione web e REST-API permetta di monitorare gli accessi all’interno di un luogo di un’organizzazione.& S \\
TIW11 & Si verifichi l’integrazione tra l’applicazione web e REST-API permetta di visualizzare i dati relativi a un’organizzazione ricevuti dal \glo{backend}.& S \\
TIW12 & Si verifichi l’integrazione tra l’applicazione web e REST-API permetta di visualizzare i dati relativi a un luogo di un’organizzazione ricevuti dal \glo{backend}.& S \\
TIW13 & Si verifichi l’integrazione tra l’applicazione web e REST-API permetta di modificare i dati relativi a un’organizzazione e inviare le modifiche al \glo{backend}.& S \\
TIW14 & Si verifichi l’integrazione tra l’applicazione web e REST-API permetta di modificare i dati relativi a un luogo di un’organizzazione e inviare le modifiche al \glo{backend}.& S \\
TIW15 & Si verifichi l’integrazione tra l’applicazione web e REST-API permetta di creare una nuova organizzazione.& S \\
TIW16 & Si verifichi l’integrazione tra l’applicazione web e REST-API permetta di creare un nuovo luogo di un’organizzazione.& S \\
TIW17 & Si verifichi l’integrazione tra l’applicazione web e REST-API permetta di creare un nuovo amministratore dell’organizzazione.& S \\
TIW18 & Si verifichi l’integrazione tra l’applicazione web e REST-API permetta di inviare al \glo{backend} una richiesta di eliminazione di un’organizzazione.& S \\
TIW19 & Si verifichi l’integrazione tra l’applicazione web e REST-API permetta di eliminare un luogo dell’organizzazione.& S \\
TIW20 & Si verifichi l’integrazione tra l’applicazione web e REST-API permetta di modificare i permessi di un amministratore dell’organizzazione e salvare le modifiche nel \glo{backend}.& S \\
TIW21 & Si verifichi l’integrazione tra l’applicazione web e REST-API permetta di eliminare un amministratore dell’organizzazione.& S \\




\end{longtable}
}