{
\rowcolors{2}{grigetto}{white}
\renewcommand{\arraystretch}{1.5}
\centering
\begin{longtable}{ c C{8.5cm} C{1cm}}
\caption{Elenco dei test di unità}\\
\rowcolor{darkblue}
\textcolor{white}{\textbf{Codice}} & \textcolor{white}{\textbf{Descrizione}} & \textcolor{white}{\textbf{Stato}}\\
\endfirsthead
\rowcolor{darkblue}
\textcolor{white}{\textbf{Codice}} & \textcolor{white}{\textbf{Descrizione}} & \textcolor{white}{\textbf{Stato}}\\
\endhead

TI & Si verifichi l’integrazione tra l’applicazione web e servizi di Firebase permetta l’autenticazione a un amministratore. & NI \\
TI & Si verifichi l’integrazione tra l’applicazione web e servizi di Firebase permetta il reset della password a un amministratore. & NI \\
TI & Si verifichi l’integrazione tra applicazione web sviluppata in Angular 2+ e Firebase.& NI \\
TI & Si verifichi l’integrazione tra l’applicazione web e il server LDAP permetta l’autenticazione.& NI \\
TI & Si verifichi l’integrazione tra REST-API e backend permetta la comunicazione fra loro.& NI \\
TI & Si verifichi l’integrazione tra l’applicazione web e REST-API permetta di recuperare la lista dei permessi che ha l’amministratore autenticato.& NI \\
TI & Si verifichi l’integrazione tra l’applicazione web e REST-API permetta di recuperare la lista delle organizzazioni che ha l’amministratore autenticato.& NI \\
TI & Si verifichi l’integrazione tra l’applicazione web e REST-API permetta di recuperare la lista dei luoghi di un’organizzazione che ha l’amministratore autenticato.& NI \\
TI & Si verifichi l’integrazione tra l’applicazione web e REST-API permetta di monitorare gli accessi all’interno di un’organizzazione.& NI \\
TI & Si verifichi l’integrazione tra l’applicazione web e REST-API permetta di monitorare gli accessi all’interno di un luogo di un’organizzazione.& NI \\
TI & Si verifichi l’integrazione tra l’applicazione web e REST-API permetta di visualizzare i dati relativi a un’organizzazione ricevuti dal backend.& NI \\
TI & Si verifichi l’integrazione tra l’applicazione web e REST-API permetta di visualizzare i dati relativi a un luogo di un’organizzazione ricevuti dal backend.& NI \\
TI & Si verifichi l’integrazione tra l’applicazione web e REST-API permetta di modificare i dati relativi a un’organizzazione e inviare le modifiche al backend.& NI \\
TI & Si verifichi l’integrazione tra l’applicazione web e REST-API permetta di modificare i dati relativi a un luogo di un’organizzazione e inviare le modifiche al backend.& NI \\
TI & Si verifichi l’integrazione tra l’applicazione web e REST-API permetta di creare una nuova organizzazione.& NI \\
TI & Si verifichi l’integrazione tra l’applicazione web e REST-API permetta di creare un nuovo luogo di un’organizzazione.& NI \\
TI & Si verifichi l’integrazione tra l’applicazione web e REST-API permetta di creare un nuovo amministratore dell’organizzazione.& NI \\
TI & Si verifichi l’integrazione tra l’applicazione web e REST-API permetta di inviare al backend una richiesta di eliminazione di un’organizzazione.& NI \\
TI & Si verifichi l’integrazione tra l’applicazione web e REST-API permetta di eliminare un luogo dell’organizzazione.& NI \\
TI & Si verifichi l’integrazione tra l’applicazione web e REST-API permetta di modificare i permessi di un amministratore dell’organizzazione e salvare le modifiche nel backend.& NI \\
TI & Si verifichi l’integrazione tra l’applicazione web e REST-API permetta di eliminare un amministratore dell’organizzazione.& NI \\




\end{longtable}
}