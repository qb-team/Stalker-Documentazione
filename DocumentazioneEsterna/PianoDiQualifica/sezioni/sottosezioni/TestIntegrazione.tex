{
\rowcolors{2}{grigetto}{white}
\renewcommand{\arraystretch}{1.5}
\centering
\begin{longtable}{ c C{13cm} C{1cm}}
\caption{Elenco dei test di unità}\\
\rowcolor{darkblue}
\textcolor{white}{\textbf{Codice}} & \textcolor{white}{\textbf{Descrizione}} & \textcolor{white}{\textbf{Stato}}\\
\endfirsthead
\rowcolor{darkblue}
\textcolor{white}{\textbf{Codice}} & \textcolor{white}{\textbf{Descrizione}} & \textcolor{white}{\textbf{Stato}}\\
\endhead

TIA1 & Si verifichi che l’integrazione tra l’applicazione Android e servizi di \glo{Firebase} permetta l’autenticazione ad un utente. & NI \\
TIA2 & Si verifichi che l’integrazione tra l’applicazione Android e servizi di \glo{Firebase} permetta la registrazione di un utente. & NI \\
TIA3 & Si verifichi che l’integrazione tra l’applicazione Android e servizi di \glo{Firebase} permetta di effettuare il logout dell'utente. & NI \\
TIA4 & Si verifichi che l’integrazione tra l’applicazione Android e il server LDAP permetta l’autenticazione. & NI \\
TIA5 & Si verifichi che l’integrazione tra l’applicazione Android e i servizi di localizzazione permettano il corretto tracciamento. & NI \\
TIA6 & Si verifichi che l’integrazione tra i servizi REST-API e \glo{backend} permetta la comunicazione fra loro. & NI \\
TIA7 & Si verifichi che l’integrazione tra l’applicazione Android e REST-API permetta di scaricare la lista delle organizzazioni dal \glo{backend}. & NI \\
TIA8 & Si verifichi che l’integrazione tra l’applicazione Android e REST-API permetta di scaricare la lista delle organizzazioni preferite dal \glo{backend}. & NI \\
TIA9 & Si verifichi che l’integrazione tra l’applicazione Android e REST-API permetta di aggiungere una organizzazione tra le organizzazioni preferite nel \glo{backend}. & NI \\
TIA10 & Si verifichi che l’integrazione tra l’applicazione Android e REST-API permetta di rimuovere una organizzazione tra le organizzazioni preferite nel \glo{backend}. & NI \\
TIA11 & Si verifichi che l’integrazione tra l’applicazione Android e REST-API permetta di aggiungere una organizzazione con credenziali LDAP tra le organizzazioni preferite nel \glo{backend}. & NI \\
TIA12 & Si verifichi che l’integrazione tra l’applicazione Android e REST-API permetta di rimuovere una organizzazione con credenziali LDAP tra le organizzazioni preferite nel \glo{backend}. & NI \\
TIA13 & Si verifichi che l’integrazione tra l’applicazione Android e REST-API permetta di avvisare il \glo{backend} qualora l'utente sia entrato in un'organizzazione preferita ed in risposta ricevere l'exit token. & NI \\
TIA14 & Si verifichi che l’integrazione tra l’applicazione Android e REST-API permetta di avvisare il \glo{backend} qualora l'utente sia uscito da un'organizzazione preferita e di riconsegnare l'exit token. & NI \\
TIA15 & Si verifichi che l’integrazione tra l’applicazione Android e REST-API permetta di avvisare il \glo{backend} qualora l'utente sia entrato in un luogo di un'organizzazione preferita ed in risposta ricevere l'exit token. & NI \\
TIA16 & Si verifichi che l’integrazione tra l’applicazione Android e REST-API permetta di avvisare il \glo{backend} qualora l'utente sia uscito da un luogo di un'organizzazione preferita e di riconsegnare l'exit token. & NI \\
TIA17 & Si verifichi che l’integrazione tra l’applicazione Android e REST-API permetta di avvisare il \glo{backend} qualora l'utente sia entrato in un'organizzazione preferita inviando il codice di autenticazione LDAP ed in risposta ricevere l'exit token. & NI \\
TIA18 & Si verifichi che l’integrazione tra l’applicazione Android e REST-API permetta di avvisare il \glo{backend} qualora l'utente sia uscito da un'organizzazione preferita inviando il codice di autenticazione LDAP e di riconsegnare l'exit token. & NI \\
TIA19 & Si verifichi che l’integrazione tra l’applicazione Android e REST-API permetta di avvisare il \glo{backend} qualora l'utente sia entrato in un luogo di un'organizzazione preferita inviando il codice di autenticazione LDAP ed in risposta ricevere l'exit token. & NI \\
TIA20 & Si verifichi che l’integrazione tra l’applicazione Android e REST-API permetta di avvisare il \glo{backend} qualora l'utente sia uscito da un luogo di un'organizzazione preferita inviando il codice di autenticazione LDAP e di riconsegnare l'exit token. & NI \\

TIW1 & Si verifichi l’integrazione tra l’applicazione web e servizi di \glo{Firebase} permetta l’autenticazione a un amministratore. & NI \\
TIW2 & Si verifichi l’integrazione tra l’applicazione web e servizi di \glo{Firebase} permetta il reset della password a un amministratore. & NI \\
TIW3 & Si verifichi l’integrazione tra applicazione web sviluppata in Angular 2+ e \glo{Firebase}.& NI \\
TIW4 & Si verifichi l’integrazione tra l’applicazione web e il server LDAP permetta l’autenticazione.& NI \\
TIW5 & Si verifichi l’integrazione tra REST-API e \glo{backend} permetta la comunicazione fra loro.& NI \\
TIW6 & Si verifichi l’integrazione tra l’applicazione web e REST-API permetta di recuperare la lista dei permessi che ha l’amministratore autenticato.& NI \\
TIW7 & Si verifichi l’integrazione tra l’applicazione web e REST-API permetta di recuperare la lista delle organizzazioni che ha l’amministratore autenticato.& NI \\
TIW8 & Si verifichi l’integrazione tra l’applicazione web e REST-API permetta di recuperare la lista dei luoghi di un’organizzazione che ha l’amministratore autenticato.& NI \\
TIW9 & Si verifichi l’integrazione tra l’applicazione web e REST-API permetta di monitorare gli accessi all’interno di un’organizzazione.& NI \\
TIW10 & Si verifichi l’integrazione tra l’applicazione web e REST-API permetta di monitorare gli accessi all’interno di un luogo di un’organizzazione.& NI \\
TIW11 & Si verifichi l’integrazione tra l’applicazione web e REST-API permetta di visualizzare i dati relativi a un’organizzazione ricevuti dal \glo{backend}.& NI \\
TIW12 & Si verifichi l’integrazione tra l’applicazione web e REST-API permetta di visualizzare i dati relativi a un luogo di un’organizzazione ricevuti dal \glo{backend}.& NI \\
TIW13 & Si verifichi l’integrazione tra l’applicazione web e REST-API permetta di modificare i dati relativi a un’organizzazione e inviare le modifiche al \glo{backend}.& NI \\
TIW14 & Si verifichi l’integrazione tra l’applicazione web e REST-API permetta di modificare i dati relativi a un luogo di un’organizzazione e inviare le modifiche al \glo{backend}.& NI \\
TIW15 & Si verifichi l’integrazione tra l’applicazione web e REST-API permetta di creare una nuova organizzazione.& NI \\
TIW16 & Si verifichi l’integrazione tra l’applicazione web e REST-API permetta di creare un nuovo luogo di un’organizzazione.& NI \\
TIW17 & Si verifichi l’integrazione tra l’applicazione web e REST-API permetta di creare un nuovo amministratore dell’organizzazione.& NI \\
TIW18 & Si verifichi l’integrazione tra l’applicazione web e REST-API permetta di inviare al \glo{backend} una richiesta di eliminazione di un’organizzazione.& NI \\
TIW19 & Si verifichi l’integrazione tra l’applicazione web e REST-API permetta di eliminare un luogo dell’organizzazione.& NI \\
TIW20 & Si verifichi l’integrazione tra l’applicazione web e REST-API permetta di modificare i permessi di un amministratore dell’organizzazione e salvare le modifiche nel \glo{backend}.& NI \\
TIW21 & Si verifichi l’integrazione tra l’applicazione web e REST-API permetta di eliminare un amministratore dell’organizzazione.& NI \\




\end{longtable}
}