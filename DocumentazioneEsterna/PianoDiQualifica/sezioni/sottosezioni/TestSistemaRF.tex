% scritto da Tommaso Azzalin, Riccardo Baratin, Christian Mattei

% test Christian R1FI1 a R1FA2(compreso) e R2FA6.1 a R2FA6.9 e R1FC3 a R1FI8 e R1FS6.1 a R1FS7.6 e R1FS10.11 a R1FS9.11

%da R1FI1 a R1FI1.6
TSF-X & Accesso all'applicazione di un utente non autenticato & \begin{itemize}
    \item Verificare che l'utente possa autenticarsi con le credenziali Stalker.
    \item Verificare che l'utente possa registrarsi creando un account Stalker.
\end{itemize}
Se l'utente vuole autenticarsi accedendo con le credenziali:
\begin{itemize}
    \item Verifica che l'utente ha inserito correttamente l'e-mail.
    \item Verifica che l'utente ha inserito correttamente la password.
\end{itemize}
Se l'utente vuole autenticarsi registrandosi:
\begin{itemize}
    \item Verifica che l'utente ha inserito una e-mail esistente.
    \item Verifica che l'utente ha inserito una password.
    \item Verifica che l'utente ha confermato l'inserimento della password.
    \item Verifica che l'utente ha accettato le condizioni generali d'uso.
\end{itemize}

%da R2FA1.8 
%TSF-X & L'utente si è dimenticato della password & \begin{itemize}

%\end{itemize}
% fine test Christian

% test Riccardo

% fine test Riccardo

% test Tommaso

% da R1FA4.1 a R1FA4.3
TSF-X & Selezione della modalità di tracciamento & \begin{itemize}
    \item Verificare che l'utente possa inserire la modalità di tracciamento anonima.
    \item Verificare che l'utente possa inserire la modalità di tracciamento autenticata.
\end{itemize}
Se l'utente si trova presso un luogo di un'organizzazione:
\begin{itemize}
    \item Verificare che nel passaggio dalla modalità di tracciamento autenticato a quella anonima venga inviata al sistema la richiesta di uscita dell'utente dal luogo e la successiva richiesta di ingresso di utente anonimo.
    \item Verificare che nel passaggio dalla modalità di tracciamento anonima a quella autenticata venga inviata al sistema la richiesta di uscita dell'utente dal luogo e la successiva richiesta di ingresso di utente riconosciuto.
\end{itemize}
% fine test Tommaso









% requisiti copiati da Analisi dei Requisiti
R1FI1 & Un utente non autenticato non può effettuare alcuna azione a meno di autenticazione e registrazione. & - \\

R1FA1.1 & L'autenticazione da parte di un utente necessita di e-mail. & - \\

R1FA1.2 & L'autenticazione da parte di un utente necessita di password.  & - \\

R1FA8.1 & L'autenticazione viene negata qualora l'utente tenti di autenticarsi con delle credenziali errate. Viene inoltre visualizzato un messaggio d'errore. & - \\

R1FA1.3 & La registrazione da parte di un utente necessita di e-mail. & - \\

R1FA8.2 & Il processo di registrazione dell'utente viene negato qualora l'e-mail inserita fosse già registrata nel sistema. Viene visualizzato inoltre un messaggio d'errore. & - \\

R1FA1.4 & La registrazione da parte di un utente necessita di password. & - \\

R1FA1.5 & La registrazione da parte di un utente necessita di conferma della password. & - \\

R1FA1.6 & Durante la registrazione viene chiesto all'utente di accettare le condizioni generali d'uso. & - \\

R1FA1.7 & Qualora l'utente dovesse rifiutare le condizioni generali d'uso deve venir interrotta la registrazione e chiusa l'applicazione. & - \\

R2FA1.8 & L'utente deve essere in grado di effettuare il reset della password qualora se la fosse dimenticata. & - \\

R1FA8.3 & Il processo di autenticazione viene negato qualora la password inserita non sia abbastanza sicura. Viene visualizzato inoltre un messaggio d'errore. & - \\

R1FA8.4 & Il processo di registrazione viene negato se password e conferma password inserita non combaciano. Viene visualizzato inoltre un messaggio d'errore. & - \\

%PERIN

R1FA2 & L'utente deve essere in grado di effettuare il logout. & - \\

R1FA3.1 & L'utente può gestire la propria lista delle organizzazioni\ap{G}. & - \\

R1FA3.2 & L'utente deve poter essere in grado di scaricare la lista di tutte le organizzazioni\ap{G}.  & - \\

R1FA8.3 & Qualora fallisca lo scaricamento della lista delle organizzazioni\ap{G} deve venire visualizzato un messaggio d'errore che lo informa di tale evento. & - \\

R1FA3.3 & L’utente deve poter essere in grado di gestire la propria lista delle organizzazioni preferite\ap{G}. & - \\

R1FA3.4 & L’utente può inserire una organizzazione\ap{G} presente nella lista delle organizzazioni\ap{G}, nella propria lista delle organizzazioni preferite\ap{G}.  & - \\

R1FA3.5 & Qualora l’utente inserisca un'organizzazione\ap{G} nella propria lista delle organizzazioni preferite\ap{G} che richiede autenticazione con credenziali LDAP, deve autenticarsi con credenziali LDAP\ap{G}. & - \\

R1FA3.6 & L’utente può rimuovere una organizzazione\ap{G} presente nella propria lista delle organizzazioni preferite\ap{G}.  & - \\

R1FA8.4 & Qualora non sia memorizzata nessuna lista delle organizzazioni\ap{G} nel dispositivo, viene informato l’utente di questo fatto.  & - \\

R1FA3.7 & L’utente ha la possibilità di aggiornare la lista delle organizzazioni\ap{G}.  & - \\

R1FA3.8 & L’utente può aggiornare la lista delle organizzazioni\ap{G} tramite refresh manuale\ap{G}.  & - \\

R1FA3.9 & L’utente può aggiornare la lista delle organizzazioni\ap{G} tramite temporizzazione\ap{G}. & - \\

R1FA3.10 & L’utente può visualizzare la lista delle organizzazioni\ap{G}. & - \\

R2FA3.11 & L’utente ha la possibilità di visualizzare la lista delle organizzazioni\ap{G} ordinate alfabeticamente, dalla A alla Z.  & - \\

R2FA3.12 & L’utente ha la possibilità di visualizzare la lista delle organizzazioni\ap{G} ordinate secondo politica FIFO\ap{G}.  & - \\

R3FA3.13 & L’utente ha la possibilità di visualizzare la lista delle organizzazioni\ap{G} che permettono il tracciamento anonimo\ap{G}. & - \\

R3FA3.14 & L’utente ha la possibilità di visualizzare la lista delle organizzazioni\ap{G} che permettono il tracciamento autenticato\ap{G}.  & - \\

R1FA3.15 & L’utente può effettuare ricerche personalizzate per cercare le organizzazioni\ap{G} presenti nella lista delle organizzazioni\ap{G}.  & - \\

R2FA3.16 & L’utente può ricercare organizzazioni\ap{G} presenti nella lista delle organizzazioni\ap{G} appartenenti alla nazione indicata dall’utente. & - \\

R1FA3.17 & L’utente può ricercare organizzazioni\ap{G} presenti nella lista delle organizzazioni\ap{G} che hanno nel nome una sottostringa scelta dall'utente.  & - \\

R2FA3.18 & L’utente può ricercare organizzazioni\ap{G} presenti nella lista delle organizzazioni\ap{G} appartenenti alla città indicata dall’utente.  & - \\

R1FA4.1 & L’utente deve poter inserire la modalità di tracciamento\ap{G} che preferisce. & - \\

R1FA4.2 & L’utente può selezionare la modalità di tracciamento anonimo\ap{G}. & - \\

R1FA4.3 & L’utente può selezionare la modalità di tracciamento autenticato\ap{G}. & - \\

R2FA5.1 & L’utente ha la possibilità di visualizzare il proprio storico degli accessi.  & - \\

R2FA5.2 & L’utente ha la possibilità di visualizzare il proprio storico degli accessi presso una organizzazione \ap{G}.  & - \\

R2FA5.3 & L'utente nella visualizzazione del proprio storico degli accessi nell'organizzazione visualizza la data per ogni accesso di quando è stato fatto. & - \\

R2FA5.4 & L'utente nella visualizzazione del proprio storico degli accessi nell'organizzazione visualizza il luogo per ogni accesso di quando è stato fatto.  & - \\

R2FA5.5 & L'utente nella visualizzazione del proprio storico degli accessi nell'organizzazione visualizza il tempo trascorso per ogni accesso di quando è stato fatto.  & - \\

R2FA5.6 & L’utente ha la possibilità di visualizzare il proprio storico degli accessi presso un luogo dell’organizzazione\ap{G} & - \\

R2FA5.7 & L'utente nella visualizzazione del proprio storico degli accessi nel luogo del organizzazione visualizza la data per ogni accesso di quando è stato fatto.  & - \\

R2FA5.8 & L'utente nella visualizzazione del proprio storico degli accessi nel luogo del organizzazione visualizza il luogo per ogni accesso di quando è stato fatto.  & - \\

R2FA5.9 & L'utente nella visualizzazione del proprio storico degli accessi nel luogo del organizzazione visualizza il tempo trascorso per ogni accesso di quando è stato fatto.  & - \\

R2FA5.10 & L’utente può visualizzare la propria lista degli accessi in una organizzazione\ap{G} ordinata per data decrescente.  & - \\

R2FA5.11 & L’utente può visualizzare la propria lista degli accessi in una organizzazione\ap{G} ordinata per data crescente.  & - \\

R3FA5.12 & L’utente può effettuare una ricerca degli accessi presso un'organizzazione\ap{G} in un giorno specifico. & - \\

R2FA5.13 & L’utente può visualizzare la propria lista degli accessi presso un luogo dell’organizzazione \ap{G} ordinata per data decrescente.  & - \\

R2FA5.14 & L’utente può visualizzare la propria lista degli accessi presso un luogo dell’organizzazione \ap{G} ordinata per data crescente.  & - \\

R3FA5.15 & L’utente può effettuare una ricerca degli accessi presso un luogo dell’organizzazione \ap{G} in un giorno specifico.  & - \\

R2FA5.16 & L’utente se si trova all’interno dell’organizzazione\ap{G} ha la possibilità di visualizzare il tempo passato all’interno dall'ultimo ingresso effettuato.  & - \\

R2FA5.17 & L’utente se si trova all’interno dell’luogo dell’organizzazione\ap{G} ha la possibilità di visualizzare il tempo passato all’interno dall'ultimo ingresso effettuato.  & - \\

R2FA8.5 & Qualora non ci sono accessi effettuati presso l'organizzazione selezionata, l'utente deve essere informato di ciò. & - \\

R2FA8.6 & Qualora non ci sono accessi effettuati presso il luogo selezionato, l'utente deve essere informato di ciò.  & - \\

R2FA6.1 & L’utente che effettua un movimento nell’organizzazione, deve essere registrato il tracciamento della sua azione.  & - \\

R2FA6.2 & Nella registrazione del tracciamento di un movimento dell’utente, deve essere memorizzata la data di quando è stato fatto.  & - \\

R2FA6.3 & Nella registrazione del tracciamento di un movimento dell’utente, deve essere memorizzata l’ora di quando è stato fatto.  & - \\

R2FA6.4 & Nella registrazione del tracciamento di un movimento dell’utente, deve essere memorizzata da chi è stata fatta.  & - \\

R2FA6.5 & L’utente che effettua un ingresso nell’organizzazione, deve essere registrato il tracciamento della sua azione secondo la modalità di tracciamento autenticata.  & - \\

R2FA6.6 & L’utente che effettua l’uscita dall’organizzazione, deve essere registrato il tracciamento della sua azione secondo la modalità di tracciamento autenticata.  & - \\

R2FA6.7 & L’utente che effettua un ingresso nell’organizzazione, deve essere registrato il tracciamento della sua azione secondo la modalità di tracciamento anonima.  & - \\

R2FA6.8 & L’utente che effettua l’uscita dall’organizzazione, deve essere registrato il tracciamento della sua azione secondo la modalità di tracciamento anonima.  & - \\

R2FA8.7 & Qualora non vengano memorizzate le informazioni necessarie per la registrazione del movimento effettuato dall’utente, deve essere notificato tale evento all’utente.  & - \\

R2FA6.9 & Deve essere notificato all’utente che è avvenuta la corretta registrazione del suo movimento.  & - \\

%Cisotto

R1FA7.1 & L'utente anonimo deve avere la possibilità di autenticarsi con le credenziali aziendali in un'organizzazione che richiede il tracciamento riconosciuto.  & - \\

R1FA8.8 & Qualora le credenziali LDAP aziendali inserite dall'utente non fossero riconosciute dal server aziendale associato viene mostrato un messaggio d'errore.  & - \\

R1FA7.2 & L'utente anonimo deve avere la possibilità di inserire l'e-mail durante l'autenticazione con le credenziali LDAP aziendali.  & - \\

R1FA7.3 & L'utente anonimo deve avere la possibilità di inserire la password durante l'autenticazione con le credenziali LDAP aziendali.  & - \\

R1FI2 & Un amministratore non autenticato non può effettuare alcuna azione a meno di autenticazione.  & - \\

R1FS1.1 & L’autenticazione da parte di un amministratore necessita di e-mail. & - \\

R1FS10.1 & L’autenticazione viene negata qualora l'amministratore tenti di autenticarsi con delle credenziali errate.  & - \\

R1FS10.2 & Qualora l'amministratore tenti di autenticarsi con le credenziali errate viene visualizzato un messaggio d’errore.  & - \\

R1FS1.2 & L’autenticazione da parte di un amministratore necessita di password. & - \\

R2FS1.3 & L'amministratore deve essere in grado di effettuare il reset della password qualora se la fosse dimenticata. & - \\

R1FS2.1 & L'amministratore deve essere in grado di effettuare il logout. & - \\

R1FC3 & L'amministratore deve poter visualizzare le organizzazioni disponibili. & - \\

R1FI3 & Deve venire mostrato il nome dell'organizzazione durante la sua visualizzazione da parte di un amministratore.  & - \\

R2FI4 & Deve venire mostrata l'immagine dell'organizzazione durante la sua visualizzazione da parte di un amministratore.  & - \\

R1FS3.1 & L'amministratore deve poter selezionare un'organizzazione tra quelle da lui visualizzate.  & - \\

R1FI5 & Deve venire mostrato il nome dell'organizzazione selezionata durante la sua visualizzazione da parte di un amministratore. & - \\

R2FI6 & Deve venire mostrata l'immagine dell'organizzazione selezionata durante la sua visualizzazione da parte di un amministratore.  & - \\

R2FI7 & Deve venire mostrata la descrizione dell'organizzazione selezionata durante la sua visualizzazione da parte di un amministratore. & - \\

R1FI8 & Deve venire mostrato l'indirizzo dell'organizzazione selezionata durante la sua visualizzazione da parte di un amministratore.  & - \\

R1FS4.1 & L'amministratore deve poter inserire un nuovo nome dell'organizzazione. & - \\

R1FS4.2 & L'amministratore deve poter modificare il nome dell'organizzazione. & - \\

R2FS4.3 & L'amministratore deve poter inserire una nuova immagine dell'organizzazione. & - \\

R2FS4.4 & L'amministratore deve poter modificare l'immagine dell'organizzazione. & - \\

R2FS4.5 & L'amministratore deve poter inserire una nuova descrizione dell'organizzazione.  & - \\

R2FS4.6 & L'amministratore deve poter modificare la descrizione dell'organizzazione.  & - \\

R1FS4.7 & L'amministratore deve poter inserire un nuovo indirizzo dell'organizzazione. & - \\

R1FS4.8 & L'amministratore deve poter modificare l'indirizzo dell'organizzazione.  & - \\

R1FS4.9 & L'amministratore deve poter inserire un nuovo indirizzo IP per l'organizzazione.  & - \\

R1FS4.10 & L'amministratore deve poter modificare l'indirizzo IP dell'organizzazione.  & - \\

R1FS10.3 & Se il nome dell'organizzazione inserito dall'amministratore non rispetta i vincoli imposti viene mostrato un messaggio d'errore.  & - \\

R1FS10.4 & Se il nome dell'organizzazione inserito dall'amministratore dovesse essere già presente nel sistema e associato ad un'altra organizzazione viene mostrato un messaggio d'errore.  & - \\

R2FS10.5 & Se l'immagine dell'organizzazione selezionata dall'amministratore non rispetta i vincoli imposti viene mostrato un messaggio d'errore. & - \\

R2FS10.6 & Se la descrizione dell'organizzazione inserita dall'amministratore non rispetta i vincoli imposti viene mostrato un messaggio d'errore. & - \\

R1FS10.7 & Se l'indirizzo dell'organizzazione inserito dall'amministratore non rispetta i vincoli imposti viene mostrato un messaggio d'errore.  & - \\

R1FS10.8 & Se l'indirizzo IP inserito dall'amministratore non rappresenta un server LDAP viene mostrato un messaggio d'errore.  & - \\

R1FS4.9 & L'amministratore deve avere la possibilità di inviare la richiesta di eliminazione per un'organizzazione.  & - \\

R3FS4.10 & L'amministratore deve poter inserire una motivazione per la richiesta di eliminazione dell'organizzazione.  & - \\

R1FS4.11 & L'amministratore deve poter annullare le modifiche che sta apportando.  & - \\



% Drago

%R1FS5 & L'amministratore deve essere in grado di modificare le funzionalità di tracciamento dell'organizzazione & \o & UCS 5 Interno & - \\
R1FS5.1 & L'amministratore può selezionare un nuovo perimetro di tracciamento dell'organizzazione. & - \\

R1FA10.9 & La modifica del perimetro dell'organizzazione viene negata qualora l'amministratore selezioni un area che non rispetta i vincoli imposti. Viene visualizzato un messaggio di errore.  & - \\

R1FS5.2 & L'amministratore deve essere in grado di creare dei nuovi luoghi di tracciamento nell'organizzazione.  & - \\

R1FS5.3 & L'amministratore deve essere in grado di modificare i luoghi di tracciamento dell'organizzazione.  & - \\

R1FS10.10 & La creazione di nuovi luoghi e la modifica dell'area di tracciamento di essi vengono negati qualora l'amministratore selezioni un area che fuoriesce dal perimetro. Viene visualizzato un messaggio di errore.  & - \\

R1FS5.4 & L'amministratore deve essere in grado di eliminare i luoghi di tracciamento dell'organizzazione.  & - \\

R1FS5.5 & L'amministratore deve essere in grado di selezionare un'area geografica per il tracciamento del luogo scelto.  & - \\

R1FS5.6 &  L'amministratore può scegliere l'area di tracciamento tramite l'inserimento delle coordinate geografiche.  & - \\

R2FS5.7 & L'amministratore può scegliere l'area di tracciamento tramite Google Maps API. & - \\

R1FS5.8 & L'amministratore deve poter annullare la selezione dell'area geografica per il tracciamento. & - \\

R1FS6.1 & L'amministratore può monitorare l'organizzazione visualizzando il numero di utenti anonimi presenti nell'organizzazione.  & - \\

R1FS6.2 & L'amministratore può monitorare l'organizzazione visualizzando il numero di utenti anonimi presenti in un specifico luogo dell'organizzazione.  & - \\

R1FS7.1 & L'amministratore può monitorato gli accessi effettuati dagli utenti riconosciuti.  & - \\

R1FS7.2 & L'amministratore può monitorare gli accessi effettuati presso una organizzazione da un specifico utente riconosciuto visualizzandone il nome, il cognome e l'orario di accesso.  & - \\

R2FS7.3 & L’amministratore può filtrare la lista degli accessi di un utente riconosciuto per data decrescente.  & - \\

R2FS7.4 & L’amministratore può filtrare la lista degli accessi di un utente riconosciuto per data crescente.  & - \\

R2FS7.5 & L'amministratore può monitorare gli accessi filtrandoli in base a una data precisa.  & - \\

R1FS7.6 & L'amministratore può monitorare gli accessi effettuati presso un luogo all'interno di una organizzazione da un specifico utente riconosciuto visualizzandone il nome, il cognome e l'orario di accesso.  & - \\

R1FS8.1 & L'amministratore può ricevere un report tabellare degli accessi ai luoghi dell'organizzazione.  & - \\

R1FS8.2 &  Tabella delle entrate e uscite degli utenti nei luoghi dell'organizzazione generabile dall'amministratore di un organizzazione a tracciamento autenticato.  & - \\

R1FS8.3 & Tabella delle ore spese dagli utenti nei luoghi dell'organizzazione generabile dall'amministratore di un organizzazione a tracciamento autenticato.  & - \\

R1FS8.4 & Tabella contenente il numero degli utenti e il totale delle ore passate da essi nei luoghi dell'organizzazione generabile dall'amministratore di un organizzazione a tracciamento autenticato o anonimo.  & - \\

%aggiungere tabelle più particolareggiate

%R3FS8.1 & Le tabelle generabili dall'amministratore possono  & \op & UCS 8.1 Interno & - \\

%Cisotto

R1FS9.1 & L'amministratore proprietario ha la possibilità di entrare nella sezione di gestione degli amministratori (per la nomina, eliminazione e modifica dei privilegi ad altri amministratori).  & - \\

R1FI9 & L'amministratore proprietario ha la possibilità di visualizzare gli amministratori da esso nominati una volta entrato nella gestione degli amministratori.  & - \\

R1FI10 & La visualizzazione di un amministratore deve mostrare la sua e-mail.  & - \\

R1FI11 & La visualizzazione di un amministratore deve mostrare i suoi privilegi.  & - \\

R1FS9.2 & L'amministratore proprietario ha la possibilità di nominare un nuovo amministratore.  & - \\

R1FS9.3 & L'amministratore proprietario deve poter inserire un'e-mail per il nuovo amministratore da nominare.  & - \\

R1FS10.11 & Viene mostrato un messaggio d'errore qualora l'e-mail del nuovo amministratore da nominare risulti già registrata nel sistema.  & - \\

R1FS10.12 & Viene mostrato un messaggio d'errore qualora la password del nuovo amministratore da nominare risulti troppo debole.  & - \\

R1FS10.13 & Viene mostrato un messaggio d'errore qualora la password non combaci con la conferma password del nuovo amministratore da nominare.  & - \\

R1FS9.4 & L'amministratore proprietario deve poter inserire una nuova password per il nuovo amministratore da nominare.  & - \\

R1FS9.5 & L'amministratore proprietario deve poter inserire la conferma della nuova password per il nuovo amministratore da nominare.  & - \\

R1FS9.6 & L'amministratore proprietario deve poter selezionare i privilegi per il nuovo amministratore da nominare.  & - \\

R1FS9.7 & L'amministratore proprietario ha la possibilità di eliminare un amministratore.  & - \\

R1FS9.8 & L'amministratore proprietario ha la possibilità di inserire la e-mail dell'account amministratore da eliminare.  & - \\

R1FS10.14 & Viene mostrato un messaggio d'errore qualora l'e-mail dell'amministratore da eliminare inserita non risulti registrata nel sistema.  & - \\

R1FS9.9 & L'amministratore proprietario ha la possibilità di modificare i privilegi di un amministratore.  & - \\

R1FS9.10 & L'amministratore proprietario ha la possibilità di inserire la e-mail dell'account amministratore a cui desidera modificare i privilegi.  & - \\

R1FS10.15 & Viene mostrato un messaggio d'errore qualora l'e-mail dell'amministratore a cui si vuole modificare i privilegi non risulti registrata nel sistema.  & - \\

R1FS9.11 & L'amministratore proprietario ha la possibilità annullare le modifiche che sta apportando agli amministratori.  & - \\
