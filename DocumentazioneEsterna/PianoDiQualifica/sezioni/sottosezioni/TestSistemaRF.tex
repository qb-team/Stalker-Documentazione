% scritto da Tommaso Azzalin, Riccardo Baratin, Christian Mattei

% test Christian R1FI1 a R1FA2(compreso) e R2FA6.1 a R2FA6.9 e R1FC3 a R1FI8 e R1FS6.1 a R1FS7.6 e R1FS10.11 a R1FS9.11

%da R1FI1 a R1FA2 (con parti splittate)
TSF-X & Accesso all'applicazione di un utente non autenticato con credenziali Stalker & \begin{enumerate}
    \item Verificare che l'utente possa autenticarsi con le credenziali Stalker.
\end{enumerate}
Se l'utente vuole autenticarsi eseguendo l'accesso:
\begin{enumerate}
    \item Verificare che l'utente abbia inserito correttamente l'e-mail.
    \item Verificare che l'utente abbia inserito correttamente la password.
    \item Verificare che l'utente visualizzi un messaggio d'errore se l'autenticazione\ap{G} viene negata per inserimento di credenziali errate.
\end{enumerate}
Se l'utente si è dimenticato della password:
\begin{itemize}
    \item Verifica che l'utente possa effettuare il reset della password qualora se la fosse dimenticata.
\end{itemize} \\

%da R1FI1 a R1FA8.4 (con parti splittate)
TSF-X & Registrazione nell'applicazione di un utente non autenticato con credenziali Stalker & \begin{itemize}
    \item Verificare che l'utente possa registrarsi creando un account Stalker.
\end{itemize}
Se l'utente vuole autenticarsi eseguendo la registrazione:
\begin{enumerate}
    \item Verificare che l'utente abbia inserito una e-mail esistente.
    \item Verificare che l'utente visualizzi un messaggio d'errore se tentasse di registrarsi con un'e-mail già usata nel sistema.
    \item Verificare che l'utente abbia inserito una password.
    \item Verificare che l'utente visualizzi un messaggio d'errore qualora abbia inserito una password non ritenuta sicura. Il processo di autenticazione\ap{G} fallisce.
    \item Verificare che l'utente abbia confermato l'inserimento della password.
    \item Verificare che l'utente visualizzi un messaggio d'errore qualora abbia inserito una conferma password diversa dalla password. Il processo di autenticazione\ap{G} fallisce.
    \item Verificare che l'utente abbia accettato le condizioni generali d'uso.
    \item Verificare che la registrazione si interrompa e che l'applicazione si chiuda nel caso che l'utente non abbia accettato le condizioni generali d'uso.
\end{enumerate} \\

% R1FA2
TSF-X & Logout\ap{G} dell'utente & \begin{itemize}
    \item Verifica che l'utente possa effettuare il logout\ap{G} dall'applicazione.
\end{itemize} \\

%da R2FA6.1 a R2FA6.9
TSF-X & Tracciamento\ap{G}dell'utente nell'organizzazione\ap{G} & \begin{enumerate}
    \item Verificare che il tracciamento\ap{G}del movimento\ap{G} dell'utente venga registrato correttamente.
    \item Verificare che l'utente abbia ricevuto la notifica della corretta registrazione del suo movimento\ap{G}.
    \item Verificare che se le informazioni necessarie per la registrazione del movimento\ap{G} non sono memorizzate, allora l'utente dovrà ricevere una notifica di tracciatura fallita.
\end{enumerate}
Durante la registrazione del tracciamento\ap{G}del movimento\ap{G} dell'utente:
\begin{enumerate}
    \item Verificare che la data sia memorizzata.
    \item Verificare che l'ora sia memorizzata.
    \item Verificare chi, in maniera generica, sia memorizzato.
\end{enumerate}
Se l'utente è in modalità di tracciamento autenticato\ap{G}:
\begin{enumerate}
    \item Verificare che abbia effettuato un ingresso nell'organizzazione\ap{G}.
    \item Verificare che abbia effettuato l'uscita nell'organizzazione\ap{G}.
\end{enumerate}
Se l'utente è in modalità di tracciamento anonimo\ap{G}:
\begin{enumerate}
    \item Verificare che abbia effettuato un ingresso nell'organizzazione\ap{G}.
    \item Verificare che abbia effettuato l'uscita nell'organizzazione\ap{G}.
\end{enumerate} \\

%da R1FC3 a R1FI8
TSF-X & Disponibilità organizzazioni\ap{G} visualizzabili dall'amministratore & \begin{itemize}
    \item Verificare che l'amministratore possa visualizzare le organizzazioni\ap{G} disponibili.
\end{itemize}
Se l'organizzazione\ap{G} visualizzata è disponibile:
\begin{enumerate}
    \item Verificare che l'amministratore possa vedere il suo nome.
    \item Verificare che l'amministratore possa vedere la sua immagine.
    \item Verificare che l'amministratore possa selezionarla.
    \item Verificare che l'amministratore possa selezionarla e vedere il suo nome.
    \item Verificare che l'amministratore possa selezionarla e vedere la sua immagine.
    \item Verificare che l'amministratore possa selezionarla e vedere la sua descrizione.
    \item Verificare che l'amministratore possa selezionarla e vedere il suo indirizzo.
\end{enumerate} \\

%da R1FS6.1 a R1FS7.6
TSF-X & Monitoraggio dell'amministratore vedendo gli utenti presenti nell'organizzazione\ap{G} & \begin{itemize}
    \item Verificare che l'amministratore possa monitorare gli utenti presenti nell'organizzazione\ap{G}.
\end{itemize}
Se gli utenti monitorati presso l'organizzazione\ap{G} sono anonimi:
\begin{enumerate}
    \item Verificare che l'amministratore possa monitorare il numero degli utenti anonimi.
    \item Verificare che l'amministratore possa monitorare il numero degli utenti anonimi in un luogo specifico.
\end{enumerate}
Se gli utenti monitorati presso l'organizzazione\ap{G} sono riconosciuti:
\begin{enumerate}
    \item Verificare che l'amministratore possa monitorare gli accessi degli utenti riconosciuti.
    \item Verificare che l'amministratore possa monitorare gli accessi effettuati da uno specifico utente riconosciuto visualizzandone il nome, cognome e l'orario di accesso.
    \item Verificare che l'amministratore possa filtrare la lista degli accessi\ap{G} di uno specifico utente riconosciuto per data decrescente.
    \item Verificare che l'amministratore possa filtrare la lista degli accessi\ap{G} di uno specifico utente riconosciuto per data crescente.
    \item Verificare che l'amministratore possa filtrare la lista degli accessi\ap{G} di uno specifico utente riconosciuto per una data precisa.
    \item Verificare che l'amministratore possa monitorare gli accessi effettuati presso un luogo da un specifico utente riconosciuto visualizzandone il nome, il cognome e l’orario di accesso.
\end{enumerate} \\

%da R1FS10.11 a R1FS9.11
%TSF-X & & \begin{itemize}

%\end{itemize}

% fine test Christian

% test Riccardo
% da R1FA3.1 a R2FA3.18
TSF-X & Visualizzazione lista delle organizzazioni & \begin{enumerate}
\item Verificare che l'utente sia in grado di scaricare la lista delle organizzazioni.
\item Verificare che nel caso in cui si verifichi un errore durante lo scaricamento della lista di organizzazioni, l'utente visualizzi un messaggio di errore.
\item Verificare che l'utente sia in grado di gestire la propria lista di organizzazioni preferite.
\item Verificare che l'utente sia in grado di inserire un'organizzazione, presente nella lista di tutte le organizzazioni,  nella propria lista delle organizzazioni.
\item Verificare che l'utente possa autenticarsi con credenziali LDAP, qualora scaricasse una organizzazione che richieda l'autenticazione con credenziali LDAP.
\item Verificare che l'utente possa rimuovere un'organizzazione dalla propria lista delle organizzazioni preferite.
\item Verificare che l'utente venga informato nel caso in cui non sia memorizzata nessuna lista delle organizzazioni del proprio dispositivo.
\item Verificare che l'utente abbia la possibilità di aggiornare la lista delle organizzazioni.
\item Verificare che l'utente possa aggiornare la lista delle organizzazioni tramite refresh manuale.
\item Verificare che l'utente possa aggiornare la lista delle organizzazioni tramite temporizzazione.
\item Verificare che l'utente possa visualizzare la lista delle organizzazioni.
\item Verificare che l'utente possa visualizzare la lista delle organizzazioni ordinate alfabeticamente, dalla A alla Z.
\item Verificare che l'utente possa visualizzare la lista delle organizzazioni ordinate secondo la politica FIFO.
\item Verificare che l'utente possa visualizzare la lista delle organizzazioni che permettono il tracciamento anonimo.
\item Verificare che l'utente possa visualizzare la lista delle organizzazioni che permettono il tracciamento autenticato.
\item Verificare che l'utente possa effettuare ricerche personalizzate per cercare le organizzazioni presenti nella lista delle organizzazioni.
\item Verificare che l'utente possa ricercare organizzazioni presenti nella lista delle organizzazioni appartenenti alle nazioni indicate dall'utente.
\item  Verificare che l'utente possa ricercare organizzazioni presenti nella lista delle organizzazioni che hanno nel nome una sottostringa scelta dall'utente.
\item  Verificare che l'utente possa ricercare organizzazioni presenti nella lista delle organizzazioni appartenenti alla città indicata dall'utente.
\end{enumerate} \\
%R1FA7.1 a R1FA7.3
TSF-X & Autenticazione con credenziali LDAP & \begin{enumerate}
\item Verificare che l'utente anonimo abbia la possibilità di autenticarsi con credenziali aziendali in un'organizzazione che richiede il tracciamento riconosciuto.
\item Verificare che l'utente visualizzi un messaggio di errore qualora le credenziali LDAP non fossero riconosciute dal server.
\item Verificare che l'utente anonimo possa inserire la propria e-mail durante l'autenticazione con le credenziali LDAP aziendali.
\item Verificare che l'utente anonimo possa inserire la propria password durante l'autenticazione con le credenziali LDAP aziendali.
\end{enumerate} \\
% R1FS4.1 a R1FS4.11
TSF-X & Modifica dei dati dell'organizzazione & \begin{enumerate}
\item Verificare che l'amministratore possa inserire un nuovo nome dell'organizzazione.
\item Verificare che l'amministratore possa modificare il nome dell'organizzazione.
\item Verificare che l'amministratore possa inserire una nuova immagine della organizzazione.
\item Verificare che l'amministratore possa modificare l'immagine della organizzazione.
\item Verificare che l'amministratore possa inserire una nuova descrizione dell’organizzazione.
\item Verificare che l'amministratore possa modificare la descrizione dell’organizzazione.
\item Verificare che l'amministratore possa inserire un nuovo indirizzo dell’organizzazione.
\item Verificare che l'amministratore possa modificare l'indirizzo dell’organizzazione.
\item Verificare che l'amministratore possa inserire un nuovo indirizzo IP per l'organizzazione.
\item Verificare che l'amministratore possa modificare l'indirizzo IP dell'organizzazione.
\item Verificare che l'amministratore visualizzi un messaggio di errore qualora il nome dell'organizzazione inserito non rispetti i vincoli imposti.
\item Verificare che l'amministratore visualizzi un messaggio di errore qualora il nome dell'organizzazione inserito sia già presente nel sistema e associato ad un'altra organizzazione.
\item Verificare che l'amministratore visualizzi un messaggio di errore qualora l'immagine dell'organizzazione inserita non rispetti i vincoli imposti.
\item Verificare che l'amministratore visualizzi un messaggio di errore qualora la descrizione dell'organizzazione inserita non rispetti i vincoli imposti.
\item Verificare che l'amministratore visualizzi un messaggio di errore qualora l'indirizzo dell'organizzazione inserito non rispetti i vincoli imposti.
\item Verificare che l'amministratore visualizzi un messaggio di errore qualora l'indirizzo IP dell'organizzazione inserito non rappresenti un server LDAP.
\item Verificare che l'amministratore possa inviare la richiesta di eliminazione per un'organizzazione.
\item Verificare che l'amministratore possa inserire una motivazione per la richiesta di eliminazione di un'organizzazione.
\item Verificare che l'amministratore possa annullare le modifiche che sta apportando ad una organizzazione.
\end{enumerate} \\
%R1FS8.1 a R1FS8.4
TSF-X & Report tabellare degli accessi ai luoghi dell'organizzazione  & \begin{enumerate}
\item Verificare che l'amministratore possa ricevere un report tabellare degli accesi ai luoghi dell'organizzazione.
\item Verificare che l'amministratore di un'organizzazione a tracciamento autenticato possa generare una tabella delle entrate e uscite degli utenti nei luoghi dell'organizzazione.
\item Verificare che l'amministratore di un'organizzazione a tracciamento autenticato possa generare una tabella delle ore delle ore spese dagli utenti nei luoghi dell'organizzazione.
\item Verificare che l'amministratore di un'organizzazione a tracciamento autenticato o anonimo possa generare una tabella tabella contenente il numero degli utenti e il totale delle ore passate da essi nei luoghi dell’organizzazione. 
\end{enumerate} \\
% fine test Riccardo

% test Tommaso

% da R1FA4.1 a R1FA4.3
TSF-X & Selezione della modalità di tracciamento\ap{G}& \begin{enumerate}
    \item Verificare che l'utente possa inserire la modalità di tracciamento anonimo\ap{G}.
    \item Verificare che l'utente possa inserire la modalità di tracciamento autenticato\ap{G}.
\end{enumerate}
Se l'utente si trova presso un luogo di un'organizzazione\ap{G}:
\begin{enumerate}
    \item Verificare che nel passaggio dalla modalità di tracciamento\ap{G}autenticato a quella anonima venga inviata al sistema la richiesta di uscita dell'utente dal luogo e la successiva richiesta di ingresso di utente anonimo.
    \item Verificare che nel passaggio dalla modalità di tracciamento anonimo\ap{G} a quella autenticata venga inviata al sistema la richiesta di uscita dell'utente dal luogo e la successiva richiesta di ingresso di utente riconosciuto.
\end{enumerate} \\

% da R2FA5.1 a R2FA5.5, da R2FA5.10 a R2FA5.12, R2FA5.16, R2FA8.5
TSF-X & Storico degli accessi\ap{G} dell'utente presso un'organizzazione\ap{G} & \begin{enumerate}
    \item Verificare che l'utente possa vedere il proprio storico accessi presso un'organizzazione\ap{G}. Ogni accesso deve mostrare la data in cui è stato compiuto, il luogo corrispondente, il tempo totale trascorso all'interno nel luogo.
    \item Come al punto 1, ma la lista degli accessi\ap{G} deve risultare ordinata per data in ordine decrescente\ap{G}.
    \item Come al punto 1, ma la lista degli accessi\ap{G} deve risultare ordinata per data in ordine crescente\ap{G}.
    \item Come al punto 1, ma della lista vengono mostrati solo gli accessi che rispettano i parametri di ricerca sul giorno cercato.
    \item Verificare che in assenza di accessi effettuati presso un'organizzazione\ap{G} l'utente visualizzi un messaggio informativo.
\end{enumerate}
Se l'utente si trova presso un luogo di un'organizzazione\ap{G}:
\begin{enumerate}
    \item Verificare che l'utente possa visualizzare il tempo trascorso all'interno di tutti i luoghi dell'organizzazione\ap{G} dall'ultimo ingresso preceduto da un'uscita da un luogo di un'altra organizzazione\ap{G} se avvenuto nella stessa giornata o dall'inizio della giornata se quest'uscita è di un giorno precedente. % NON SONO SICURO, MOLTO COMPLESSO E FORSE NON FATTIBILE
\end{enumerate}
Se l'utente si trova presso un luogo di un'organizzazione\ap{G} e sta visualizzando lo storico accessi presso la stesso luogo:
\begin{enumerate}
    \item Verificare che l'utente possa visualizzare il tempo trascorso all'interno dall'ultimo ingresso effettuato.
\end{enumerate} \\

% da R2FA5.6 a R2FA5.9, da R2FA5.13 a R2FA5.15, R2FA5.17, R2FA8.6
TSF-X & Storico degli accessi\ap{G} dell'utente presso un luogo di un'organizzazione\ap{G} & \begin{enumerate}
    \item Verificare che l'utente possa vedere il proprio storico accessi presso il luogo di un'organizzazione\ap{G}. Ogni accesso deve mostrare la data in cui è stato compiuto, il luogo corrispondente, il tempo totale trascorso all'interno nel luogo.
    \item Come al punto 1, ma la lista degli accessi\ap{G} deve risultare ordinata per data in ordine decrescente\ap{G}.
    \item Come al punto 1, ma la lista degli accessi\ap{G} deve risultare ordinata per data in ordine crescente\ap{G}.
    \item Come al punto 1, ma della lista vengono mostrati solo gli accessi che rispettano i parametri di ricerca sul giorno cercato.
    \item Verificare che in assenza di accessi effettuati presso il luogo di un'organizzazione\ap{G} selezionato l'utente visualizzi un messaggio informativo.
\end{enumerate}
Se l'utente si trova presso lo stesso luogo:
\begin{enumerate}
    \item Verificare che l'utente possa visualizzare il tempo trascorso all'interno del luogo dall'ultimo ingresso effettuato.
\end{enumerate} \\

% da R1FI2 a R1FS1.3
TSF-X & Accesso al server di un amministratore non autenticato & \begin{enumerate}
    \item Verificare che l'amministratore possa inserire l'e-mail correttamente.
    \item Verificare che l'amministratore possa inserire correttamente la password.
    \item Verificare che l'amministratore visualizzi un messaggio d'errore se l'autenticazione\ap{G} viene negata per inserimento di credenziali errate.
\end{enumerate}
Se l'amministratore si è dimenticato della password:
\begin{enumerate}
    \item Verifica che l'amministratore possa effettuare il reset della password qualora se la fosse dimenticata.
\end{enumerate} \\

% R1FS2.1
TSF-X & Logout\ap{G} dell'amministratore & \begin{enumerate}
    \item Verifica che l'amministratore possa effettuare il logout\ap{G} dal server.
\end{enumerate} \\

% da R1FS5.1 a R1FS5.5
TSF-X & Modifica della lista dei luoghi di tracciamento\ap{G}di un'organizzazione\ap{G} & \begin{enumerate}
    \item Verificare che l'amministratore possa essere in grado di aggiungere un nuovo luogo in cui effettuare il tracciamento.
    \item Verificare che l'amministratore possa essere in grado di modificare un luogo dell'organizzazione\ap{G}.
    \item Verificare che l'amministratore possa essere in grado di eliminare un luogo dell'organizzazione\ap{G}.
    \item Verificare che venga negata la selezione di un'area che non rispetta i vincoli imposti per l'organizzazione\ap{G}, visualizzando un messaggio d'errore.
    \item Verificare che venga negato l'inserimento di un luogo se viene selezionata un'area che fuoriesce dal perimetro imposto per l'organizzazione\ap{G}, visualizzando un messaggio d'errore.
\end{enumerate} \\

% da R1FS5.5 a R1FS5.8
TSF-X & Modifica di un luogo di un'organizzazione\ap{G} & \begin{enumerate}
    \item Verificare che l'amministratore possa selezionare l'area geografica in cui effettuare il tracciamento\ap{G}mediante l'inserimento di coordinate geografiche.
    \item Verificare che l'amministratore possa selezionare l'area geografica in cui effettuare il tracciamento\ap{G}mediante l'inserimento di marcatori su una mappa interattiva.
    \item Verificare che l'amministratore possa annullare l'operazione di modifica del luogo dell'organizzazione\ap{G}.
\end{enumerate} \\

% da R1FS9.1 a R1FI11 (richiamo altri test????)
TSF-X & Gestione degli amministratori nominati da un amministratore & \begin{enumerate}
    \item Verificare che l'amministratore possa visualizzare gli amministratori che ha precedentemente nominato, di cui si devono visualizzare e-mail, privilegi.
    \item Verificare che l'amministratore possa nominare un nuovo amministratore.
    \item Verificare che l'amministratore proprietario possa modificare i privilegi di un altro amministratore, inserendo il suo indirizzo e-mail.
    \item Verificare che l'amministratore proprietario possa eliminare un amministratore, inserendo il suo indirizzo e-mail.
    \item Verificare che se non è presente un amministratore con l'indirizzo e-mail inserito dall'amministratore proprietario venga visualizzato un messaggio d'errore.
    \item Verificare che se non è presente un amministratore con l'indirizzo e-mail inserito dall'amministratore proprietario venga visualizzato un messaggio d'errore.
    \item Verificare che l'amministratore proprietario possa annullare l'operazione di modifica dei privilegi di un amministratore.
\end{enumerate} \\

% da R1FS9.2 a R1FS9.6
TSF-X & Nomina di un nuovo amministratore da parte di un altro amministratore per la stessa organizzazione\ap{G} & \begin{enumerate}
    \item Verificare che l'amministratore possa inserire l'indirizzo e-mail per il nuovo amministratore.
    \item Verificare che l'amministratore possa inserire la password per il nuovo amministratore.
    \item Verificare che l'amministratore possa inserire la conferma della password (che dev'essere uguale alla password).
    \item Verificare che l'amministratore possa selezionare i privilegi per il nuovo amministratore.
    \item Verificare che se l'indirizzo e-mail del nuovo amministratore è già presente nel sistema venga visualizzato un messaggio d'errore.
    \item Verificare che se la password risulta troppo debole venga visualizzato un messaggio d'errore.
    \item Verificare che se la conferma della password non combacia con la password venga visualizzato un messaggio d'errore.
\end{enumerate} \\
% fine test Tommaso