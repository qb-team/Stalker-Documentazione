{
\rowcolors{2}{grigetto}{white}
\renewcommand{\arraystretch}{1.5}
\centering
\begin{longtable}{ c C{13cm} C{1cm}}
\caption{Elenco dei test di unità}\\
\rowcolor{darkblue}
\textcolor{white}{\textbf{Codice}} & \textcolor{white}{\textbf{Descrizione}} & \textcolor{white}{\textbf{Stato}}\\
\endfirsthead
\rowcolor{darkblue}
\textcolor{white}{\textbf{Codice}} & \textcolor{white}{\textbf{Descrizione}} & \textcolor{white}{\textbf{Stato}}\\
\endhead

TUA1 & Si verifichi che l'utente non \glo{autenticato} possa inserire l'indirizzo e-mail. & NI \\
TUA2 & Si verifichi che l'utente non \glo{autenticato} possa inserire la password. & NI \\
TUA3 & Si verifichi che l'utente non \glo{autenticato} possa ricevere un messaggio di errore se l'\glo{autenticazione} viene negata per inserimento di credenziali errate. & NI \\
TUA4 & Si verifichi che l'utente non \glo{autenticato} possa effettuare il reset della password qualora se la fosse dimenticata. & NI \\
TUA5 & Si verifichi che l’utente non \glo{autenticato} possa inserire l'indirizzo e-mail. & NI \\
TUA6 & Si verifichi che l’utente non \glo{autenticato} possa ricevere un messaggio di errore se tentasse di registrarsi con un'e-mail già usata nel sistema.  & NI \\
TUA7 & Si verifichi che l’utente non \glo{autenticato} possa inserire una password. & NI \\
TUA8 & Si verifichi che l’utente non \glo{autenticato} possa inserire nuovamente la password come conferma. & NI \\
TUA9 & Si verifichi che l’utente non \glo{autenticato} possa ricevere un messaggio di errore qualora abbia inserito una password non ritenuta sicura. Il processo di \glo{autenticazione} deve fallire. & NI \\
TUA10 & Si verifichi che l’utente non \glo{autenticato} possa ricevere un messaggio di errore qualora abbia inserito una conferma password diversa dalla password. Il processo di \glo{autenticazione} deve fallire. & NI \\
TUA11 & Si verifichi che l’utente non \glo{autenticato} possa accettare le condizioni generali d'uso. & NI \\
TUA12 & Si verifichi che la registrazione si interrompa e che l'applicazione si chiuda nel caso che l'utente non autenticato non abbia accettato le condizioni generali d'uso. & NI \\
TUA13 & Si verifichi che l'utente anonimo possa effettuare il \glo{logout} dall'applicazione. & NI \\
TUA14 & Si verifichi che l'utente anonimo possa scaricare la lista di tutte le organizzazioni. & NI \\
TUA15 & Si verifichi che l'utente anonimo riceva un messaggio di errore qualora lo scaricamento della lista di tutte le organizzazioni non vada a buon fine. & NI \\
TUA16 & Si verifichi che l'utente anonimo ossa aggiornare la lista delle organizzazioni tramite \glo{refresh manuale}. & NI \\
TUA17 & Si verifichi che l'utente anonimo possa aggiornare la lista delle organizzazioni tramite \glo{temporizzazione}. & NI \\
TUA18 & Si verifichi che l'utente anonimo possa visionare la lista delle organizzazioni ordinate alfabeticamente. & NI \\
TUA19 & Si verifichi che l'utente anonimo possa visionare la lista delle organizzazioni ordinate secondo la politica \glo{FIFO}. & NI \\
TUA20 & Si verifichi che l'utente anonimo possa visionare la lista delle organizzazioni che permettono il tracciamento anonimo. & NI \\
TUA21 & Si verifichi che l'utente anonimo possa visionare la lista delle organizzazioni che permettono il \glo{tracciamento autenticato}. & NI \\
TUA22 & Si verifichi che l'utente anonimo possa ricercare organizzazioni presenti nella lista delle organizzazioni appartenenti alle nazioni indicate dall'utente. & NI \\
TUA23 & Si verifichi che l'utente anonimo possa ricercare organizzazioni presenti nella lista delle organizzazioni che hanno nel nome una sottostringa scelta dall'utente. & NI \\
TUA24 & Si verifichi che l'utente anonimo possa ricercare organizzazioni presenti nella lista delle organizzazioni appartenenti alla città indicata dall'utente. & NI \\
TUA25 & Si verifichi che l'utente anonimo possa inserire un'organizzazione, presente nella lista di tutte le organizzazioni, nella propria lista delle organizzazioni preferite. & NI \\
TUA26 & Si verifichi che l'utente anonimo possa rimuovere un'organizzazione dalla propria lista delle organizzazioni preferite. & NI \\
TUA27 & Si verifichi che l'utente anonimo venga informato nel caso in cui non sia memorizzata nessuna lista delle organizzazioni del proprio dispositivo. & NI \\
TUA28 & Si verifichi che l'utente riconosciuto possa inserire la modalità di \glo{tracciamento anonimo}. & NI \\
TUA29 & Si verifichi che l'utente riconosciuto possa inserire la modalità di \glo{tracciamento autenticato}. & NI \\
TUA30 & Si verifichi che nel passaggio dalla modalità di \glo{tracciamento autenticato} a quella anonima venga inviata al sistema la richiesta di uscita dell'utente riconosciuto dal luogo e la successiva richiesta di ingresso di utente anonimo. & NI \\
TUA31 & Si verifichi che nel passaggio dalla modalità di \glo{tracciamento anonimo} a quella autenticata venga inviata al sistema la richiesta di uscita dell'utente anonimo dal luogo e la successiva richiesta di ingresso di utente riconosciuto. & NI \\
TUA32 & Si verifichi che l'utente anonimo possa vedere il proprio storico accessi presso un'\glo{organizzazione}. & NI \\
TUA33 & Si verifichi che ogni accesso deve mostrare la data in cui è stato compiuto. & NI \\
TUA34 & Si verifichi che ogni accesso deve mostrare il luogo corrispondente.  & NI \\
TUA35 & Si verifichi che ogni accesso deve mostrare il tempo totale trascorso all'interno nel luogo. & NI \\
TUA36 & Si verifichi che la \glo{lista degli accessi} deve risultare ordinata \glo{per data in ordine decrescente}. & NI \\
TUA37 & Si verifichi che la \glo{lista degli accessi} deve risultare ordinata \glo{per data in ordine crescente}. & NI \\
TUA38 & Si verifichi che nella lista vengono mostrati solo gli accessi che rispettano i parametri di ricerca sul giorno cercato. & NI \\
TUA39 & Si verifichi che l’utente anonimo riceva un messaggio informativo in assenza di accessi effettuati presso un'\glo{organizzazione}. & NI \\
TUA40 & Si verifichi che l’utente anonimo possa visionare il nome dello specifico luogo in cui si trova. & NI \\
TUA41 & Si verifichi che l’utente anonimo possa visionare il tempo trascorso da quando ha fatto l'ultimo ingresso in uno specifico luogo. & NI \\
TUA42 & Si verifichi che l'utente anonimo possa vedere il proprio storico accessi presso il luogo di un'\glo{organizzazione}. & NI \\
TUA43 & Si verifichi che ogni accesso deve mostrare la data in cui è stato compiuto. & NI \\
TUA44 & Si verifichi che ogni accesso deve mostrare il luogo corrispondente. & NI \\
TUA45 & Si verifichi che ogni accesso deve mostrare il tempo totale trascorso all'interno nel luogo. & NI \\
TUA46 & Si verifichi che la \glo{lista degli accessi} deve risultare ordinata \glo{per data in ordine decrescente}. & NI \\
TUA47 & Si verifichi che la \glo{lista degli accessi} deve risultare ordinata \glo{per data in ordine crescente}. & NI \\
TUA48 & Si verifichi che nella lista vengono mostrati solo gli accessi che rispettano i parametri di ricerca sul giorno cercato. & NI \\
TUA49 & Si verifichi che l'utente anonimo, in assenza di accessi effettuati presso il luogo di un'\glo{organizzazione} selezionato, visualizzi un messaggio informativo. & NI \\
TUA50 & Si verifichi che l'utente anonimo possa visualizzare il tempo trascorso all'interno del luogo dall'ultimo ingresso effettuato. & NI \\
TUA51 & Si verifichi che l’utente anonimo riceva la notifica della corretta registrazione se il tracciamento del suo \glo{movimento} in/da un \glo{luogo} ha avuto successo. & NI \\
TUA52 & Si verifichi che l’utente anonimo riceva un messaggio di errore qualora il tracciamento del \glo{movimento} non sia andato a buon fine. & NI \\
TUA53 & Si verifichi che durante la registrazione del \glo{tracciamento} del \glo{movimento} dell'utente anonimo/riconosciuto, venga memorizzato il \glo{timestamp} in cui è avvenuto il \glo{movimento}. & NI \\
TUA54 & Si verifichi che se l'utente riconosciuto è in modalità di \glo{tracciamento autenticato}, venga verificata la correttezza delle credenziali \glo{LDAP}. & NI \\
TUA55 & Si verifichi che se l'utente riconosciuto è in modalità di \glo{tracciamento autenticato}, possa effettuare un ingresso in un luogo dell'\glo{organizzazione}. & NI \\
TUA56 & Si verifichi che se l'utente riconosciuto è in modalità di \glo{tracciamento anonima}, possa effettuare un ingresso in un luogo dell'\glo{organizzazione}. & NI \\
TUA57 & Si verifichi che se l'utente riconosciuto è in modalità di \glo{tracciamento anonima}, possa effettuare un'uscita da un luogo dell'\glo{organizzazione}. & NI \\
TUA58 & Si verifichi che l’utente anonimo possa autenticarsi con credenziali aziendali LDAP in un'organizzazione che richiede il tracciamento riconosciuto. & NI \\
TUA59 & Si verifichi che l’utente anonimo riceva un messaggio di errore qualora le credenziali \glo{LDAP} non fossero riconosciute dal server. & NI \\
TUA60 & Si verifichi che l’utente anonimo possa inserire il proprio nome utente durante l'autenticazione con le credenziali \glo{LDAP} aziendali. & NI \\
TUA61 & Si verifichi che l’utente anonimo possa inserire la propria password utente durante l'autenticazione con le credenziali \glo{LDAP} aziendali. & NI \\
TUS1 & Si verifichi che l’utente non \glo{autenticato} possa inserire l'e-mail correttamente. & NI \\
TUS2 & Si verifichi che l’utente non \glo{autenticato} possa inserire correttamente la password. & NI \\
TUS3 & Si verifichi che l’utente non \glo{autenticato} riceva un messaggio d'errore se l'\glo{autenticazione} viene negata per inserimento di credenziali errate. & NI \\
TUS4 & Si verifichi che l’utente non \glo{autenticato} possa effettuare il reset della password qualora se la fosse dimenticata. & NI \\
TUS5 & Si verifichi che l'amministratore \glo{autenticato} possa effettuare il \glo{logout} dalla applicazione web. & NI \\
TUS6 & Si verifichi che l’amministratore visualizzatore possa selezionare un’organizzazione dalla sua lista delle organizzazioni. & NI \\
TUS7 & Si verifichi che l'amministratore visualizzatore possa visualizzare il nome dell'organizzazione selezionata. & NI \\
TUS8 & Si verifichi che l'amministratore visualizzatore possa visualizzare l’immagine dell'organizzazione selezionata. & NI \\
TUS9 & Si verifichi che l'amministratore visualizzatore possa visualizzare la descrizione dell'organizzazione selezionata. & NI \\
TUS10 & Si verifichi che l'amministratore visualizzatore possa visualizzare l’indirizzo dell'organizzazione selezionata. & NI \\
TUS11 & Si verifichi che l'amministratore visualizzatore possa visualizzare l’indirizzo IP dell'organizzazione selezionata. & NI \\
TUS12 & Si verifichi che l'amministratore visualizzatore possa visualizzare le coordinate geografiche dell'organizzazione selezionata. & NI \\
TUS13 & Si verifichi che l'amministratore gestore possa modificare il nome dell'organizzazione selezionata. & NI \\
TUS14 & Si verifichi che l'amministratore gestore possa modificare l’immagine dell'organizzazione selezionata. & NI \\
TUS15 & Si verifichi che l'amministratore gestore possa modificare la descrizione dell'organizzazione selezionata. & NI \\
TUS16 & Si verifichi che l'amministratore gestore possa modificare l’indirizzo dell'organizzazione selezionata. & NI \\
TUS17 & Si verifichi che l'amministratore gestore possa modificare l’indirizzo IP dell'organizzazione selezionata. & NI \\
TUS18 & Si verifichi che l'amministratore gestore possa modificare le coordinate geografiche dell'organizzazione selezionata. & NI \\
TUS19 & Si verifichi che l'amministratore gestore riceva un messaggio di errore qualora il nome dell'organizzazione inserito non rispetti i vincoli imposti. & NI \\
TUS20 & Si verifichi che l'amministratore gestore possa riceva un messaggio di errore qualora il nome dell'organizzazione inserito sia già presente nel sistema e associato ad un'altra organizzazione. & NI \\
TUS21 & Si verifichi che l'amministratore gestore possa riceva un messaggio di errore qualora l'immagine dell'organizzazione inserita non rispetti i vincoli imposti. & NI \\
TUS22 & Si verifichi che l'amministratore gestore possa riceva un messaggio di errore qualora la descrizione dell'organizzazione inserita non rispetti i vincoli imposti. & NI \\
TUS23 & Si verifichi che l'amministratore gestore possa riceva un messaggio di errore qualora l'indirizzo dell'organizzazione inserito non rispetti i vincoli imposti. & NI \\
TUS24 & Si verifichi che l'amministratore gestore possa riceva un messaggio di errore qualora l'indirizzo IP dell'organizzazione inserito non rappresenti un server \glo{LDAP}. & NI \\
TUS25 & Si verifichi che l'amministratore gestore possa inviare la richiesta di eliminazione per un'organizzazione. & NI \\
TUS26 & Si verifichi che l'amministratore gestore possa inserire una motivazione per la richiesta di eliminazione di un'organizzazione. & NI \\
TUS27 & Si verifichi che l'amministratore gestore possa annullare le modifiche che sta apportando ad una organizzazione. & NI \\
TUS28 & Si verifichi che l’amministratore visualizzatore possa selezionare un luogo di un’organizzazione. & NI \\
TUS29 & Si verifichi che l'amministratore visualizzatore possa visualizzare il nome del luogo di un’organizzazione selezionata. & NI \\
TUS30 & Si verifichi che l'amministratore visualizzatore possa visualizzare le coordinate geografiche del luogo di un’organizzazione selezionata. & NI \\
TUS31 & Si verifichi che l'amministratore gestore possa modificare il nome del luogo di un’organizzazione selezionata. & NI \\
TUS32 & Si verifichi che l'amministratore gestore possa selezionare l'area geografica in cui effettuare il \glo{tracciamento} mediante l'inserimento di coordinate geografiche. & NI \\
TUS33 & Si verifichi che l'amministratore gestore possa selezionare l'area geografica in cui effettuare il \glo{tracciamento} mediante l'inserimento di marcatori su una mappa interattiva. & NI \\
TUS34 & Si verifichi che l'amministratore gestore possa eliminare un luogo di un’organizzazione. & NI \\
TUS35 & Si verifichi che l'amministratore gestore possa annullare le modifiche che sta apportando a un luogo di un’organizzazione.  & NI \\
TUS36 & Si verifichi che l’amministratore visualizzatore possa monitorare il numero degli utenti anonimi all’interno di una organizzazione. & NI \\
TUS37 & Si verifichi che l’amministratore visualizzatore possa monitorare il numero degli utenti anonimi all’interno di un luogo specifico. & NI \\
TUS38 & Si verifichi che l’amministratore visualizzatore possa monitorare gli accessi effettuati da uno specifico utente riconosciuto visualizzandone il nome. & NI \\
TUS39 & Si verifichi che l’amministratore visualizzatore possa monitorare gli accessi effettuati da uno specifico utente riconosciuto visualizzandone il cognome. & NI \\
TUS40 & Si verifichi che l’amministratore visualizzatore possa monitorare gli accessi effettuati da uno specifico utente riconosciuto visualizzandone l’orario di accesso. & NI \\
TUS41 & Si verifichi che l’amministratore visualizzatore possa filtrare la \glo{lista degli accessi} di uno specifico utente riconosciuto per data decrescente. & NI \\
TUS42 & Si verifichi che l’amministratore visualizzatore possa filtrare la \glo{lista degli accessi} di uno specifico utente riconosciuto per data crescente. & NI \\
TUS43 & Si verifichi che l’amministratore visualizzatore possa filtrare la \glo{lista degli accessi} di uno specifico utente riconosciuto per una data precisa. & NI \\
TUS44 & Si verifichi che l’amministratore visualizzatore possa monitorare gli accessi effettuati presso un luogo da uno specifico utente riconosciuto visualizzandone il nome. & NI \\
TUS45 & Si verifichi che l’amministratore visualizzatore possa monitorare gli accessi effettuati presso un luogo da uno specifico utente riconosciuto visualizzandone il cognome. & NI \\
TUS46 & Si verifichi che l’amministratore visualizzatore possa monitorare gli accessi effettuati presso un luogo da uno specifico utente riconosciuto visualizzandone l’orario di accesso. & NI \\
TUS47 & Si verifichi che si possa ottenere un report tabellare degli accesi ai luoghi dell'organizzazione. & NI \\
TUS48 & Si verifichi che si possa generare una tabella contenente il numero degli utenti. & NI \\
TUS49 & Si verifichi che si possa generare una tabella contenente il totale delle ore passate dagli utenti nei luoghi dell’organizzazione. & NI \\
TUS50 & Si verifichi che si possa generare una tabella delle entrate e uscite degli utenti nei luoghi dell'organizzazione. & NI \\
TUS51 & Si verifichi che si possa generare una tabella delle ore spese dagli utenti nei luoghi dell'organizzazione. & NI \\
TUS52 & Si verifichi che l’amministratore proprietario possa visionare gli amministratori che ha precedentemente nominato, di cui si devono visionare la e-mail. & NI \\
TUS53 & Si verifichi che l’amministratore proprietario possa visionare gli amministratori che ha precedentemente nominato, i privilegi. & NI \\
TUS54 & Si verifichi che l’amministratore proprietario possa modificare i privilegi di un altro amministratore, inserendo il suo indirizzo e-mail. & NI \\
TUS55 & Si verifichi che l’amministratore proprietario possa eliminare un amministratore, inserendo il suo indirizzo e-mail. & NI \\
TUS56 & Si verifichi che l’amministratore proprietario riceva un messaggio d'errore se non è presente un amministratore con l'indirizzo e-mail inserito dall'amministratore proprietario. & NI \\
TUS57 & Si verifichi che l’amministratore proprietario possa inserire un nuovo amministratore inserendo l’e-mail. & NI \\
TUS58 & Si verifichi che l’amministratore proprietario possa inserire un nuovo amministratore inserendo la password. & NI \\
TUS59 & Si verifichi che l’amministratore proprietario possa selezionare i privilegi per il nuovo amministratore. & NI \\
TUS60 & Si verifichi che l’amministratore proprietario riceva un messaggio d'errore se l'indirizzo e-mail del nuovo amministratore è già presente nel sistema. & NI \\
TUS61 & Si verifichi che l’amministratore proprietario riceva un messaggio d'errore se la password risulta troppo debole. & NI \\
TUS62 & Si verifichi che l’amministratore proprietario riceva un messaggio d'errore se la conferma della password non combacia con la password. & NI \\
TUS63 & Si verifichi che l’amministratore proprietario possa annullare l'operazione di modifica dei privilegi di un amministratore.  & NI \\
\end{longtable}
}