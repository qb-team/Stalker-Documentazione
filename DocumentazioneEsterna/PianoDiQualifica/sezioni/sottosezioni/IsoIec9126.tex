\subsection{ISO/IEC 9126}
Lo standard ISO/IEC 9126 si occupa di presentare le caratteristiche di qualità di un prodotto software e gli attributi che le compongono.

\subsubsection{Metriche esterne}
Le metriche relative alla qualità "esterna" indirizzano le caratteristiche esteriori del software, cioè quelle rilevabili direttamente dagli utenti e dagli operatori.

\subsubsection{Metriche interne}
 Le metriche della qualità "interne" del software sono utilizzate durante la fase di sviluppo e permettono di valutare il comportamento del software dal punto di vista degli sviluppatori e di predire quello che sarà il punto di vista esterno degli utenti.

\subsubsection{Funzionalità}
Capacità del prodotto software di fornire funzioni adeguate al contesto di applicazione.
\begin{itemize}
\item \textbf{Adeguatezza}: capacità di fornire un insieme di funzioni che permettano agli utenti del software di poter svolgere i loro compiti.
\item \textbf{Accuratezza}: capacità di fornire i risultati attesi dall’utente con la precisione richiesta.
\item \textbf{Interoperabilità}: capacità di interagire con uno o più sistemi specificati.
\item \textbf{Sicurezza (Security)}: capacità di proteggere le informazioni e i dati dell’utente da persone non autorizzate ad accedervi.
\item \textbf{Aderenza alla funzionalità}: capacità di aderire a standard, norme, convenzioni e regolamenti sulle funzionalità.
\end{itemize}

\subsubsection{Affidabilità}
Capacità del prodotto software di mantenere il livello di prestazione quando usato in condizioni specificate.
E’ limitata da errori nei requisiti, nella progettazione e nel codice del software.
\begin{itemize}
\item \textbf{Maturità}: capacità di evitare che si verifichino errori.
\item \textbf{Tolleranza a guasti}: capacità di mantenere il livello di prestazioni in caso di errori o violazione delle interfacce. Assieme alla Maturità, descrivono l’attributo Disponibilità, non specificato in quanto formato da questi due.
\item \textbf{Recuperabilità}: capacità di ripristinare il livello di prestazioni e i dati in caso di errori e malfunzionamenti.
\item \textbf{Aderenza all’affidabilità}: capacità di aderire a standard, norme, convenzioni e regolamenti sull’affidabilità.
\end{itemize}

\subsubsection{Usabilità}
Capacità del prodotto software di essere comprensibile, di poter essere studiato.
\begin{itemize}
\item \textbf{Comprensibilità}: capacità di permettere all’utente di capire le funzionalità e come usarle con successo.
\item \textbf{Apprendibilità}: capacità di permettere all’utente di imparare l’applicazione.
\item \textbf{Operabilità}: capacità di permettere all’utente di usare il software e controllarlo.
\item \textbf{Attrattività}: capacità di risultare attraente (ossia possedere una interfaccia utente accattivante).
\item \textbf{Aderenza all’usabilità}: capacità di aderire a standard, norme, convenzioni e regolamenti sull’usabilità.
\end{itemize}

\subsubsection{Efficienza}
Capacità del prodotto software di realizzare le funzioni richieste nel minor tempo possibile.
\begin{itemize}
\item \textbf{Comportamento rispetto al tempo}: capacità di fornire in tempi adeguati risposte per l’utente.
\item \textbf{Utilizzo risorse}: capacità di utilizzare un appropriato numero e tipo di risorse, eseguendo le funzionalità previste.
\item \textbf{Aderenza all’efficienza}: capacità di aderire a standard, norme, convenzioni e regolamenti sull’efficienza.
\end{itemize}

\subsubsection{Manutenibilità}
Capacità del prodotto software di essere modificato.
\begin{itemize}
\item \textbf{Analizzabilità}: capacità di poter diagnosticare errori e individuare malfunzionamenti.
\item \textbf{Modificabilità}: capacità di consentire lo sviluppo di modifiche al software originale.
\item \textbf{Stabilità}: capacità di evitare effetti non desiderati.
\item \textbf{Provabilità}: capacità di consentire la verifica delle funzionalità del prodotto software.
\item \textbf{Aderenza alla manutenibilità}: capacità di aderire a standard, norme, convenzioni e regolamenti sulla manutenibilità.
\end{itemize}

\subsubsection{Portabilità}
Capacità di essere trasportato da un ambiente ad un altro.
\begin{itemize}
\item \textbf{Adattabilità}: verrà descritta se ce ne sarà il bisogno.
\item \textbf{Installabilità}: verrà descritta se ce ne sarà il bisogno.
\item \textbf{Coesistenza}: verrà descritta se ce ne sarà il bisogno.
\item \textbf{Aderenza alla portabilità}: verrà descritta se ce ne sarà il bisogno.
\end{itemize}

\subsubsection{Qualità in uso}
\begin{itemize}
\item \textbf{Efficacia}: capacità per l’utente del prodotto software di raggiungere obiettivi specifici con accuratezza e completezza.
\item \textbf{Produttività}: capacità di permettere all’utente di impegnare un numero definito di risorse, in relazione all’efficienza raggiunta. Queste risorse possono essere tempo, materiali e costi.
\item \textbf{Sicurezza fisica (Safety)}: capacità di raggiungere un livello accettabile di rischio per dati, business e persone. I rischi sono tipicamente correlati a difetti in progettazione o analisi o codifica.
\item \textbf{Soddisfazione}: capacità di soddisfare gli utenti in uno specifico contesto.
\end{itemize}