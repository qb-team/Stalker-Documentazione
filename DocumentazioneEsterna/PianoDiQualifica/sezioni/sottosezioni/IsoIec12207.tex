\subsection{ISO IEC 12207}
ISO/IEC 12207 è uno standard utilizzato per misurare la qualità dei processi. Questa normativa è suddivisa in 3 parti,
ed ognuna contiene dei sotto processi e delle attività. I processi e le attività usate sono riportate in seguito. 

\subsubsection{Processi primari}
Sono i processi\ap{G} e le attività che fanno parte dello sviluppo del software e hanno lo scopo di soddisfare tutti i requisiti concordati con il cliente.

\paragraph{Sviluppo}
Il processo ha lo scopo di sviluppare un prodotto software, o un sistema basato sul software, che indirizzi le esigenze del cliente.
\begin{itemize}
    \item \textbf{Analisi dei requisiti:}
    Lo sviluppatore deve valutare i requisiti software in base ai criteri elencati di seguito. I risultati delle valutazioni devono essere documentati.
        \begin{itemize}
            \item tracciabilità dei requisiti di sistema e progettazione del sistema;
            \item coerenza esterna con i requisiti di sistema;
            \item coerenza interna;
            \item testabilità;
            \item fattibilità della progettazione del software;
            \item fattibilità di funzionamento e manutenzione.
        \end{itemize}
    \item \textbf{Pianificazione di dettaglio:}\\
    Lo sviluppatore deve sviluppare un progetto dettagliato per ciascun componente software. I componenti del software 
    devono essere perfezionati in livelli inferiori contenenti unità software che possono essere codificate, compilate 
    e testate. È necessario garantire che tutti i requisiti software siano assegnati dai componenti software 
    alle unità software. La pianificazione di dettaglio deve essere documentato.
    
    \item \textbf{Codfica:}
    Lo sviluppatore deve valutare il codice del software e i risultati dei test considerando i criteri elencati sotto. I risultati delle valutazioni devono essere documentati.
    \begin{itemize}
        \item Tracciabilità ai requisiti e alla progettazione dell'articolo software;
        \item coerenza esterna con i requisiti e il design dell'articolo software;
        \item coerenza interna tra i requisiti dell'unità;
        \item testare la copertura delle unità;
        \item adeguatezza dei metodi e delle norme di codifica utilizzati;
        \item fattibilità dell'integrazione e dei test del software;
        \item Fattibilità di funzionamento e manutenzione.   
    \end{itemize}
\end{itemize}
\subsubsection{Processi di supporto}
Sono i processi\ap{G} e le attività che aiutano gli altri processi\ap{G} nel raggiungimento del successo e nella qualità del progetto.

\paragraph{Documentazione}
Il processo di "Gestione della documentazione" garantisce lo sviluppo e la manutenzione delle informazioni prodotte e registrate relativamente al prodotto software.
\\
\textbf{Implementazione}: identifica i documenti da produrre durante il ciclo di vita del prodotto software,
deve essere sviluppato, documentato e implementato. Per ogni documento identificato, è necessario quanto segue
essere indirizzato:
\begin{itemize}
    \item titolo o nome;
    \item scopo;
    \item pubblico previsto;
    \item Procedure e responsabilità per input, sviluppo, revisione, modifica, approvazione, produzione, stoccaggio, distribuzione, manutenzione e gestione della configurazione;
    \item Programma per le versioni intermedie e finali.
\end{itemize}

\paragraph{Garanzia di qualità}
Il processo di "Assicurazione qualità" ha lo scopo di assicurare che tutti i prodotti di fase (work product) siano conformi con i piani e gli standard definiti.
\paragraph{Verifica}
Il processo di "Verifica" ha lo scopo di confermare che ciascun work product o servizio realizzato da un processo soddisfi i requisiti specificati. 
Il processo di Verifica deve essere integrato nei processi di Sviluppo, Fornitura e Manutenzione. Se la verifica viene eseguita da terzi, questa viene definita come "Processo di verifica indipendente".
\\
Il processo deve essere verificato considerando i criteri elencati di seguito:
\begin{itemize}
    \item I requisiti di pianificazione del progetto sono adeguati e tempestivi.
    \item I processi selezionati per il progetto sono adeguati, implementati, eseguiti come previsto, e conforme al contratto.
    \item Gli standard, le procedure e gli ambienti per i processi del progetto sono adeguati.
    \item Il progetto è composto da personale e personale addestrato come richiesto dal contratto.
\end{itemize}

\subsubsection{Processi organizzativi}
Sono i processi\ap{G} e le attività che coprono gli aspetti organizzativi e di gestione delle risorse.

\paragraph{Gestione}
Il processo di Gestione garantisce lo sviluppo e la manutenzione delle informazioni prodotte e registrate relativamente 
al prodotto software. Il "gestore" prepara i piani per l'esecuzione del processo.
I piani associati all'esecuzione del processo devono contenere descrizioni: delle attività, dei compiti associati e
identificazioni dei prodotti software che verranno forniti. Questi piani devono rispettare i seguenti punti:
\begin{itemize}
    \item programmi per il completamento tempestivo dei compiti;
    \item risorse adeguate necessarie per eseguire i compiti;
    \item assegnazione di compiti;
    \item assegnazione di responsabilità;
    \item quantificazione dei rischi associati ai compiti o al processo stesso;
    \item misure di controllo della qualità da applicare durante l'intero processo;
    \item costi associati all'esecuzione del processo;
    \item Fornitura di ambiente e infrastruttura.
\end{itemize}
