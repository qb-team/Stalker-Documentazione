\section{Resoconto delle attività di verifica}
In questa sezione vengono descritti ed analizzati gli esiti delle attività di verifica su tutti i documenti destinati alla consegna.

\subsection{Revisione dei requisiti}

\subsubsection{Analisi statica dei documenti}
I membri del gruppo \Gruppo{} hanno analizzato i documenti mediante la tecnica di walkthrough che ha portato all'individuazione di 
alcuni errori ortografici e grammaticali, grazie anche agli strumenti di controllo ortografico integrati negli editor per la produzione
della documentazione che il gruppo ha deciso di utilizzare.

\subsubsection{Esiti verifiche automatizzate}
Attualmente, l'unico valore che può essere calcolato per verificare\ap{G} se la garanzia di qualità che il gruppo ritiene fornire è
soddisfatta, è data dall'Indice di Gulpease (MPC6).
Per poter calcolare questo indice, viene utilizzato come strumento il "Calcolatore dell'Indice Gulpease" ospitato a \href{https://farfalla-project.org/readability_static/}{questo indirizzo}.
Non vengono contati per l'indice di Gulpease il testo presente nelle seguenti parti o sezioni del documento:
\begin{itemize}
    \item Copertina;
    \item Indice;
    \item Registro delle modifiche;
    \item Riferimenti.
\end{itemize}
Nella seguente tabella sono riportati i risultati degli indici di Gulpease ottenuti dai documenti.

{
\rowcolors{2}{grigetto}{white}
\renewcommand{\arraystretch}{1.5}
\centering
\begin{longtable}{ C{4cm} | C{4cm} | C{4cm} }
\caption{Elenco dei test di sistema}\\
\rowcolor{darkblue}
\textcolor{white}{\textbf{Documento}} & \textcolor{white}{\textbf{Indici di Gulpease}} & \textcolor{white}{\textbf{Esito}} \\
\hline
\endhead
\AdRv{1.0.0} &  & \\
\PdPv{1.0.0} &  & \\
\PdQ{1.0.0} &  & \\

\NdP{1.0.0} &  & \\
\SdF{1.0.0} &  & \\

\Glossariov{1.0.0} &  & \\

VI\_2019\_12\_18 & & \\
VE\_2019\_12\_16 & & \\
VI\_2019\_12\_13 & & \\
VI\_2019\_12\_10 & & \\
VI\_2019\_12\_10 & & \\
VI\_2019\_12\_06 & & \\
VI\_2019\_12\_03 & & \\
VI\_2019\_12\_02 & & \\
VI\_2019\_11\_29 & & \\
VI\_2019\_11\_27 & & \\
VI\_2019\_11\_20 & & \\

\end{longtable}
}