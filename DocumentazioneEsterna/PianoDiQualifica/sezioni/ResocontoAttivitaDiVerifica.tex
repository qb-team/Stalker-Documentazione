\section{Resoconto delle attività di verifica}
In questa sezione vengono descritti ed analizzati gli esiti delle attività di verifica su tutti i documenti destinati alla consegna.

\subsection{Revisione dei requisiti}

\subsubsection{Analisi statica dei documenti}
I membri del gruppo \Gruppo{} hanno analizzato i documenti mediante la tecnica di walkthrough che ha portato all'individuazione di 
alcuni errori ortografici e grammaticali, grazie anche agli strumenti di controllo ortografico integrati negli editor per la produzione
della documentazione che il gruppo ha deciso di utilizzare.

\subsubsection{Esiti verifiche automatizzate}
Attualmente, l'unico valore che può essere calcolato per \glo{verificare} se la garanzia di qualità che il gruppo ritiene fornire è
soddisfatta, è data dall'Indice di Gulpease (MPC6).
Per poter calcolare questo indice, viene utilizzato come strumento il "Calcolatore dell'Indice Gulpease" ospitato a \href{https://farfalla-project.org/readability_static/}{questo indirizzo}.
Non vengono contati per l'indice di Gulpease il testo presente nelle seguenti parti o sezioni del documento:
\begin{itemize}
    \item Copertina;
    \item Indice;
    \item Registro delle modifiche;
    \item Riferimenti.
\end{itemize}
Nella seguente tabella sono riportati i risultati degli indici di Gulpease ottenuti dai documenti per ogni periodo, successivamente ci sarà un grafico che riporterà l'andamento degli indici di Gulpease per ogni documento a seconda del periodo.

\paragraph{Legenda}
\begin{itemize}
	\item \textbf{Documento}: Nome del documento;
	\item \textbf{RR}: Periodo di revisione dei requisiti;
	\item \textbf{RP}: Periodo di revisione di progettazione;
	\item \textbf{RQ}: Periodo di revisione di qualifica;
	\item \textbf{RA}: Periodo di revisione di accettazione;
	\item \textbf{-}: Valore inesistente;
	\item  \textbf{Colore giallo}: Indica che il valore ottenuto è accettabile;
	\item  \textbf{Colore verde}: Indica che il valore ottenuto è ottimo. 
\end{itemize}

{
\rowcolors{2}{grigetto}{white}
\renewcommand{\arraystretch}{1.5}
\centering
\begin{longtable}{C{4cm} C{1cm} C{1cm} C{1cm} C{1cm}}
\caption{Elenco dei indici di Gulpease }\\
\rowcolor{darkblue}
\textcolor{white}{\textbf{Documento}} & \textcolor{white}{\textbf{RR}} &
\textcolor{white}{\textbf{RP}} & \textcolor{white}{\textbf{RQ}} & 
\textcolor{white}{\textbf{RA}} \\
\hline
\endhead
\AdRv{1.0.0} & \textcolor{verde}{\textbf{95}} & - & - & -\\
\PdPv{1.0.0} & \textcolor{verde}{\textbf{100}} & - & - & -\\
\PdQv{1.0.0} & \textcolor{verde}{\textbf{100}} & - & - & - \\

\NdPv{1.0.0} & \textcolor{giallo}{\textbf{75}} & - & - & -\\
\SdFv{1.0.0} & \textcolor{verde}{\textbf{94}} & - & - & -\\

\Glossariov{1.0.0} & \textcolor{giallo}{\textbf{67}} & - & - & -\\

VI\_2019\_12\_18 & \textcolor{verde}{\textbf{100}} & - & - & -\\
VE\_2019\_12\_16 & \textcolor{giallo}{\textbf{65}} & - & - & -\\
VI\_2019\_12\_13 & \textcolor{verde}{\textbf{92}} & - & - & -\\
VI\_2019\_12\_10 & \textcolor{verde}{\textbf{100}} & - & - & -\\
VI\_2019\_12\_06 & \textcolor{verde}{\textbf{83}} & - & - & -\\
VI\_2019\_12\_03 & \textcolor{verde}{\textbf{83}} & - & - & -\\
VI\_2019\_11\_27 & \textcolor{verde}{\textbf{100}} & - & - & -\\
VI\_2019\_11\_20 & \textcolor{verde}{\textbf{87}} & - & - & -\\

\end{longtable}
}