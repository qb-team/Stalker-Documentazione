\section{Qualità di prodotto}
La qualità di un prodotto software è valutata secondo criteri semplici e comprensibili a tutti, utenti e sviluppatori, operatori e addetti alla manutenzione.
Per valutare tale qualità il gruppo \Gruppo{} ha deciso di far riferimento allo standard ISO/IEC 9126, per soddisfare alcuni degli standard di qualità tra quelli proposti dal medesimo, ovvero quelli che il gruppo ha ritenuto necessari per il progetto; La selezione è riportata nelle \NdP{}. In particolare, lo standard ISO/IEC 9126 descrive le seguenti norme:
\begin{itemize}
    \item Un modello di qualità del software;
    \item Le caratteristiche che determinano la qualità del software;
    \item Le metriche per la misurazione della qualità del software.
\end{itemize}

\subsection{Metriche interne}
%Le metriche della qualità "interne" del software sono utilizzate durante la fase di sviluppo e permettono di valutare il comportamento del software dal punto di vista degli sviluppatori e di predire quello che sarà il punto di vista esterno degli utenti.
\subsubsection{Usabilità}
Capacità del prodotto software di essere comprensibile, di poter essere usato e compreso facilmente, in ogni sua parte, da qualsiasi utente che lo voglia usare.\\
\vspace{0.3cm}
\subsubsection{Manutenibilità}
Capacità di predire il livello di impegno richiesto per modificare il prodotto software dal punto di vista degli sviluppatori.
  \rowcolors{2}{grigetto}{white}
  \renewcommand{\arraystretch}{1.5}
\begin{longtable}{ c C{4cm} c C{3.5cm} C{3.5cm}}
	\caption{Tabella metriche per la manutenibilità}\\
	\rowcolor{darkblue}
	\textcolor{white}{\textbf{Metrica}} & \textcolor{white}{\textbf{Nome}} & \textcolor{white}{\textbf{Sigla}} & \textcolor{white}{\textbf{Range Accettabile}} & \textcolor{white}{\textbf{Range Ottimale}}\\
	MPD1 & Complessità Ciclomatica del Software & $CCS $ & $1 \leq CCS \leq 7 $ & $1 \leq CCS \leq 4$\\
\end{longtable}
\subsection{Metriche esterne}
%Le metriche relative alla qualità "esterna" indirizzano le caratteristiche esteriori del software, cioè quelle rilevabili direttamente dagli utenti e dagli operatori.

\subsubsection{Affidabilità}
Capacità del prodotto software di dimostrare un adeguato livello di affidabilità quando opererà nel sistema in cui è previsto debba operare.
\rowcolors{2}{grigetto}{white}
\renewcommand{\arraystretch}{1.5}
\begin{longtable}{ c C{4cm} c C{3.5cm} C{3.5cm}}
	\caption{Tabella metriche per la affidabilità}\\
	\rowcolor{darkblue}
	\textcolor{white}{\textbf{Metrica}} & \textcolor{white}{\textbf{Nome}} & \textcolor{white}{\textbf{Sigla}} & \textcolor{white}{\textbf{Range Accettabile}} & \textcolor{white}{\textbf{Range Ottimale}}\\
	MPD2 & Presenza di Code Smell & $CS$ & $CS \leq 50 $ & $CS = 10 $\\
	MPD3 & Presenza di Vulnerabilità & $VLN$ & $VLN = 2$ & $VLN = 0 $\\
	MPD4 & Presenza di Bug & $BUG$ & $BUG = 20 $ & $BUG = 0 $\\
\end{longtable}

\subsubsection{Usabilità}
Capacità del prodotto software di essere facilmente comprensibile, apprendibile ed utilizzabile per ogni utente intenzionato a usarlo.
\rowcolors{2}{grigetto}{white}
\renewcommand{\arraystretch}{1.5}
\begin{longtable}{ c C{4cm} c C{3.5cm} C{3.5cm}}
	\caption{Tabella metriche per l'usabilità}\\
	\rowcolor{darkblue}
	\textcolor{white}{\textbf{Metrica}} & \textcolor{white}{\textbf{Nome}} & \textcolor{white}{\textbf{Sigla}} & \textcolor{white}{\textbf{Range Accettabile}} & \textcolor{white}{\textbf{Range Ottimale}}\\
	 MPD5 & Profondità Strutturale dell'Interfaccia & $PSI$ & $1 \leq PSI \leq 5$ &$1 \leq PSI \leq 3$\\
\end{longtable}
