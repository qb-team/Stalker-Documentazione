
\section{Qualità di prodotto}
La qualità di un prodotto software è valutata secondo criteri semplici e comprensibili per tutti, utenti e sviluppatori, operatori e addetti alla manutenzione.
Per valutare tale qualità il gruppo "Qbteam" ha deciso di far riferimento allo standard ISO/IEC 9126, le quali norme descrivono:

\begin{itemize}
\item Un modello di qualità del software;
\item Le caratteristiche che determinano la qualità del software;
\item Le metriche per la misurazione della qualità del software;
\end{itemize}

In questa sezione di documento verranno riportate solo alcune delle caratteristiche definite dallo standard, ovvero quelle ritenute più inerenti ai fini del progetto.

\subsection{Metriche interne}
 Le metriche della qualità "interne" del software sono utilizzate durante la fase di sviluppo e permettono di valutare il comportamento del software dal punto di vista degli sviluppatori e di predire quello che sarà il punto di vista esterno degli utenti.
  \subsubsection{Funzionalità}
      Capacità del prodotto software di soddisfare i requisiti funzionali e le necessità degli utenti.\\

              \paragraph{Metrica - Aderenza delle funzioni e/o delle interfacce} 
              \begin{itemize}
          \item  \textbf{Codice:} MPD-01
        \item    \textbf{Descrizione:} Misurare il livello di aderenza delle funzioni e delle interfacce sviluppate rispetto agli standard, alle normative e alla regolamentazioni
          \item  \textbf{Attributo di riferimento:} Aderenza alle funzionalità 
        \item    \textbf{Sigla:} AFI
         \item   \textbf{Formula:} \begin{math}AFI = \frac{A}{B}\end{math}\\ \\
              A = numero di funzioni(e/o interfacce) sviluppate che risultano aderenti a standard, regole e normative emesse al riguardo; \\
              B = numero totale di funzioni(e/o interfacce) che devono essere aderenti a tali regole come descritto nelle specifiche;
                   \item \textbf{Range di valori che può assumere:}
        \begin{itemize}
            \item \textbf{Accettabile:} 
            \item \textbf{Ottimale:} 
        \end{itemize}
       \end{itemize}
              
  \subsubsection{Affidabilità} 
  Capacità di predire se il prodotto software in questione potrà soddisfare i requisiti prescritti per l'affidabilità dal punto di vista degli sviluppatori.\
                \paragraph{Metrica - Rilevamento dei difetti} 
                  \begin{itemize}
         \item   \textbf{Codice:} MPD-02
         \item   \textbf{Descrizione:} Misurare l'efficacia nel rilevare i difetti presenti nel software durante le diverse fasi di sviluppo del prodotto
        \item    \textbf{Attributo di riferimento:} Maturità
        \item    \textbf{Sigla:} RDF
        \item    \textbf{Formula:} \begin{math} RDF = \frac{A}{B}\end{math}\\ \\
             A = numero di difetti rilevati nelle revisioni tecniche, ispezioni e test del prodotto in ciascuna fase di sviluppo;\\
              B = numero totale di difetti previsti nella fase di sviluppo;
             \item \textbf{Range di valori che può assumere:}
        \begin{itemize}
            \item \textbf{Accettabile:} 
            \item \textbf{Ottimale:} 
        \end{itemize}
       \end{itemize}
              
                  
              
           
\subsubsection{Usabilità} 
Capacità del prodotto software di essere comprensibile, di poter essere usato e compreso facilmente, in ogni sua parte, da qualsiasi utente che lo voglia usare. \\

	 \paragraph{Metrica - Validità dei dati d'input} 
	    \begin{itemize}
          \item  \textbf{Codice: } MPD-03
           \item \textbf{Descrizione:} Misurare il livello di correttezza dei dati forniti in input all'applicazione
         \item   \textbf{Attributo di riferimento:} Operabilità
          \item  \textbf{Sigla:} VDI
         \item   \textbf{Formula:}\begin{math} VDI = \frac{A}{B}\end{math}\\ \\
             A = numero dei dati di input di cui si effettua il controllo di validità;\\
             B = numero totale di dati di input previsti;
                   \item \textbf{Range di valori che può assumere:}
        \begin{itemize}
            \item \textbf{Accettabile:} 
            \item \textbf{Ottimale:} 
        \end{itemize}
       \end{itemize}
              
          
           
                   \paragraph{Metrica - Attrattività delle interfacce utente} 
                      \begin{itemize}
          \item  \textbf{Codice: } MPD-04
          \item  \textbf{Descrizione:} Misurare quanto attrattive risultino le interfacce agli utenti dal punto di vista grafico 
          \item  \textbf{Attributo di riferimento:} Attrattività
          \item  \textbf{Sigla:} AIU
           \item \textbf{Formula:}\begin{math}AIU = V (q) \end{math}\\ \\
             Valore medio dei risultati di un questionario compilato da almeno tre utenti.\\
         Può essere utilizzata una scala a quattro valori: Molto attrattivo, Attrattivo, Poco attrattivo, Non Attrattivo.
           \end{itemize}
         
      
      
      
        \subsubsection{Manutenibilità} 
    Capacità di predire il livello di impegno richiesto per modificare il prodotto software dal punto di vista degli sviluppatori.
    
        \paragraph{Metrica - Diagnostica} 
           \begin{itemize}
          \item  \textbf{Codice:} MPD-05
          \item  \textbf{Descrizione:} Misurare il livello di diagnostica che il prodotto consente tramite le apposite funzioni 
          \item  \textbf{Attributo di riferimento:} Analizzabilità
         \item   \textbf{Sigla:} D
          \item  \textbf{Formula:} \begin{math}D = \frac{A}{B}\end{math}\\ \\
            A = numero di funzioni di diagnostica sviluppate;\\
            B = numero totale di funzioni di diagnostica previste nelle specifiche;
                 \item \textbf{Range di valori che può assumere:}
        \begin{itemize}
            \item \textbf{Accettabile:} 
            \item \textbf{Ottimale:} 
        \end{itemize}
       \end{itemize}
              
           
           \paragraph{Metrica - Complessità del software} 
              \begin{itemize}
         \item   \textbf{Codice:} MPD-06
         \item   \textbf{Descrizione:} Misurare la complessità ciclomatica dei singoli moduli sviluppati
          \item  \textbf{Attributo di riferimento:} Modificabilità
          \item  \textbf{Sigla:} CF
         \item   \textbf{Formula:} \begin{math}CF = v(G) = e - n + 2 \end{math}\\ \\
            G = grafo del modulo;\\
            e = cammino;\\
            n = nodo; 
               \item \textbf{Range di valori che può assumere:}
        \begin{itemize}
            \item \textbf{Accettabile:} 
            \item \textbf{Ottimale:} 
        \end{itemize}
       \end{itemize}
              

\paragraph{Metrica - Registrazione delle modifiche} 
   \begin{itemize}
          \item  \textbf{Codice:} MPD-07
         \item   \textbf{Descrizione:} Misurare se tutte le modifiche apportate al software sono commentate nei singoli moduli e nella documentazione tecnica
         \item   \textbf{Attributo di riferimento:} Modificabilità
         \item   \textbf{Sigla:} RM
         \item   \textbf{Formula:} \begin{math}RM = \frac{A}{B}\end{math}\\ \\
            A = numero di modifiche commentate nel codice e nella documentazione tecnica;\\
            B = numero totale di modifiche eseguite;
               \item \textbf{Range di valori che può assumere:}
        \begin{itemize}
            \item \textbf{Accettabile:} 
            \item \textbf{Ottimale:} 
        \end{itemize}
       \end{itemize}
              
 
           
\subsection{Metriche esterne}
Le metriche relative alla qualità "esterna" indirizzano le caratteristiche esteriori del software, cioè quelle rilevabili direttamente dagli utenti e dagli operatori.
   \subsubsection{Funzionalità}
   Capacità del prodotto software di fornire funzioni adeguate al contesto di applicazione.
   
      

                   \paragraph{Metrica - Completezza delle funzionalità sviluppate} 
            \begin{itemize}
            \item  \textbf{Codice:} MPD-08
            \item  \textbf{Descrizione:} Misurare il livello di completezza delle funzioni sviluppate 
            \item  \textbf{Attributo di riferimento:} Adeguatezza
            \item  \textbf{Sigla:} CFS
            \item  \textbf{Formula:} \begin{math}CFS = 1- \frac{A}{B}\end{math}\\ \\
            A = numero di funzioni omesse;\\
            B = numero totale di funzioni previste nelle specifiche;
            \item \textbf{Range di valori che può assumere:}
        \begin{itemize}
            \item \textbf{Accettabile:} 
            \item \textbf{Ottimale:} 
        \end{itemize}
       \end{itemize}
       
             
       
              
          \subsubsection{Affidabilità}
   Capacità del prodotto software di dimostrare un adeguato livello di affidabilità quando opererà nel sistema in cui è previsto debba operare.
   
       
                  \paragraph{Metrica - Maturità dei test} 
            \begin{itemize}
           \item   \textbf{Codice:} MPD-09
            \item  \textbf{Descrizione:} Misurare la percentuale di casi di test eseguiti con successo rispetto al numero totale previsto per garantire piena copertura dei requisiti sia funzionali che qualitativi(usabilità, affidabilità, efficienza)
              \item   \textbf{Attributo di riferimento:} Maturità
          \item    \textbf{Sigla:} AF
           \item   \textbf{Formula:} \begin{math}AF = \frac{A}{B}\end{math}\\ \\
            A = numero di casi di test eseguiti con successo;\\
            B = numero totale di casi di test previsto;
            \item \textbf{Range di valori che può assumere:}
        \begin{itemize}
            \item \textbf{Accettabile:} 
            \item \textbf{Ottimale:} 100\%
        \end{itemize}
       \end{itemize}
       
              \subsubsection{Usabilità}
   Capacità del prodotto software di essere facilmente comprensibile, apprendibile ed operabile per ogni utente intenzionato a usarlo.
   
                  \paragraph{Metrica - Completezza della descrizione funzionale} 
            \begin{itemize}
           \item   \textbf{Codice:} MPD-10
           \item   \textbf{Descrizione:} Misurare la percentuale di funzioni comprese dall'utente dopo aver letto la descrizione del prodotto(es. Manuale utente, specifiche funzionali) 
           \item    \textbf{Attributo di riferimento:} Accuratezza
           \item   \textbf{Sigla:} CDF
           \item   \textbf{Formula:} \begin{math}CDF = \frac{A}{B}\end{math}\\ \\
            A = numero di funzioni comprese;\\
            B = numero totale di funzioni disponibili;
            \item \textbf{Range di valori che può assumere:}
        \begin{itemize}
            \item \textbf{Accettabile:} 
            \item \textbf{Ottimale:} 100\%
        \end{itemize}
       \end{itemize}
       
              
                
       
                 
       
       
