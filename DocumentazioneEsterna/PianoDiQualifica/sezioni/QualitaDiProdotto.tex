\section{Qualità di prodotto}
La qualità di un prodotto software è valutata secondo criteri semplici e comprensibili a tutti, utenti e sviluppatori, operatori e addetti alla manutenzione.
Per valutare tale qualità il gruppo \Gruppo{} ha deciso di far riferimento allo standard ISO/IEC 9126, le quali norme descrivono:
\begin{itemize}
    \item Un modello di qualità del software; 
    \item Le caratteristiche che determinano la qualità del software;
    \item Le metriche per la misurazione della qualità del software.
\end{itemize}

\subsection{Metriche interne}
Le metriche della qualità "interne" del software sono utilizzate durante la fase di sviluppo e permettono di valutare il comportamento del software dal punto di vista degli sviluppatori e di predire quello che sarà il punto di vista esterno degli utenti.

\subsubsection{Funzionalità}
Capacità del prodotto software di soddisfare i requisiti funzionali e le necessità degli utenti.
\paragraph{Metrica - Aderenza agli Standard di Funzioni o Interfacce} 
\begin{itemize}
    \item \textbf{Codice:} MPD1;
    \item \textbf{Descrizione:} Misura in percentuale il livello di aderenza delle funzioni e delle interfacce sviluppate rispetto agli standard, alle normative e alla regolamentazioni;
    \item \textbf{Attributo di riferimento:} Aderenza alle funzionalità;
    \item \textbf{Sigla:} $ASFI$ 
    \item \textbf{Formula:} $$ASFI = {|funzioni \; o \; interfacce \; aderenti \; a \; standard| \over |funzioni \; o \; interfacce \; che \; devono \; aderire \; a \; standard|} \cdot 100 $$ 
    \item \textbf{Range di valori che può assumere:}
    \begin{itemize}
        \item \textbf{Accettabile:} $ASFI \geq 80\% $
        \item \textbf{Ottimale:} $ASFI = 100\%$
    \end{itemize}
\end{itemize}
              
\subsubsection{Affidabilità} 
Capacità di predire se il prodotto software in questione potrà soddisfare i requisiti prescritti per l'affidabilità dal punto di vista degli sviluppatori.

\paragraph{Metrica - Rilevamento dei Difetti} 
\begin{itemize}
    \item \textbf{Codice:} MPD2
    \item \textbf{Descrizione:} Misurare in percentuale l'efficacia nel rilevare i difetti presenti nel software durante lo sviluppo del prodotto;
    \item \textbf{Attributo di riferimento:} Maturità\ap{G}
    \item \textbf{Sigla:} $RD$
    \item \textbf{Formula:} $$RD = {|difetti \; rilevati \; nei \; test \; del \; prodotto| \over |difetti \; previsti \; durante \; lo \; sviluppo|} \cdot 100 $$
    \item \textbf{Range di valori che può assumere:}
    \begin{itemize}
        \item \textbf{Accettabile:} $RD \geq 80\% $
        \item \textbf{Ottimale:} $RD = 100\%$
    \end{itemize}
\end{itemize}

\subsubsection{Usabilità} 
Capacità del prodotto software di essere comprensibile, di poter essere usato e compreso facilmente, in ogni sua parte, da qualsiasi utente che lo voglia usare.\\
\paragraph{Metrica - Validità dei Dati d'Input} 
\begin{itemize}
    \item \textbf{Codice:} MPD3
    \item \textbf{Descrizione:} Misurare in percentuale la correttezza dei dati forniti in input all'applicazione;
    \item \textbf{Attributo di riferimento:} Operabilità\ap{G};
    \item \textbf{Sigla:} $VDI$
    \item \textbf{Formula:} $$VDI = {|dati \; di \; input \; controllati \; e \; validitati| \over |dati \; di \; input \; previsti|} \cdot 100$$  
    \item \textbf{Range di valori che può assumere:}
    \begin{itemize}
        \item \textbf{Accettabile:} $VDI \geq 90\%$
        \item \textbf{Ottimale:} $VDI = 100\%$
    \end{itemize}
\end{itemize}

\paragraph{Metrica - Attrattività della User Interface (UI)} 
\begin{itemize}
    \item \textbf{Codice: } MPD4
    \item \textbf{Descrizione:} Misurare quanto attrattive risultino le interfacce agli utenti dal punto di vista grafico.
    Gli utenti dovranno poi compilare un questionario in base all'esperienza che hanno fatto;
    Il valore medio di tale valutazione è ritenuto valido se almeno tre utenti hanno compilato il questionario. 
    Viene utilizzata una scala a quattro valori: Molto attrattivo, Attrattivo, Poco attrattivo, Non Attrattivo;
    \item \textbf{Attributo di riferimento:} Attrattività\ap{G};
    \item \textbf{Sigla:} $AUI$
    \item \textbf{Formula:}$$AUI = V(q) $$
    con:
        \begin{itemize}
        \item $V$ = Valore medio dei risultati;
        \item $q$ = Questionario compilato;
        \end{itemize}
    \item \textbf{Range di valori che può assumere:}
        \begin{itemize}
            \item \textbf{Accettabile:} $AUI = Attrattivo$ 
            \item \textbf{Ottimale:} $AUI = Molto \; attrattivo$
        \end{itemize}
\end{itemize}

\subsubsection{Manutenibilità} 
Capacità di predire il livello di impegno richiesto per modificare il prodotto software dal punto di vista degli sviluppatori.           
\paragraph{Metrica - Complessità Ciclomatica del Software} 
    \begin{itemize}
    \item \textbf{Codice:} MPD5
    \item \textbf{Descrizione:} Misurare la complessità ciclomatica dei singoli moduli sviluppati;
    \item \textbf{Attributo di riferimento:} Modificabilità\ap{G}
    \item \textbf{Sigla:} $CCS$
    \item \textbf{Formula:} $$CCS = e - n + p$$
    con:
    \begin{itemize}
        \item $G$ = grafo del modulo;
        \item $e$ = numero di congiunzioni tra statement (corrispondenti agli archi di un grafo);
        \item $n$ = numero di statement (nodi presenti nel grafo);
        \item $p$ = numero delle componenti connesse da ogni nodo (per esecuzione sequenziale: p=2, essendovi un predecessore e un successore);
    \end{itemize}

    \item \textbf{Range di valori che può assumere:}
    \begin{itemize}
        \item \textbf{Accettabile:} $1 \leq CS \leq 7 $
        \item \textbf{Ottimale:} $ 1 \leq CS \leq 4 $
    \end{itemize}
\end{itemize}

\paragraph{Metrica - Unità Documentate} 
\begin{itemize}
    \item \textbf{Codice:} MPD6
    \item \textbf{Descrizione:} Misurare in percentuale il numero di unità di codice con documentazione tecnica;
    \item \textbf{Attributo di riferimento:} Modificabilità\ap{G};
    \item \textbf{Sigla:} $UD$
    \item \textbf{Formula:} $$UD = {|unit\grave{a} \; di \; codice \; con \; documentazione \; tecnica| \over |unit\grave{a} \; di \; codice|} \cdot 100$$
    \item \textbf{Range di valori che può assumere:}
    \begin{itemize}
        \item \textbf{Accettabile:} $UD \geq 80\%  $
        \item \textbf{Ottimale:} $UD = 100\%$
    \end{itemize}
\end{itemize}
              
       
\subsection{Metriche esterne}
Le metriche relative alla qualità "esterna" indirizzano le caratteristiche esteriori del software, cioè quelle rilevabili direttamente dagli utenti e dagli operatori.

\subsubsection{Affidabilità}
Capacità del prodotto software di dimostrare un adeguato livello di affidabilità quando opererà nel sistema in cui è previsto debba operare.
    
\paragraph{Metrica - Maturità dei Test} 
\begin{itemize}
    \item \textbf{Codice:} MPD7
    \item \textbf{Descrizione:} Misurare la percentuale di casi di test eseguiti con successo rispetto al numero totale previsto per garantire piena copertura dei requisiti sia funzionali che qualitativi(usabilità, affidabilità, efficienza);
    \item \textbf{Attributo di riferimento:} Maturità\ap{G};
    \item \textbf{Sigla:} $MT$
    \item \textbf{Formula:} $$MT = {|casi \; di \; test \; eseguiti \; con \; successo| \over |casi \; di \; test \; previsti|} \cdot 100$$
    \item \textbf{Range di valori che può assumere:}
    \begin{itemize}
        \item \textbf{Accettabile:} $MT = 100\% $
        \item \textbf{Ottimale:} $MT = 100\% $
    \end{itemize}
\end{itemize}
       
\subsubsection{Usabilità}
Capacità del prodotto software di essere facilmente comprensibile, apprendibile ed operabile per ogni utente intenzionato a usarlo.

\paragraph{Metrica - Profondità Strutturale dell'Interfaccia}
\begin{itemize}
    \item \textbf{Codice:} MPD8
    \item \textbf{Descrizione:} È sconsigliato per le interfacce utente l'utilizzo di una struttura troppo profonda per questioni di usabilità. L'utente non deve fare troppi passaggi per raggiungere la funzionalità desiderata;
    \item \textbf{Attributo di riferimento:} Operabilità\ap{G};
    \item \textbf{Sigla:} $PSI$
    \item \textbf{Range di valori che può assumere:}
    \begin{itemize}
        \item \textbf{Accettabile:} $1 \leq PSI \leq 5$
        \item \textbf{Ottimale:} $1 \leq PSI \leq 3$
    \end{itemize}
\end{itemize}

\rowcolors{2}{grigetto}{white}
\renewcommand{\arraystretch}{1.5}
\begin{longtable}{ c C{4cm} c c c}
\caption{Tabella metriche interne del prodotto}\\
\rowcolor{darkblue}
\textcolor{white}{\textbf{Metrica}} & \textcolor{white}{\textbf{Nome}} & \textcolor{white}{\textbf{Sigla}} & \textcolor{white}{\textbf{Range Accettabile}} & \textcolor{white}{\textbf{Range Ottimale}}\\
    MPD1 & Aderenza agli Standard di Funzioni o Interfacce & $ASFI$ & $ASFI = 100\%$ & $ASFI = 100\%$\\
    MPD2 & Rilevamento dei Difetti & $RD$ & $ RD \geq 80\% $ & $RD = 100\%$\\
    MPD3 & Validità dei Dati d'Input & $VDI$ &  $VDI \geq 90\%$ &  $VDI = 100\%$\\
    MPD4 & Attrattività della User Interface (UI) & $AUI$ & $AUI = Attrattivo$ &  $AUI = Molto \; attrattivo$\\
    MPD5 & Complessità Ciclomatica del Software & $CCS $ & $1 \leq CCS \leq 7 $ & $1 \leq CCS \leq 4$\\
    MPD6 & Unità Documentate & $UD$ & $UD \geq 90\%$ & $UD = 100\%$\\
\end{longtable} 

\newpage %adattamento pagina pdf

\rowcolors{2}{grigetto}{white}
\renewcommand{\arraystretch}{1.5}
\begin{longtable}{ c C{4cm} c c c}
\caption{Tabella metriche esterne del prodotto}\\
\rowcolor{darkblue}
\textcolor{white}{\textbf{Metrica}} & \textcolor{white}{\textbf{Nome}} & \textcolor{white}{\textbf{Sigla}} & \textcolor{white}{\textbf{Range Accettabile}} & \textcolor{white}{\textbf{Range Ottimale}}\\
    MPD7 & Maturità dei Test & $MT$ & $MT = 100\%$ & $MT = 100\%$\\
    MPD8 & Profondità Strutturale dell'Interfaccia & $PSI$ & $1 \leq PSI \leq 5$ &$1 \leq PSI \leq 3$\\
\end{longtable}
