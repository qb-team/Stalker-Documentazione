
\section{Qualità di prodotto}
\subsection{Metriche interne}
 
  \subsubsection{Funzionalità}
      Capacità del prodotto software di soddisfare i requisiti funzionali e le necessità degli utenti.\\
      \paragraph{Metrica -  Accuratezza delle funzioni sviluppate} 
      \begin{itemize}
          \item  \textbf{Codice:} MPD-01
            \item  \textbf{Descrizione:} Misurare il livello di accuratezza con cui sono sviluppate le funzionalità richieste
            \item  \textbf{Attributo di riferimento:} Accuratezza 
           \item   \textbf{Sigla:} AFS
           \item   \textbf{Formula:} \begin{math}AFS = \frac{A}{B}\end{math}\\ \\
              A = numero di funzioni sviluppate con l'accuratezza richiesta;\\
              B = numero totale di funzioni sviluppate;
                    \item \textbf{Range di valori che può assumere:}
        \begin{itemize}
            \item \textbf{Accettabile:} 
            \item \textbf{Ottimale:} 
        \end{itemize}
       \end{itemize}
              
              \paragraph{Metrica - Aderenza delle funzioni e/o delle interfacce} 
              \begin{itemize}
          \item  \textbf{Codice:} MPD-02
        \item    \textbf{Descrizione:} Misurare il livello di aderenza delle funzioni e delle interfacce sviluppate rispetto agli standard, alle normative e alla regolamentazioni
          \item  \textbf{Attributo di riferimento:} Aderenza alle funzionalità 
        \item    \textbf{Sigla:} AFI
         \item   \textbf{Formula:} \begin{math}AFI = \frac{A}{B}\end{math}\\ \\
              A = numero di funzioni(e/o interfacce) sviluppate che risultano aderenti a standard, regole e normative emesse al riguardo; \\
              B = numero totale di funzioni(e/o interfacce) che devono essere aderenti a tali regole come descritto nelle specifiche;
         \item  \textbf{Range di valori che può assumere:}
               \end{itemize}
  \subsubsection{Affidabilità} 
  Capacità di predire se il prodotto software in questione potrà soddisfare i requisiti prescritti per l'affidabilità dal punto di vista degli sviluppatori.\
                \paragraph{Metrica - Rilevamento dei difetti} 
                  \begin{itemize}
         \item   \textbf{Codice:} MPD-03
         \item   \textbf{Descrizione:} Misurare l'efficacia nel rilevare i difetti presenti nel software durante le diverse fasi di sviluppo del prodotto
        \item    \textbf{Attributo di riferimento:} Maturità
        \item    \textbf{Sigla:} RDF
        \item    \textbf{Formula:} \begin{math} RDF = \frac{A}{B}\end{math}\\ \\
             A = numero di difetti rilevati nelle revisioni tecniche, ispezioni e test del prodotto in ciascuna fase di sviluppo;\\
              B = numero totale di difetti previsti nella fase di sviluppo;
      \item    \textbf{Range di valori che può assumere:}      
                  \end{itemize}
                  
                  \paragraph{Metrica - Livello di controllo dei guasti}
                     \begin{itemize}
        \item    \textbf{Codice:} MPD-04
        \item    \textbf{Descrizione:} Misurare il numero di condizioni di errore messe sotto controllo per evitare i guasti al prodotto
         \item   \textbf{Attributo di riferimento:} Tolleranza ai guasti
       \item     \textbf{Sigla:} LCG
       \item     \textbf{Formula:} \begin{math} LCG = \frac{A}{B}\end{math}\\ \\
             A = numero di condizione di errore gestite correttamente;\\
              B = numero totale di condizioni di errori possibili nel sistema;
       \item    \textbf{Range di valori che può assumere:}  
       \end{itemize}
           
\subsubsection{Usabilità} 
Capacità del prodotto software di essere comprensibile, di poter essere usato e compreso facilmente, in ogni sua parte, da qualsiasi utente che lo voglia usare. \\

	 \paragraph{Metrica - Validità dei dati d'input} 
	    \begin{itemize}
          \item  \textbf{Codice: } MPD-05
           \item \textbf{Descrizione:} Misurare il livello di correttezza dei dati forniti in input all'applicazione
         \item   \textbf{Attributo di riferimento:} Operabilità
          \item  \textbf{Sigla:} VDI
         \item   \textbf{Formula:}\begin{math} VDI = \frac{A}{B}\end{math}\\ \\
             A = numero dei dati di input di cui si effettua il controllo di validità;\\
             B = numero totale di dati di input previsti;
           \item  \textbf{Range di valori che può assumere:}
           \end{itemize}
             
                \paragraph{Metrica - Completezza delle funzioni per gli utenti} 
                   \begin{itemize}
         \item   \textbf{Codice: } MPD-06
         \item   \textbf{Descrizione:} Misurare quale percentuale di funzioni siano comprensibili agli utenti
         \item   \textbf{Attributo di riferimento:} Comprensibilità
          \item  \textbf{Sigla:} CFU
          \item  \textbf{Formula:}\begin{math}CFU = \frac{A}{B}\end{math}\\ \\
             A = numero di di funzioni presenti nelle interfacce utente e giudicate comprensibili da parte loro;\\
              B = numero totale di funzioni previste nelle interfacce utente;
        \item   \textbf{Range di valori che può assumere:}
        \end{itemize}
           
                   \paragraph{Metrica - Attrattività delle interfacce utente} 
                      \begin{itemize}
          \item  \textbf{Codice: } MPD-07
          \item  \textbf{Descrizione:} Misurare quanto attrattive risultino le interfacce agli utenti dal punto di vista grafico 
          \item  \textbf{Attributo di riferimento:} Attrattività
          \item  \textbf{Sigla:} AIU
           \item \textbf{Formula:}\begin{math}AIU = V (q) \end{math}\\ \\
             Valore medio dei risultati di un questionario compilato da almeno tre utenti.
          \item \textbf{Range di valori che può assumere:} Può essere utilizzata una scala a quattro valori: Molto attrattivo, Attrattivo, Poco attrattivo, Non Attrattivo.
           \end{itemize}
           
 \subsubsection{Efficienza}
 Capacità del prodotto software di realizzare le funzioni richieste nel minor tempo possibile. Inoltre con la misurazione dell’efficienza si vuole anche ridurre il numero di risorse usate dal software per eseguire le funzionalità offerte.
 
            \paragraph{Metrica - Tempo di risposta } 
               \begin{itemize}
          \item  \textbf{Codice:} MPD-08
          \item  \textbf{Descrizione:} Capacità del prodotto software di realizzare le funzioni richieste nel minor tempo possibile
         \item   \textbf{Attributo di riferimento:} Comportamento rispetto al tempo
         \item   \textbf{Sigla:} TR
         \item   \textbf{Formula:} Viene calcolato prima in base alle caratteristiche del sistema operativo e di quelle del progetto e dopo in base al codice del prodotto software.
         \item  \textbf{Range di valori che può assumere:}
            \end{itemize}
            
               \paragraph{Metrica - Utilizzo delle Risorse} 
                  \begin{itemize}
         \item   \textbf{Codice:} MPD-09
         \item   \textbf{Descrizione:}Misurare la quantità di risorse utilizzate dal sistema per completare una singola attività o un lavoro completo(insieme di attività)
          \item  \textbf{Attributo di riferimento:} Utilizzo di risorse
          \item  \textbf{Sigla:} UR
           \item \textbf{Formula:} Viene calcolato come numero e dimensione delle risorse impiegate per svolgere una determinata attività o un insieme di attività.
        \item   \textbf{Range di valori che può assumere:}
      \end{itemize}
      
               \paragraph{Metrica - Aderenza all'efficienza} 
                  \begin{itemize}
          \item  \textbf{Codice:} MPD-10
          \item  \textbf{Descrizione:} Misurare il livello di aderenza del prodotto sviluppato in relazione agli standard, alle normative e alle regolamentazioni previsti per l'efficienza
          \item  \textbf{Attributo di riferimento:} Aderenza all'efficienza 
          \item  \textbf{Sigla:} AE
          \item  \textbf{Formula:} \begin{math}AE = \frac{A}{B}\end{math}\\ \\
            A = numero di elementi sviluppati che risultano essere aderenti a tali standard, regole e normative;\\
            B = numero totale di elementi previsti nelle specifiche che vedono aderenti a tali regole;
         
         \item  \textbf{Range di valori che può assumere:}
      \end{itemize}
      
      
      
      
           
    \subsubsection{Manutenibilità} 
    Capacità di predire il livello di impegno richiesto per modificare il prodotto software dal punto di vista degli sviluppatori.
    
        \paragraph{Metrica - Diagnostica} 
           \begin{itemize}
          \item  \textbf{Codice:} MPD-11
          \item  \textbf{Descrizione:} Misurare il livello di diagnostica che il prodotto consente tramite le apposite funzioni 
          \item  \textbf{Attributo di riferimento:} Analizzabilità
         \item   \textbf{Sigla:} D
          \item  \textbf{Formula:} \begin{math}D = \frac{A}{B}\end{math}\\ \\
            A = numero di funzioni di diagnostica sviluppate;\\
            B = numero totale di funzioni di diagnostica previste nelle specifiche;
         \item  \textbf{Range di valori che può assumere:}
           \end{itemize}
           
           \paragraph{Metrica - Complessità del software} 
              \begin{itemize}
         \item   \textbf{Codice:} MPD-12
         \item   \textbf{Descrizione:} Misurare la complessità ciclomatica dei singoli moduli sviluppati
          \item  \textbf{Attributo di riferimento:} Modificabilità
          \item  \textbf{Sigla:} CF
         \item   \textbf{Formula:} \begin{math}CF = v(G) = e - n + 2 \end{math}\\ \\
            G = grafo del modulo;\\
            e = cammino;\\
            n = nodo; 
         \item  \textbf{Range di valori che può assumere:}
\end{itemize}

\paragraph{Metrica - Registrazione delle modifiche} 
   \begin{itemize}
          \item  \textbf{Codice:} MPD-13
         \item   \textbf{Descrizione:} Misurare se tutte le modifiche apportate al software sono commentate nei singoli moduli e nella documentazione tecnica
         \item   \textbf{Attributo di riferimento:} Modificabilità
         \item   \textbf{Sigla:} RM
         \item   \textbf{Formula:} \begin{math}RM= \frac{A}{B}\end{math}\\ \\
            A = numero di modifiche commentate nel codice e nella documentazione tecnica;\\
            B = numero totale di modifiche eseguite;
       \item    \textbf{Range di valori che può assumere:}
           \end{itemize}
           
           \paragraph{Metrica - Impatto delle modifiche} 
              \begin{itemize}
         \item   \textbf{Codice:} MPD-14
          \item  \textbf{Descrizione:} Misurare l'impatto negativo sulla corretta esecuzione del software procurato dalle modifiche al codice
         \item   \textbf{Attributo di riferimento:} Stabilità
         \item   \textbf{Sigla:} IM
         \item   \textbf{Formula:} \begin{math}IM= \frac{A}{B}\end{math}\\ \\
            A = numero di modifiche che hanno procurato un malfunzionamento del software o che hanno influito negativamente sulle prestazioni;\\
            B = numero totale di modifiche eseguite;
          \item \textbf{Range di valori che può assumere:}
          \end{itemize}
           
\subsection{Metriche esterne}
   \subsubsection{Funzionalità}
        \paragraph{Metrica - } 
            \textbf{Codice:}
            \textbf{Descrizione:}
            \textbf{Attributo di riferimento:} (completezza)
            \textbf{Sigla:}
            \textbf{Formula:}
           \textbf{Range di valori che può assumere:}
           

           

