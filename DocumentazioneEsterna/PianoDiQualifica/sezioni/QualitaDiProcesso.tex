\section{Qualità di processo}
Per garantire la qualità dei \glo{processi} si utilizza come riferimento lo standard ISO/IEC 12207:1995. Dopo uno studio dettagliato di tale documento sono stati scelti i \glo{processi}
e le attività da utilizzare. Il tutto è stato semplificato ed adattato in base alle esigenze del progetto. Come è previsto nello standard, tutti i \glo{processi} e le attività sono raccolti 
nei \glo{processi}: primari, di supporto ed organizzativi. Le attività fanno parte a dei sotto \glo{processi} rispetto a quelli appena elencati e per questo motivo la loro 
appartenenza è resa chiara nella sezione: \glo{Processo} di riferimento.

\subsection{Processi primari}

\subsubsection{Analisi dei requisiti}
   \rowcolors{2}{grigetto}{white}
   \renewcommand{\arraystretch}{1.5}
   \begin{longtable}{ c C{4cm} c C{3.5cm} C{3.5cm}}
   	\caption{Tabella metriche per l'analisi dei requisiti}\\
   	\rowcolor{darkblue}
   	\textcolor{white}{\textbf{Metrica}} & \textcolor{white}{\textbf{Nome}} & \textcolor{white}{\textbf{Sigla}} & \textcolor{white}{\textbf{Range Accettabile}} & \textcolor{white}{\textbf{Range Ottimale}}\\
   	MPC1 & Percentuale Requisiti Soddisfatti & $PRS$ & $PRS = 100\%$ & $PRS = 100\%$ \\
   \end{longtable}
\vspace{0.3cm}
\subsubsection{Progettazione architetturale}

\paragraph{Metrica MPC2 - Structural Fan-In}
\begin{itemize}
	\item \textbf{Descrizione:} Numero di procedure che chiama questa procedura;
\end{itemize}
\paragraph{Metrica MPC3 - Structural Fan-Out}
\begin{itemize}
	\item \textbf{Descrizione:} Numero di procedure che questa procedura chiama;
\end{itemize}
\vspace{0.3cm}
\subsubsection{Progettazione di dettaglio}
\rowcolors{2}{grigetto}{white}
\renewcommand{\arraystretch}{1.5}
\begin{longtable}{ c C{4cm} c C{3.5cm} C{3.5cm}}
	\caption{Tabella metriche per progettazione di dettaglio}\\
	\rowcolor{darkblue}
	\textcolor{white}{\textbf{Metrica}} & \textcolor{white}{\textbf{Nome}} & \textcolor{white}{\textbf{Sigla}} & \textcolor{white}{\textbf{Range Accettabile}} & \textcolor{white}{\textbf{Range Ottimale}}\\
    MPC2 & Coupling Between Objects & $CBO$ & $0 \leq CBO \leq 4$ & $0 \leq CBO \leq 2$ \\
    MPC3 & Livello Profondità Gerarchia & $LPG$ &  $1 \leq LPG \leq 3$ &  $1 \leq LPG \leq 2$ \\
\end{longtable}
\newpage
\subsubsection{Codifica}  
     \rowcolors{2}{grigetto}{white}
     \renewcommand{\arraystretch}{1.5}
     \begin{longtable}{ c C{4cm} c C{3.5cm} C{3.5cm}}
     	\caption{Tabella metriche per la codifica}\\
     	\rowcolor{darkblue}
     	\textcolor{white}{\textbf{Metrica}} & \textcolor{white}{\textbf{Nome}} & \textcolor{white}{\textbf{Sigla}} & \textcolor{white}{\textbf{Range Accettabile}} & \textcolor{white}{\textbf{Range Ottimale}}\\
		MPC4 & Numero di Parametri per Metodo & $NPM$ & $0 < NPM < 8$ & $ 0 < NPM < 4$ \\
		MPC5 & Linee di Codice per Linee di Commento & $LCLC$ & $LCLC \geq 0.25$ & $LCLC \geq 0.30$ \\
	\end{longtable}
\vspace{0.3cm}
\subsection{Processi supporto}

\subsubsection{Documentazione}
    \rowcolors{2}{grigetto}{white}
    \renewcommand{\arraystretch}{1.5}
    \begin{longtable}{ c C{4cm} c C{3.5cm} C{3.5cm}}
    	\caption{Tabella metriche per la documentazione}\\
    	\rowcolor{darkblue}
    	\textcolor{white}{\textbf{Metrica}} & \textcolor{white}{\textbf{Nome}} & \textcolor{white}{\textbf{Sigla}} & \textcolor{white}{\textbf{Range Accettabile}} & \textcolor{white}{\textbf{Range Ottimale}}\\
    	MPC6 & Indice di Gulpease & $IG$ & $40 < IG < 100$ & $80 < IG < 100$ \\
    \end{longtable}
\vspace{0.3cm}
\subsubsection{Verifica}
    \rowcolors{2}{grigetto}{white}
    \renewcommand{\arraystretch}{1.5}
    \begin{longtable}{ c C{4cm} c C{3.5cm} C{3.5cm}}
    	\caption{Tabella metriche per la verifica}\\
    	\rowcolor{darkblue}
    	\textcolor{white}{\textbf{Metrica}} & \textcolor{white}{\textbf{Nome}} & \textcolor{white}{\textbf{Sigla}} & \textcolor{white}{\textbf{Range Accettabile}} & \textcolor{white}{\textbf{Range Ottimale}}\\
    	MPC7 & Code Coverage & $CC$ & $CC = 80\%$ & $CC = 100\%$  \\
    \end{longtable}
\vspace{0.3cm}
\subsection{Processi organizzativi}

\subsubsection{Pianificazione}
\rowcolors{2}{grigetto}{white}
\renewcommand{\arraystretch}{1.5}
\begin{longtable}{ c C{4cm} c C{3.5cm} C{3.5cm}}
	\caption{Tabella metriche per la pianificazione}\\
	\rowcolor{darkblue}
	\textcolor{white}{\textbf{Metrica}} & \textcolor{white}{\textbf{Nome}} & \textcolor{white}{\textbf{Sigla}} & \textcolor{white}{\textbf{Range Accettabile}} & \textcolor{white}{\textbf{Range Ottimale}}\\
		MPC8 & Actual Cost of Work Performed & $ACWP$ & $0 \leq ACWP \leq BCWS$ & $0 \leq ACWP \leq B_{tot}$ \\
		MPC9 & Budgeted Cost of Work Scheduled & $BCWS$ & $BCWS \geq 0$ &  $BCWS \geq 0$ \\
		MPC10 & Budgeted Cost of Work Performed & $BCWP$ & $BCWP \geq 0$ & $BCWP \geq 0$ \\
		MPC11 & Schedule Variance & $SV$ & $SV = 0$ & $SV > 0$  \\	
		MPC12 & Budget Variance & $BV$ & $0 \leq BV < ACWP$ & $0 \leq BV \leq B_{tot}$  \\
	\end{longtable}