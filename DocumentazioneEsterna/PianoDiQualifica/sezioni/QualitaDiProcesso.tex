\section{Qualità di processo}
Per garantire la qualità dei \glo{processi} si utilizza come riferimento lo standard ISO/IEC 12207:1995. Dopo uno studio dettagliato di tale documento sono stati scelti i \glo{processi}
e le attività da utilizzare. Il tutto è stato semplificato ed adattato in base alle esigenze del progetto. Come è previsto nello standard, tutti i \glo{processi} e le attività sono raccolti 
nei \glo{processi}: primari, di supporto ed organizzativi. Le attività fanno parte a dei sotto \glo{processi} rispetto a quelli appena elencati e per questo motivo la loro 
appartenenza è resa chiara nella sezione: \glo{Processo} di riferimento.

\subsection{Processi primari}

\subsubsection{Analisi dei requisiti}
    \paragraph{Metrica MPC1 - Percentuale requisiti obbligatori soddisfatti}
    \begin{itemize}
        \item \textbf{Descrizione:} È la percentuale dei requisiti che devono essere soddisfatti;
        \item \textbf{Formula:} $$PRS = {|requisiti \; soddisfatti| \over |requisiti \; totali|}\; \cdot \; 100$$
        \item \textbf{Range di valori che può assumere:}
        \begin{itemize}
            \item \textbf{Accettabile:} $PRS = 100\%$
            \item \textbf{Ottimale:} $PRS = 100\%$
        \end{itemize}
    \end{itemize} 
%\subsubsection{Progettazione architetturale}

%\paragraph{Metrica MPC2 - Structural Fan-In}
%\begin{itemize}
%	\item \textbf{Descrizione:} Numero di procedure che chiama questa procedura;
%\end{itemize}
%\paragraph{Metrica MPC3 - Structural Fan-Out}
%\begin{itemize}
%	\item \textbf{Descrizione:} Numero di procedure che questa procedura chiama;
%\end{itemize}
\subsubsection{Progettazione di dettaglio}
    \paragraph{Metrica MPC2 - Incapsulamento CBO}
    \begin{itemize}
        \item \textbf{Descrizione:} Il "Coupling Between Objects" misura il numero delle classi correlate ad una in esame al di fuori dalla gerarchia di ereditarietà. Più è alto il grado di coupling della classe in esame e più il sistema è difficile da mantenere;
        \item \textbf{Formula:} $$CBO = {\sum_{i=1}^{N} C_i}$$
        con:
        \begin{itemize}
            \item $N$ = numero classi non appartenenti alla gerarchia di ereditarietà della classe in esame;
            \item $C_i$ =
            \begin{math} {
                \begin{cases}
                    1, & la \; $i$-esima \; classe \; \grave{e} \; correlata \; a \; quella \; in \; esame \\
                    0, & la \; $i$-esima \; classe \; non \; \grave{e} \; correlata \; a \; quella \; in \; esame
                \end{cases}
            }
            \end{math}
        \end{itemize}
        \item \textbf{Range di valori che può assumere:}
        \begin{itemize}
            \item \textbf{Accettabile:} $0 \leq{} CBO \leq 4$
            \item \textbf{Ottimale:} $0 \leq{} CBO \leq 2$
        \end{itemize}
    \end{itemize}

    \paragraph{Metrica MPC3 - Livello profondità gerarchia}
    \begin{itemize}
        \item \textbf{Descrizione:} È il valore intero che indica la profondità della gerarchia formata tra classi. Se una gerarchia è formata da una classe allora il valore è uguale a 1;
        \item \textbf{Range di valori che può assumere:}
        \begin{itemize}
            \item \textbf{Accettabile:} $1 \leq{} LPG \leq 3$
            \item \textbf{Ottimale:} $1 \leq{} LPG \leq 2$
        \end{itemize}
    \end{itemize}

\subsubsection{Codifica}  
    \paragraph{Metrica MPC4 - Numero di parametri per metodo} 
    \begin{itemize}
        \item \textbf{Descrizione:} Un numero elevato di parametri per metodo può indicare il bisogno di ridurre funzionalità associate a tale metodo. Più è grande questo valore e più la possibilità aumenta nel commettere errori progettuali;
        \item \textbf{Range di valori che può assumere:}
        \begin{itemize}
            \item \textbf{Accettabile:} $0 \leq{} NPM \leq 8$
            \item \textbf{Ottimale:} $0 \leq{} NPM \leq 4$
        \end{itemize}
    \end{itemize}

    \paragraph{Metrica MPC5 - Linee di commento per linee di codice}
    \begin{itemize}
        \item \textbf{Descrizione:} È il rapporto tra linee di commento e linee di codice. Per le linee di codice si intende Logical SLOC il numero di linee di codice effettive che corrispondono al numero di statement;
        \item \textbf{Formula:}$$LCLC = {|linee \; di \; commento| \over |linee \; di \; codice|}$$
        \item \textbf{Range di valori che può assumere:}
        \begin{itemize}
            \item \textbf{Accettabile:} $LCLC \geq 0.25$
            \item \textbf{Ottimale:} $LCLC \geq 0.30$
        \end{itemize}
    \end{itemize}

\subsection{Processi supporto}

\subsubsection{Documentazione}
    \paragraph{Metrica MPC6 - Indice di Gulpease}
    \begin{itemize}
        \item \textbf{Descrizione:} È l'indice di leggibilità di un determinato testo. Calcola la lunghezza delle parole e delle frasi rispetto al numero totale delle lettere. Il valore è un intero da 0 a 100; se esso è inferiore a 80 sarà difficile da leggere per chi ha la licenza elementare, mentre se è inferiore a 40 sarà difficili da leggere per chi ha un diploma superiore;
        \item \textbf{Formula:} $$IG = 89 + {{300 \; \cdot \; |frasi| \; - \; 10 \; \cdot \; |lettere|}\over |parole|}$$
        \item \textbf{Range di valori che può assumere:}
        \begin{itemize}
            \item \textbf{Accettabile:} $40 < IG < 100$
            \item \textbf{Ottimale:} $80 < IG < 100$
        \end{itemize}
    \end{itemize}

\subsubsection{Verifica}
    \paragraph{Metrica MPC7 - Code Coverage}
    \begin{itemize}
        \item \textbf{Descrizione:} È la percentuale di copertura del codice attraversato dai test rispetto al totale del codice di base. Per dare una misurazione in termini di grandezza si adoperano le linee di codice come riferimento;
        \item \textbf{Formula:} $$CC = {|linee \; di \; codice \; percorse \; dai  \; test| \over |linee \; di \; codice \; totali|} \; \cdot \; 100$$
        \item \textbf{Range di valori che può assumere:}
        \begin{itemize}
            \item \textbf{Accettabile:} $CC = 80\%$
            \item \textbf{Ottimale:} $CC = 100\%$
        \end{itemize}
    \end{itemize}

\subsection{Processi organizzativi}

\subsubsection{Pianificazione}

\paragraph{Metrica MPC8 - Actual Cost of Work Performed}
\begin{itemize}
	\item \textbf{Descrizione:} Denaro speso fino al momento del calcolo;
	\item \textbf{Formula:} $$ACWP = {Somma\; delle\; ore\; lavorative\; moltiplicate\; con\; il\; corrispondente\; costo\; orario}$$
	con:
	\begin{itemize}
		\item $B_{tot}$ = Budget totale.
	\end{itemize}
	 \begin{itemize}
		\item \textbf{Accettabile:}  $0 \leq ACWP \leq BCWS$
		\item \textbf{Ottimale:} $0 \leq ACWP \leq B_{tot}$
	\end{itemize}
\end{itemize}

\paragraph{Metrica MPC9 - Budgeted Cost of Work Scheduled}
\begin{itemize}
	\item \textbf{Descrizione:} Costo pianificato per realizzare le attività di progetto fino al momento del calcolo;
	\item \textbf{Formula:} $$BCWS = {B_{tot} * \% di\; lavoro\; pianificato}$$
	\begin{itemize}
		\item $B_{tot}$ = Budget totale.
	\end{itemize}
	\begin{itemize}
		\item \textbf{Accettabile:} $BCWS \geq 0$
		\item \textbf{Ottimale:} $BCWS \geq 0$
	\end{itemize}
\end{itemize}

\paragraph{Metrica MPC10 - Budgeted Cost of Work Performed}
\begin{itemize}
	\item \textbf{Descrizione:} Valore del lavoro fatto fino al momento del calcolo, ovvero la variazione del numero di requisiti soddisfatti nel periodo in cui la metrica viene calcolata;
	\item \textbf{Formula:} $$BCWP = {B_{tot} * \% di\; lavoro\; effettivamento\; fatto}$$
	 	\begin{itemize}
	 	\item $B_{tot}$ = Budget totale.
	 \end{itemize}
	 \begin{itemize}
	 	\item \textbf{Accettabile:} $BCWP \geq 0$
	 	\item \textbf{Ottimale:} $BCWP \geq 0$
	 \end{itemize}

\end{itemize}

    \paragraph{Metrica MPC11 - Schedule Variance}
    \begin{itemize}
        \item \textbf{Descrizione:} È il valore che indica se si è in linea ($=0$), in anticipo ($>0$) oppure in ritardo ($<0$) rispetto alla schedulazione delle attività di progetto pianificate nella \glo{baseline};
        \item \textbf{Formula:} $$SV = {BCWP \; - \; BCWS}$$
        con:
        \begin{itemize}
            \item $BCWP$ = Budgeted Cost of Work Performed (valore delle attività eseguite nella data corrente);
            \item $BCWS$ = Budgeted Cost of Work Scheduled (costo pianificato per la realizzazione delle attività di progetto alla data corrente);
        \end{itemize}
        \item \textbf{Range di valori che può assumere:}
        \begin{itemize}
            \item \textbf{Accettabile:} $SV = 0$
            \item \textbf{Ottimale:} $SV > 0$
        \end{itemize}
    \end{itemize}

    \paragraph{Metrica MPC12 - Budget Variance}
        \begin{itemize}
            \item \textbf{Descrizione:} È il valore che indica se alla data corrente si è speso di più ($>0$) o di meno ($<0$) rispetto a quanto pianificato dal budget totale $B_{tot}$;
            \item \textbf{Formula:} $$BV = {BCWS \; - \; ACWP}$$
            con:
            \begin{itemize}
                \item $BCWS$ = Budgeted Cost of Work Scheduled (costo pianificato per la realizzazione delle attività di progetto alla data corrente);
                \item $ACWP$ = Actual Cost of Work Performed (costo effettivamente sostenuto alla data corrente);
                \item $B_{tot}$ = Budget totale.
            \end{itemize}
            \item \textbf{Range di valori che può assumere:}
            \begin{itemize}
                \item \textbf{Accettabile:} $0 \leq BV < ACWP$
                \item \textbf{Ottimale:} $0 \leq BV \leq B_{tot}$
            \end{itemize}
        \end{itemize}

\subsection{Tabella riassuntiva metriche dei processi}
    \rowcolors{2}{grigetto}{white}
    \renewcommand{\arraystretch}{1.5}
    \begin{longtable}{ c C{4cm} c c c}
    \caption{Tabella metriche dei processi}\\
    \rowcolor{darkblue}
    \textcolor{white}{\textbf{Metrica}} & \textcolor{white}{\textbf{Nome}} & \textcolor{white}{\textbf{Sigla}} & \textcolor{white}{\textbf{Range Accettabile}} & \textcolor{white}{\textbf{Range Ottimale}}\\
    MPC1 & Percentuale Requisiti Soddisfatti & $PRS$ & $PRS = 100\%$ & $PRS = 100\%$ \\
    MPC2 & Coupling Between Objects & $CBO$ & $0 \leq CBO \leq 4$ & $0 \leq CBO \leq 2$ \\
    MPC3 & Livello Profondità Gerarchia & $LPG$ &  $1 \leq LPG \leq 3$ &  $1 \leq LPG \leq 2$ \\
    MPC4 & Numero di Parametri per Metodo & $NPM$ & $0 < NPM < 8$ & $ 0 < NPM < 4$ \\
    MPC5 & Linee di Codice per Linee di Commento & $LCLC$ & $LCLC \geq 0.25$ & $LCLC \geq 0.30$ \\
    MPC6 & Indice di Gulpease & $IG$ & $40 < IG < 100$ & $80 < IG < 100$ \\
    MPC7 & Code Coverage & $CC$ & $CC = 80\%$ & $CC = 100\%$  \\
    MPC8 & Actual Cost of Work Performed & $ACWP$ & $0 \leq ACWP \leq BCWS$ & $0 \leq ACWP \leq B_{tot}$ \\
    MPC9 & Budgeted Cost of Work Scheduled & $BCWS$ & $BCWS \geq 0$ &  $BCWS \geq 0$ \\
    MPC10 & Budgeted Cost of Work Performed & $BCWP$ & $BCWP \geq 0$ & $BCWP \geq 0$ \\
    MPC11 & Schedule Variance & $SV$ & $SV = 0$ & $SV > 0$  \\	
    MPC12 & Budget Variance & $BV$ & $0 \leq BV < ACWP$ & $0 \leq BV \leq B_{tot}$  \\
    \end{longtable}