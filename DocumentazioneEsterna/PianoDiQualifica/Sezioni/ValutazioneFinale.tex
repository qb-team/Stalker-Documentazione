\section{Valutazione finale sulla vita del progetto}
Analizzando a posteriori tutto il nostro percorso effettuato durante il ciclo di vita del progetto è possibile riscontrare che ogni membro del gruppo \Gruppo{} è maturato acquisendo maggiori competenze, conoscenze e abilità. La situazione di partenza di ogni membro del gruppo era di totale inesperienza sia nello svolgere i compiti per ogni ruoli e sia nella attuazione dei processi. Grazie all'adozione di buone prassi e con l'autovalutazione su ciò che si effettuava, è stato possibile acquisire maggiore dimestichezza nello svolgere in modo corretto ogni ruolo. L'utilizzo del metriche è stato all'inizio un po' travagliato, dato che si erano adottate delle metriche poco utili e l'utilizzo di esse, non era entrato appieno nel \glo{way of working}, dato che venivano considerate più un onere che un strumento utile. Ad ogni modo il gruppo, con il maturare dell'esperienza, ha capito l'utilità essenziale di dotarsi di buone metriche per tenere sotto controllo l'andamento del prodotto. Altra nozione che è stata difficile da apprendere inizialmente dal gruppo è il funzionamento del modello di sviluppo incrementale, infatti inizialmente gli incrementi erano mal pianificati e i progressi che venivano fatti non venivano verificati, rischiando di trasformare l'incremento in un iterazione. Con il tempo il gruppo ha capito i proprio sbagli, infatti per correggere questa lacune, il gruppo ha attuato una corretta pianificazione e verificato ogni sviluppo fatto. In conclusione grazie alla maturazione fatta con la pratica e con lo studio è stato possibile adottarsi di buone pratiche e dei giusti processi.