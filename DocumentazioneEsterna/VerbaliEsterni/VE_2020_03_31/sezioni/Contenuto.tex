\section{Informazioni Generali}
\begin{itemize}
\item \textbf{Luogo:} Hangouts;
\item \textbf{Data:} \Data;
\item \textbf{Ora:} 13.35-14.10;
\item \textbf{Partecipanti del gruppo:}
	\begin{itemize}
	\item \AT{}; 
	\item \CE{}; 
	\item \DF{};
	\item \LD{};
	\item \PF{};
	\item \SE{};
	\item \BR{};
	\item \MC{}.
	\end{itemize} 
\item \textbf{Segretario:} \LD{}.
\end{itemize}


\section{Ordine del Giorno}
\begin{itemize}
	\item Chiamata con il prof \CR{} tramite Hangouts;
	\item Discussione e chiarimenti sulle correzioni al documento \AdR{} riportate dopo la RP;
\end{itemize}

\section{Resoconto}
\subsection{Relazioni errate}
Durante la chiamata sono stati chiariti numerosi errori presenti nella \AdR{}.
Per quanto riguarda le relazioni da rivedere bisogna tenere a mente che le estensioni non fanno verificare lo scenario principale del caso 
d'uso da cui partono, quindi nel nostro caso bisogna rivederne alcune cercando di modellarle come generalizzazioni in modo da non escludere il corretto 
funzionamento del caso d'uso generale.\\
Un'altra cosa da segnalare è la precondizione nelle generalizzazioni deve essere più stringente rispetto al caso d'uso generale.\\
\subsection{UCA 6.1.1}
Serve un sotto caso d'so diverso, perchè non stiamo aggiungendo funzionlità ma stiamo specificando più in dettaglio come funziona.
\subsection{Relazioni di inclusione}
Come va intesa l'inclusione: ho uno scenario principale di un caso d'uso A, se includo B allora prima eseguo lo scenario principale di A e poi quello di B;
nel nostro caso invece se intendiamo che lo scenario principale si compone di 3 passi non possiamo usare inclusioni ma bisogna modellarli come 3 sotto-casi d'uso diversi.
Quindi UCS 5.2.1 deve essere un sotto-caso di UCS 5.2, perché specifico come è fatto il suo scenario principale\\
In realtà potrebbe anche essere rimosso e inglobato nello scenario principale perché troppo semplice per essere un sotto-caso separato.

\subsection{Angular}
Per fare i diagrammi identifichiamo i component come classi\\
Per l'aggiunta dinamica di elemneti grafici tramite i tag, dato che fanno parte della UI, non serve farci i diagrammi perché non portano vantaggi.
\subsection{differenza tra aggregazione e associazione}
Sono abbastanza intercambiabili, dipende dai vincoli che imponiamo sull'architettura.
Nella composizione si aggiunge che le istanze sono locali e non condivisibili.
 




\clearpage