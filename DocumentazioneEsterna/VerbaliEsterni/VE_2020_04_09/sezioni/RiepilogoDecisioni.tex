 
\section{Riepilogo delle decisioni}
{
\rowcolors{2}{white}{grigetto}
\renewcommand{\arraystretch}{1.5}
\centering
\begin{longtable}{ >{\centering}p{0.20\textwidth} >{}p{0.70\textwidth}}

\caption{Risposte e consensi ai dubbi affrontati durante la riunione esterna del \Data}\\

\rowcolor{darkblue}

\textcolor{white}{\textbf{Codice}} & \textcolor{white}{\textbf{Decisione}} \\	
		
VE\_\Data.1 & Se i rischi a cui andiamo incontro sono asincroni (inaspettati) vanno affrontati nel modo in cui abbiamo descritti nel \PdP{} nella sezione di Analisi dei Rischi (con la speranza che siano indicati), in modo manualistico, per poterli risolvere nel modo più corretto. Rimane il fatto che bisogna fare un check periodico per intercettare correttamente i rischi attesi durante tutto il corso del progetto. \\
		
VE\_\Data.2 & Lo sviluppo incrementale, per essere considerato utile, deve avere un periodo di attuazione degli incrementi relativamente breve, in modo da ottenere un feedback utile sull'andamento del prodotto.
Ciò comporta una pianificazione nel breve periodo ed incrementi di breve durata. Uno degli scopi principali del feedback relativo agli incrementi è che permettere di capire se sono stati dimensionati bene o male gli incrementi e se vengono usate correttamente le tecnologie. \\

VE\_\Data.3 & Mettere gli standard nei riferimenti normativi non è corretto. È più ragionevole pensare ad ispirarsi ad essi e inserirli nei riferimenti informativi. \\

VE\_\Data.4 & Per quanto riguarda la giusta posizione su dove collocare le varie metriche è bene capire il vero significato del \PdQ{}. Esso si divide in due parti. La prima decide quali valori vanno a cruscotto ovvero quali metriche si è intenzionati ad usare con eventuali indicatori. La seconda parte, invece, inoltra i dati al cruscotto ovvero i valori delle metriche e valutazioni per migliorare. \\

VE\_\Data.5 & La descrizione delle metriche utilizzate non necessarie è nel \PdQ{}, nel cruscotto va inserito solamente a cosa serve tale metrica senza ulteriori dettagli. Quest'ultimi vanno inseriti nell'apposita sezione del documento \NdP{}. \\

VE\_\Data.6 & Per quanto riguarda il manuale utente relativo al giusto utilizzo dell'applicazione va bene fornire video tutorial. Per il manuale sviluppatore pare saggia la decisione di presentare una documentazione delle API. Quest'ultima deve essere redatta con buon senso poiché potrebbe tornare utile a futuri sviluppatori qualora volessero riutilizzare il codice prodotto dal gruppo. \\
		
\end{longtable}
}