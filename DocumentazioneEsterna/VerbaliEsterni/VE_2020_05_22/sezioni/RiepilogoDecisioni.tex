 
\section{Riepilogo delle decisioni}
{
\rowcolors{2}{white}{grigetto}
\renewcommand{\arraystretch}{1.5}
\centering
\begin{longtable}{ >{\centering}p{0.20\textwidth} >{}p{0.70\textwidth}}

\caption{Risposte e consensi ai dubbi affrontati durante la riunione esterna del \Data}\\

\rowcolor{darkblue}

\textcolor{white}{\textbf{Codice}} & \textcolor{white}{\textbf{Decisione}} \\	
		
VE\_\Data.1 &  Si è deciso di rivedere il modo in cui vengono registrati gli ingressi e le uscite da una organizzazione(al momento viene utilizzato un booleano). \\
		
VE\_\Data.2 & Davide Zanetti ci ha chiesto di riflettere riguardo alla comunicazione tramite polling implementata nel prodotto in quanto prevede molte chiamate tra front-end e back-end. Davide ci ha suggerito di optare per una comunicazione di tipo web socket in modo da avere un flusso di scambio di dati continuo e in tempo reale tra front-end e back-end. \\

VE\_\Data.3 & Per quanto riguarda l'applicazione mobile Davide Zanetti ci ha chiesto di riorganizzare alcune funzionalità relative alla gestione della lista delle organizzazioni. Per esempio ci ha suggerito di inserire la funzionalità di ordinamento alfabetico della lista delle organizzazioni nel menù in alto a desta dell'applicazione in modo da renderlo più intuitivo ed accessibile. Inoltr  si è deciso di inserite uno storico degli accessi in modo da raggruppare tutte le entrare e uscite da parte di un utente presso un'organizzazione in una interfaccia dedicata. \\

VE\_\Data.2 & Nel complesso il feedback restituitoci da Davide Zanetti è per lo più positivo, rivelando una buona maturità del prodotto software. \\		
\end{longtable}
}