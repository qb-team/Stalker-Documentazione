\section{Informazioni Generali}
\begin{itemize}
\item \textbf{Luogo:} Hangouts;
\item \textbf{Data:} \Data;
\item \textbf{Ora:} 18:30 - 19:20;
\item \textbf{Partecipanti del gruppo:}
	\begin{itemize}
	\item \AT{}; 
	\item \CE{}; 
	\item \DF{};
	\item \LD{};
	\item \PF{};
	\item \SE{};
	\item \BR{};
	\item \MC{}.
	\end{itemize} 
\item \textbf{Segretario:} \MC{}.
\end{itemize}


\section{Ordine del Giorno}
\begin{itemize}
	\item Chiamata con il proponente tramite Hangouts;
	\item Discussione con il proponente sulle tecnologie da utilizzare e richiesta di spiegazioni ai punti non chiari da parte dei membri del gruppo.
\end{itemize}

\section{Resoconto}
Nella discussione sono state affrontate svariate tematiche riguardanti alcune domande o dubbi legate alle tecnologie, in seguito ci sono tutte le risposte del proponente:

\subsection{Quale versione del SDK di Android si dovrebbe utilizzare?}
Ha consigliato di vedere il sito di Android per le best practices (\href{https://developer.android.com/distribute/best-practices}{link al sito}) dove è possibile visionare quale versione \glo{API} di Android sarebbe opportuna da utilizzare. 
Consiglia, al minimo, le \glo{API} 28 corrispondenti alla versione 9 di Android per poter supportare a pieno i servizi Google Play.

\subsection{Si possono utilizzano le API di Google per la geolocalizzazione?}
Le si possono usare tranquillamente, cercando però di non utilizzarle costantemente, soprattutto per fare test: bisogna stare attenti nell'utilizzo frequente della posizione perché si rischia di esaurire il numero di richieste gratuite.
Sarebbe opportuno non inviare richieste di controllo di posizione al di sotto del secondo ed in caso valutare di utilizzare dei mock (oggetti simulati che riproducono il comportamento degli oggetti reali in modo controllato).

\subsection{Accertamenti sull'autenticazione}
Il gruppo intende creare le \glo{API} di \glo{autenticazione} appoggiandosi al servizio esterno Firebase di Google, per poter far utilizzare l'applicazione agli utenti ed avere accesso a qualsiasi funzionalità dell'applicazione.
Per chi è amministratore e non usa l'applicazione Android, ma la web-app, dovrà gestire l'\glo{organizzazione} e per questo scopo utilizzerà sempre l'autenticazione che si appoggia a Firebase. I dipendenti che devono venire 
\glo{tracciati} tramite \glo{tracciamento autenticato} (come richiesto nel capitolato) richiedono un account aggiuntivo in fase di inserimento dell'\glo{organizzazione}, che è quello che usano nella propria organizzazione (account \glo{LDAP}).\\
Il proponente ha detto che va bene e consiglia di far \glo{autenticare} l'utente in maniera diretta (direttamente al servizio e non attraverso le nostre API), perché cosi la password dell'utente non viene inviata in chiaro. In generale la password in chiaro non deve mai girare.

\subsection{Si possono utilizzare Swagger e la specifica OpenAPI per realizzare il backend?}
Il proponente afferma che va bene usarlo come tipologia di approccio \textit{contract first}, ovvero prima definire il contratto delle \glo{API} e poi implementarlo. Il proponente 
consiglia di prepararsi il file YAML con Swagger e successivamente di utilizzare il plugin per Maven "OpenAPI Generator" a partire da quel file per generare il codice automaticamente, ma in maniera decisamente più controllata di quella propria di Swagger Hub.
Swagger Hub infatti genera codice automatico senza il controllo dell'utente, rendendolo troppo espansivo. 
Questo può essere comodo nel momento in cui bisogna fare delle modifiche al codice (ad esempio per errori di battitura), permettendo di modificare in un singolo punto cose che altrimenti sarebbero state da modificare in molteplici file.

\subsection{Un'autenticazione con LDAP fornisce un token come per gli altri sistemi di autenticazione o è una cosa totalmente diversa?}
Il proponente conferma che con LDAP è possibile ottenere l'access token, che supporta OAuth.

\subsection{La base di dati da realizzare è relazionale e le scelte da utilizzare sono MySQL o PostgreSQL, possono andare bene?}
Il proponente conferma e allo stesso tempo consiglia MySQL perché molto più semplice di PostgreSQL. Successivamente fa notare che i problemi possono sorgere quando bisogna tenere conto dei \glo{tracciamenti}: visionarli in tempo reale su delle tabelle potrebbe essere eccessivamente dispendioso con un database relazionale.
Consiglia fortemente di usare software come Apache Kafka o Redis per questi casi, permettendo di avere l'accesso a queste informazioni come stream asincrono e in maniera tale che l'informazione rimanga memorizzata su file e 
sotto forma di log e non tabellare. Quindi le varie entry vengono segnalate come movimenti degli utenti e dopo, dall'altro lato queste informazioni vengono "scodate" e vengono inserite nel database relazionale appena possibile.

\subsection{Kafka lo si può usare anche per gestire ad esempio le richieste dell'applicazione?}
Kafka dovrebbe essere utilizzato esclusivamente per le richieste asincrone e non per quelle sincrone. Utilizzare Kafka per richieste sincrone non avrebbe alcun senso.

\subsection{Kafka serve per risolvere i problemi che sorgono nel pattern Publisher/Subscriber, per quale motivo lo nominate se poi viene utilizzato così di rado nel sistema?}
Il pattern Publisher/Subscriber non lo si usa soltanto di rado, è un pattern che lo si usa per disaccoppiare determinate operazioni e permette di registrarsi in determinati canali per fare queste operazioni.
In questo contesto lo si utilizzerebbe fortemente per gestire tutti i dati mandati dalle varie \glo{API} che forniscono informazioni di \glo{tracciamento}; un sistema 
che non riesce a gestire tutto questo traffico rischierebbe di collassare proprio per il fatto che l'informazione arriverebbe in modo sincrono.

\subsection{Per altre operazioni come ad esempio lo storico degli accessi o la generazione di report tabellari, sarebbero operazioni sincrone quindi Kafka risulterebbe poco pratico, sarebbe meglio prendere i dati da MySQL?}
Il proponente conferma.

\subsection{Come PoC pensavamo di mostra l'autenticazione con Firebase e forse con LDAP, se si entra in un luogo di un'\glo{organizzazione} gli amministratori vedono il \glo{tracciamento} effettuato. Potrebbe andare bene?}
Il proponente ha fatto capire che già riuscire a fare tutto questo sarebbe un buon punto di partenza.

\clearpage