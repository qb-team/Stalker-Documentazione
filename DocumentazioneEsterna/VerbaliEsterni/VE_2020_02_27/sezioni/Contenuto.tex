\section{Informazioni Generali}
\begin{itemize}
\item \textbf{Luogo:} Hangouts;
\item \textbf{Data:} \Data;
\item \textbf{Ora:} 18:30 - 19:20;
\item \textbf{Partecipanti del gruppo:}
	\begin{itemize}
	\item \AT{}; 
	\item \CE{}; 
	\item \DF{};
	\item \LD{};
	\item \PF{};
	\item \SE{};
	\item \BR{};
	\item \MC{}.
	\end{itemize} 
\item \textbf{Segretario:} \MC{}.
\end{itemize}


\section{Ordine del Giorno}
\begin{itemize}
	\item Chiamata con il proponente tramite Hangouts;
	\item Discussione con il proponente sulle tecnologie da utilizzare e richiesta di spiegazioni ai punti non chiari da parte dei membri del gruppo.
\end{itemize}

\section{Resoconto}
Nella discussione sono state affrontate svariate tematiche riguardanti alcune domande o dubbi legate alle tecnologie, in seguito ci sono tutte le risposte del proponente:

\subsection{Quale versione di SDK Android si dovrebbe utilizzare?}
Ha consigliato di vedere il developer Android best practies dove è possibile visionare quale versione \glo{API} di Android sarebbe opportuna da utilizzare. 
Come minimo deve essere l'\glo{API} 28 versione 9 di Android per poter supportare Google Play.

\subsection{Si possono utilizzano le API di Google per la geolocalizzazione?}
Le si possono usare tranquillamente cercando di non andare subito nella \glo{API} di Google; bisogna stare attenti nella richiesta frequente della posizione perché si rischia 
di esaurire le chiamate gratuite. Quindi sarebbe opportuno non utilizzare una richiesta di posizione che va al di sotto dei secondi ed in caso valutare di utilizzare dei 
mock (oggetti simulati che riproducono il comportamento degli oggetti reali in modo controllato).

\newpage
\subsection{Va bene utilizzare l'autenticazione con Google per gli utenti che desiderano entrare nelle fiere/organizzazioni pubbliche mentre per gli account dei dipendenti 
di organizzazioni aziendali  credenziali LDAP?}
Si è possibile agganciare l'account dell'applicazione con quella di Google e \glo{LDAP} per i dipendenti aziendali.

\subsection{Accertamenti sull'autenticazione}
Stiamo pensando di creare noi le \glo{API} di \glo{autenticazione} che si appoggia al servizio esterno Firebase di Google, cosi è possibili far utilizzare  l'applicazione agli utenti 
che serve per avere accesso a qualsiasi funzionalità dell'applicazione e anche per chi è amministratore che dovrà gestire l'\glo{organizzazione}. I dipendenti che devono venire 
\glo{tracciati} tramite \glo{autenticazione} (come richiesto nel capitolato) ne richiediamo un account aggiuntivo in fase di inserimento dell'\glo{organizzazione} che è quello che usano nella 
loro azienda (\glo{LDAP}). \\
Il proponente ha detto che va bene e consiglia di far \glo{autenticare} l'utente in maniera diretta, perché cosi la password all'utente non la date. In generale la password in chiaro non 
deve mai girare.

\subsection{Si può utilizzare Swagger OpenAPI per realizzare il backend?}
Ha detto che va bene usarlo come tipologia di approccio il contract first, ossia prima definisci il contratto delle \glo{API} e poi si implementa la parte di codice.  Il proponente 
consiglia di prepararsi il Yaml con Swagger  e dopo di utilizzare OpenAPI Generator Maven \glo{Plugin} su questo Yaml per generare il codice come le classi, interfacce, ecc. 
Questo può essere comodo nel momento in cui bisogna fare delle modifiche al codice, che lo si farà direttamente sul Yaml e non sull'output del \glo{Plugin}.

\subsection{Un'autenticazione su LDAP si ottiene un token come nelle altre autenticazioni o è una cosa totalmente diversa?}
Si è possibile ottenere l'access token anche da \glo{LDAP} e supporta OAuth. Quindi nell'ambito su Firebase è possibile richiedere all'utente di \glo{autenticarsi} tramite \glo{LDAP} ottenendo 
l'access token ed in seguito di restituirlo per verificare che sul \glo{LDAP} sia tutto valido.

\subsection{La base di dati che vogliamo utilizzare è relazionale e le scelte da utilizzare sono MySQL o PostgreSQL, andrebbero bene?}
Ha detto che va bene ma i problemi possono sorgere quando bisogna tenere conto dei \glo{tracciamenti} e visionarli in tempo reale su delle tabelle. Consiglia fortemente di usare 
Kafka o Redis per questi casi, permettendo di avere l'accesso di queste informazioni come stream asincrono in maniera tale che l'informazione rimanga persistita su un file e 
quindi lo si ritrova come una sorta di log. Quindi si ha tutte le varie entry e vengono segnalate le entrate ed uscite degli utenti e dopo, dall'altro lato queste informazioni 
vengono scodate e vengono inserite nel database relazionale con la dovuta calma. 

\subsection{Kafka lo si può usare anche per gestire ad esempio le richieste dell'applicazione?}
Kafka viene utilizzato esclusivamente per le richieste asincrone e non per quelle sincrone, ottenendo cosi dei risultati istantanei dall'altro lato. 
Se ci fosse un disaccoppiatore nel mezzo non funzionerebbe.

\subsection{Kafka serve per risolvere i problemi che sorgono nel pattern publisher subscriber, per quale motivo lo nominate se poi viene utilizzato cosi di rado nel sistema?}
Il publisher subscriber non lo si usa soltanto di rado, esso è un pattern che lo si usa per disaccoppiare determinate operazioni e permette di registrarti in determinati canali 
per fare queste operazioni. In questo contesto lo si utilizzerebbe fortemente per gestire tutti i dati mandati dalle varie \glo{API} che forniscono informazioni di \glo{tracciatura}; un sistema 
che non riesce a gestire tutto questo traffico rischierebbe di collassare proprio per il fatto che l'informazione arriverebbe in modo sincrono.

\subsection{Per altre operazioni come ad esempio lo storico degli accessi o mostrami un report tabellare, sarebbero operazioni sincrone quindi Kafka risulterebbe poco pratico, 
sarebbe meglio prendere i dati da MySQL?}
Il proponente ha detto che va bene usarlo per queste operazioni.

\subsection{Come PoC pensavamo di usare FireBase e forse il LDAP, mentre per l'applicazione trasmette le coordinate di posizione e se si entra in un luogo di un'\glo{organizzazione} gli 
amministratori vedono il \glo{tracciamento} effettuato. Potrebbe andare bene?}
Il proponente ci ha fatto capire che se riusciamo già a fare tutte queste procedure sarebbe ottimo.

\clearpage