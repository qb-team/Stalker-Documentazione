\section{Informazioni Generali}
\begin{itemize}
\item \textbf{Luogo:} Aula 1A150 presso Torre Archimede;
\item \textbf{Data:} \Data;
\item \textbf{Ora:} 14:30 - 16:30;
\item \textbf{Partecipanti del gruppo:}
	\begin{itemize}
	\item Azzalin Tommaso; 
	\item Cisotto Emanuele; 
	\item Drago Francesco;
	\item Lazzaro Davide;
	\item Perin Federico;
	\item Salmaso Enrico;
	\item Baratin Riccardo;
	\item Mattei Christian.
	\end{itemize} 
\item \textbf{Segretario:} Azzalin Tommaso.
\end{itemize}

\clearpage

\section{Ordine del Giorno}
\begin{itemize}
	\item Presentazione dei membri del gruppo al proponente;
	\item Discussione con il proponente di idee sul progetto e richiesta di spiegazioni a punti non chiari de parte dei membri del gruppo.
\end{itemize}

~\\

\section{Resoconto}
\subsection{Presentazione dei membri del gruppo al proponente}
Il gruppo si è presentato al proponente, l'azienda Imola Informatica, e ha esposto la propria volontà di portare avanti il progetto \NomeProgetto.

\subsection{Discussione con il proponente di idee sul progetto \NomeProgetto e richiesta di spiegazioni a punti non chiari de parte dei membri del gruppo}
Il gruppo si era preparato per l'incontro le seguenti domande (che aveva precedentemente comunicato tramite mail al proponente):
\begin{enumerate}
	\item Quali sono i riferimento normativi sulla privacy da tenere in considerazione?
	\item Quali sono le differenze fra tracciamento anonimo e autenticato? Perché un'organizzazione che richiede il tracciamento autenticato avrebbe bisogno di un pulsante per abilitare il tracciamento anonimo?
	\item Qual è la licenza più opportuna per lo sviluppo del progetto software?
	\item Spiegazione più in dettaglio delle tecniche per il tracciamento degli utenti della posizione degli utenti, utilizzando le varie tecnologie indicate nella presentazione del capitolato.
	\item Spiegazione di alcune tecniche relative alla scalabilità del server, all'autenticazione degli utenti nell'app e nel server e al protocollo LDAP.
\end{enumerate}
\subsubsection*{Riferimenti normativi sulla privacy da tenere in considerazione}
Sicuramente è da tenere in considerazione la recente normativa a livello europeo in tema di privacy, il GDPR. \\
Il GDPR presta molta attenzione al trattamento dei dati personali e richiede ad esempio che, se una base di dati aziendale è affidata a terzi, questa non deve risultare accessibile se non dal proprietario della stessa.
Inoltre, tutte le informazioni sulle attività di un utente e i dati personali dell'utente devono essere strettamente separati e non accessibili assieme. Il proponente ha fatto un semplice esempio: se viene usato un database
NoSQL in cui tipicamente i dati sono salvati in file di testo in un formato specifico (per esempio JSON), non è possibile che siano presenti assieme i dati personali di un utente e attività da lui svolte, che in questo progetto
sarebbero gli accessi ai luoghi delle organizzazioni.
Il proponente ha consigliato di fare prima uno studio sugli attori e sulle informazioni necessarie alla loro definizione e, una volta che è chiaro cosa bisogna memorizzare per identificare tutti questi, di pensare al tipo di database.
Successivamente ha consigliato, una volta aver definito i casi d'uso, di verificare che essi rispettino il GDPR. Questo è possibile farlo seguendo l'elenco presente nel seguente \href{https://gdpr.eu/checklist/}{sito}.

\subsubsection*{Differenze fra tracciamento anonimo e autenticato. Motivazione della necessità di avere un pulsante per attivare il tracciamento anonimo in organizzazioni con tracciamento autenticato}
La differenza fra le due modalità di tracciamento è stata confermata essere quella indicata nel documento di presentazione del capitolato.\\
La motivazione per il pulsante di cui sopra è stata illustrata con due esempi:
\begin{itemize}
	\item un dipendente di un'azienda che ha le chiavi della sede, può entrare quando vuole per svolgere delle attività, ma non vuole essere visto come disponibile per essere contattato;
	\item un dipendente di una strttura fieristica che si trova a un evento presso il suo stesso luogo di lavoro come visitatore e non come lavoratore, non vuole essere visto come disponibile (per lo stesso motivo del precedente).
\end{itemize}
Quando un utente passa nella modalità di tracciamento anonimo ed è ancora presso un luogo di un'organizzazione, viene terminato il suo accesso e genera un accesso con tracciamento anonimo presso lo stesso luogo.
Il proponente richiede che in generale le scelte implementative non specificate nel capitolato vanno documentate, analizzando i pro e i contro di ognuna. Tiene anche a specificare che lo scopo del progetto non è
tracciare gli orari dei dipendenti e che quindi non è fondamentale concentrarsi troppo sull'aspetto dell'orario negli accessi ai luoghi.

\subsubsection*{Licenza per lo sviluppo del progetto}
Il proponente consiglia di guardare fra le seguenti licenze:
\begin{itemize}
	\item GNU GPL;
	\item LGPL;
	\item MIT.
\end{itemize}
L'importante, dice, è che sia una licenza open source. Consiglia di visitare il seguente \href{https://opensource.org/licenses}{link}.

\subsubsection*{Tecniche di tracciamento degli utenti}
Il dipendente dell'azienda proponente presente afferma di non essere la persona più adatta per rispondere a questa domanda e di chiedere al suo collega (di cui ci fornisce l'indirizzo mail) maggiori informazioni a riguardo.

\subsubsection*{Tecniche relative alla scalabilità del server, all'autenticazione degli utenti nell'app e nel server e al protocollo LDAP}
Riguardo all'autenticazione degli utenti nell'app, il proponente afferma che la sua idea è che quando un utente accede per la prima volta nell'applicazione, debba autenticarsi con delle credenziali per la piattaforma.
Le credenziali di questa utenza dovrebbe essere cosa diversa dalle credenziali di un'organizzazione: lo scopo di queste è poterle riutilizzare per l'accesso attraverso un altro telefono all'applicazione e avere già
memorizzate tutte le informazioni precedentemente salvate nell'applicazione (usando l'altro telefono).\\
\clearpage