\section{Informazioni Generali}
\begin{itemize}
\item \textbf{Luogo:} Aula 1A150 presso Torre Archimede;
\item \textbf{Data:} \Data;
\item \textbf{Ora:} 14:30 - 16:30;
\item \textbf{Partecipanti del gruppo:}
	\begin{itemize}
	\item Azzalin Tommaso; 
	\item Cisotto Emanuele; 
	\item Drago Francesco;
	\item Lazzaro Davide;
	\item Perin Federico;
	\item Salmaso Enrico;
	\item Baratin Riccardo;
	\item Mattei Christian.
	\end{itemize} 
\item \textbf{Segretario:} \Redattori.
\end{itemize}

\clearpage

\section{Ordine del Giorno}
\begin{itemize}
	\item Presentazione dei membri del gruppo al proponente;
	\item Discussione con il proponente di idee sul progetto e richiesta di spiegazioni a punti non chiari de parte dei membri del gruppo.
\end{itemize}

~\\

\section{Resoconto}
\subsection{Presentazione dei membri del gruppo al proponente}
Il gruppo si è presentato al proponente, l'azienda Imola Informatica, e ha esposto la propria volontà di portare avanti il progetto \NomeProgetto.

\subsection{Discussione con il proponente di idee sul progetto \NomeProgetto  e richiesta di spiegazioni su punti non chiari de parte dei membri del gruppo}
Il gruppo ha preparato per l'incontro le seguenti domande (precedentemente comunicate per e-mail al proponente):
\begin{enumerate}
	\item Quali sono i riferimento normativi sulla privacy da tenere in considerazione?
	\item Quali sono le differenze fra tracciamento anonimo e autenticato? Perché un'organizzazione che richiede il tracciamento autenticato dovrebbe aver bisogno di un pulsante per abilitare il tracciamento anonimo?
	\item Qual è la licenza più opportuna per lo sviluppo del progetto software?
	\item Spiegazione più dettagliata delle tecniche per il tracciamento della posizione degli utenti, utilizzando le varie tecnologie indicate nella presentazione del capitolato.
	\item Spiegazione di alcune tecniche relative alla scalabilità del server, all'autenticazione degli utenti nell'app e nel server e al protocollo LDAP.
\end{enumerate}
Inoltre, essendo l'incontro stato fatto assieme alle componenti degli altri gruppi interessati al capitolato, sono stati discussi altri punti descritti in seguito.

\subsubsection*{Riferimenti normativi sulla privacy da tenere in considerazione}
È certamente da tenere in considerazione la recente normativa a livello europeo in tema di privacy, il GDPR. \\
Il GDPR presta molta attenzione al trattamento dei dati personali e richiede ad esempio che, se una base di dati aziendale è affidata a terzi, questa non risultari accessibile se non dal proprietario della stessa.
Inoltre, tutte le informazioni sulle attività di un utente e i suoi dati personali devono essere strettamente separati e non accessibili assieme. Il proponente ha fatto un semplice esempio: se viene usato un database
NoSQL in cui tipicamente i dati sono salvati in file di testo in un formato specifico (per esempio JSON), non è possibile che siano presenti assieme i dati personali di un utente e attività da lui svolte, che in questo progetto
sarebbero gli accessi ai luoghi delle organizzazioni.
Il proponente ha consigliato di fare prima uno studio sugli attori e sulle informazioni necessarie alla loro definizione e, una volta che è chiaro cosa bisogna memorizzare per identificare tutti questi, di pensare al tipo di database.
Successivamente ha consigliato, una volta aver definito i casi d'uso, di verificare che essi rispettino il GDPR. Questo è possibile farlo seguendo l'elenco presente nel seguente \href{https://gdpr.eu/checklist/}{sito}.

\subsubsection*{Differenze fra tracciamento anonimo e autenticato e motivazione della necessità di avere un pulsante per attivare il tracciamento anonimo in organizzazioni con tracciamento autenticato}
La differenza fra le due modalità di tracciamento è stata confermata essere quella indicata nel documento di presentazione del capitolato.\\
La motivazione per il pulsante di cui sopra è stata illustrata con due esempi:
\begin{itemize}
	\item un dipendente di un'azienda che ha le chiavi della sede, può entrare quando vuole per svolgere delle attività, ma non vuole essere visto come disponibile per essere contattato;
	\item un dipendente di una struttura fieristica che si trova a un evento presso il suo stesso luogo di lavoro come visitatore e non come lavoratore, non vuole essere visto come disponibile (per lo stesso motivo del precedente).
\end{itemize}
Quando un utente passa nella modalità di tracciamento anonimo ed è ancora presso un luogo di un'organizzazione, viene terminato il suo accesso e genera un accesso con tracciamento anonimo presso lo stesso luogo.
Il proponente richiede che in generale le scelte implementative non specificate nel capitolato vengano documentate, analizzando i pro e i contro di ognuna. Tiene anche a specificare che lo scopo del progetto non è
tracciare gli orari dei dipendenti e che quindi non è fondamentale concentrarsi troppo sull'aspetto dell'orario negli accessi ai luoghi.

\subsubsection*{Licenza per lo sviluppo del progetto}
Il proponente consiglia di guardare fra le seguenti licenze:
\begin{itemize}
	\item GNU GPL;
	\item LGPL;
	\item MIT.
\end{itemize}
L'importante è, dice, che sia una licenza open source. Consiglia di visitare il seguente \href{https://opensource.org/licenses}{link}.

\subsubsection*{Tecniche di tracciamento degli utenti}
Il dipendente dell'azienda proponente presente afferma di non essere la persona più adatta per rispondere a questa domanda e di chiedere al suo collega (di cui ci fornisce l'indirizzo mail) maggiori informazioni a riguardo.

\subsubsection*{Tecniche relative alla scalabilità del server, all'autenticazione degli utenti nell'app e nel server e al protocollo LDAP}
Riguardo all'autenticazione degli utenti nell'app, il proponente afferma che la sua idea è che quando un utente accede per la prima volta all'applicazione, egli debba autenticarsi con delle credenziali per la piattaforma.
Le credenziali di questa utenza dovrebbe essere cosa diversa dalle credenziali di un'organizzazione: lo scopo di queste è poterle riutilizzare per l'accesso attraverso un altro telefono all'applicazione e avere già
memorizzate tutte le informazioni precedentemente salvate nell'applicazione (usando l'altro telefono).\\
Riguardo al protocollo LDAP da utilizzare per autenticare gli utenti a cui viene richiesto il tracciamento autenticato, il proponente non vuole che essa sia fornita dal sistema da realizzare.
In altre parole, non è compito del sistema autenticare gli utenti mediante LDAP: esso si deve limitare a fornire l'indirizzo a cui effettuare l'autenticazione e richiedere il risultato del processo.
Quindi è compito dell'amministratore di un'organizzazione inserire le informazioni sul proprio server LDAP (per esempio quello aziendale) all'interno del sistema.
Per testare il funzionamento di un server LDAP, il proponente consiglia il software \href{https://www.openldap.org/}{OpenLDAP}.\\
Riguardo alla scalabilità del server non si è riusciti ad ottenere una risposta per mancanza di tempo. In caso di problemi su questo tema, verrà nuovamente contattato il proponente.

\subsection*{Precisione richiesta per l'area che definisce un luogo di un'organizzazione}
La domanda è stata posta da un componente di un altro gruppo.\\
Il proponente a riguardo dice che è sufficiente approssimarla ad una figura geometrica (per esempio un quadrilatero), che per la maggior parte dei casi dovrebbe risultare più che sufficiente.
L'importante è avere una precisione accettabile per il conteggio degli accessi, tenendo conto delle limitazioni delle tecnologie per il tracciamento (come il GPS, che è attorno ai 5 m circa).
Un componente di un gruppo fa sapere che sono disponibili API (cita \href{https://www.openstreetmap.org/}{OpenStreetMap API]) che permettono di sapere se un punto appartiene o meno ad un contorno delimitato da una poli-linea.

\clearpage