\section{Riepilogo delle decisioni}
{
\rowcolors{2}{white}{grigetto}
\renewcommand{\arraystretch}{1.5}
\centering
\begin{longtable}{ >{\centering}p{0.20\textwidth} >{}p{0.70\textwidth}}

\caption{Risposte e consensi ai dubbi affrontati durante la riunione esterna del \Data}\\

\rowcolor{darkblue}
	\textcolor{white}{\textbf{Codice}} & \textcolor{white}{\textbf{Decisione}} \\
	VE\_\Data.1 & Utilizzare metriche che devono essere professionalmente utili e indispensabili al gruppo; \\	
	VE\_\Data.2 & Costruire delle buone prassi che facciano da guida e che mostrino che tutto sia standardizzato; \\
	VE\_\Data.3 & Inserire nelle norme i pattern architetturali utilizzati; \\
	VE\_\Data.4 & Scrivere la prassi che utilizziamo per comunicare in questo periodo di emergenza dovuta al virus; \\
	VE\_\Data.5 & Nelle norme ci va il livellamento all'alto del modo in cui lavoriamo, cioè le nostre migliori prassi.;\\
	VE\_\Data.6 & Tutto è un albero in cui alla radice c'è il prodotto e nei vari rami ci sono le varie parti del prodotto (software e documenti); \\
	VE\_\Data.7 & Si può inserire nella repository solo l'esito della verifica su ciò che si vuole inserire è positivo cioè rispetta le regole di verifica definite ; \\
	VE\_\Data.8 & Il numero di versione deve riflettere il livello di profondità all'interno della repository; \\
	VE\_\Data.9 & Viene riflesso nell'indice di prodotto la storia del prodotto, mostrando pero le modifiche più rilevanti non tutte; \\
	VE\_\Data.10 & È sconsigliato associare a una milestone un numero di versione oltre il quale non si può andare fino a quando non scade la milestone \\
	VE\_\Data.11 & Non si devono applicare diverse regole di numerazione per i numeri di versione del documento e del prodotto software, ma si devono usare le stesse regole; \\
	VE\_\Data.12 & Si posso scrivere regole che prevedono cambi unitari o cambi asincroni; \\
	VE\_\Data.13 & Se il nostro intento è di fare le cose per bene è consentito un eventuale slittamento della revisione. \\
	
\end{longtable}
}

