\section{Informazioni Generali}
\begin{itemize}
\item \textbf{Luogo:} Piattaforma Zoom;
\item \textbf{Data:} \Data;
\item \textbf{Ora:} 16:00 - 17:00;
\item \textbf{Partecipanti del gruppo:}
	\begin{itemize}
	\item \AT{}; 
	\item \CE{}; 
	\item \DF{};
	\item \LD{};
	\item \PF{};
	\item \SE{};
	\item \BR{};
	\item \MC{}.
	\end{itemize} 
\item \textbf{Partecipanti esterni:}
\begin{itemize}
	\item \VT{};
\end{itemize} 

\item \textbf{Segretario:} \PF{}.

\end{itemize}


\section{Ordine del Giorno}
\begin{itemize}
	\item Chiamata con il \VT{};
	\item Discussione con il docente sulle segnalazioni individuate nella fase di Revisione di Progettazione e richiesta di spiegazioni ai punti non chiari da parte dei membri del gruppo.
\end{itemize}

\section{Resoconto}
Nella discussione sono state poste alcune domande riguardanti legate alle segnalazioni fatte nell'esito della fase di RP in modo da chiarire alcuni dubbi, in seguito ci sono tutte le risposte del docente:

\subsection{Presentazione: Le metriche non del tutto convincenti? Cosa significa?}
Usiamo le metriche per oggettivare le nostre affermazioni. Nel nostro caso le metriche applicate trasmetto l'idea che il gruppo le adotti perché viene chiesto dal proponente di farlo, senza che queste vengano adottate per migliorare la qualità del prodotto. Nella nostra presentazione le metriche esposte non erano particolarmente professionalmente utili a noi.  

\subsection{Norme di Progetto: Aspettativa normazione tecnica relativa alla progettazione}
Resta debole perché non ci sono regole che normano la Product Baseline ovvero la progettazione, le norme di progetto sono utili se si riescono a misurare, nel nostro caso non c'è nulla da misurare perché non sono buone. Man mano che si acquisisce esperienza si posso inserire norme riguardanti la progettazione. Serve qualcosa che dica la nostra prassi comune costruita sulla base nella nostra esperienza, all'inizio si avranno delle bozze di norme che nel tempo vengono corrette e migliorate, grazie alla esperienza acquisita nel tempo dal team. Riguardo ai pattern architetturali è importante essere coerenti, cioè nel dichiarare quali pattern vengono utilizzati, si deve riflettere nel codice la corretta implementazione, è importante capirne il contesto di utilizzo del pattern, quali obblighi porta ad avere e talvolta l'utilizzo di certe tecnologie impongono l'utilizzo di specifici pattern, perché già implementanti in essi oppure che funzionano solo con specifici pattern.

\subsection{Norme di Progetto: Distanziamento sociale, che norme?}
Noi utilizziamo un modo di comunicare funzionante ma, al momento non è scritto nelle norme, o meglio c'è scritto ma è una versione non aggiornata rispetto ai tempi attuali (quarantena obbligatoria). Nelle norme ci va il livellamento all'alto del modo in cui lavoriamo, cioè le nostre migliori prassi. 

\subsection{Norme di Progetto/Piano di Progetto: Cosa si intende per versionamento sia per i documenti sia per il prodotto?}
Nella realta ogni sviluppatore non ha un singolo repository ma ne ha più di uno, e dentro a questi repository contiene tutto ciò che il cliente ha chiesto nel contratto. Noi abbiamo un contratto con il proponente che ci chiede documenti e prodotto software, idealmente si dovrebbe avere un singolo repository con dentro tutti gli aggregati del prodotto richiesto, pero tutto non può stare in una sola repository ma vengono messi da altre parti. Il concetto è "tutto è un albero in cui alla radice c'è il prodotto e nei vari rami ci sono le varie parti del prodotto (software e documenti)", in questa repository (l'albero) si possono inserire documenti o parti di software se l'esito della verifica o approvazione è positivo, secondo a regole di verifica applicate su cioè che deve essere inserito. Quando viene inserito nella repository qualcosa di nuovo, anche una modifica, questo cambia il prodotto il quale assume un nuovo numero di versione che riflette il cambiamento.\\
Nel concreto si devono avere delle regole che verifica ogni cosa(item) che si inserisci all'interno della repository. Quando Il numero di versione cambia riflette la profondità della modifica sulla repository. Viene riflesso nell'indice di prodotto la storia del prodotto, mostrando pero le modifiche più rilevanti non tutte. Il numero di versione del prodotto evolve riflettendo le modifiche fatte al suo repository, anche a tutte le parti del prodotto, secondo qualche regola. Si può associare un numero di versione a una milestone in modo tale che il numero di versione evolva nel periodo che va fino alla milestone e che resti sempre al di sotto del numero di versione della milestone, per via della nostra inesperienza non possiamo applicare tale regole perché possiamo fare una grossa modifica che scatta prima delle 4 milestone. Non si devono applicare diverse regole di numerazione per i numeri di versione del documento e del prodotto software, ma si devono usare le stesse regole, perché fanno tutti parte del prodotto finale e inoltre, non c'è più la relazione tra documenti richiesti e software da fare. Si posso scrivere regole che prevedono cambi unitari o cambi asincroni.

\subsection{Eventuale slittamento Revisione di Qualifica}
Se il nostro intento è di fare le cose per bene è consentito un eventuale slittamento della revisione.

\clearpage